\chapter{Introduction}

\section{What is $\Psi$lab}
\index{signals!saving and loading}

 Since the introduction of the ``classic'' Fortran MATLAB 
by C. Moler in 1982 there have been a number of interactive software 
packages which have been developed for the study of systems theory and 
signal processing.

$\Psi$lab is an extension of the classic MATLAB, developed at INRIA.
It is freely distributed in source code format (see the file {\tt notice.tex}). 
%It is in part based on a previous software package, BASILE. 
The code which is provided is made for a Unix/Xwindow environment. 
Its libraries and most of the interpreter are written in Fortran to keep 
a better compatibility with numerical librairies. 
The graphic facilities and the Unix interface are written in C. 
$\Psi$lab is made of three distinct parts: an interpreter, 
librairies of macros ($\Psi$lab procedures) and libraries of Fortran 
and C routines. 
These routines (which, strictly speaking, do not belong to $\Psi$lab but
are interactively called by the interpreter) are of 
independent interest and most of them are available through netlib. 
A few a them have been slightly modified for a better compatibility
with $\Psi$lab's interpreter.
A very useful tool which is distributed with $\Psi$lab is {\tt intersci} which
is a set of routines which allow to add easily new primitives into
$\Psi$lab i.e. to add new modules of Fortran or C code into $\Psi$lab.
 Thus the facilities which are available may change from one machine
to an other: $\Psi$lab is organized in such a way that it is easily
customized. The code which is
distributed should be seen as an example; since $\Psi$lab is distributed with
the complete source code as well as the Makefile files which build $\Psi$lab,
it should be easy to adapt $\Psi$lab to your needs and in particular to
add new Netlib routines to it.
 
A key feature of the MATLAB syntax is its ability to handle matrices: 
basic matrix manipulations such as concatenation, 
extraction or transpose are immediately performed as well as basic operations
such as addition or mutiplication. 
$\Psi$lab's aims are the following: first to use the MATLAB syntax 
for more complex objects than numerical matrices,
(e.g. control people may want to manipulate transfer matrices) and
second to be an open interface to numerical libraries (e.g. a specific  
routine can be either called dynamically from $\Psi$lab or included 
in the package as a new primitive). 

$\Psi$lab is an interactive, interpreted software package with 
the MATLAB syntax which has a number of powerful features.  
%
\begin{itemize}
	\item Lists
	\item Symbolic Manipulation of Polynomials and Matrix Polynomials
        \item Symbolic Manipulation of Linear and Non-Linear systems
	\item Non-Linear Calculation: Simulation and Optimization
	\item Easy Interfacing with Numeric Subprograms
\end{itemize}
%
Each one of these features are essential to the development 
of complicated systems analyses.

 The list structure allows a natural symbolic representation of complicated
mathematical objects such as transfer functions and linear systems
(see Section~\ref{s2.5}).

	Polynomials, matrix polynomials and transfer matrices 
are also defined and 
$\Psi$lab permits the definition and manipulation of these objects in a
natural, symbolic fashion (see Section~\ref{s2.4}).  In addition,
the syntax used for manipulating these matrices
is identical to that used for manipulating constant vectors and matrices.

	$\Psi$lab provides a variety of powerful primitives for
the analysis of non-linear systems.  
Integration of explicit and implicit systems can be accomplished 
numerically.  In addition there are numerical optimization 
facilities available for non linear optimization (including
non differentiable optimization), quadratic optimization or 
linear optimization.

	$\Psi$lab has an open programming environment where the
creation of macros and libraries of macros is completely in the
hands of the user (see Chapter~\ref{ch4}).    
Macros are recognized as data objects in $\Psi$lab and, thus, can be 
manipulated or created as other data objects.  For example, macros
can be passed as the arguments of primitives or other macros.

In addition $\Psi$lab supports a character chain data type 
which, in particular, allows the automatic creation of macros.
Matrices of character strings are also manipulated with the same
syntax or ordinary matrices.
	Finally, $\Psi$lab is easily interfaced with Fortran or C 
subprograms (see Chapter~\ref{intersci}).  This permits use of standardized 
packages and libraries in the interpreted environment of $\Psi$lab.

	The general philosophy of $\Psi$lab is to provide the following
sort of computing environment:
\begin{itemize}
   \item To have data types which are varied and flexible.
   \item To have a syntax which is natural and easy to use.
   \item To provide a reasonable set of primitives which serve
	   as a basis for a wide variety of calculations.
   \item To have an open programming environment where new
	   primitives are easily added.
   \item To support library development through ``toolboxes'' of
         macros devoted to specific
	   applications (linear control, signal processing, 
	   networks analysis, non-linear control, etc.)
\end{itemize}

	The objective of this introduction manual is to give the user 
an idea of what $\Psi$lab can do. On line documentation for all the
$\Psi$lab functions is also available.

\section{Sample Session}

$\Psi$lab is called by typing {\tt scilab} (or {\tt scilab}).
These shell scripts  (in the directory {\tt SCIDIR/bin}) run $\Psi$lab 
in an X-window environment or in a pure Unix-environment respectively.

A first contact with $\Psi$lab can be made by clicking on ``demos''
and running the introduction demo.

	To give the user an idea of some of the capabilities of $\Psi$lab
we give the following sample session in $\Psi$lab.\\
\bigskip
{\tt scilab}
\begin{verbatim}

     ===========
     S C I L A B        31-DEC-1990 at 23:28:13 
     ===========

 Scilab Version 1.0 (01-01-1990) Copyright (C) INRIA
   
 -->

\end{verbatim}
After the initial call to $\Psi$lab the $\Psi$lab banner is displayed
and the prompt represented by the {\tt -->}.  Several libraries
(see the {\tt SCIDIR/scilab.star} file) are automatically loaded and 
now the user can evaluate expressions in $\Psi$lab.

	Now for some simple commands.

\noindent.\dotfill.
\begin{verbatim}

-->a=1;
\end{verbatim}
Give the value of 1 to the variable {\tt a}.  The semi-colon
at the end of the command suppresses the display of the result.

\noindent.\dotfill.
\begin{verbatim}
 
-->c=[1 2];b=1.5
 b         =
 
    1.5  
\end{verbatim}
Two commands are processed and the second result is displayed because
it is not followed by a semi-colon.

\noindent.\dotfill.
\begin{verbatim}
 
-->a=1;b=1.5;
 
-->2*a+b^2
 ans       =
 
    4.25  
\end{verbatim}
Here an example of an expression which mixes constants with existing
variables.  The result is retained in the standard default variable 
called {\tt ans}\index{ans@{\tt ans}}.

\noindent.\dotfill.
\begin{verbatim}
 
-->sqrt([4  -4])
 ans       =
 
!   2.    2.i !
\end{verbatim} 
Call to a function (or primitive) with a vector argument.  The response
is a complex vector.

\noindent.\dotfill.
\begin{verbatim}
 
-->p=poly([1 2 3],'s','coeff')
 p         =
 
               2  
    1 + 2s + 3s   
 
-->//p is the polynomial in z with coefficients 1,2,3.
\end{verbatim}
A command a little more complicated which creates a polynomial.  The
chain of characters which follow the {\tt //} is not interpreted by
$\Psi$lab.

\noindent.\dotfill.
\begin{verbatim}

-->M=[p, p-1; p+1 ,2]
m         =
 
!              2             2    !
!   1 + 2s + 3s       2s + 3s     !
!                                 !
!              2                  !
!   2 + 2s + 3s         2         !
 

-->det(m)

ans      =

          2     3    4  
    2 - 4s - 12s - 9s   

\end{verbatim}
Definition of a polynomial matrix. The syntax for polynomial matrices
is the same as that for matrices of constants. Calculation of the
determinant of the polynomial matrix by the {\tt det} function. 
Note that {\tt scilab} is not case-sensitive.

\noindent.\dotfill.
\begin{verbatim}
 
-->z=poly(0,'z');
 
-->f=[1/z        ,(z+1)/(1-z)
     z/p        ,   z^2   ]
 f         =
 
!   1               1 + z  !
!  ---              ------ !
!   z               1 - z  !
!                          !
!                      2   !
!       z             z    !
!   ---------        ---   !
!             2            !
!  1 + 2z + 3z        1    !
 
\end{verbatim}
Definition of a matrix of rational polynomials.  The internal representation
of {\tt f} is as a list {\tt list('r',num,den)} where {\tt num} and 
{\tt den} are two matrix polynomials.  
\noindent.\dotfill.
\begin{verbatim}
 
-->pause
 
-1->pt=return(z*p)
 
-->pt
 pt        =
 
          2    3  
    z + 2z + 3z   
 
\end{verbatim}
Here we push into a new environment using the command 
{\tt pause}\index{pause@{\tt pause}}
and we obtain the new prompt {\tt -1->} which indicates the level
of the new environment (level 1).  All variables that are available
in the first environment are available in the new environment.  Variables
created in the new environment can be returned to the original environment
by using {\tt return}\index{return@{\tt return}}.  
Use of {\tt return} without an argument 
destroys all the variables created in the new environment before returning
to the old environment. The {\tt pause} facility is very useful 
for debugging macros.

\noindent.\dotfill.
\begin{verbatim}
 
-->f21=f(2,1);v=0:0.01:%pi;frequencies=exp(%i*v);
 
-->response=freq(f21(2),f21(3),frequencies);
 
-->plot2d(v',abs(response)',[-1],'011',' ',[0,0,3.5,0.7],[5,4,5,7]);
 
-->xtitle(' ','radians','magnitude');
\end{verbatim}

Definition of a rational polynomial by extraction of an element
of the matrix {\tt f} defined above.  This is followed by the evaluation
of the rational polynomial at the vector of complex frequency values defined
by {\tt frequencies}.  The evaluation of the polynomial is done by
the primitive {\tt freq}. {\tt f21(2)} is the numerator
polynomial and {\tt f21(3)} is the denominator polynomial.
The  visualization of the resulting evaluation
is made by using the command {\tt plot2d} (see Figure~\ref{f1.1}).

\noindent.\dotfill.
\begin{verbatim}
 
-->w=(1-z)/(1+z);f=1/p
 f         =
 
         1        
    ------------  
               2  
    1 + 2z + 3z   
 
-->horner(f,w)
 ans       =
 
              2   
    1 + 2z + z    
    -----------   
               2  
    6 - 4z + 2z   
\end{verbatim}  
The macro {\tt horner} allows the
user to make a change of variables for a polynomial (for example, to
perform the bilinear transformation as seen above).

\noindent.\dotfill.
\begin{verbatim}
 
-->A=[-1,0;1,2];B=[1,2;2,3];C=[1,0];
 
-->Sl=syslin('c',A,B,C);
 
-->ss2tf(Sl)
 ans       =
 
!     1         2    !
!   ------    ------ !
!   1 + s     1 + s  !

\end{verbatim}
Definition of a linear system in state-space representation.
The macro {\tt syslin} defines here the continuous time ({\tt 'c'}) system
{\tt Sl} with state-space matrices ({\tt A,B,C}). The macro
{\tt ss2tf} transforms {\tt Sl} into transfer matrix representation.

\noindent.\dotfill.
\begin{verbatim}
 
-->s=poly(0,'s');
 
-->R=[1/s,s/(1+s),s^2]
 r         =
 
!                    2 !
!   1       s       s  !
!  ---     ---     --- !
!   s     1 + s     1  !
 
-->Sl=syslin('c',R);
 
-->tf2ss(sl)
 ans       =
 
 
       ans(1)   (state-space system:)
 
 lss   
 
       ans(2) = A matrix = 
 
! - 0.5  - 0.5 !
! - 0.5  - 0.5 !
 
       ans(3) = B matrix = 
 
! - 1.    1.    0. !
!   1.    1.    0. !
 
       ans(4) = C matrix = 
 
! - 1.    0. !
 
       ans(5) = D matrix = 
 
!               2 !
!   0    1     s  !
 
       ans(6) = X0 (initial state) = 
 
!   0. !
!   0. !
 
       ans(7) = Time domain = 
 
 c   
 
\end{verbatim}
Definition of the rational matrix {\tt R}. {\tt Sl} is the
continuous-time linear system with (improper) transfer matrix
R. {\tt tf2ss} puts {\tt Sl} in state-space representation with a
polynomial {\tt D} matrix. Note that linear systems are represented
by lists.

\noindent.\dotfill.
\begin{verbatim}
 
-->sl1=[sl;2*sl+eye]
 sl1       =
 
!                        2  !
!   1           s       s   !
!  ---         ---     ---  !
!   s         1 + s     1   !
!                           !
!                         2 !
!   2 + s      2s       2s  !
!   ------     ---      --- !
!     s       1 + s     1   !

-->size(sl1)
  ans       =
 
!   2.    3. !

-->size(tf2ss(sl1))
 ans       =
 
!   2.    3. !

\end{verbatim}

{\tt sl1} is the linear system in transfer matrix representation
obtained by the parallel inter-connection of {\tt sl} and {\tt 2*sl +eye}.
The same syntax is valid with {\sl} in state-space representation.

\noindent.\dotfill.
\begin{verbatim}
 
-->deff('[Cl]=compen(Sl,Kr,Ko)',[ '[A,B,C,D]=abcd(Sl);';
       'A1=[A-B*Kr ,B*Kr; 0*A ,A-Ko*C]; Id=eye(A);';
       'B1=[Id     ,0*Ko; Id  ,-Ko   ];';
       'C1=[C ,0*C];Cl=syslin(''c'',A1,B1,C1)' ])
 
-->comp(compen)
\end{verbatim}
On-line definition of a macro which calculates the state space representation
({\tt Cl}) of a linear system controlled by an observer with gain {\tt Ko}
and a controller with gain {\tt Kr}.  Note that matrices are constructed
in block form using other matrices.  The macro {\tt compen} is then
compiled by {\tt comp}.

\noindent.\dotfill.
\begin{verbatim}
 
-->A=[1,1 ;0,1];B=[0;1];C=[1,0];Sl=syslin('c',A,B,C);
 
-->Cl=compen(Sl,ppol(A,B,[-1,-1]),...
                     ppol(A',C',[-1+%i,-1-%i])');
 
-->f=Cl(2),spec(f)
 f         =
 
!   1.    1.    0.    0. !
! - 4.  - 3.    4.    4. !
!   0.    0.  - 3.    1. !
!   0.    0.  - 5.    1. !
 ans       =
 
! - 1.       !
! - 1.       !
! - 1. + i   !
! - 1. - i   !
\end{verbatim}
Call to the macro {\tt compen} defined above where the gains were
calculated by a call to the primitive {\tt ppol} which performs pole
placement.
The resulting {\tt f} matrix is displayed and the placement
of its poles is verified using the primitive {\tt spec} which calculates
the eigen-values of a matrix.

\noindent.\dotfill.
\begin{verbatim}

-->save('myfile')
\end{verbatim}
Saves the environment in a file named {\tt mufile}.

\noindent.\dotfill.
\begin{verbatim}

-->unix_x('more myfile')
\end{verbatim}
Request to the host system to perform a system command.

\noindent.\dotfill.
\begin{verbatim}

-->q
  /--error 
 undefined variable: q       
\end{verbatim}
An error message: command is not interpretable by the system since the
variable {\tt q} is unknown.

\noindent.\dotfill.
\begin{verbatim}
-->foo=['      subroutine foo(a,b,c)';
        '      c=a+b';
        '      end'  ];

-->write('foo.f',foo);

-->unix_s('make foo.o') 

-->link('foo.o','foo')  

-->deff('[c]=myplus(a,b)',
   'c=fort(''foo'',a,1,''r'',b,2,''r'',''out'',[1,1],3,''r'')')

-->myplus(5,7)
ans

 12
\end{verbatim}
Definition of a column vector of character chains defining a Fortran
subroutine. The routine is compiled, dynamically loaded into $\Psi$lab,
and interactivelly called by the macro {\tt myplus}.

\noindent.\dotfill.
\begin{verbatim}
 
-->deff('[ydot]=f(t,y)','ydot=[a-y(2)*y(2) -1;1 0]*y')
 
-->a=1;comp(f);y0=[1;0];t0=0;instants=0:0.02:20;
 
-->y=ode(y0,t0,instants,f);
 
-->plot2d(y(1,:)',y(2,:)',[-1],'011',' ',[-3,-3,3,3],[10,2,10,2])
 
-->xtitle('Van der Pol')
\end{verbatim}
Definition of a macro which calculates a first order vector differential
{\tt f(t,y)}.  This is followed by the definition of the constant {\tt a}
used in the macro and the macro is compiled.  The primitive {\tt ode}
then integrates the differential equation defined by {\tt f(t,y)}
for $y(0)=\langle 1;0\rangle$ at $t=0$ and where the solution is given
at the time values $t=0,.02,.04,\ldots,20$.  The result is plotted in
Figure~\ref{f1.2} where the first element of the integrated vector is 
plotted against the second element of this vector.

\noindent.\dotfill.
\begin{verbatim}
 
-->m=['a' 'cos(b)';'sin(a)' 'c']
 m         =
 
!a       cos(b)  !
!                !
!sin(a)  c       !
 
-->m*m'
     !--error    43 
not implemented in scilab....
 
 
-->deff('[x]=%cmc(a,b)',['[l,m]=size(a);[m,n]=size(b);x=[];';
  'for j=1:n,y=[];';
  'for i=1:l,t='' '';';
  'for k=1:m;';
  'if k>1 then t=t+''+(''+a(i,k)+'')*''+''(''+b(k,j)+'')'';';
  'else t=''('' + a(i,k) + '')*'' + ''('' + b(k,j) + '')'';';
  'end,end;';
  'y=[y;t],end;';
  'x=[x y],end,'])
 
--> m*m'
 ans       =
 
!(a)*(a)+(cos(b))*(cos(b))  (a)*(sin(a))+(cos(b))*(c)  !
!                                                      !
!(sin(a))*(a)+(c)*(cos(b))  (sin(a))*(sin(a))+(c)*(c)  !
 
\end{verbatim}
Definition of a matrix containing character chains. The 
operation of symbolic multiplication of two matrices of character
strings is not defined in $\Psi$lab.  The (on-line)
macro definition for {\tt \%cmc} defines the multiplication of 
matrices of character chains (Note that the double quote is necessary
because the body of the {\tt deff} contains quotes inside of quotes).
The {\tt \%} which begins the macro definition for {\tt \%cmc}
allows the definition of an operation which did not previously 
exist in $\Psi$lab, and the name {\tt cmc} means ``chain multiply chain''.

\noindent.\dotfill.
\begin{verbatim}
 
-->deff('[y]=calcul(x,method)','z=method(x),y=poly(z,''x'')')
 
-->deff('[z]=meth1(x)','z=x')
 
-->deff('[z]=meth2(x)','z=2*x')
 
-->calcul([1,2,3],meth1)
 ans       =
 
                2   3  
  - 6 + 11x - 6x + x   
 
-->calcul([1,2,3],meth2)
 ans       =
 
                  2   3  
  - 48 + 44x - 12x + x   
 
\end{verbatim}
A simple example which illustrates the passing of a macro as an argument
to another macro. $\Psi$lab macros are objects which may be defined, loaded,
or manipulated as other objects such as matrices or lists.

\noindent.\dotfill.
\begin{verbatim}

-->quit

\end{verbatim}
Exit from $\Psi$lab.

\noindent.\dotfill.
\input{\Figdir d1.7.tex}
\dessin{A Simple Response}{f1.1}

\input{\Figdir d1.14.tex}
\dessin{Phase Plot}{f1.2}

\section{Software Organization}

$\Psi$lab is divided into a set of directories. The main directory
\verb!SCIDIR! contains the files scilab.star (startup file), the
copyright file notice.tex, and the file \verb!Imake.incl! (see(\ref{install})).
The subdirectories are the following:
\begin{itemize}
\item{{\tt macros} contains the libraries of main $\Psi$lab macros
which are available on-line. A new library can be easily added 
(see the Makefile). This directory is divided into a number of subdirectories
which contain Toolboxes for control, signal processing, etc..}

\item{{\tt demos} is the directory of $\Psi$lab demos. The file 
{\tt alldems.dem} allows to add a new demo which can be run by 
clicking in ``demos''.}

\item{{\tt man} is the manual (Unix manual), divided into submanuals.
The manual is given in Latex format. The \LaTeX code is produced by a 
translation of the Unix format $\Psi$lab manual (see the subdirectory
{\tt Man-General}).
To get information about an item enter {\tt help item} in $\Psi$lab
or use the help window facility obtained with help button.
To get functions corresponding to a key word enter {\tt apropos
key-word}
or use apropos filed of the help window facility.


\item{{\tt routines} is a directory with contains the source code of all
the numerical routines. The subdirectory {\tt default} contains the
source code of routines which are useful to customize $\Psi$lab.}
In particular ``external'' routines for ODE solvers or optimization
should be included here (see e.g. thee file {\tt fydot.f}.
\item{{\tt intersci} is the facility provided to add new Fortran or C 
primitives to $\Psi$lab.}

\item{{\tt doc} is documentation: \LaTeX, dvi and Postscript files. 
This documentation is {\tt SCIDIR/doc/intro/intro.tex}. See 
also the manual in the directory {\tt SCIDIR/man}}

\item{{\tt libs} contains the $\Psi$lab libraries (compiled code).}

\item{{\tt scripts} is the directory which contains the source code of 
shell scripts files.}

\item{{\tt xmetanet} is the directory which contains {\tt xmetanet}, a 
graphic display for networks. Enter {\tt metanet()} in $\Psi$lab to use it.}

\item{{\tt imp}} is the directory where Postscipt files ar managed for
print.

\end{itemize}
%The file {\tt make.incl} must be adapted to your machine. It  
%contains thee pathname if the directory where $\Psi$lab is
%installed.

\section{Installing $\Psi$lab}
\label{install}
$\Psi$lab is given in source code format, with a collection
of {\tt Imakefile} to compile and install it.
The files {\tt Imake.incl , Path.incl} should be adapted to your machine \index{install}.  .
See the README file in main Scilab directory.
