\chapter{Graphics}
This section introduces graphics in Scilab. 

\section{The Graphics Window}
It is possible to use several graphics windows {\tt ScilabGraphicx} x being
the number used for the management of the windows, but at any time only
one window is active. On the main Scilab window the button
 {\tt Graphic Window x} is used to manage the windows : x denotes the
number of the active window, and we can set (create), raise or delete 
the window numbered x : in particular we can directly create the 
graphics window numbered 10.
The execution of a plotting command automatically creates a window if 
necessary.

We will see later that Scilab uses a {\tt graphics environment}
defining some parameters of the plot, these parameters have default
values and can be changed by the user; every graphics window has its
specific context so the same plotting command van give different
results on different windows.

There are 4 buttons on the graphics window:
\begin{itemize}
	\item \verb+3D Rot.+: for applying a rotation with the mouse to a
3D plot. This button is inhibited for a 2D plot. For the help of
manipulations (rotation with specific angles ...) the rotation angles 
are given at the top of the window.
	\item \verb+2D Zoom+: zooming on a 2D plot. This command can be
recursively invoked. For a 3D plot this button is not inhibited but it
has no effect.
	\item \verb+UnZoom+: return to the initial plot (not to the plot
corresponding to the previous zoom in case of multiple zooms).

These 3 buttons affecting the plot in the window are not always in
use; we will see later that there are different choices for the 
underlying device and zoom and rotation need the record of the
plotting commands which is one of the possible choices (this is the
default).

	
	\item \verb+File+: this button opens different commands and menus.

The first one is simple : {\tt Clear} simply rubs out the window
(without affecting the graphics context of the window).

The command {\tt Print...} opens a selection panel for printing. 
The printers are defined in the main scilab script 
{\tt SCIDIR/bin/scilab} (obtained by ``make all'' from the origin 
file {\tt SCIDIR/bin/scilab.g}).

The {\tt Export} command opens a panel selection for getting a copy of the 
plot on a file with a specified format (Postscript, Postscript-Latex, Xfig).

The {\tt save} command directly saves the plot on a file with a
specified name. This file can be loaded later in Scilab for replotting.

The {\tt Close} is the same command than the previous {\tt Delete
Graphic Window} of the menu of the main window, but simply 
applied to its window (the graphic context is, of course deleted). 

\end{itemize}

 
\section{The Media}
There are different graphics devices in Scilab which can be used to send
graphics to windows or paper. The default for the output is 
\verb+ScilabGraphic0+ window .

\noindent The different drivers are:

%
\begin{itemize}
	\item \verb+X11+	: graphics driver for the X11 window system 
	\item \verb+Rec+	: an X Window driver (X11) which also
	records all the graphic commands. This is the default
	(required for the zoom and rotate).
	\item \verb+Wdp+	: an X11 driver without recorded
	graphics; the graphics are done on a pixmap and are send to
	the graphic window with the command  {\tt xset("wshow")}. The 
	pixmap is cleared with the command {\tt xset("wwpc")} or with the
	usual command {\tt xbasc()}
	\item \verb+Pos+	: graphics driver for Postscript printers 
	\item \verb+Fig+	: graphics driver for the Xfig system
\end{itemize}
%

In the 3 first cases the 'implicit' device is a graphics window
(existing or created by the plot). For the 2 last cases we will see
later how to affect a specific device to the plot : a file where the
plot will be recorded in the Postscript or Xfig format.

The basic Scilab graphics commands are :
%
\begin{itemize}
	\item \verb+driver+: selects a graphic driver

The next 3 commands are specific of the X-drivers :

	\item \verb+xclear+: clears one or more graphic windows; does
	not affect the graphics context of these windows.
	\item \verb+xbasc+: clears a graphic window and erase the
	recorded graphics; does not affect the graphics context of  
	the window.
	\item \verb+xpause+: a pause in milliseconds
	\item \verb+xselect+: raises the current graphic window
	(for X-drivers)
	\item \verb+xclick+: waits for a mouse click
	\item \verb+xbasr+: redraws the plot of a graphic window
	\item \verb+xdel+: deletes a graphic window (equivalent to the
	{\tt Close} button

The following commands are specific of the Postscript and Xfig 
drivers :
	
	\item \verb+xinit+: initializes a graphic device
	simply opens a graphics window for the X-drivers this command
	is necessary for Postscript and Xfig drivers.
	\item \verb+xend+: closes a graphic session (and the
	associated device).
\end{itemize}
%

In fact, the regular driver for a common use is {\tt Rec} and there
are special commands in order to avoid a change of driver; in many
cases, one can ignore the existence of drivers and use the
functions \verb+xbasimp+, \verb+xs2fig+ in order to send a graphic 
to a printer or in a file for the \verb+Xfig+ system. For example with
:

\begin{verbatim}
-->driver('Pos')
 
-->xinit('foo.ps')
 
-->plot(1:10)
 
-->xend()
 
-->driver('Rec')
 
-->plot(1:10)
 
-->xbasimp(0,'foo1.ps')

\end{verbatim}

\noindent we get two identical Postscript files : {\tt 'foo.ps'} and {\tt 'foo1.ps.0'}
(the appending 0 is the number of the active window where the plot has been 
done).

The default for plotting is the superposition; this means that between
2 different plots one of the 2 following command is needed :
{\tt xbasc(window-number)} which clears the window and erase the
recorded Scilab graphics command associated with the window 
\verb+window-number+ or {\tt xclear}) which simply clears the window.

If you enlarge a graphic window, the command \verb+xbasr(window-number)+
is executed by Scilab. This command clears the graphic window
\verb+window-number+ and replays the graphic commands associated with it. One 
can call this function manually, in order to verify the associated
recorded graphics commands. 

Any number of graphics windows can be created with buttons 
or with the commands \verb+xset+ or \verb+xselect+. The environment 
variable DISPLAY can be used to specify an X11 Display or one can use
the \verb+xinit+ function in order to open a graphic window on a
specific display. 

\section{Global Parameters of a Plot}
\subsubsection{Graphics Context}

Some parameters of the graphics are controlled by a graphic context 
( for example the line thickness) and others are controlled through
graphics arguments of a plotting command.
The graphics context has a default definition and can be change by the
command {\tt xset} : the command without argument i.e. {\tt xset()}
opens the {\tt Scilab Toggles Panel} and the user can changes the
parameters by simple mouse clickings. We give here different
parameters controlled by this command :
%
\begin{itemize}
	\item  {\tt xset}		: set graphic context values.

(i)-{\tt xset("font",fontid,fontsize)} 	: fix the current font and 
its current size.

(ii)-{\tt xset("mark",markid,marksize)}	: set the current mark 
and current mark size.

(iii)-{\tt xset("use color",flag)} 	: change to color or gray plot according to 
the values (1 or 0) of {\tt flag}.

(iv)-{\tt xset("colormap",cmap)} 	: set the colormap as a m x 3
matrix. m is the number of colors.  Color number i is given as a
3-uple cmap[i,1],cmap[i,2], cmap[i,3] corresponding respectively to Red,
Green and Blue intensity between 0 and 1. Calling again {\tt xset()} 
shows the colormap with the indices of the colors.

(v)-{\tt xset("window",window-number)}	: sets the current window to the window 
{\tt window-number} and creates the window if it doesn't exist.

(vi)-{\tt xset("wpos",x,y)}	: fixes the position of the upper left point of 
the graphic window.

Many other choices are done by {\tt xset} :
 
-use of a pixmap : the plot can be directly displayed on the screen or
executed on a pixmap and then expose by the command 
{\tt xset("wshow")}; this is the usual way for animation effect.

-logical function for drawing : this parameter can be changed for 
specific effects (superposition or adding or substracting of colors).
Looking at the successive plots of the following simple commands give
an example of 2 possible effects of this parameter :

\begin{verbatim}
xset('default');
plot3d();
plot3d();
xset('alufunction',7);
xset('window',0);
plot3d();
xset('default');
plot3d();
xset('alufunction',6);
xset('window',0);
plot3d();
\end{verbatim} 

We have seen that some choices exist for the fonts and this choice can
be extended by the command:
        \item  {\tt xlfont}	: to load a new family of fonts from
the XWindow Manager


There exists the function ``reciprocal'' to {\tt xset} :

	\item  {\tt xget}	: to get informations about the
	current graphic context.

All the values of the parameters fixed by {\tt xset} can be obtained by 
{\tt xget}. An example :

\begin{verbatim}

-->pos=xget("wpos")
 pos  =
 
!   105.    121. !

\end{verbatim}

{\tt pos} is the position of the upper left point of the graphic window.
\end{itemize}
%

\subsubsection{Some Manipulations}

Coordinates transforms:
\begin{itemize}
	\item \verb+isoview+	: isometric scale without window change 

allows an isometric scale in the window of previous plots without changing
the window size:

\begin{verbatim}

t=(0:0.1:2*%pi)';
plot2d(sin(t),cos(t));
xbasc()
isoview(-1,1,-1,1);
plot2d(sin(t),cos(t),-1,'001');

\end{verbatim}

	\item \verb+square+	: isometric scale with resizing the window

the window is resized according to the parameters of the command.

        \item \verb+scaling+	: scaling on data
        \item \verb+rotate+	: rotation

\verb+scaling+ and \verb+rotate+ executes respectively an affine transform and
a geometric rotation of a 2-lines-matrix corresponding to the {\tt (x,y) }
values of a set of points.
 
        \item \verb+xgetech, xsetech+	: change of scale inside the graphic window

The current graphic scale can be fixed by a high level plot command. You may
want to get this parameter or to fix it directly : this is the role of 
\verb+xgetech, xsetech+ . {\tt xsetech } is a simple way to cut the
window in differents parts for different plots :

\begin{verbatim}

t=(0:0.1:2*%pi)';
xsetech([0.,0.,0.6,0.3],[-1,1,-1,1]);
plot2d(sin(t),cos(t));
xsetech([0.5,0.3,0.4,0.6],[-1,1,-1,1]);
plot2d(sin(t),cos(t));

\end{verbatim}

\end{itemize}
%


\section{2D Plotting}

\subsection{Basic 2D Plotting}

The simplest 2D plot is {\tt plot(x,y)} or {\tt plot(y)}: this is the plot of 
{\tt y} as function of {\tt x} where {\tt x} and {\tt y} are 2 vectors; if 
{\tt x} is missing, it is replaced by the vector {\tt (1,...,size(y))}.
If {\tt y} is a matrix, its rows are plotted. There are optional arguments.


A first example is given by the following commands and one of the results is
represented on figure \ref{d7-1}:

\begin{verbatim}

t=(0:0.05:1)';
ct=cos(2*%pi*t);
// plot the cosine
plot(t,ct);
// xset() opens the toggle panel and 
// some parameters can be changed with mouse clicks
// given by commands for the demo here 
xset("font",5,4);xset("thickness",3);
// plot with captions for the axis and a title for the plot
// if a caption is empty the argument ' ' is needed
plot(t,ct,'Time','Cosine','Simple Plot');
// click on a color of the xset toggle panel and do the previous plot again
// to get the title in the chosen color

\end{verbatim}

\input{figures/d7-1.tex}
\caption{\label{d7-1}First example of plotting}
\end{figure}

\noindent The generic 2D multiple plot is 


\noindent {\tt  plot2di(str,x,y,[style,strf,leg,rect,nax])}
 
\begin{itemize}

\item index of \verb+plot2d+	: {\tt i=missing,1,2,3,4}.

For the different values of {\tt i} we have:

	{\tt i=missing}	: piecewise linear plotting

	{\tt i=1}	: as previous with possible logarithmic scales

	{\tt i=2}	: piecewise constant drawing style

	{\tt i=3}	: vertical bars

	{\tt i=4}	: arrows style (e.g. ode in a phase space)

\end{itemize}

\begin{verbatim}
t=(1:0.1:8)';xset("font",2,3);
xsetech([0.,0.,0.5,0.5],[-1,1,-1,1]);
plot2d([t t],[1.5+0.2*sin(t) 2+cos(t)]);
xtitle('Plot2d');
titlepage('Piecewise linear');
//
xsetech([0.5,0.,0.5,0.5],[-1,1,-1,1]);
plot2d1('oll',t,[1.5+0.2*sin(t) 2+cos(t)]);
xtitle('Plot2d1');
titlepage('Logarithmic scale(s)');
//
xsetech([0.,0.5,0.5,0.5],[-1,1,-1,1]);
plot2d2('onn',t,[1.5+0.2*sin(t) 2+cos(t)]);
xtitle('Plot2d2');
titlepage('Piecewise constant');
//
xsetech([0.5,0.5,0.5,0.5],[-1,1,-1,1]);
plot2d3('onn',t,[1.5+0.2*sin(t) 2+cos(t)]);
xtitle('Plot2d3')
titlepage('Vertical bar plot')
xset('default')
\end{verbatim}

  \begin{figure}[hbtp]
  \begin{center}
  %If you prefer cm use the followin two lines
  %\setlength{\unitlength}{1mm}
  %\fbox{\begin{picture}(105,74)
%BEGIN IMAGE	
  \fbox{\begin{picture}(300.0,212.0)
  \special{psfile=\Figdir nouv1.epsf hscale=100.0 vscale=100.0}
  \end{picture}}
%END IMAGE
%HEVEA\imageflush
  \end{center}
  \caption{Different 2D plotting}\label{nouv1}
  \end{figure}

\dessin{Different 2D plotting}{nouv1}

\begin{itemize}

\item Parameter {\tt str}  	: it is the string {\tt "abc"} :
 
	{\tt str} is empty if {\tt i} is missing.

	{\tt a=e} : means empty; the values of {\tt x} are not used;
	   (The user must give a dummy value to {\tt x}).

	{\tt a=o} : means one; the x-values are the same for all the curves

	{\tt a=g} : means general. 

	{\tt b=l} : a logarithmic scale is used on the X-axis

	{\tt c=l} : a logarithmic scale is used on the Y-axis

-Parameters {\tt x,y} : two matrices of the same size {\tt [nl,nc]} ({\tt nc} 
is the number of curves and {\tt nl} is the number of points of each
curve).

For a single curve the vector can be row or column : 
{\tt plot2d(t',cos(t)')  plot2d(t,cos(t))} are equivalent.


\item Parameter {\tt style}  	:it is a real vector of size {\tt
(1,nc)}; the style 
to use for curve j is defined by {\tt size(j)} (when only one curve is drawn 
{\tt style} can specify the style and a position to use for the
caption).

\begin{verbatim}
x=0:0.1:2*%pi;
u=[-0.8+sin(x);-0.6+sin(x);-0.4+sin(x);-0.2+sin(x);sin(x)];
u=[u;0.2+sin(x);0.4+sin(x);0.6+sin(x);0.8+sin(x)]';   
//start trying the color with the 2 following lines
//sty=[-9,-8,-7,-6,-5,-4,-3,-2,-1,0];
//plot2d1('onn',x',u,sty,"111"," ",[0,-2,2*%pi,3],[2,10,2,10]);
plot2d1('onn',x',u,...
  [9,8,7,6,5,4,3,2,1,0],"011"," ",[0,-2,2*%pi,3],[2,10,2,10]);
x=0:0.2:2*%pi; 
v=[1.4+sin(x);1.8+sin(x)]'; 
xset("mark",1,5);
plot2d1('onn',x',v,[7,8],"011"," ",[0,-2,2*%pi,3],[2,10,2,10]);
xset('default');
\end{verbatim}

  \long\def\Checksifdef#1#2#3{%
     \expandafter\ifx\csname #1\endcsname\relax#2\else#3\fi}
  \Checksifdef{Figdir}{\gdef\Figdir{}}{}
  \def\dessin#1#2{
  \begin{figure}[hbtp]
  \begin{center}
  %If you prefer cm use the followin two lines
  %\setlength{\unitlength}{1mm}
  %\fbox{\begin{picture}(105,74)
  \fbox{\begin{picture}(300.0,212.0)
  \special{psfile=\Figdir nouv2.epsf hscale=100.0 vscale=100.0}
  \end{picture}}
  \end{center}
  \caption{\label{#2}#1}
  \end{figure}}

\dessin{Black and white plotting styles}{nouv2}

\item Parameter {\tt strf}  	: it is a string of length 3 {\tt
"xyz"} corresponding to :

	    {\tt x=1} : captions  displayed 

	    {\tt y=1} : the argument {\tt rect} is used to specify the 
		boundaries of the plot.\\  
		{\tt rect=[xmin,ymin,xmax,ymax]}

	    {\tt y=2}	: the boundaries of the	plot are computed 

	    {\tt y=0}	: the current boundaries 

	    {\tt z=1}	: an axis is drawn and  the number of tics can be specified by the {\tt nax} argument

	    {\tt z=2}	: the plot is only surrounded by a box

\item Parameter {\tt leg } 	 : it is the string of the captions for the different 
plotted curves . This string is composed of fields separated by the {\tt @} 
symbol: for example  {\tt ``module@phase''} (see example below). These 
strings are
displayed under the plot with  small segments recalling the styles of the 
corresponding curves.

\item Parameter {\tt rect } 	: it is a vector of 4 values specifying the boundaries 
of the plot {\tt rect=[xmin,ymin,xmax,ymax]}.

\item Parameter {\tt nax }  	: it is a vector [nx,Nx,ny,Ny] where nx (ny) is 
the number of subgrads on the x (y) axis and Nx (Ny) is the number of 
graduations  on the x (y) axis.

\end{itemize}

\begin{verbatim}

//captions for identifying the curves
//controlling the boundaries of the plot and the tics on axes
x=-%pi:0.3:%pi;
y1=sin(x);y2=cos(x);y3=x;
X=[x;x;x]; Y=[y1;y2;y3];
plot2d1("gnn",X',Y',[1 2 3]',"111","caption1@caption2@caption3",...
[-3,-3,3,2],[2,20,5,5]);

\end{verbatim}

  \begin{figure}[hbtp]
  \begin{center}
  %If you prefer cm use the followin two lines
  %\setlength{\unitlength}{1mm}
  %\fbox{\begin{picture}(105,74)
%BEGIN IMAGE	
  \fbox{\begin{picture}(300.0,212.0)
  \special{psfile=\Figdir nouv3.epsf hscale=100.0 vscale=100.0}
  \end{picture}}
%END IMAGE
%HEVEA\imageflush
  \end{center}
  \caption{Box, captions and tics}\label{nouv3}
  \end{figure}

\dessin{Box, captions and tics}{nouv3}

For different plots the simple commands without any argument show a demo 
(e.g {\tt plot2d3()} ).

\subsection{Captions and Presentation}
%
\begin{itemize}
	\item \verb+xgrid+	: adds a grid on a 2D graphic; the
calling parameter is the number of the color.
	\item \verb+xtitle+	: adds	title above the plot and axis 
names on a 2D graphic
	\item \verb+titlepage+	: graphic title page in the middle of
the plot

\begin{verbatim}

//Presentation
x=-%pi:0.3:%pi;
y1=sin(x);y2=cos(x);y3=x;
X=[x;x;x]; Y=[y1;y2;y3];
plot2d1("gnn",X',Y',[1 2 3]',"111","caption1@caption2@caption3",...
[-3,-3,3,2],[2,20,2,5]);
xtitle(["General Title";"(with xtitle command)"],"x-axis title","y-axis title (with xtitle command)");
xgrid();
xclea(-2.7,1.5,1.5,1.5);
titlepage("Titlepage");
xstring(0.6,.45,"(with titlepage command)");
xstring(0.05,.7,["xstring command after";"xclea command"],0,1);

\end{verbatim}

  \begin{figure}[hbtp]
  \begin{center}
  %If you prefer cm use the followin two lines
  %\setlength{\unitlength}{1mm}
  %\fbox{\begin{picture}(105,74)
%BEGIN IMAGE	
  \fbox{\begin{picture}(300.0,212.0)
  \special{psfile=\Figdir nouv4.epsf hscale=100.0 vscale=100.0}
  \end{picture}}
%END IMAGE
%HEVEA\imageflush
  \end{center}
  \caption{Grid, Title eraser and comments}\label{nouv4}
  \end{figure}

\dessin{Grid, Title eraser and comments}{nouv4}

	\item \verb+plotframe+	: graphic frame with scaling and grid
\end{itemize}
%

We have seen that it is possible to control the tics on the axes,
choose the size of the rectangle for the plotand add a grid.
This operation can be prepared once and then used for a sequence of
different plots. One of the most useful aspect is to get  graduations 
by choosing the number of graduations and getting rounded numbers.

\begin{verbatim}

rect=[-%pi,-1,%pi,1];
tics=[2,10,4,10];
plotframe(rect,tics,[%t,%t],...
['Plot with grids and automatic bounds','angle','velocity']);

\end{verbatim}

\begin{itemize}
	\item \verb+graduate+	: a simple tool for computing pretty
axis graduations before a plot.
\end{itemize}
%

\subsection{Specialized 2D Plottings}
%
\begin{itemize}
        \item \verb+champ+	: vector field in $R^{2}$

\begin{verbatim}
//try champ
x=[-1:0.1:1];y=x;u=ones(x);
fx=x.*.u';fy=u.*.y';
champ(x,y,fx,fy);
xset("font",2,3);
xtitle(['Vector field plot';'(with champ command)']);
//with the color (and a large stacksize)
x=[-1:0.004:1];y=x;u=ones(x);
fx=x.*.u';fy=u.*.y';
champ1(x,y,fx,fy);
\end{verbatim}

  \begin{figure}[hbtp]
  \begin{center}
  %If you prefer cm use the followin two lines
  %\setlength{\unitlength}{1mm}
  %\fbox{\begin{picture}(105,74)
%BEGIN IMAGE
  \fbox{\begin{picture}(300.0,212.0)
  \special{psfile=\Figdir nouv5.epsf hscale=100.0 vscale=100.0}
  \end{picture}}
%END IMAGE
%HEVEA\imageflush
  \end{center}
  \caption{Vector field in the plane}\label{nouv5}
  \end{figure}

\dessin{Vector field in the plane}{nouv5}


	\item \verb+fchamp+	: for a vector field in $R^{2}$ defined by a 
function. The same plot than {\tt champ} for a vector field defined
for example by a scilab program.

	\item \verb+fplot2d+	: 2D plotting of a curve described by a 
function. This function plays the same role for {\tt plot2d} than the 
previous for {\tt champ}.

	\item \verb+grayplot+	: 2D plot of a surface using gray
levels; the surface being defined by the matrix of the values
for a grid.

	\item \verb+fgrayplot+	: the same than the previous for a
surface defined by a function (scilab program).

In fact these 2 functions can be replaced by a usual color plot with
an appropriate colormap where the 3 RGB components are the same.

\begin{verbatim}
R=[1:256]/256;RGB=[R' R' R'];
xset('colormap',RGB);
deff('[z]=surf(x,y)','z=-((abs(x)-1)**2+(abs(y)-1)**2)');
fgrayplot(-1.8:0.02:1.8,-1.8:0.02:1.8,surf,"111",[-2,-2,2,2]);
xset('font',2,3);
xtitle(["Grayplot";"(with fgrayplot command)"]);
//the same plot can be done with a ``unique'' given color
R=[1:256]/256;
G=0.1*ones(R);
RGB=[R' G' G'];
xset('colormap',RGB);
fgrayplot(-1.8:0.02:1.8,-1.8:0.02:1.8,surf,"111",[-2,-2,2,2]);
\end{verbatim}

  \long\def\Checksifdef#1#2#3{%
     \expandafter\ifx\csname #1\endcsname\relax#2\else#3\fi}
  \Checksifdef{Figdir}{\gdef\Figdir{}}{}
  \def\dessin#1#2{
  \begin{figure}[hbtp]
  \begin{center}
  %If you prefer cm use the followin two lines
  %\setlength{\unitlength}{1mm}
  %\fbox{\begin{picture}(105,74)
  \fbox{\begin{picture}(300.0,212.0)
  \special{psfile=\Figdir nouv6.epsf hscale=100.0 vscale=100.0}
  \end{picture}}
  \end{center}
  \caption{\label{#2}#1}
  \end{figure}}

\dessin{Gray plot with a gray colormap}{nouv6}

	\item \verb+errbar+	: creates a plot with error bars
\end{itemize}
%


\subsection{Plotting Some Geometric Figures}
\subsubsection{Polylines Plotting}
\begin{itemize}
	\item \verb+xsegs+	: draws	a set of unconnected segments 
	\item \verb+xrect+	: draws	a single rectangle
	\item \verb+xfrect+	: fills a single rectangle
	\item \verb+xrects+	: fills or draws a set	of rectangles
	\item \verb+xpoly+	: draws	a polyline
	\item \verb+xpolys+	: draws a set of polylines
	\item \verb+xfpoly+	: fills a polygon
	\item \verb+xfpolys+	: fills a set of polygons
	\item \verb+xarrows+	: draws a set of unconnected arrows
	\item \verb+xfrect+	: fills a single rectangle
	\item \verb+xclea+	: erases a rectangle on a graphic window
\end{itemize}
%

\subsubsection{Curves Plotting}
\begin{itemize}
	\item \verb+xarc+	: draws	an ellipsis
	\item \verb+xfarc+	: fills	an ellipsis
	\item \verb+xarcs+	: fills	or draws a set of ellipsis
\end{itemize}
%

\subsection{ Writting by Plotting}
%
\begin{itemize}
	\item  \verb+xstring+	: draws a string or a matrix of strings
        \item  \verb+xstringl+	: computes a rectangle which surrounds
        a string
	\item  \verb+xstringb+	: draws a string in a specified box
	\item  \verb+xnumb+	: draws a set of numbers
\end{itemize}
%

We give now the sequence of the commands for obtaining the 
figure~ \ref{fgeom}.

\begin{verbatim}
//	initialize default environment variables
xset('default');
xset("use color",0);
plot([1:10]);
xbasc()
xrect(0,1,3,1)
xfrect(3.1,1,3,1)
xstring(0.5,0.5,"xrect(0,1,3,1)")
xstring(4.,0.5,"xfrect(3.1,1,3,1)")
xset("alufunction",6)
xstring(4.,0.5,"xfrect(3.1,1,3,1)")
xset("alufunction",3)
xv=[0 1 2 3 4]
yv=[2.5 1.5 1.8 1.3 2.5]
xpoly(xv,yv,"lines",1)
xstring(0.5,2.,"xpoly(xv,yv,""lines"",1)")
xa=[5 6 6 7 7 8 8 9 9 5]
ya=[2.5 1.5 1.5 1.8 1.8 1.3 1.3 2.5 2.5 2.5]
xarrows(xa,ya)
xstring(5.5,2.,"xarrows(xa,ya)")
xarc(0.,5.,4.,2.,0.,64*300.)
xstring(0.5,4,"xarc(0.,5.,4.,2.,0.,64*300.)")
xfarc(5.,5.,4.,2.,0.,64*360.)
//xset("alufunction",6)
xclea(5.6,4.4,2.8,0.8);
xstring(5.8,4.,"xfarc  and then  xclea")
//xset("alufunction",3)
xstring(0.,4.5,"WRITING-BY-XSTRING()",-22.5) 
xnumb([5.5 6.2 6.9],[5.5 5.5 5.5],[3 14 15],1)
isoview(0,12,0,12)
xarc(-5.,12.,5.,5.,0.,64*360.)
xstring(-4.5,9.25,"isoview + xarc",0.)
xset("font",4,5)
A=["  1" "       2" "  3";"  4" "       5" "  6";"68" "  17.2" "  9"];
xstring(7.,10.,A);
rect=xstringl(7,10,A);
xrect(rect(1),rect(2),rect(3),rect(4));
\end{verbatim}

  \long\def\Checksifdef#1#2#3{%
     \expandafter\ifx\csname #1\endcsname\relax#2\else#3\fi}
  \Checksifdef{Figdir}{\gdef\Figdir{}}{}
  \def\dessin#1#2{
  \begin{figure}[hbtp]
  \begin{center}
  %If you prefer cm use the followin two lines
  %\setlength{\unitlength}{1mm}
  %\fbox{\begin{picture}(105,74)
  \fbox{\begin{picture}(300.0,212.0)
  \special{psfile=\Figdir nouv7.epsf hscale=100.0 vscale=100.0}
  \end{picture}}
  \end{center}
  \caption{\label{#2}#1}
  \end{figure}}

\dessin{Geometric Graphics and Comments}{fgeom}

e have seen that some parameters of the graphics are controlled by a 
graphic context 
( for example the line thickness) and others are controlled through
graphics arguments . 
%
\begin{itemize}
	\item  {\tt xset}	: to set graphic context values.
Some examples of the use of {\tt xset} : 

(i)-{\tt xset("use color",flag)} changes to color or gray plot according to 
the values (1 or 0) of {\tt flag}.

(ii)-{\tt xset("window",window-number)} sets the current window to the window 
{\tt window-number} and creates the window if it doesn't exist.

(iii)-{\tt xset("wpos",x,y)} fixes the position of the upper left point of 
the graphic window.

The choice of the font, the width and height of the window, the driver...
can be done by {\tt xset}.

	\item  {\tt xget}	: to get informations about the current graphic context.
All the values of the parameters fixed by {\tt xset} can be obtained by 
{\tt xget}.

        \item  {\tt xlfont}	: to load a new family of fonts from the XWindow Manager
\end{itemize}
%
\subsection{Some Classical Graphics for Automatic Control}

\begin{itemize}

	\item  \verb+bode+	: plot magnitude and phase of the 
frequency response of a linear system.

	\item  \verb+gainplot+	: same as bode but plots only the 
magnitude of the frequency response.

	\item  \verb+nyquist+	: plot of imaginary part versus real 
part of the frequency response of a linear system.

	\item  \verb+m_circle+	: M-circle plot used with nyquist plot.

	\item  \verb+chart+	: plot the Nichols'chart 

	\item  \verb+black+	: plot the Black's diagram (Nichols'chart)
for a linear system.  

	\item  \verb+evans+	: plot the Evans root locus for a
linear system.

	\item  \verb+plzr+	: pole-zero plot of the linear system

\end{itemize}

\begin{verbatim}
s=poly(0,'s');
h=syslin('c',(s^2+2*0.9*10*s+100)/(s^2+2*0.3*10.1*s+102.01));
h1=h*syslin('c',(s^2+2*0.1*15.1*s+228.01)/(s^2+2*0.9*15*s+225));
//bode
xsetech([0.,0.,0.5,0.5],[-1,1,-1,1]);
gainplot([h1;h],0.01,100);
//nyquist
xsetech([0.5,0.,0.5,0.5],[-1,1,-1,1]);
nyquist([h1;h])
//chart and black 
xsetech([0.,0.5,0.5,0.5],[-1,1,-1,1]);
black([h1;h],0.01,100,['h1';'h'])
chart([-8 -6 -4],[80 120],list(1,0));
//evans
xsetech([0.5,0.5,0.5,0.5],[-1,1,-1,1]);
H=syslin('c',352*poly(-5,'s')/poly([0,0,2000,200,25,1],'s','c'));
evans(H,100)
\end{verbatim}

  \begin{figure}[hbtp]
  \begin{center}
  %If you prefer cm use the followin two lines
  %\setlength{\unitlength}{1mm}
  %\fbox{\begin{picture}(105,74)
%BEGIN IMAGE
  \fbox{\begin{picture}(300.0,212.0)
  \special{psfile=\Figdir nouv8.epsf hscale=100.0 vscale=100.0}
  \end{picture}}
%END IMAGE
%HEVEA\imageflush

  \end{center}
  \caption{Some Plots in Automatic Control}{\label nouv8}
  \end{figure}

\dessin{Some Plots in Automatic Control}{nouv8}

%
\subsection{Miscellaneous}
%
\begin{itemize}
	\item  \verb+edit_curv+ : interactive graphic curve editor.

	\item  \verb+gr_menu+ 	: simple interactive graphic editor. It
is a Xfig-like simple editor with a flexible use for a nice
presentation of graphics : the user can superpose the elements of
\verb+gr_menu+ and use it with the usual possibilities of {\tt xset}.

	\item  \verb+locate+ 	: to get the coordinates of one or more 
points selected with the mouse on a graphic window.

\end{itemize}
% 

\begin{verbatim}

\end{verbatim}
  \begin{figure}[hbtp]
  \begin{center}
  %If you prefer cm use the followin two lines
  %\setlength{\unitlength}{1mm}
  %\fbox{\begin{picture}(105,74)]
%BEGIN IMAGE
  \fbox{\begin{picture}(300.0,212.0)
  \special{psfile=\Figdir scifi2.epsf hscale=100.0 vscale=100.0}
  \end{picture}}
%END IMAGE
%HEVEA\imageflush

  \end{center}
  \caption{Presentation of Plots}{\label scifi2}
  \end{figure}

\dessin{Presentation of Plots}{scifi2}


\section{3D Plotting}
\subsection{Generic 3D Plotting}
%
\begin{itemize}
	\item  \verb+plot3d+ : 3D plotting of a matrix of points : plot3d(x,y,z) with x,y,z 3 matrices, z being the values for the points with coordinates x,y. Other
arguments are optional
	\item   \verb+plot3d1+ : 3d plotting	of a matrix of points with gray levels
	\item   \verb+fplot3d+ : 3d plotting	of a surface described by a function; z is given by an external z=f(x,y)
	\item   \verb+fplot3d1+ : 3d	plotting of a surface described	by a function with gray	levels
\end{itemize}
%
 
\subsection{Specialized 3D Plotting}
\begin{itemize}
	\item \verb+param3d+	: plots parametric curves in 3d space
	\item \verb+contour+	: level curves for a 3d function given by a 
matrix
	\item \verb+grayplot10+ : gray level on a 2d plot
	\item \verb+fcontour10+ : level curves for a 3d function given by a 
function
	\item \verb+hist3d+	: 3d histogram
	\item \verb+secto3d+	: conversion of a surface description from 
sector to plot3d compatible data
	\item \verb+eval3d+	: evaluates a function on a regular grid. 
(see also feval)
\end{itemize}
%
 

\subsection{Mixing 2D and 3D graphics}

When one uses 3D plotting function, default graphic boundaries are
fixed, but in $R^3$. If one wants to use graphic primitives 
to add informations on 3D graphics, the \verb+geom3d+ function can be
used to convert 3D coordinates to 2D-graphics coordinates. The
figure~ \ref{d710} illustrates this feature.

%\def\exemple{d710}
\input{figures/d7-10.vtex}

  \long\def\Checksifdef#1#2#3{%
     \expandafter\ifx\csname #1\endcsname\relax#2\else#3\fi}
  \Checksifdef{Figdir}{\gdef\Figdir{}}{}
  \def\dessin#1#2{
  \begin{figure}[hbtp]
  \begin{center}
  %If you prefer cm use the followin two lines
  %\setlength{\unitlength}{1mm}
  %\fbox{\begin{picture}(105,74)
  \fbox{\begin{picture}(300.0,212.0)
  \special{psfile=\Figdir d710.epsf hscale=100.0 vscale=100.0}
  \end{picture}}
  \end{center}
  \caption{\label{#2}#1}
  \end{figure}}

\dessin{2D and 3D plot}{d710}

\subsection{Sub-windows}
%\def\exemple{d7-8}

It is also possible to make multiple plotting in the same 
graphic window (Figure~\ref{d7-8}).

\input{figures/d7-8.vtex}

\input{figures/d7-8.tex}
\caption{\label{d7-8}Use of xsetech}
\end{figure}


\subsection{A Set of Figures}
%\def\exemple{d7a11}

In this next example we give a brief summary of different plotting
functions for 2D or 3D graphics. The figure~ \ref{d7a11} is 
obtained and inserted in this document with the help of the command 
\verb+Blatexprs+.
 
\input{figures/d7a11.vtex}

\input{figures/d7a11.tex}
\caption{\label{d7a11}Group of figures}
\end{figure}

\section{Printing and Inserting Scilab Graphics in \LaTeX}

We describe here the use of programs (Unix shells) for handling Scilab 
graphics and printing the results. These programs are located in the
sub-directory {\tt bin} of Scilab.

\subsection{Window to Paper}
The simplest command to get a paper copy of a plot is to click on the
{\tt print} button of the ScilabGraphic window.

\subsection{Creating a Postscript File}
We have seen at the beginning of this chapter that the simplest way to get 
a Postscript file containing a Scilab plot is :

\begin{verbatim}
-->driver('Pos')
 
-->xinit('foo.ps')
 
-->plot3d1();
 
-->xend()
 
-->driver('Rec')
 
-->plot3d1()
 
-->xbasimp(0,'foo1.ps')

\end{verbatim}


The Postscript files ({\tt foo.ps } or {\tt foo1.ps }) generated by Scilab 
cannot be directly sent to a
Postscript printer, they need a preamble. Therefore, printing is done
through the use of Unix scripts or programs which are provided with
Scilab. The program \verb+Blpr+ is used to print a set of Scilab Graphics 
on a single sheet of paper and is used as follows~:

\begin{verbatim}
 Blpr string-title file1.ps file2.ps > result
\end{verbatim}

You can then print the file  {\tt result} with the classical Unix command :

\begin{verbatim}
lpr -Pprinter-name result
\end{verbatim}

\noindent or use the \verb+ghostview+ Postscript interpreter on your Unix
workstation to see the result.

You can avoid the file {\tt result} with a pipe, replacing {\tt  > result}
by the printing command {\tt | lpr} or the previewing command 
{\tt | ghostview -}.

The best result (best sized figures) is obtained when printing two 
pictures on a single page. 


\subsection{Including a Postscript File in \LaTeX}

The \verb|Blatexpr| Unix shell and the programs \verb+Batexpr2+ and 
\verb+Blatexprs+ are provided in order to help inserting Scilab graphics 
in \LaTeX .

Taking the previous file {\tt foo.ps} and  typing the following statement 
under a Unix shell~:
\begin{verbatim}
Blatexpr 1.0 1.0 foo.ps
\end{verbatim}
creates two files \verb+foo.epsf+ and \verb+foo.tex+. The original
Postscript file is left unchanged. 
To include the figure in a \LaTeX~ document you should insert the following 
\LaTeX~ code in your \LaTeX~ document~:

\begin{verbatim}
\input foo.tex
\dessin{The caption of your picture}{The-label}
\end{verbatim}

You can also see your figure by using  the  Postscript
previewer \verb+ghostview+.

The program \verb+Blatexprs+ does the same thing: it is used to insert
a set of Postscript figures in one \LaTeX picture.

In the following example, we begin by using the Postscript driver {\tt Pos}
and then initialize successively 4 Postscript files 
{\tt fig1.ps, ..., fig4.ps} for
4 different plots and at the end return to the driver {\tt Rec} (X11 driver
with record).

\input{diary/mulplot.dia}

Then we execute the command :

\begin{verbatim}
  Blatexprs multi fig1.ps fig2.ps fig3.ps fig4.ps
\end{verbatim}

\noindent and we get 2 files {\tt multi.tex} and  {\tt multi.ps} and you can 
include the result in a \LaTeX~ source file by :

\begin{verbatim}
\input multi.tex
\dessin{The caption of your picture}{The-label}
\end{verbatim}

Note that the second line {\tt dessin...} is absolutely necessary and you
have of course to give the absolute path for the input file if you are 
working in another directory (see below). The file {\tt  multi.tex} is only 
the definition of the command {\tt dessin} with 2 parameters : the caption 
and the label; the command {\tt dessin} can be used with one or two
empty arguments {\tt `` ``} if you want to avoid the caption or the label.   

The Postscipt files are inserted in \LaTeX~ with the help of the
\verb+\special+ command and with a syntax that works with the
\verb+dvips+ program.

The program \verb+Blatexpr2+ is used when you want two pictures side
by side.
\begin{verbatim}
  Blatexpr2 Fileres file1.ps file2.ps 
\end{verbatim}

%\def\exemple{d7-12}
\begin{figure}
\begin{center}
\setlength{\unitlength}{1cm}
\fbox{\begin{picture}(15.00,5.00)
\special{psfile=figures/d7.12.ps}
\end{picture}}
\end{center}

\caption{\label{d7-12}Blatexp2 Example}
\end{figure}

It is sometimes convenient to have a main \LaTeX~ document in a directory
and to store all the figures in a subdirectory. The proper way to
insert a  picture file in the main document, when the picture 
is stored in the subdirectory \verb+figures+, is the following~:

\begin{verbatim}
\def\Figdir{figures/} % My figures are in the {\tt figures/ } subdirectory.
\input{figures/fig.tex}
\dessin{The caption of you picture}{The-label}
\end{verbatim}

The declaration \verb+\def\Figdir{figures/}+ is used twice, first to
find the file {\tt fig.tex} (when you use \verb+latex+), and
 second to produce a correct pathname for the
\verb+special+ \LaTeX~ command  found in \verb+fig.tex+. (used at
dvips level).

-WARNING : the default driver is {\tt Rec}, i.e. all the graphic commands are
recorded, one record corresponding to one window. The {\tt xbasc()} command
erases the plot on the active window and all the records corresponding to
this window. The {\tt  clear} button has the same effect; the {\tt xclear}
command erases the plot but the record is preserved. So you almost never need
to use the {\tt xbasc()} or {\tt  clear} commands. If you use such a command 
and if you re-do a plot you may have a surprising result (if you forget
that the environment is wiped out); the scale only is preserved and so you 
may have the ``window-plot'' and the ``paper-plot'' completely different.


\subsection{Postscript by Using Xfig}

Another useful way to get a Postscript file for a plot is to use Xfig.
By the simple command {\tt xs2fig(active-window-number,file-name)} you get
a file in Xfig syntax. 

This command needs the use of the driver {\tt Rec}.

The window ScilabGraphic0 being  active, if you enter :

\begin{verbatim}

-->t=-%pi:0.3:%pi;
 
-->plot3d1(t,t,sin(t)'*cos(t),35,45,'X@Y@Z',[2,2,4]);
 
-->xs2fig(0,'demo.fig');   

\end{verbatim}

you get the file {\tt demo.fig} which contains the plot 
of window 0. 

Then you can use Xfig and after the modifications you want, get a Postscript 
file that you can insert in a \LaTeX~ file. The following figure is the result
of Xfig after adding some comments.

\begin{figure}
\fbox{\hspace*{0.cm} \begin{picture}(400.0,350.0)(100.0,200.0)
\hspace{-0.8cm}
\special{psfile=figures/demo.ps}
\end{picture}}
\caption{Encapsulated Postscript by Using Xfig}
\label{xfig2ps}
\end{figure}


\subsection{Encapsulated Postscript Files}
As it was said before, the use of {\tt Blatexpr} creates 2 files : a 
{\tt .tex} file to be inserted in the \LaTeX~ file and a {\tt .epsf} file.

It is possible to get the encapsulated Postscript file corresponding to a 
{\tt .ps} file by using the command {\tt BEpsf}.

Notice that the {\tt .epsf} file generated by {\tt Blatexpr} is not an 
encapsulated Postscript file : it has no bounding box and {\tt BEpsf}
generates a {\tt .eps} file which is an encapsulated Postscript file with 
a bounding box.
