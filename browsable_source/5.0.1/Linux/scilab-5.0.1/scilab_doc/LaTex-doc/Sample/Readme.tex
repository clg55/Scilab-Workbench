\documentstyle[11pt,verbatim,../tr2latex/troffman]{article}
             \textheight=660pt 
             \textwidth=15cm
             \topmargin=-27pt 
             \oddsidemargin=0.7cm
             \evensidemargin=0.7cm
             \marginparwidth=60pt
             \title{Tr2Tex for Scilab} 
             \author{J.Ph. Chancelier\thanks{Cergrene. Ecole Nationale des Ponts et Chauss\'ees, La Courtine  93167 Noisy le Grand C\'{e}dex }}
	
\begin{document}\maketitle
\def\verbatok#1{\expandafter\begin{verbatim}
\input{#1}}
	
\section{Modifications from the original tr2tex}
I've slightly modified the tr2tex program as follows~:
\begin{itemize}
\item The translation for \verb+.TH+ and \verb+.SH+ have been modified.
\verb+.SH NAME+ has a special translation, the line following this
statement must be of type 
\begin{verbatim}
<one-word> - <sentence>
\end{verbatim}
\item \verb+.TP+ and \verb+.IP+ are translated so as to obtain
description or itemize list in \LaTeX. The tag for \verb+.TP+ 
is by default in verb mode (tt font). One can't mix \verb+.TP+ and
\verb+.IP+ in the same enumeration ( the resulting code would not be 
properly indented )
\item \verb+.RS+ and \verb+.RE+ are used to produce indented sequences
of description or itemize~:
\begin{verbatim}
	.TP
	x
	1234567890
	.TP
	x
	1234567890
	.RS
	.IP x
	1234567890
	.IP y
	1234567890
	.RE
	.TP
	z
	1234567890
\end{verbatim}
\item troff Comments beginning with \verb+\"LaTeX+ are recognized and the
rest of the line is transmitted verbatim to \LaTeX. This allows the 
 writer to provide his own translations for the \LaTeX version of
the documentation and to add whatever statement he wants to the LaTeX 
documentation. \verb+.LA+ behaves like \verb+\"LaTeX+~:
\begin{verbatim}
	\"LaTeX \[ x=\sum_{k=0}^n \alpha_k^2 \]
	.LA \[ x=\sum_{k=0}^n \alpha_k^2 \]
\end{verbatim}
\item lines between \verb+.nf+ and \verb+.ni+ are supposed to be verbatim
statements and are translated as such. If the region surrounded with 
\verb+.nf+ and \verb+.ni+ is too large tr2tex can fail,
just cut your region into pieces.
\item \verb+.HR.+ is ignored in nroff and will produce an horizontal
line in \LaTeX
\item \verb+\fV+ will switch to \verb+tt+ font inside text \verb+.Vb+
will put is argument in \verb+tt+ font.
\end{itemize}
\section{An example : \LaTeX translation}
% -*-LaTeX-*-
% Converted automatically from troff to LaTeX by tr2tex on Wed Jul 16 16:46:52 1997
% tr2tex was written by Kamal Al-Yahya at Stanford University
% (Kamal%Hanauma@SU-SCORE.ARPA)


%\documentstyle[troffman]{article}

%
% input file: example.1
%
\phead{macroname}{2}{April\ 1993}{Scilab\ Group}{Scilab\ Function}

\input man1/sci.an
\Sdoc{macroname}{ keywords for whatis}\index{macroname}\label{macroname}

\Shead{CALLING SEQUENCE}
\begin{verbatim}
[out1,out2] = macroname(input1,input2, [optional])
\end{verbatim}
\Shead{PARAMETERS}
\begin{scitem}
\item[{\tt input1}]
: real matrix, meaning.
\item[{\tt input2}]
: character string, meaning.
\item[{\tt out1}]
: polynomial matrix
\end{scitem}% end Env
\Shead{DESCRIPTION}
This is an example of a documentation item.
Here, what the macro is doing. %
\tt X %
\rm (will appear in verb mode in Latex) 
is something ... %
\tt out1 %
\rm is something else
and the result is again something else. If %
\tt input1 %
\rm is a list then 
something else is made and blablabla.
 \ignore{
\begin{verbatim}
          --n
          \\
       x= |  alpha(k)**2
          /  
          --k=0 
\end{verbatim}
 }
\ignore{
\begin{verbatim}
          --n
          \\
       x= |  alpha(k)**2
          /  
          --k=0 
\end{verbatim}
}
 \[ x=\sum_{k=0}^n \alpha_k^2 \]
 \[ x=\sum_{k=0}^n \alpha_k^2 \]\par
\undertext{underline text}
\par
{\bf bold} text
\par
{\it italic} text
\par
\par
\verb? verb in Latex x\y*%pi?
\par
Here a new paragraph which is indented in latex. 
This paragraph contains various  comments.
\par\noindent
Here a new paragraph which is not indented in latex.
\Shead{USING TP AND IP}
IP and TP are used to produce description or itemize or enumerate in
 \LaTeX.
The  tag for TP is translated in verb mode. One can mix sequences 
of TP or IP and indented levels are obtained with .RS 
in verb mode.
\begin{scitem}
\item[{\tt verb-tag}]
: the item description in TP
\item[{\tt verb-tag}]
: the item description in TP
\begin{scitem}
\begin{scitem}
\item[{1}]
an IP-list inserted in TP and surrounded with RS,RE.
\item[{2}]
second IP-item description. IP is used to produce enumerate itemize or 
description
\end{scitem}
\end{scitem}%
\item[{\tt verb-tag}]
: the item description in TP
\end{scitem}% end Env
\par\noindent
\begin{scitem}
\item[{Roman}]
: the item description of an IP-list 
\item[{%
\it Italic}]
: the item description of an IP-list 
\item[{%
\bf Bod}]
: the item description of an IP-list 
\item[{%
\tt Verb}]
: the item description of an IP-list 
\end{scitem}\par\noindent
\Shead{REMARK}
Note that ...
\Seealso{SEE ALSO}
{\verb?name1?} \pageref{name1},{\verb?name2?} \pageref{name2},{\verb?nam_e3?} \pageref{name3}
\Sauthor{AUTHOR}
your name


\section{An example : Troff code}
\verbatok{example.1}
\end{verbatim}


\end{document}
