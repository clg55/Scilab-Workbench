\documentstyle[modif.article]{article}
\textheight=660pt 
\textwidth=470pt 
\topmargin=-27pt 
\oddsidemargin=0pt 
\evensidemargin=0pt 
\title{Travaux dirig\'es 2}
\author{J.Ph. Chancelier et M. Cohen de Lara}
\begin{document}
\maketitle 
\def\cmarg{\hspace{1cm}}
\def\tit#1{ \begin{center} \fbox{{\bf  #1}} \end{center}}


\def\Mhlp{
\begin{flushleft}
{\sl 
\cmarg //$<$$>$=hlp()\\ 
\cmarg \verb@// Travaux diriges : bibliotheque de macros @\\ 
\cmarg \verb@// macros du T.P 2@\\ 
\cmarg \verb@//@\\ 
\cmarg \verb@// tdinit   :  initialise les donnees du td @\\ 
\cmarg \verb@//@\\ 
\cmarg \verb@//    PREMIERE PARTIE @\\ 
\cmarg \verb@// pp       :  Un modele proie-predateur avec insecticide@\\ 
\cmarg \verb@// equilpp  :  calcule un point d'equilibre du systeme pp@\\ 
\cmarg \verb@//             pour la commande constante u@\\ 
\cmarg \verb@//@\\ 
\cmarg \verb@//    DEUXIEME PARTIE @\\ 
\cmarg \verb@// compet   :  Un modele de competition @\\ 
\cmarg \verb@// equilcom :  calcule un point d'equilibre du systeme compet@\\ 
\cmarg \verb@// lincomp  :  linearise le modele compet autour@\\ 
\cmarg \verb@//               du point d'equilibre donne par equilcom(ue)@\\ 
\cmarg \verb@// gaincom  :  calcul de la matrice k de gain pour @\\ 
\cmarg \verb@//               la commande du modele de competition @\\ 
\cmarg \verb@// gainobs  :  calcul de la matrice l de gain pour @\\ 
\cmarg \verb@//               l'observateur du modele de competition@\\ 
\cmarg \verb@// boucle   :  simule les trajectoires d'un systeme observe-controle@\\ 
\cmarg \verb@//          de dimension 2 avec bruit d'observation@\\ 
\cmarg \verb@//          utilise pour le modele de competition @\\ 
\cmarg \verb@// obscont  :  renvoit un systeme observe-controle.@\\ 
\cmarg \verb@// @\\ 
\cmarg \verb@// test     :  dynamique linearisee du systeme precedent @\\ 
\cmarg \verb@//               observateur-controleur (voir boucle)@\\ 
\cmarg \verb@//@\\ 
\cmarg \verb@//    TROISIEME PARTIE @\\ 
\cmarg \verb@// comric   :  calcul d'un gain de commande a l'aide @\\ 
\cmarg \verb@//          d'une equation de riccati@\\ 
\cmarg \verb@// obsric   :  calcul d'un gain d'observateur (de filtre)@\\ 
\cmarg \verb@//          a l'aide d'une equation de riccati@\\ 
\cmarg \verb@// exemple  : un essai de calcul de gains avec riccati@\\ 
\cmarg \verb@//          et simulations.@\\ 
\cmarg \verb@//@\\ 
\cmarg \verb@//!@\\ 
\cmarg 0;\\ 
\cmarg //end}
\end{flushleft}}




\def\Mtdinit{
\begin{flushleft}
{\sl 
\cmarg //$<$$>$=tdinit()\\ 
\cmarg //$<$$>$=tdinit()\\ 
\cmarg \verb@// constantes pour le modele proie-predateur@\\ 
\cmarg \verb@// et le modele de competition  @\\ 
\cmarg \verb@//!@\\ 
\cmarg ppr=1/100;\\ 
\cmarg ppk=1000;\\ 
\cmarg ppa=1/20000;\\ 
\cmarg ppb=1/10000;\\ 
\cmarg ppm=1/100;\\ 
\cmarg pps=1/200;\\ 
\cmarg ppl=500;\\ 
\cmarg $<$ppr,ppk,ppa,ppb,ppm,ppl,pps$>$={\bf resume}(ppr,ppk,ppa,ppb,ppm,ppl,pps);\\ 
\cmarg //end}
\end{flushleft}}



\def\Mpp{
\begin{flushleft}
{\sl 
\cmarg //$<$xdot$>$=pp(t,x,u)\\ 
\cmarg //$<$xdot$>$=pp(t,x,$[$u$]$)\\ 
\cmarg \verb@// Un modele proie-predateur avec insecticide@\\ 
\cmarg \verb@// x(1) : la proie (insecte nuisible)@\\ 
\cmarg \verb@// x(2) : le predateur (un gentil insecte)@\\ 
\cmarg \verb@// u    : taux de mortalite du a l'insecticide @\\ 
\cmarg \verb@//        qui agit a la fois sur la proie et le predateur@\\ 
\cmarg \verb@//        par defaut u=0 (pas d'insecticide)@\\ 
\cmarg \verb@//!@\\ 
\cmarg $<$lhs,rhs$>$={\bf argn}(0);\\ 
\cmarg {\bf if} rhs=2,u=0,{\bf end}\\ 
\cmarg xdot(1) = ppr$\star$x(1)$\star$(1$-$x(1)/ppk) $-$ ppa$\star$x(1)$\star$x(2) $-$ u$\star$x(1);\\ 
\cmarg xdot(2) = $-$ppm$\star$x(2)             + ppb$\star$x(1)$\star$x(2) $-$ u$\star$x(2);\\ 
\cmarg //end}
\end{flushleft}}



\def\Mequilpp{
\begin{flushleft}
{\sl 
\cmarg //$<$xe$>$=equilpp(ue)\\ 
\cmarg //$<$xe$>$=equilpp($[$ue$]$)\\ 
\cmarg \verb@// calcule un point d'equilibre xe du systeme pp@\\ 
\cmarg \verb@// pour la commande constante ue a choisir lors de l'appel de la macro@\\ 
\cmarg \verb@//  ue=0 par defaut@\\ 
\cmarg \verb@//!@\\ 
\cmarg $<$lhs,rhs$>$={\bf argn}(0);{\bf if} rhs=0,ue=0,{\bf end}\\ 
\cmarg xe(1) =  (ppm+ue)/ppb;\\ 
\cmarg xe(2) =  (ppr$\star$(1$-$xe(1)/ppk)$-$ue)/ppa;\\ 
\cmarg //end}
\end{flushleft}}



\def\Mcompet{
\begin{flushleft}
{\sl 
\cmarg //$<$xdot$>$=compet(t,x,u)\\ 
\cmarg //$<$xdot$>$=compet(t,x,$[$u$]$) u est optionnel\\ 
\cmarg \verb@// Un modele de competition @\\ 
\cmarg \verb@// x(1),x(2) deux populations vivant sur une meme ressource @\\ 
\cmarg \verb@// 1/u est le niveau de la ressource (vaut 1 par defaut)@\\ 
\cmarg \verb@// ex : deux souches de levure @\\ 
\cmarg \verb@// ex : des crustaces sur une algue @\\ 
\cmarg \verb@//!@\\ 
\cmarg $<$lhs,rhs$>$={\bf argn}(0);\\ 
\cmarg {\bf if} rhs=2,u=1,{\bf end}\\ 
\cmarg xdot=0$\star${\bf ones}(2,1);\\ 
\cmarg xdot(1) = ppr$\star$x(1)$\star$(1$-$x(1)/ppk) $-$ u$\star$ppa$\star$x(1)$\star$x(2) ,\\ 
\cmarg xdot(2) = pps$\star$x(2)$\star$(1$-$x(2)/ppl) $-$ u$\star$ppb$\star$x(1)$\star$x(2) ,\\ 
\cmarg //end}
\end{flushleft}}



\def\Mequilcom{
\begin{flushleft}
{\sl 
\cmarg //$<$xe$>$=equilcom(ue)\\ 
\cmarg //$<$xe$>$=equilcom($[$ue$]$)\\ 
\cmarg \verb@// calcule un point d'equilibre du systeme compet@\\ 
\cmarg \verb@// pour un niveau de ressource ue  (vaut 1 par defaut)@\\ 
\cmarg \verb@//!@\\ 
\cmarg $<$lhs,rhs$>$={\bf argn}(0);\\ 
\cmarg {\bf if} rhs=0,ue=1,{\bf end}\\ 
\cmarg mat=$<$  ppr/ppk , ue$\star$ppa; ue$\star$ppb , pps/ppl $>$\\ 
\cmarg cte=$<$ppr;pps $>$\\ 
\cmarg xe= {\bf inv}(mat)$\star$cte;\\ 
\cmarg //end}
\end{flushleft}}



\def\Mlincomp{
\begin{flushleft}
{\sl 
\cmarg //$<$f,g,h,linsy$>$=lincomp(ue)\\ 
\cmarg //$<$f,g,h,linsy$>$=lincomp(ue)\\ 
\cmarg \verb@// fournit les matrices f,g et h du modele compet avec sortie@\\ 
\cmarg \verb@// linearise  autour du point d'equilibre donne par equilpp(ue).@\\ 
\cmarg \verb@// la sortie y est la premiere composante x(1) du systeme.@\\ 
\cmarg \verb@// linsy : est une macro qui donne la dynamique du syt\ `eme @\\ 
\cmarg \verb@// lin\ 'eaire tangent @\\ 
\cmarg \verb@//!@\\ 
\cmarg $<$f,g,linsy$>$=tangent('compet',equilcom(ue),ue);\\ 
\cmarg h=$<$1,0$>$;\\ 
\cmarg //end}
\end{flushleft}}




\def\Mgaincom{
\begin{flushleft}
{\sl 
\cmarg //$<$$>$=gaincom(pole,ue)\\ 
\cmarg //$<$$>$=gaincom(pole,$[$ue$]$)\\ 
\cmarg \verb@// calcul de la matrice k de gain pour @\\ 
\cmarg \verb@// la commande du mod\ `ele de competition @\\ 
\cmarg \verb@// de sorte que la matrice f -g*k ait pour p\ \^oles @\\ 
\cmarg \verb@// le vecteur colonne pole@\\ 
\cmarg \verb@//!@\\ 
\cmarg $<$lhs,rhs$>$={\bf argn}(0);\\ 
\cmarg {\bf if} rhs=1,ue=1,{\bf end}\\ 
\cmarg $<$f,g,h$>$=lincomp(ue);\\ 
\cmarg k={\bf ppol}(f,g,pole)\\ 
\cmarg $<$f,g,h,k$>$={\bf resume}(f,g,h,k)\\ 
\cmarg //end}
\end{flushleft}}



\def\Mgainobs{
\begin{flushleft}
{\sl 
\cmarg //$<$$>$=gainobs(pole,ue)\\ 
\cmarg //$<$$>$=gainobs(pole,$[$ue$]$)\\ 
\cmarg \verb@// calcul de la matrice l de gain pour @\\ 
\cmarg \verb@// l'observateur du modele de competition@\\ 
\cmarg \verb@// de sorte que la matrice f -l*h ait pour poles @\\ 
\cmarg \verb@// le vecteur colonne pole @\\ 
\cmarg \verb@//!@\\ 
\cmarg $<$lhs,rhs$>$={\bf argn}(0);\\ 
\cmarg {\bf if} rhs=1,ue=1,{\bf end}\\ 
\cmarg $<$f,g,h$>$=lincomp(ue);\\ 
\cmarg l={\bf ppol}(f',h',pole)\\ 
\cmarg l=l'\\ 
\cmarg $<$f,g,h,l$>$={\bf resume}(f,g,h,l)\\ 
\cmarg //end}
\end{flushleft}}



\def\Mboucle{
\begin{flushleft}
{\sl 
\cmarg //$<$$>$=boucle(fch,abruit,xdim,npts,farrow)\\ 
\cmarg //$<$$>$=boucle(fch,$[$abruit,xdim,npts,farrow$]$)\\ 
\cmarg \verb@// Donne le portrait de phase et des trajectoires du syst\ `eme @\\ 
\cmarg \verb@// dynamique fch suppose etre un systeme dynamique observe-controle@\\ 
\cmarg \verb@// avec sortie bruit\ 'ee  de dimension d'etat 4 ( x:2 , xchap:2)@\\ 
\cmarg \verb@// zdot=fch(t,z) dans le cadre xdim=<xmin,ymin,ymax,ymax>@\\ 
\cmarg \verb@// @\\ 
\cmarg \verb@// Dans la forme d'appel par d\ 'efaut : boucle(fch) les valeurs@\\ 
\cmarg \verb@// du cadre  et les pas d'int\ 'egration sont demand\ 'es interactivement.@\\ 
\cmarg \verb@// La macro boucle  va chercher dans l'environnement @\\ 
\cmarg \verb@// global les valeurs de (xe,ue) le point d'equilibre@\\ 
\cmarg \verb@//   f,g,h les matrices du systeme linearise @\\ 
\cmarg \verb@//   l et k  les deux matrices de gain @\\ 
\cmarg \verb@// Arguments:@\\ 
\cmarg \verb@// fch  : nom du systeme a integrer. @\\ 
\cmarg \verb@//       si c'est une chaine de caract\ `ere, on peut lui donner les valeurs@\\ 
\cmarg \verb@//       'bcomp' : pour mod\ `ele de competition non-lineaire @\\ 
\cmarg \verb@//               observe-commande @\\ 
\cmarg \verb@//       'lcomp' : pour mod\ `ele linearise observe-comande@\\ 
\cmarg \verb@//       sinon on peut lui donner le nom d'une macro que l'on aura cree@\\ 
\cmarg \verb@//       avec la commande obscont @\\ 
\cmarg \verb@//  @\\ 
\cmarg \verb@// abruit : amplitude du bruit rajoute sur les observations@\\ 
\cmarg \verb@//    @\\ 
\cmarg \verb@// npts=<nombre-de-points,pas> ->  sert \ `a donner le nombre de points et @\\ 
\cmarg \verb@//          le pas pour l'int\ 'egration num\ 'erique.@\\ 
\cmarg \verb@// @\\ 
\cmarg \verb@// xdim=<xmin,ymin,xmax,ymax> -> sert \ `a donner le cadre du dessin @\\ 
\cmarg \verb@// farrow vaur 't' ou 'f' : s'il vaut 't' on rajoute des fleches@\\ 
\cmarg \verb@//  le long des trajectoires @\\ 
\cmarg \verb@//!@\\ 
\cmarg $<$lhs,rhs$>$={\bf argn}(0);\\ 
\cmarg \verb@// appel minimal @\\ 
\cmarg {\bf if} rhs$<$=4,farrow='f';{\bf end};\\ 
\cmarg {\bf if} rhs$<$=3,{\bf write}(\%io(2),'integration : nb points,pas du systeme '),\\ 
\cmarg \hspace{0.5cm}npts={\bf read}(\%io(1),1,2),\\ 
\cmarg {\bf end}\\ 
\cmarg {\bf if} rhs $<$= 2,\\ 
\cmarg \hspace{0.5cm}{\bf write}(\%io(2),'cadre du dessin : xmin,ymin,xmax,ymax'),\\ 
\cmarg \hspace{0.5cm}xdim={\bf read}(\%io(1),1,4),\\ 
\cmarg \verb@// Test sur le cadre @\\ 
\cmarg \hspace{0.5cm}{\bf if} xdim(3) $<$= xdim(1),\\ 
\cmarg \hspace{0.5cm}{\bf write}(\%io(2),'Erreur:  xmin $>$ xmax '),{\bf return},{\bf end}\\ 
\cmarg \hspace{0.5cm}{\bf if} xdim(4) $<$= xdim(2),\\ 
\cmarg \hspace{0.5cm}{\bf write}(\%io(2),'Erreur:  ymin $>$ ymax '),{\bf return},{\bf end}\\ 
\cmarg {\bf end} \\ 
\cmarg {\bf if} rhs $<$=1,abruit=0.0;{\bf end} \\ 
\cmarg tcal=npts(2)$\star$(0:npts(1))\\ 
\cmarg {\bf rand}('normal')\\ 
\cmarg br={\bf sqrt}(abruit)$\star${\bf rand}(1,npts(1)+1);\\ 
\cmarg {\bf if} {\bf type}(fch)=10;\\ 
\cmarg \verb@// Passage des constantes a un programme Fortran d'initialisation @\\ 
\cmarg \hspace{0.5cm}idisp=0;\\ 
\cmarg \hspace{0.5cm}pp\_c=$<$ppr,ppk,ppa,ppb,ppm,pps,ppl$>$\\ 
\cmarg \hspace{0.5cm}{\bf fort}('icomp',xe,1,'r',ue,2,'r',f,3,'r',g,4,'r',h,5,'r',...\\ 
\cmarg \hspace{1.5cm}k,6,'r',l,7,'r',br,8,'r',npts(2),9,'r',npts(1),10,'i',...\\ 
\cmarg \hspace{1.5cm}pp\_c,11,'r',idisp,12,'i','{\bf sort}');\\ 
\cmarg {\bf end} \\ 
\cmarg xset("window",0);xselect();xclear();\\ 
\cmarg \verb@// Boucle sur les points de depart @\\ 
\cmarg \hspace{0.5cm}goon=1\\ 
\cmarg \hspace{0.5cm}{\bf while} goon=1,\\ 
\cmarg \hspace{1.8cm}ftest=1;\\ 
\cmarg \hspace{1.8cm}{\bf while} ftest=1,\\ 
\cmarg \hspace{2.5cm}addtitle(fch);\\ 
\cmarg \hspace{2.5cm}plot2d($<$xdim(1);xdim(1);xdim(3)$>$,$<$xdim(2);xdim(4);xdim(4)$>$)\\ 
\cmarg \hspace{2.5cm}plot2d($<$xe(1)$>$,$<$xe(2)$>$,$<$2,4$>$,"111",...\\ 
\cmarg \hspace{3.5cm}"Point d''equilibre pour ue='+{\bf string}(ue),xdim);\\ 
\cmarg \hspace{2.5cm}{\bf write}(\%io(2),'Utilisez la souris : ');\\ 
\cmarg \hspace{2.5cm}{\bf write}(\%io(2),' $-$$>$ Bouton de droite pour quiter ');\\ 
\cmarg \hspace{2.5cm}{\bf write}(\%io(2),' $-$$>$ Bouton du milieu ou de gauche ');\\ 
\cmarg \hspace{2.5cm}{\bf write}(\%io(2),'      pour indiquer x0 ');\\ 
\cmarg \hspace{2.5cm}$<$n,x,y$>$=xclick()\\ 
\cmarg \hspace{2.5cm}{\bf if} n=2,goon=0;{\bf return};{\bf end}\\ 
\cmarg \hspace{2.5cm}$<$xx0,yy0,recc$>$=xchange(x,y,'i2f');\\ 
\cmarg \hspace{2.5cm}x0=$<$xx0,yy0$>$;\\ 
\cmarg \hspace{2.5cm}{\bf write}(\%io(2),'Utilisez la souris : ');\\ 
\cmarg \hspace{2.5cm}{\bf write}(\%io(2),' $-$$>$ Bouton de droite pour quiter ');\\ 
\cmarg \hspace{2.5cm}{\bf write}(\%io(2),' $-$$>$ Bouton du milieu ou de gauche ');\\ 
\cmarg \hspace{2.5cm}{\bf write}(\%io(2),'      pour indiquer xchap0 (observateur) ');\\ 
\cmarg \hspace{2.5cm}$<$n,x,y$>$=xclick()\\ 
\cmarg \hspace{2.5cm}{\bf if} n=2,goon=0;{\bf return};{\bf end}\\ 
\cmarg \hspace{2.5cm}$<$xx0,yy0,recc$>$=xchange(x,y,'i2f');\\ 
\cmarg \hspace{2.5cm}xchap0=$<$xx0,yy0$>$;\\ 
\cmarg \hspace{2.5cm}{\bf if} {\bf type}(fch)=10,\\ 
\cmarg \hspace{3.2cm}ftest=desorb1($<$x0,xchap0$>$',npts,fch,farrow,xdim);\\ 
\cmarg \hspace{2.5cm}{\bf else} \\ 
\cmarg \hspace{3.2cm}ftest=desorb1($<$x0,xchap0$>$',npts,{\bf list}(fch,abruit,...\\ 
\cmarg \hspace{6.2cm}npts(2),npts(1)),farrow,xdim);             \\ 
\cmarg \hspace{2.5cm}{\bf end}\\ 
\cmarg \hspace{2.5cm}{\bf if} ftest=1;{\bf write}(\%io(2),'conditions initiales hors du cadre'),{\bf end}\\ 
\cmarg \hspace{1.8cm}{\bf end}\\ 
\cmarg \hspace{0.5cm}{\bf end}\\ 
\cmarg //end}
\end{flushleft}}



\def\Mdesorbu{
\begin{flushleft}
{\sl 
\cmarg //$<$res$>$=desorb1(x0,n1,fch,farrow,xdim);\\ 
\cmarg //$<$res$>$=desorb1(x0,n1,fch,farrow,xdim);\\ 
\cmarg \verb@//!@\\ 
\cmarg res=0\\ 
\cmarg {\bf write}(\%io(2),'Calculs en cours')\\ 
\cmarg tcal=n1(2)$\star$(0:n1(1))\\ 
\cmarg xxx={\bf ode}(x0,0,tcal,fch);\\ 
\cmarg $<$nn1,nn2$>$={\bf size}(tcal);\\ 
\cmarg comcom=$-$k$\star$(xxx(3:4,:)$-$xe$\star${\bf ones}(1,nn2));\\ 
\cmarg \verb@//dessin de l'evolution conjointe de la deuxieme@\\ 
\cmarg \verb@//composante de l'etat et de son estimee (observateur)@\\ 
\cmarg xset("window",1);xclear();\\ 
\cmarg plot2d($<$tcal;tcal$>$',xxx($<$2,4$>$,:)',$<$$-$1,$-$2$>$,"111",...\\ 
\cmarg \hspace{1.8cm}"x2(t) @observateur de x2(t)",$<$0,xdim(2),n1(1)$\star$n1(2),xdim(4)$>$)\\ 
\cmarg xset("window",2);xclear();\\ 
\cmarg \verb@//dessin de la commande lineaire@\\ 
\cmarg plot2d($<$tcal$>$',$<$comcom$>$',$<$$-$1$>$,"121",...\\ 
\cmarg \hspace{1.8cm}"commande lineaire en fonction du temps (ecart par rapport a ue)")\\ 
\cmarg xset("window",0);xclear();\\ 
\cmarg \verb@//portrait de phase @\\ 
\cmarg plot2d(xxx($<$1,3$>$,:)',xxx($<$2,4$>$,:)',$<$$-$1,$-$2$>$,"111","(x1,x2)@observateur ",...\\ 
\cmarg xdim);\\ 
\cmarg //end }
\end{flushleft}}



\def\Mobscont{
\begin{flushleft}
{\sl 
\cmarg //$<$macr$>$=obscont(sysn)\\ 
\cmarg //$<$macr$>$=obscont(sysn)\\ 
\cmarg \verb@// @\\ 
\cmarg \verb@//cette macro renvoit le systeme observe-commande @\\ 
\cmarg \verb@//construit a partir du systeme sysn linearise autour de (xe,ue)@\\ 
\cmarg \verb@//@\\ 
\cmarg \verb@// sysn : chaine de caractere donnant le nom du systeme a commander @\\ 
\cmarg \verb@// gaincom,gainobs : vecteurs colonnes des gains demandes @\\ 
\cmarg \verb@//@\\ 
\cmarg \verb@// Retour : une nouvelle macro donnant le syst\ `eme observe-commande@\\ 
\cmarg \verb@// <x1dot>=macr(t,x1,abruit,pas,n)@\\ 
\cmarg \verb@//    x1=<x;xchap>,@\\ 
\cmarg \verb@// pour l'appel il faudra creer un vecteur de bruit de nom br @\\ 
\cmarg \verb@// d'autre part la macro cr\ 'ee va chercher dans l'environnement @\\ 
\cmarg \verb@// global les valeurs de (xe,ue) le point d'equilibre@\\ 
\cmarg \verb@//   f,g,h les matrices du systeme linearise @\\ 
\cmarg \verb@//   l et k  les deux matrices de gain @\\ 
\cmarg \verb@// exemple :@\\ 
\cmarg \verb@//    pas=10;n=200;@\\ 
\cmarg \verb@//    br=rand(1,n)@\\ 
\cmarg \verb@//    ode(<x0;xchap0>,0,pas*(0:n),list(macr,1.0,pas,n));@\\ 
\cmarg \verb@//!@\\ 
\cmarg {\bf deff}('$<$zdot$>$=macr(t,z,abr,pas,n)',...\\ 
\cmarg \hspace{1.0cm}$<$'u=ue$-$k(1)$\star$(z(3)$-$xe(1))$-$k(2)$\star$(z(4)$-$xe(2))';\\ 
\cmarg \hspace{1.2cm}'ff\_brui=abr$\star$br({\bf ent}({\bf mini}(t/pas+1,n)))';\\ 
\cmarg \hspace{1.2cm}'y=h(1)$\star$z(1)+h(2)$\star$z(2)+ff\_brui';\\ 
\cmarg \hspace{1.2cm}'xdot='+sysn+'(t,z(1:2),u)';\\ 
\cmarg \hspace{1.2cm}'xdot1=f$\star$(z(3:4)$-$xe) +g$\star$(u$-$ue)$-$l$\star$(h$\star$z(3:4)$-$y)';\\ 
\cmarg \hspace{1.2cm}'zdot=$<$xdot;xdot1$>$';$>$);\\ 
\cmarg //end}
\end{flushleft}}





\def\Mcomric{
\begin{flushleft}
{\sl 
\cmarg //$<$$>$=comric(q1,q2,r)\\ 
\cmarg //$<$$>$=comric(q1,q2,r)\\ 
\cmarg \verb@// pi est la solution de l'equation de riccati stationnaire @\\ 
\cmarg \verb@// f'*pi+pi*f+ q*eye(2)-pi*g*(1/r)*g'*pi=0@\\ 
\cmarg \verb@// pour calculer le gain r\^-1*g'*pi de la commande.@\\ 
\cmarg \verb@// ici, q et r sont des matrices de ponderation d'une fonction @\\ 
\cmarg \verb@// cout quadratique.@\\ 
\cmarg \verb@//!@\\ 
\cmarg $<$pi$>$={\bf ricc}(f,(1/r)$\star$g$\star$g',$<$q1,0;0,q2$>$,'cont');\\ 
\cmarg k= (1/r) $\star$g'$\star$pi;\\ 
\cmarg $<$k$>$={\bf resume}(k);\\ 
\cmarg //end}
\end{flushleft}}



\def\Mobsric{
\begin{flushleft}
{\sl 
\cmarg //$<$$>$=obsric(q1,q2,r)\\ 
\cmarg //$<$$>$=obsric(q1,q2,r)\\ 
\cmarg \verb@// p est la solution de l'equation de riccati stationnaire @\\ 
\cmarg \verb@// f*p+p*f'+ q*eye(2)-(1/r)*p*h'*h*p=0@\\ 
\cmarg \verb@// pour calculer le gain r\^-1*p*h' de l'observateur@\\ 
\cmarg \verb@// confondu avec le filtre de Kalman.@\\ 
\cmarg \verb@// ici, q est la variance du bruit d'etat, @\\ 
\cmarg \verb@//      r la variance du bruit de mesure. @\\ 
\cmarg \verb@//!@\\ 
\cmarg $<$p$>$={\bf ricc}(f',(1/r)$\star$h'$\star$h,$<$q1,0;0,q2$>$,'cont');\\ 
\cmarg l= (1/r) $\star$p$\star$h'\\ 
\cmarg $<$l$>$={\bf resume}(l);\\ 
\cmarg //end}
\end{flushleft}}




\def\Mexemple{
\begin{flushleft}
{\sl 
\cmarg //$<$$>$=exemple()\\ 
\cmarg \verb@// un essai de calcul de gains avec riccati@\\ 
\cmarg \verb@// et simulation.@\\ 
\cmarg \verb@//!@\\ 
\cmarg ue=1;\\ 
\cmarg xe=equilcom(ue);\\ 
\cmarg $<$f,g,h,vv$>$=lincomp(ue);\\ 
\cmarg obsric(3,1,1);\\ 
\cmarg comric(3,1,1);\\ 
\cmarg ue=1;\\ 
\cmarg xe=equilcom(ue);\\ 
\cmarg boucle('bcomp',0.1,$<$0,150,50,210$>$,$<$100,10$>$);\\ 
\cmarg //end}
\end{flushleft}}




\def\Mtest{
\begin{flushleft}
{\sl 
\cmarg //$<$f1,f2$>$=test(ue)\\ 
\cmarg //$<$f1,f2$>$=test(ue)\\ 
\cmarg \verb@// f1 est la dynamique linearisee @\\ 
\cmarg \verb@//autour du point d'equilibre donne par equilcom(ue)@\\ 
\cmarg \verb@//du systeme precedent observateur-controleur (voir simulcomp)@\\ 
\cmarg \verb@// f2 est la valeur theorique de f1 (voir cours)@\\ 
\cmarg \verb@//!@\\ 
\cmarg {\bf deff}('$<$yy,zdot$>$=fff(z,vv)',...\\ 
\cmarg \hspace{1.0cm}$<$'u=1$-$k$\star$z(3:4)';\\ 
\cmarg \hspace{1.2cm}'yy=0';\\ 
\cmarg \hspace{1.2cm}'xdot(1,1) = ppr$\star$z(1)$\star$(1$-$z(1)/ppk) $-$ u$\star$ppa$\star$z(1)$\star$z(2)';\\ 
\cmarg \hspace{1.2cm}'xdot(2,1) = pps$\star$z(2)$\star$(1$-$z(2)/ppl) $-$ u$\star$ppb$\star$z(1)$\star$z(2)';\\ 
\cmarg \hspace{1.2cm}'xdot1 = (f$-$l$\star$h$-$g$\star$k)$\star$z(3:4) + l$\star$(z(1)$-$xe(1))';\\ 
\cmarg \hspace{1.2cm}'zdot=$<$xdot;xdot1$>$'$>$);\\ 
\cmarg xe=equilcom(ue)\\ 
\cmarg $<$f1,g1,h1,k1$>$=lin(fff,$<$xe;0;0$>$,0),\\ 
\cmarg f2=$<$ f, $-$g$\star$k; l$\star$h,f$-$l$\star$h$-$g$\star$k$>$,\\ 
\cmarg //end}
\end{flushleft}}




\centerline{{\bf Le fichier des macros de la s\'eance de T.D. 2}}

\Mhlp

La macro {\bf tdinit} doit \^etre appel\'ee par {\bf tdinit()} pour initialiser
le T.D.

\Mtdinit

\section{Premi\`ere partie : un mod\`ele proie/pr\'edateur}

\tit{pp}

Cette macro d\'efinit le champ de vecteurs correspondant \`a une dynamique
proie/pr\'edateur, avec pour param\`etre le taux de mortalit\'e $u$ d\^u \`a 
un insecticide qui agit \`a la fois sur la proie et le pr\'edateur.
\medskip

\Mpp

\tit{equilpp}

Cette macro calcule les points d'\'equilibre du champ de vecteurs
pr\'ec\'edent {\bf pp} pour un taux de mortalit\'e $ue$ donn\'e.
\medskip

\Mequilpp

\centerline{{\sc Questions}}

\begin{itemize}
\item Taper la commande {\bf equilpp(0)} puis, en fonction du
r\'esultat, choisir un cadre convenable et visualiser le champ de
vecteurs {\bf pp} par une commande du type 
{\bf fchamp(pp,0,?:?:?,?:?:?)}. Visualiser quelques trajectoires
gr\^ace \`a la commande {\bf portrait(pp)}.
\item Comment se d\'eplace le point d'\'equilibre lorsque la commande $u$
cro\^{\i}t~? Commenter~!  Pour visualiser des trajectoires ou des champs de vecteurs correspondant \`a des valeurs non nulle de $u$,
 on remplacera {\bf pp} par {\bf list(pp,u)} dans l'appel de portrait 
ou de fchamp en donnant \`a u la valeur num\'erique souhait\'ee.
\item Pour certaines valeurs de la commande $u$, on obtient un point 
d'\'equilibre \`a composante n\'egative. Commenter.
\end{itemize}



\centerline{{\sc Questions}}

Le macro {\bf tangent} permet d'obtenir les matrices $f,g$
du mod\`ele {\bf pp}
lin\'earis\'e autour du point d'\'equilibre donn\'e par {\bf equilpp$(ue)$}. 
De plus cette macro renvoie comme dernier argument une nouvelle macro 
qui donne la dynamique
du syst\`eme lin\'earis\'e autour de {\bf (equilpp$(ue)$,$ue$)}.

\begin{itemize}
\item Utiliser la macro {\bf tangent}  pour d\'efinir un champ de vecteurs
lin\'eaire {\bf pplin} qui repr\'esente la dynamique lin\'earis\'ee
autour d'un point d'\'equilibre fix\'e {\bf $xe$ = equilpp$(ue)$}. 
La syntaxe en est $<f,g,pplin>=$ {\bf tangent('pp',equilpp$(ue)$,$ue$)}.

\item Superposer le champ {\bf pp} et le champ {\bf pplin}
autour du point d'\'equilibre retenu. On fera attention au moment de l'appel
 de {\bf champ} ou de {\bf portrait} \`a utiliser {\bf list(pplin,$ue$)} 
pour donner la valeur de la commande {\bf ue}.
Visualiser quelques trajectoires
gr\^ace \`a la commande {\bf portrait}. Commenter.
\end{itemize}

\section{Deuxi\`eme partie : un mod\`ele de comp\'etition}

\subsection{Le mod\`ele}

\tit{compet}

Cette macro d\'efinit le champ de vecteurs correspondant \`a une dynamique
de comp\'etition pour une m\^eme ressource entre deux esp\`eces, 
avec pour param\`etre $u$ l'inverse du niveau de la ressource.
\medskip

\Mcompet

\tit{equilcom}

Cette macro calcule les points d'\'equilibre du champ de vecteurs
pr\'ec\'edent {\bf compet} pour un niveau de ressource $1/u$.
\medskip

\Mequilcom

\centerline{{\sc Questions}}

\begin{itemize}
\item Taper la commande {\bf equilcom(1)} puis, en fonction du
r\'esultat, choisir un cadre convenable et visualiser le champ de
vecteurs {\bf compet} par une commande du type 
{\bf fchamp(list(compet,1),0,?:?:?,?:?:?)}. Visualiser quelques trajectoires
gr\^ace \`a la commande {\bf portrait(list(compet,1),\ldots)} et interpr\'eter.
\item Observer le d\'eplacement du point d'\'equilibre 
et la modification du champ de vecteurs lorsque la commande $u$
varie. Commenter.
\end{itemize}

\subsection{Le mod\`ele lin\'earis\'e }

\tit{lincomp}

\Mlincomp

\centerline{{\sc Questions}}
Taper la commande $<f,g,h,complin>=${\bf lincomp$(ue)$} puis
tester la commandabilit\'e et l'observabilit\'e du triplet
$f$,$g$,$h$ pour diff\'erentes valeurs de $ue$ gr\^ace \`a la
commande {\bf contr($f,g$)} (faire {\bf help contr}). 
La macro {\bf complin} donne la dynamique du syst\`eme 
lin\'earis\'e autour du point d'\'equilibre ({\bf equilcom}$(ue)$,$ue$) 
et on peut visualiser les trajectoires 
et le champ associ\'es avec les commandes 
{\bf portrait(list(complin,$ue$),\ldots)} et 
{\bf fchamp(list(complin,$ue$),\ldots)}.


\subsection{Commande lin\'eaire par placement de p\^oles}

\tit{gaincom}

\Mgaincom

\tit{gainobs}

\Mgainobs

\tit{simulcom}
Cette macro simule le syst\`eme boucl\'e par observateur-contr\^oleur
et affiche la deuxi\`eme composante de l'\'etat et son estim\'ee
au cours du temps.

\Mboucle

\centerline{{\sc Questions}}
\begin{itemize}
 \item Apr\`es avoir cr\'e\'e les variables globales $xe$,$ue$,$f$,$g$,$h$,
r\'egler les deux gains $k$ et $l$ de fa\c{c}on ``satisfaisante''
( p\^oles de l'ordre de $- 1/100$).
  \item Taper {\bf boucle('bcomp',\ldots)}, pour visualiser le syst\`eme 
  observ\'e comand\'e. On choisira dans un premier temps n=200,pas=10.
  Observer l'influence des non-lin\'earit\'es suivant le choix du point initial
$x0$.
 \item Bruiter la sortie en tapant {\bf boucle('bcomp',0.5)}
Qu'observe-t-on~?
 \item Changer les valeurs des gains sur l'observateur et la commande et 
reprendre les questions ci-dessus.
 \item La macro obscont permet de cr\'eer un syst\`eme dynamique 
  observ\'e-control\'e \`a partir d'un syst\`eme donn\'e. 
Taper {\bf compet1=obscont('compet')} et visualiser compet1 avec 
{\bf disp(compet1)}. La macro {\bf boucle}
int\`egre en fait une version en Fortran de la macro cr\'e\'ee et
on peut maintenant utiliser {\bf compet1 } dans la macro {\bf boucle}.
\end{itemize}

\Mobscont	

\tit{test}
Cette macro fournit comme premier argument la partie lin\'eaire
du syst\`eme boucl\'e par observateur-contr\^oleur calcul\'ee
 par lin\'earisation. Son deuxi\`eme argument en est la valeur th\'eorique.
\medskip

\Mtest

\centerline{{\sc Questions}}
Tester~!

\end{document}

