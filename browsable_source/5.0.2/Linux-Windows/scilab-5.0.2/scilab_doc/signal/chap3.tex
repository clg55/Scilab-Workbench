% Copyright INRIA

\chapter[FIR Filters]{Design of Finite Impulse Response Filters}
\label{fir}
\index{FIR filters}
\section{Windowing Techniques} 
\index{windowing!FIR filters}
\index{FIR filter design!windowing technique}
\label{s1}

In theory,  the design of FIR filters is straightforward.
One takes the inverse Fourier transform of the desired frequency response
and obtains the discrete time impulse response of the
filter according to (\ref{e1})
%
%%%%%%%%%%%%%%%%%%%%%%%%%%%%%%%%%%%%%%%%%%%
\begin{equation}
\begin{array}{cc}
h(n)=\displaystyle{\frac{1}{2\pi}\int_{-\pi}^{\pi}
H(\omega)e^{j\omega n}d\omega} 
&  -\infty<n<\infty
\end{array}
\label{e1}
\end{equation}
%%%%%%%%%%%%%%%%%%%%%%%%%%%%%%%%%%%%%%%%%%%
%
The problem, in practice, is that for many filters of interest
the resulting impulse response is infinite and non-causal.
An example of this is the low pass filter which, given its cut-off
frequency, $\omega_c$,  is defined by
%
%%%%%%%%%%%%%%%%%%%%%%%%%%%%%%%%%%%%%%%%%%%
\begin{equation}
H(\omega|\omega_c)=\left\{ \begin{array}{ll}
                     1, & \mbox{$|\omega|\leq\omega_c$}\\
                     0, & \mbox{otherwise}
                \end{array}
       \right.
\label{e2}
\end{equation}
%%%%%%%%%%%%%%%%%%%%%%%%%%%%%%%%%%%%%%%%%%%
%
The associated impulse response is obtained by applying (\ref{e1}) to 
(\ref{e2})
which yields
%
%%%%%%%%%%%%%%%%%%%%%%%%%%%%%%%%%%%%%%%%%%%
\begin{equation}
\begin{array}{cc}
h(n|\omega_c)={\displaystyle{\frac{1}{\pi n}\sin(\omega_cn)}} &  
-\infty<n<\infty
\end{array}
\label{e3}
\end{equation}
%%%%%%%%%%%%%%%%%%%%%%%%%%%%%%%%%%%%%%%%%%%
%
A technique for obtaining a finite length
implementation to (\ref{e3}) is to take the N elements of
$h(n)$ which are centered around $n=0$ and to discard 
all the remaining elements.  This operation can
be represented by multiplying the sequence in (\ref{e3}) by an 
appropriately shifted version of a rectangular window of the form
%
%%%%%%%%%%%%%%%%%%%%%%%%%%%%%%%%%%%%%%%%%%%
\begin{equation}
R_N(n)=\left\{ \begin{array}{ll}
                     1, & \mbox{$0 \leq n \leq N-1$}\\
                     0, & \mbox{otherwise}
                \end{array}
       \right.
\label{e4}
\end{equation}
%%%%%%%%%%%%%%%%%%%%%%%%%%%%%%%%%%%%%%%%%%%
%

The magnitude of the resulting windowed sequence frequency response 
is depicted in Figure~\ref{f1} superimposed on the ideal frequency
response (the dotted curve).  The filter illustrated in Figure~\ref{f1} 
has length $N=33$ and
a cut-off frequency of $\omega_c=.2$.  As can be seen, the 
%
\input{\Figdir fir1.tex}
\dessin{{\tt exec('fir1.code')} Rectangularly windowed low-pass filter}{f1}
%
approximation is marked by a ripple in both the pass
and stop bands.  This ripple finds its greatest deviations near the discontinuity
at $\omega_c$.  The observed ripple is due to the convolution of the
ideal frequency response given by (\ref{e2}) with the frequency response
of the rectangular window.  For many applications the ripples in the frequency 
response of Figure~\ref{f1} are unacceptable.

It is possible to decrease the amount of rippling by using different 
types of windows.
The performance of a window is governed by its frequency response.
Since the
frequency response of the window is convolved with the 
desired frequency response the 
objective is to find a window which has a frequency response which
is as impulsive as possible.  That is,  the frequency response 
should have a narrow main lobe with most of the energy
in this lobe and side lobes which are as small as possible.
The width of the main lobe governs, for example, the width
of the transition band between the pass and stop 
bands of a low pass filter.  The side lobes govern the amount 
ripple in the pass and stop bands.  The area under the main lobe
governs the amount of rejection available in
the stop bands. 

The choice of window in the design process is a matter
of trading off, on one hand, the effects of 
transition band width, ripple, and rejection in the stop band with,
on the other hand, the filter length and the window type.

\subsection{Filter Types}
\label{s2}

The Scilab function {\tt wfir} designs four different types
of FIR linear phase filters: low pass, high pass,
band pass, and stop band filters.  The impulse response of the three
latter filters can be obtained from the low pass filter by
simple relations and the impulse response of the
low pass filter is given in (\ref{e3}).

To show the relationship between the four filter types we
first examine Figure~\ref{f2} which illustrates the frequency
response of a low pass filter with cut off frequency denoted
by $\omega_l$.
%
\input{\Figdir fir2.tex}
\dessin{{\tt exec('fir2\_5.code')} Frequency response of a low pass filter}
{f2}
%
The frequency response of a 
high pass filter is illustrated in Figure~\ref{f3}
%
\input{\Figdir fir3.tex}
\dessin{{\tt exec('fir2\_5.code')} Frequency response of a high pass filter}
{f3}
%
where $\omega_h$ denotes, also, the cut off frequency.
Taking the functional form of the low pass filter to be
$H(\omega|\omega_l)$ and that of the high pass filter
to be $G(\omega|\omega_h)$, the relationship between
these two frequency responses is 
%
%%%%%%%%%%%%%%%%%%%%%%%%%%%%%%%%%%%%%%%%%%%
\begin{equation}
G(\omega|\omega_h)=1-H(\omega|\omega_h).
\label{e5}
\end{equation}
%%%%%%%%%%%%%%%%%%%%%%%%%%%%%%%%%%%%%%%%%%%
%
Using the result in (\ref{e3}), the impulse response of the high pass
filter is given by
%
%%%%%%%%%%%%%%%%%%%%%%%%%%%%%%%%%%%%%%%%%%%
\begin{eqnarray}
g(n|\omega_h) &=& \delta(n)-h(n|\omega_h) \\
              &=& \delta(n)-\frac{1}{n\pi}\sin(\omega_hn)
\label{e6}
\end{eqnarray}
%%%%%%%%%%%%%%%%%%%%%%%%%%%%%%%%%%%%%%%%%%%
%
where $\delta(n)=1$ when $n=0$ and is zero otherwise.

For a band pass filter, as illustrated in 
Figure~\ref{f4},
%
\input{\Figdir fir4.tex}
\dessin{{\tt exec('fir2\_5.code')} Frequency response of a band pass filter}
{f4}
%
the functional form of the frequency response,
$F(\omega|\omega_l,\omega_h)$ can be obtained by shifting the
low pass filter two times as follows
%
%%%%%%%%%%%%%%%%%%%%%%%%%%%%%%%%%%%%%%%%%%%
\begin{eqnarray}
F(\omega|\omega_l,\omega_h) &=& H(\omega-\omega_1|\omega_2)+H(\omega+\omega_1|\omega_2)\\
                  \omega_1  &=& \frac{1}{2}(\omega_l+\omega_h)\\
                  \omega_2  &=& \frac{1}{2}(\omega_l-\omega_h).
\label{e7}
\end{eqnarray}
%%%%%%%%%%%%%%%%%%%%%%%%%%%%%%%%%%%%%%%%%%%
%
Thus, the impulse response of the band pass filter is
%
%%%%%%%%%%%%%%%%%%%%%%%%%%%%%%%%%%%%%%%%%%%
\begin{eqnarray}
f(n|\omega_l,\omega_h) &=& e^{j\omega_1n}h(n|\omega_2)+e^{-j\omega_1n}h(n|\omega_2) \\
                       &=& \frac{2}{n\pi}\cos(\omega_1n)\sin(\omega_2n).
\label{e8}
\end{eqnarray}
%%%%%%%%%%%%%%%%%%%%%%%%%%%%%%%%%%%%%%%%%%%
%

	Finally, the stop band filter illustrated in Figure~\ref{f5}
%
\input{\Figdir fir5.tex}
\dessin{{\tt exec('fir2\_5.code')} Frequency response of a stop band filter}
{f5}
%
can be obtained from the band pass filter by the relation
%
%%%%%%%%%%%%%%%%%%%%%%%%%%%%%%%%%%%%%%%%%%%
\begin{equation}
D(\omega|\omega_l,\omega_h)=1-F(\omega|\omega_l,\omega_h)
\label{e9}
\end{equation}
%%%%%%%%%%%%%%%%%%%%%%%%%%%%%%%%%%%%%%%%%%%
%
where $D(\omega|\omega_l,\omega_h)$ is the frequency response of the stop
band filter.  The impulse response of this filter is
%
%%%%%%%%%%%%%%%%%%%%%%%%%%%%%%%%%%%%%%%%%%%
\begin{eqnarray}
d(n|\omega_l,\omega_h) &=& \delta(n)-f(n|\omega_l,\omega_h) \\
                       &=& \delta(n)-\frac{2}{n\pi}\cos(\omega_1n)\sin(\omega_2n).
\label{e10}
\end{eqnarray}
%%%%%%%%%%%%%%%%%%%%%%%%%%%%%%%%%%%%%%%%%%%
%
\subsection{Choice of Windows}
\label{s3}

Four types of windows are discussed
here.  They are the triangular, generalized Hamming,
Kaiser, and Chebyshev windows.  As was noted in the introduction it is the
frequency response of  a window
which governs its efficacy in filter design.  Consequently, for
each window type, we try to give its frequency response and a
qualitative analysis of its features with respect to the frequency 
response of the rectangular window.

	The frequency response of the rectangular window
\index{windows!rectangular} is obtained
by taking the Fourier transform of (\ref{e4})
%
%%%%%%%%%%%%%%%%%%%%%%%%%%%%%%%%%%%%%%%%%%%
\begin{eqnarray}
R_N(\omega) &=& \sum_{n=0}^{N-1}e^{-j\omega n} \\
            &=& \frac{\sin(\omega N/2)}{\sin(\omega/2)}e^{-j(N-1)\omega/2}.
\label{e11}
\end{eqnarray}
%%%%%%%%%%%%%%%%%%%%%%%%%%%%%%%%%%%%%%%%%%%
%
The magnitude of (\ref{e11}) is plotted as the solid line in Figure~\ref{f6}.
Evaluating (\ref{e11}) at $\omega=0$ yields the height of the 
main lobe which is $R_N(0)=N$.  The zeros of $R_N(\omega)$ are
located at $\omega=\pm2\pi n/N$, $n=1,2,\ldots$, and, consequently,
the base of the main lobe has width $4\pi/N$.  The area
under the main lobe can be bounded from above by the area of a
rectangle (depicted by a dotted curve in Figure~\ref{f6})
of area $4\pi$ and from
below by that of a triangle (also shown in Figure~\ref{f6}) of area $2\pi$.
Thus, the area under the main lobe is essentially independent
of the value of $N$ and the percentage area under the 
main lobe decreases with increasing $N$.  This fact is
important  because it illustrates 
that the rectangular window is limited
in its ability to perform like an impulse.

%
\input{\Figdir fir6.tex}
\dessin{{\tt exec('fir6.code')} Magnitude of rectangular window}
{f6}
%


	By comparison the percentage area under the main lobe of 
the triangular window
\index{windows!triangular} is
approximately constant as the value of $N$ increases.  
The impulse response of the triangular window is
%
%%%%%%%%%%%%%%%%%%%%%%%%%%%%%%%%%%%%%%%%%%%
\begin{equation}
T_{2N-1}(n)=
               \left\{ \begin{array}{ll}
                     (n+1)/N, & \mbox{$0\le n\le N-1$}\\
                     (2N-1-n)/N, & \mbox{$N\le n\le 2N-2$}\\
                     0, & \mbox{otherwise}.
                \end{array}
       \right.
\label{e12}
\end{equation}
%%%%%%%%%%%%%%%%%%%%%%%%%%%%%%%%%%%%%%%%%%%
%
Since the impulse response for the triangular window
can be obtained by scaling the rectangular window by $1/\sqrt{N}$
and convolving it with itself, the frequency response, $T_{2N-1}(\omega)$,
is the square of $R_N(\omega)/N$ or
%
%%%%%%%%%%%%%%%%%%%%%%%%%%%%%%%%%%%%%%%%%%%
\begin{equation}
T_{2N-1}(\omega)=\frac{\sin^2(\omega N/2)}{N\sin^2(\omega/2)}e^{-j(N-1)\omega}.
\label{e13}
\end{equation}
%%%%%%%%%%%%%%%%%%%%%%%%%%%%%%%%%%%%%%%%%%%
%
As can be seen from (\ref{e13}), the width of the main
lobe of the triangular window is the same width as that
of the rectangular window (i.e. $4\pi/N$).  However, the impulse response
of the triangular window is twice as long as that of the 
rectangular window.  Consequently, the triangularly
windowed filter shows less ripple but broader transition bands than
the rectangularly windowed filter.  

	The Hamming window
\index{windows!Hamming} is like the triangular window
in that its main lobe is about twice as wide as that of
a rectangular window for an equal length impulse response.
All but $.04\%$ of the Hamming windows energy is in the
main lobe.  The Hamming window is defined by
%
%%%%%%%%%%%%%%%%%%%%%%%%%%%%%%%%%%%%%%%%%%%
\begin{equation}
H_N(n)=
        \left\{ \begin{array}{ll}
 \alpha+(1-\alpha)\cos(\frac{2\pi n}{N}), & \mbox{$-(N-1)/2\le n\le (N-1)/2$}\\
             0, & \mbox{otherwise}.
                \end{array}
       \right.
\label{e14}
\end{equation}
%%%%%%%%%%%%%%%%%%%%%%%%%%%%%%%%%%%%%%%%%%%
%
where $\alpha=.54$.  Other values for $\alpha$ are possible.
For example when $\alpha=.5$ then  (\ref{e14}) is known as the
Hanning window.. The frequency response of (\ref{e14})
can be obtained by noting that $H_N(n)$ is a
rectangularly windowed version of the constant $\alpha$ and an
infinite length cosine.  Thus
%
%%%%%%%%%%%%%%%%%%%%%%%%%%%%%%%%%%%%%%%%%%%
\begin{eqnarray}
H_N(\omega) &=&
R_N(\omega)*\\
&&[\alpha\delta(\omega)+\frac{1}{2}(1-\alpha)\delta(\omega
-\frac{2\pi}{N})+\frac{1}{2}(1-\alpha)\delta(\omega+\frac{2\pi}{N})]\\
&=& \alpha R_N(\omega)+(\frac{1-\alpha}{2})R_N(\omega+\frac{2\pi}{N})+
(\frac{1-\alpha}{2})R_N(\omega-\frac{2\pi}{N}).
\label{e15}
\end{eqnarray}
%%%%%%%%%%%%%%%%%%%%%%%%%%%%%%%%%%%%%%%%%%%
%
where the ``$*$'' symbol denotes convolution.

	The Kaiser window
\index{windows!Kaiser} is defined as
%
%%%%%%%%%%%%%%%%%%%%%%%%%%%%%%%%%%%%%%%%%%%
\begin{equation}
K_N(n)=
        \left\{ \begin{array}{ll}
             \frac{I_o(\beta\sqrt{1-[2n/(N-1)]^2})}{I_o(\beta)},
                &  \mbox{$-(N-1)/2\le n\le(N-1)/2$}\\
             0, &  \mbox{otherwise}.
                \end{array}
       \right.
\label{e16}
\end{equation}
%%%%%%%%%%%%%%%%%%%%%%%%%%%%%%%%%%%%%%%%%%%
%
where $I_o(x)$ is the modified zeroth-order Bessel function and
$\beta$ is a constant which controls the trade-off of the side-lobe
heights and the width of the main lobe.  The Kaiser 
window yields an optimal window in the sense that
the side lobe ripple is minimized in the least squares
sense for a certain main lobe width.
A closed form for the frequency response of the Kaiser window
is not available.

	The Chebyshev window\index{windows!Chebyshev}
is obtained as the inverse DFT of
a Chebyshev polynomial evaluated at equally spaced intervals
on the unit circle.  The Chebyshev window uniformly minimizes the amount
of ripple in the side lobes for a given main lobe width and filter
length.  A useful aspect of the design procedure for the Chebyshev
window is that given any two of the three parameters: the
window length, $N$; half the main lobe width, $\delta_f$; the side lobe
height, $\delta_p$, the third can be determined analytically using the formulas
which follow.	For $\delta_f$ and $\delta_p$ known, $N$ is obtained from
%
%%%%%%%%%%%%%%%%%%%%%%%%%%%%%%%%%%%%%%%%%%%
\begin{equation}
N \ge  1+\frac{\cosh^{-1}((1+\delta_p)/(\delta_p))}{\cosh^{-1}(1/(\cos(\pi\delta_f)))}.
\label{e17}
\end{equation}
%%%%%%%%%%%%%%%%%%%%%%%%%%%%%%%%%%%%%%%%%%%
%
For $N$ and $\delta_p$ known, $\delta_f$ is obtained from
%
%%%%%%%%%%%%%%%%%%%%%%%%%%%%%%%%%%%%%%%%%%%
\begin{equation}
\delta_f=\frac{1}{\pi}\cos^{-1}(1/\cosh(\cosh^{-1}((1+\delta_p)/\delta_p)/(N-1))).
\label{e18}
\end{equation}
%%%%%%%%%%%%%%%%%%%%%%%%%%%%%%%%%%%%%%%%%%%
%
Finally, for $N$ and $\delta_f$ known, $\delta_p$ is obtained from
%
%%%%%%%%%%%%%%%%%%%%%%%%%%%%%%%%%%%%%%%%%%%
\begin{equation}
\delta_p=[\cosh((N-1)\cosh^{-1}(1/\cos(\pi\delta_f)))-1]^{-1}.
\label{e19}
\end{equation}
%%%%%%%%%%%%%%%%%%%%%%%%%%%%%%%%%%%%%%%%%%%
%
\subsection{How to use {\tt wfir}}
\index{FIR filter design!function syntax}
\index{function syntax!wfir@{\tt wfir}}
\label{s4}

	The syntax for the function {\tt wfir} is as follows can take 
two formats.  The first format is as follows:
\begin{verbatim}
--> [wft,wfm,fr]=wfir()
\end{verbatim}
where the parentheses are a required part of the name.  
This format of the function is interactive and will prompt the user for 
required input parameters such as the filter type (lp='low pass',
hp='high pass', bp='band pass', sb='stop band'),
filter length (an integer $n>2$), window type (re='rectangular',
tr='triangular', hm='hamming', kr='kaiser', ch='chebyshev')
and other special parameters such as $\alpha$ for the
the generalized Hamming window ($0<\alpha<1$) and $\beta$
for the Kaiser window ($\beta>0$).
The three returned arguments are:
%
\begin{itemize}
\item{wft:}
  A vector containing the windowed filter coefficients for a filter of length n.
\item{wfm:}
  A vector of length 256 containing the frequency response of the windowed 
  filter.
\item{fr:}
  A vector of length 256  containing the frequency axis values 
  ($0\le$ fr$\le .5$) associated to the values contained in wfm.
\end{itemize}
%
The second format of the function is as follows:
\begin{verbatim}
--> [wft,wfm,fr]=wfir(ftype,forder,cfreq,wtype,fpar)
\end{verbatim}
This format of the function is not interactive and, consequently, all
the input parameters must be passed as arguments to the function.  
The first argument {\tt ftype} indicates the type of filter to
be constructed and can take the values {\tt 'lp'}, {\tt 'hp'},
{\tt 'bp'}, and {\tt sb'} representing, respectively the filters
low-pass, high-pass, band-pass, and stop-band.  
The argument {\tt forder} is a positive
integer giving the order of the desired filter.  The argument
{\tt cfreq} is a two-vector for which only the first element is
used in the case of low-pass and high-pass filters.  Under these
circumstances {\tt cfreq(1)} is the cut-off frequency (in normalized
Hertz) of the desired filter. For band-pass and stop-band filters
both elements of {\tt cfreq} are used, the first being the low
frequency cut-off and the second being the high frequency cut-off
of the filter.  Both values of {\tt cfreq} must be in the range
$[0,.5)$ corresponding to the possible values of a discrete
frequency response.  The argument {\tt wtype} indicates the type
of window desired and can take the values {\tt 're'}, {\tt 'tr'},
{\tt 'hm'}, {\tt 'hn'}, {\tt 'kr'}, and {\tt 'ch'} representing,
respectively, the windows rectangular, triangular, Hamming, Hanning,
Kaiser, and Chebyshev.  Finally, the argument {\tt fpar} is a two-vector
for which only the first element is used in the case of Kaiser window
and for which both elements are used in the case of a Chebyshev window.
In the case of a Kaiser window the first element of {\tt fpar} 
indicates the relative trade-off between the main lobe of the window
frequency response and the side-lobe height and must be a positive integer.
For more on this parameter see \cite{rabiner}.  For the case of the
Chebyshev window one can specify either the width of the window's main
lobe or the height of the window sidelobes.  The first element of
{\tt fpar} indicates the side-lobe height and must take a value
in the range $[0,1)$ and the second element 
gives the main-lobe width and must take a value in the range $[0,.5)$.
The unspecified element of the {\tt fpar}-vector is indicated by
assigning it a negative value.  Thus, {\tt fpar=[.01,-1]} means
that the Chebyshev window will have side-lobes of height $.01$ and
the main-lobe width is left unspecified.

	Note: Because of the properties of FIR linear phase filters it is
not possible to design an even length high pass or stop band filter.
 
\subsection{Examples}
\index{FIR filter design!examples}
\label{s5}

	This section gives several examples of windowed filter design.
In the first example we choose a low pass filter of length $n=33$
using a Kaiser window with parameter $\beta=5.6$.  The resulting magnitude
of the windowed filter is 
plotted in Figure~\ref{f7} where the magnitude axis is given on a log scale.
%
\input{\Figdir fir7.tex}
\dessin{{\tt exec('fir7.code')} Low pass filter with Kaiser window, $n=33$, $\beta=5.6$}
{f7}
%

	The second example is a stop band filter of length 127
using a Hamming window with parameter $\alpha=.54$.  The resulting
magnitude of the windowed filter is plotted in Figure~\ref{f8} where
the magnitude is given on a log scale.
%
\input{\Figdir fir8.tex}
\dessin{{\tt exec('fir8.code')} Stop band filter with Hamming window, $n=127$, $\alpha=.54$}
{f8}
%

	The third example is a band pass filter of length 55
using a Chebyshev window with parameter $dp=.001$ and $df=.0446622$.  The resulting
magnitude of the windowed filter is plotted in Figure~\ref{f9} where
the magnitude is given on a log scale.
%
\input{\Figdir fir9.tex}
\dessin{{\tt exec('fir9.code')} Band pass filter with Chebyshev window, $n=55$, $dp=.001$, $df=.0446622$}
{f9}
%


\section{Frequency Sampling Technique}
\index{FIR filter design!frequency sampling}

This technique is based on specification of a set of samples
of the desired frequency response at $N$ uniformly spaced points around the unit circle, where $N$ is the filter length.
The z-transform of an FIR filter is easily shown to be :
%%%%%%%%%%%%%%%%
\begin{eqnarray}
H(z)=\frac{1-z^{-N}}{N}\sum_{k=0}^{N-1}\frac{H(k)}{(1-z^{-1}e^{j(2\pi/N)k})}
\label{e.fs.1}
\end{eqnarray}
%%%%%%%%%%%%%%%%
This means that one way of approximating any continuous frequency
response is to {\em sample in frequency}, at $N$ equi-spaced points around the 
unit circle (the frequency samples), and interpolate between them to obtain
the continuous frequency response. Thus, the approximation error will
be exactly zero at the sampling frequencies and finite between them.
This fact has to be related to the reconstruction of a continuous
function from its samples, as exposed in section~\ref{c2.sampling} 
for the case of a continuous-time signal.

	The interpolation formula for an FIR filter, that is its frequency response, is obtained by evaluating (\ref{e.fs.1}) on the unit circle:
%%%%%%%%%%%%%%%
\begin{eqnarray}
H(e^{j\omega}) &=& \frac{e^{-j\omega(N-1)/2}}{N}\sum_{k=0}^{N-1}\frac{H(k)e^{-jk\pi/N}\sin(N\omega/2)}{\sin(\omega/2-k\pi/N)}\nonumber\\
&=& \frac{e^{-j\omega(N-1)/2}}{N}\sum_{k=0}^{N-1}H(k)S(\omega,k)
\end{eqnarray}
%%%%%%%%%%%%%%%
where
%%%%%%%%%%%%%%%
\begin{eqnarray}
S(\omega,k) &=& e^{-jk\pi/N}\frac{\sin(N\omega/2)}{\sin(\omega/2-k\pi/N)}\nonumber\\
&=&\pm e^{-jk\pi/N}\frac{\sin(N(\omega/2)-k\pi/N)}{\sin(\omega/2-k\pi/N)}
\end{eqnarray}
%%%%%%%%%%%%%%
are the interpolating functions.
Thus, the contribution of every frequency sample to the continuous
frequency response is proportional to the interpolating function
$\sin(N\omega/2)/\sin(\omega/2)$
shifted by \(k\pi/N\) in frequency.
	The main drawback of this technique is the lack of flexibility
in specifying the transition band width, which is equal to the number of samples the user decides to put in times \(\pi/N\), and thus is 
strongly related to $N$.
Moreover, the specification of frequency samples in transition bands, giving minimum ripple near the band edges, is not immediate.
Nevertheless, it will be seen, in a later chapter on filter optimization
techniques, that simple linear programming techniques can be used to drastically reduce the error approximation by optimizing only those samples located in the transition bands. To illustrate this point, Figure~\ref{f.fs1} shows the response obtained for a type 1 band pass filter with length 65 : 
first with no sample in the transition bands and second (dashed curve) with one sample 
of magnitude .5 in each of these bands. It is worth noting at this point 
that the linear-FIR design problem with arbitrary frequency response 
specification is more efficiently solved using a minmax approximation 
approach, which is exposed in the next section. \\
%%%%%%%%%%%
\input{\Figdir fstyp121.tex}
\dessin{{\tt exec('fstyp121.code')}Type 1 band pass filter with no sample or one sample in each transition band}{f.fs1}
%%%%%%%%%%
Finally, depending on where the initial frequency sample occurs, two distinct sets of frequency samples can be given, corresponding to the so-called
 type 1 and type 2 FIR filters :\\

\begin{eqnarray}
f_k&=&\frac{k}{N} \; \; k=0, \ldots ,N-1 \mbox{ for type 1 filters}\nonumber\\
f_k&=&\frac{k+1/2}{N}\; \; k=0,\ldots ,N-1 \mbox{ for type 2 filters}\nonumber
\end{eqnarray}
The type of design is at user's will and depends on the application:
for example, a band edge may be closer to a type 1 than to a type 2 frequency
sampling point. This point is illustrated in Figure~\ref{f.fs2}
for the case of a low pass filter with length 64 and no sample in 
the transition band.
%%%%%%%%%%%
\input{\Figdir fstyp122.tex}
\dessin{{\tt exec('fstyp122.code')}Type 1 and type 2 low pass filter}{f.fs2}
%%%%%%%%%%
The full line (resp. the dashed line) gives the approximated response
for the type 1 (resp. type 2) FIR linear filter.
	We give now the way the two previous examples have been 
generated and the code of the function {\tt fsfir} which calculates 
the approximated response.
Figure~\ref{f.fs1} was obtained with the following set of instructions :\\
\verbatok{\Diary fstyp121.dia}
\end{verbatim}
and Figure~\ref{f.fs2} with :
\verbatok{\Diary fstyp122.dia}
\end{verbatim}

\section{Optimal filters}
\index{FIR filter design!minimax optimization}

     The design of FIR linear phase filters
is a topic addressed in some detail in the section on windowed
filter design.  The essential idea behind the
techniques of windowed filter design is to obtain
a filter which is close to a minimum squared error 
approximation to the desired filter.  
 This section is devoted to the description of a filter
design function which seeks to
optimize such an alternative  criterion : the  minimax or Chebyshev 
error approximation.


\subsection{Minimax Approximation}
\index{minimax approximation}
\index{Chebyshev approximation}
\index{optimal FIR filter design}

   To illustrate 
the problem of minimax approximation
we propose an overspecified system of $N$ linear equations
in $M$ unknowns where $N>M$.  If $x$ represents the
unknown $M$-vector then the system of equations can be written
as
%
\begin{equation}
Ax=b
\label{e5.01}
\end{equation}
%
where $A$ is an $N\times M$ matrix and $b$ is an $N$-vector.
In general, no solution will exist for (\ref{e5.01}) and,
consequently, it is reasonable to seek an approximation
to $x$ such that the error vector
%
\begin{equation}
r(x)=Ax-b
\label{e5.02}
\end{equation}
%
is in some way minimized with respect to $x$.

     Representing the $N$ components of $r(x)$ as
$r_k(x)$, $k=1,2,\ldots,N$ the minimax approximation
problem for the system of linear equations in (\ref{e5.01}) can
be posed as follows.  The minimax approximation, $\hat{x}_{\infty}$,
is obtained by finding the solution to
%
\begin{equation}
\hat{x}_{\infty}=\arg\min_x ||r_k(x)||_{\infty}
\label{e5.03}
\end{equation}
%
where
%
\begin{equation}
||r_k(x)||_{\infty}=\max_k |r_k(x)|.
\label{e5.04}
\end{equation}
%
Equation (\ref{e5.04}) defines the supremum norm of the vector 
$r(x)$.  The supremum norm of $r(x)$ for a particular
value of $x$ is the component of $r(x)$ (i.e., the $r_k(x)$)
which is the largest.  The minimax approximation
in (\ref{e5.03}) is the value of $x$ which, out of all possible
values for $x$, makes (\ref{e5.04}) the smallest.

     The minimax approximation can be 
contrasted by the minimum squared error approximation,
$\hat{x}_2$, as defined by
%
\begin{equation}
\hat{x}_2=\arg\min_x ||r(x)||_2
\label{e5.05}
\end{equation}
%
where
%
\begin{equation}
||r(x)||_2=[\sum_{k=1}^{N}{r_k}^2(x)]^{1/2}.
\label{e5.06}
\end{equation}
%
There is a relationship between (\ref{e5.04}) and (\ref{e5.06}) which
can be seen by examining the class of norms defined on $r(x)$ by
%
\begin{equation}
||r(x)||_n=[\sum_{k=1}^{N}{r_k}^n(x)]^{1/n}.
\label{e5.07}
\end{equation}
%
For $n=2$ the expression in (\ref{e5.07}) is the squared error
norm in (\ref{e5.06}) and for $n\rightarrow \infty$ the norm in (\ref{e5.07})
becomes the supremum norm in (\ref{e5.04}) (which explains the 
notation $||\cdot||_{\infty}$).  If
$r(x)$ was a continuous function instead of a discrete
component vector then the sum in (\ref{e5.06}) would become an
integral and the interpretation of the approximation in 
(\ref{e5.05}) would be that
the best approximation was the one which minimized
the area under the magnitude of the error function $r(x)$.
By contrast the interpretation of the approximation in (\ref{e5.03})
would be that the best approximation is the one which minimizes the maximum
magnitude of $r(x)$.

	As an example, consider the system of four linear
equations in one unknown:
%
\begin{eqnarray}
x&=&2\nonumber\\
\frac{1}{3}x&=&1\nonumber\\
x&=&4\nonumber\\
\frac{6}{15}x&=&3
\label{e5.08}
\end{eqnarray}
%
The plot of the magnitude of the four error functions
$|r_k(x)|$, $k=1,2,3,4$ is shown in Figure~\ref{f5.01}.
%
\input{\Figdir remez1.tex}
\dessin{{\tt exec('remez1.code')} Minimax Approximation for Linear Equations}
{f5.01}
%
Also shown in Figure~\ref{f5.01} is a piece-wise continuous
function denoted by the cross-hatched segments of the $r_k(x)$.  This
is the function which represents $||r(x)||_{\infty}$ as a function
of $x$.  Consequently, it is easy to see which value
of $x$ minimizes $||r(x)||_{\infty}$ for this problem.  It is
the value of $x$ which lies at the cross-hatched intersection of the
functions $|x-2|$ and $|\frac{6}{15}x-3|$,  that is 
$\hat{x}_{\infty}=3.571$.  The maximum error at this value is 
$1.571$.

	By comparison the mean squared error
approximation for a system of linear equations as in 
(\ref{e5.01}) is
%
\begin{equation}
\hat{x}_2=(A^TA)^{-1}A^Tb
\label{e5.09}
\end{equation}
%
where $T$ denotes the transpose  (and assuming that $A^TA$ is
invertible).  Consequently, for the example in (\ref{e5.08})
we have that $A=[1, \frac{1}{3}, 1, \frac{6}{15}]^T$ and
that $b=[2, 1, 4, 3]^T$ and thus $\hat{x}_2=3.317$.  The maximum 
error here is $1.673$.  As expected the maximum error for
the approximation $\hat{x}_2$ is bigger than that for the
approximation $\hat{x}_{\infty}$.

\subsection{The Remez Algorithm}
\index{Remez algorithm}


	The Remez algorithm seeks to uniformly minimize
the magnitude of an error function $E(f)$ on an interval $[f_0,f_1]$.
In the following discussion
the function $E(f)$ takes the form of a weighted difference of two functions
%
\begin{equation}
E(f)=W(f)(D(f)-H(f))
\label{e5.1}
\end{equation}
%
where $D(f)$ is a single-valued function
which is to be approximated by $H(f)$, and $W(f)$ is a 
positive weighting function.  The Remez algorithm iteratively searches 
for the $H^*(f)$ such that
%
\begin{equation}
H^*(f)=\arg \min_{H(f)}\|E(f)\|_{\infty}
\label{e5.2}
\end{equation}
%
where
%
\begin{equation}
\|E(f)\|_{\infty}=\max_{f_0\le f\le f_1}|E(f)|
\label{e5.3}
\end{equation}
%
is known as both the Chebyshev and the minimax norm of $E(f)$.
The details of the Remez algorithm can be found in \cite{cheney}.

	The function $H(f)$ is 
constrained, for our purposes, to the class of 
functions
%
\begin{equation}
H(f)=\sum_{n=0}^{N}a_n\cos(2\pi fn).
\label{e5.4}
\end{equation}
%
Furthermore, we take the interval of approximation to
be $[0,.5]$.  Under these conditions the posed problem
corresponds to digital filter design where the functions
$H(f)$ represent the discrete Fourier transform of an
FIR, linear phase filter of odd length and even symmetry.
Consequently, the function $H(f)$ can be written
%
\begin{equation}
H(f)=\sum_{n=-N}^{N}h_n e^{-j2\pi fn}
\label{e5.5}
\end{equation}
%
The relationship between the 
coefficients in (\ref{e5.4}) and (\ref{e5.5})
is $a_n=2h_n$ for $n=1,2,\ldots,N$ and $a_0=h_0$.

   With respect to the discussion in the previous section the problem
posed here can be viewed as an overspecified system
of linear equations in the  $N+1$ unknowns, $a_n$, where the
number of equations is uncountably infinite.  The Remez
algorithm seeks to solve this overspecified system of linear equations
in the minimax sense.  The next section describes the Scilab
function {\tt remezb} and how it can be used to design FIR filters.

\subsection{Function {\tt remezb}}
\index{function syntax!remezb@{\tt remezb}}
\index{optimal FIR filter design!function syntax}

	The syntax for the function {\tt remezb} is as 
follows:
{\tt
\begin{verbatim}
--> an=remezb(nc,fg,df,wf)
\end{verbatim}}
\noindent where {\tt df} and {\tt wf} are vectors which are sampled values
of the functions $D(f)$ and $W(f)$ (see the previous section for definition
of notation), respectively.  The sampled
values of $D(f)$ and $W(f)$ are taken on a grid of points
along the $f$-axis in the interval $[0,.5]$.
The values of the frequency grid are in the vector 
{\tt fg}.  
The values of {\tt fg} are not obliged to be equi-spaced in the interval
$[0,.5]$.  In fact,
it is very useful, for certain problems, to specify an
{\tt fg} which has elements which are equi-spaced in only certain
sub-intervals of $[0,.5]$ (see the examples in the following section).
The value of {\tt nc} is the number
of cosine functions to be used in the
evaluation of the approximating function
$H(f)$ (see (\ref{e5.4})).  
The value of the variable {\tt nc} must be a positive, odd integer
if the problem is to correspond to an FIR filter.
The {\tt an} are the values of the
coefficients in (\ref{e5.4}) which correspond to the optimal
$H(f)$.

	To obtain the coefficients of the corresponding
FIR filter it suffices to create a vector {\tt hn}
using the Scilab commands:
\verbatok{\Diary remez8.code}
\end{verbatim}

Even length filters can be implemented as follows.
For an even length filter to have linear phase the 
filter must have even symmetry about the origin.  Consequently,
it follows that the filter must take values at the points
$n=\pm\frac{1}{2},\pm\frac{3}{2},\ldots,\pm\frac{N-1}{2}$ 
and that the frequency
response of the filter has the form
%
\begin{equation}
H(f)=\sum_{n=-N-\frac{1}{2}}^{N+\frac{1}{2}}h_ne^{-j2\pi fn}.
\label{e5.001}
\end{equation}
%
Due to the even symmetry of the frequency response, $H(f)$, 
(\ref{e5.001}) can be put into the form
%
\begin{equation}
H(f)=\sum_{n=1}^{N}b_n\cos[2\pi(n-\frac{1}{2})f]
\label{e5.002}
\end{equation}
%
where the relationship between the $h_n$ in (\ref{e5.001}) and the 
$b_n$ in (\ref{e5.002}) is $h(n)=\frac{1}{2}b(N-n)$ for
$n=1,2,\ldots,N$.

	The expression for $H(f)$ in (\ref{e5.002}) can be rewritten
so that
%
\begin{equation}
H(f)=\cos(\pi f)\sum_{n=0}^{N-1}\tilde{b}_n\cos(2\pi nf).
\label{e5.003}
\end{equation}
%   	
where $b(n)=\frac{1}{2}[\tilde{b}(n-1)+\tilde{b}(n)]$
for $n=2,3,\ldots,N-1$ and $b(1)=\tilde{b}(0)+\frac{1}{2}\tilde{b}(1)$ and
$b(N)=\frac{1}{2}\tilde{b}(N-1)$.
Since the expression in (\ref{e5.003}) is identical to that in
(\ref{e5.4}) all that is required to make the function {\tt remezb}
work is a change in the values of the desired and weight vectors
by the factor $\cos^{-1}(\pi f)$.  That is, the arguments given
to the function {\tt remezb} are {\tt ddf} and {\tt wwf} where
$\mtt{ ddf}=\mtt{ df}/\cos(\pi f)$ and $\mtt{ wwf}=\mtt{ wf}\cos(\pi f)$.
Caution must be used in choosing the values of {\tt fg} since
for $f=.5$ the division of {\tt df} by $\cos(\pi f)=0$ is not acceptable.
The output, {\tt an}, of the function can be converted to the filter coefficients
{\tt hn} by using the Scilab commands
\verbatok{\Diary remez2.code}
\end{verbatim}

     Noting that the form of (\ref{e5.003}) has the term
$\cos(\pi f)$ as a factor, it can be seen that $H(.5)=0$ regardless
of the choice of filter coefficients.  Consequently, the user should
not attempt to design filters which are of even length and which 
have non-zero magnitude at $f=.5$.


\subsection{Examples Using the function {\tt remezb}}
\index{optimal FIR filter design!examples}

     Several examples are presented in this section.  These 
examples show the capabilities and properties of the function
{\tt remezb}.  The first example is that of a low-pass filter
with cut-off frequency .25.  The number of cosine functions used
is 21.  The input data to the function are first created and then 
passed to the function {\tt remezb}. The subsequent output of cosine
coefficients is displayed below.

     Notice that the frequency grid {\tt fg} is a vector 
of frequency values which are equally spaced in the interval $[0,.5]$.
The desired function {\tt ds} is a vector of the same length as
{\tt fg} and which takes the value $1$ in the interval $[0,.25]$
and the value $0$ in $(.25,.5]$.  The weight function {\tt wt}
is unity for all values in the interval. 

\verbatok{\Diary remez4.dia}
\end{verbatim}
As described in the previous section the cosine coefficients {\tt an}
are converted into the coefficients for a even symmetry FIR filter
which has frequency response as illustrated in Figure~\ref{f5.1}.
%
\input{\Figdir remez2.tex}
\dessin{{\tt exec('remez2\_4.code')} Low Pass Filter with No Transition Band}
{f5.1}
%

The error of the solution illustrated in Figure 3.13 is very large;
it can become reasonable by 
leaving a transition band when giving the specification of the 
frequency grid.  The following example shows how this is done; 
{\tt remezb} is specified as follows :
\verbatok{\Diary remez5.dia}
\end{verbatim}
Here the frequency grid {\tt fg} is specified in the intervals
$[0,.24]$ and $[.26,.5]$ leaving the interval $[.24,.26]$
as an unconstrained transition band.  The frequency magnitude 
response of the resulting filter is illustrated in Figure~\ref{f5.2}.
As can be seen the response in Figure~\ref{f5.2} is much more
acceptable than that in Figure~\ref{f5.1}.
%
\input{\Figdir remez3.tex}
\dessin{{\tt exec('remez2\_4.code')} Low Pass Filter with Transition Band $[.24,.26]$}
{f5.2}
%

     A third and final example using the function {\tt remezb}
is illustrated below.  In this example the desired function
is triangular in shape.  The input data was created using 
the following Scilab commands
\verbatok{\Diary remez6.dia}
\end{verbatim}
The resulting frequency magnitude response  is illustrated 
in Figure~\ref{f5.3}.This example illustrates the strength of the 
function {\tt remezb}.
The function is not constrained to standard filter design problems
such as the class of band pass filters.  The function is capable
of designing linear phase FIR filters of any desired magnitude 
response.

%
\input{\Figdir remez4.tex}
\dessin{{\tt exec('remez2\_4.code')} Triangular Shaped Filter}
{f5.3}
%

\subsection{Scilab function {\tt eqfir}}
\index{function syntax!eqfir@{\tt eqfir}}
\index{optimal FIR filter design!function syntax}

 For the design of piece-wise constant filters (such as
band pass, low pass, high pass, and stop band filters) with
even or odd length the user may use the function {\tt eqfir} which
is of simpler manipulation.
	Three examples are presented here.  The first two
examples are designs for a stopband filter.  The third example is for
a design of a high pass filter.

	The first design for the stop band filter uses the following
Scilab commands to create the input to the function {\tt eqfir}:
\verbatok{\Diary eqfir1.dia}
\end{verbatim}
\noindent The resulting magnitude response  of the filter
coefficients is shown in Figure~\ref{f5.5}.
As can be seen the design is very bad.  This is due to the fact that
the design is made with an even length filter and at the same time
requires that the frequency response at $f=.5$ be non-zero.

%
\input{\Figdir remez5.tex}
\dessin{{\tt exec('remez5\_7.code')} Stop Band Filter of Even Length}
{f5.5}
%

	The same example with $\mtt{nf}=33$ is run with the result
shown in Figure~\ref{f5.6}.
%
\input{\Figdir remez6.tex}
\dessin{{\tt exec('remez5\_7.code')} Stop Band Filter of Odd Length}
{f5.6}
%

      The final example is that of a high pass filter whose
input parameters were created as follows:
\verbatok{\Diary eqfir2.dia}
\end{verbatim}
The result is displayed in Figure~\ref{f5.7}.
%
\input{\Figdir remez7.tex}
\dessin{{\tt exec('remez5\_7.code')} High Pass Filter Design}
{f5.7}
%

%\section{Linear Programming Design of FIR filters}
%\section{Hilbert Transform}
%\end{document}
