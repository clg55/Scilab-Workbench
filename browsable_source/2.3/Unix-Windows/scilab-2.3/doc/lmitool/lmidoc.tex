\documentclass{article}

\usepackage{psfig}

\textwidth=6.65in
\textheight=9.5in 
\oddsidemargin=-.08in
\evensidemargin=0in
\topmargin=-.6in
\parskip.1cm

\begin{document}
\def\lmitool{{\tt LMITOOL}}
\def\dsp{\displaystyle}
\def\reals{{\bf R}}
\def\BEQ{\begin{equation}}
\def\EEQ{\end{equation}}
\def\eg{{\em e.g.}}
\def\ie{{\em i.e.}}
\def\Tr{{\mathop{\bf Tr}}}
\def\diag{{\mathop{\bf diag}}}
\def\eqbydef{\mathrel{\stackrel{\Delta}{=}}}

\title{\lmitool:
a Package \\ for LMI Optimization in Scilab\\[0.5in]
User's Guide}
\date{}
\author{R. Nikoukhah\thanks{{\tt Ramine.Nikoukhah@inria.fr}}
\and
F. Delebecque\thanks{\tt Francois.Delebecque@inria.fr}.  
\and
L. El Ghaoui\thanks{ENSTA, 32, Bvd. Victor, 75739 Paris, France.
Internet: {\tt elghaoui@ensta.fr}.  Research supported in part by DRET
under contract  92017-BC14.}
}

\maketitle

\begin{abstract}
This report describes a user-friendly Scilab package, and in
particular its two main functions {\tt lmisolver} and
{\tt lmitool} for solving
Linear Matrix Inequalities problems.  This package
uses Scilab function {\tt semidef}, an interface to the program 
Semidefinite Programming {\bf SP} (Copyright \copyright 1994 by Lieven
Vandenberghe and Stephen Boyd) distributed with Scilab. 
\end{abstract}

\tableofcontents

\newpage

\section{Purpose}
Many problems in systems and control can be formulated as follows 
(see \cite{BEFB:94}): 
\[
\Sigma : \;\; \left\{
\begin{array}{ll}
\mbox{minimize}    &  f(X_1,\ldots,X_M) \\
\mbox{subject to}  & \left\{ \begin{array}{l}
               G_i(X_1,\ldots,X_M) = 0, \;\;i=1,2,...,p,\\
                H_j(X_1,\ldots,X_M) \geq 0, \;\;j=1,2,..,q. 
                              \end{array} \right.
\end{array}
\right.
\]
where 
\begin{itemize}
\item
$X_1,\ldots,X_M$ are unknown real matrices, referred to as the {\em
unknown matrices,} 
\item
$f$ is a real linear scalar function of the entries of the unknown
matrices $X_1,\ldots,X_M$; it is referred to as the {\em objective function},
\item
$G_i$'s are real matrices with entries which are affine functions of the 
entries of the unknown matrices, $X_1,\ldots,X_M$;
they are referred to as  ``Linear Matrix Equality'' (LME) functions, 
\item
$H_j$'s are real symmetric matrices with entries which are affine functions
of the entries of the unknown matrices $X_1,\ldots,X_M$; they are referred to as  
``Linear Matrix Inequality'' (LMI) functions.  (In this report, 
the  $V \geq 0$ stands for $V$ positive semi-definite unless stated otherwise).  
\end{itemize}
The purpose of \lmitool\ is to solve problem $\Sigma$ in a user-friendly manner
in Scilab, using the code SP \cite{sp}. This code is intended for
small and medium-sized problems (say, up to a few hundred variables).

\section{Function {\tt lmisolver}}

\lmitool\ is built around the Scilab function {\tt lmisolver}.  This
function computes the solution $X_1,\ldots,X_M$ of
problem~$\Sigma$, given functions $f$, $G_i$ and $H_j$. To solve
$\Sigma$, user must provide an evaluation function which 
``evaluates'' $f$, $G_i$ and $H_j$ as a function the unknown matrices,
as well as an initial guess on the values of the unknown matrices.  User can either 
invoke {\tt lmisolver} directly, by providing the necessary information in a
special format or he can use the interactive
function {\tt lmitool} described in Section~\ref{s_lmitool}.

\subsection{Syntax} 
\begin{verbatim}
[XLISTF[,OPT]] = lmisolver(XLIST0,EVALFUNC[,options])
\end{verbatim}
where
\begin{itemize}

\item {\tt XLIST0:} a list structure including matrices and/or list of matrices.
It contains initial guess on the values of the unknown matrices. In general, the
ith element of {\tt XLIST0} is the initial guess on the value of the
unknown matrix $X_i$. In some cases
however it is more convenient to define one or more elements of
{\tt XLIST0} to be lists (of unknown matrices) themselves. This is a
useful feature when the number of unknown matrices is not fixed a priori
(see Example of Section~\ref{ex2}).

The values of the matrices in {\tt XLIST0}, if compatible with the LME functions, 
are used as intial condition for the optimization algorithm; they are
ignored otherwise. The size and structure of {\tt XLIST0} are used to
set up the problem and determine the size and structure of the output {\tt XLISTF}.

\item
{\tt EVALFUNC:} a Scilab function called {\em evaluation function}
(supplied by the user)
which evaluates the LME, LMI and objective functions, given the values of the
unknown matrices. The syntax is:
\begin{verbatim}
[LME,LMI,OBJ]=EVALFUNC(XLIST)
\end{verbatim}
where
\begin{itemize}
\item {\tt XLIST:} a list, identical in size and structure to {\tt XLIST0}.

\item {\tt LME:} a list of matrices containing values of the LME
functions $G_i$'s
for $X$ values in {\tt XLIST}. {\tt LME} can be a matrix in case
there is only one LME function to be evaluated (instead of a list
containing this matrix as unique element). It can also be a list
of a mixture of matrices and lists which in turn contain values of
LME's, and so on.

\item {\tt LMI:} a list of matrices containing the values of the LMI
functions $H_j$'s
for $X$ values in {\tt XLIST}. {\tt LMI} can also be a matrix (in case
there is only one LMI function to be evaluated). It can also be a list
of a mixture of matrices and lists which in turn contain values of
of LMI's, and so on.

\item {\tt OBJ:} a scalar equal to the value of the objective function $f$
for $X$ values in {\tt XLIST}.
\end{itemize}
If the $\Sigma$ problem has no equality constraints then {\tt LME}
should be {\tt []}. Similarly for {\tt LMI} and {\tt OBJ}.

\item
{\tt options:} a $5 \times 1$ vector containing optimization
parameters {\tt Mbound}, {\tt abstol}, {\tt nu}, {\tt maxiters},
and {\tt reltol}, see  manual page for {\tt semidef} for details ({\tt Mbound}
is a multiplicative coefficient for {\tt M}). This argument is optional,
if omitted, default parameters are used.


\item {\tt XLISTF:} a list, identical in size and structure to {\tt XLIST0}
containing the solution of the problem (optimal values of the unknown matrices).

\item {\tt OPT:} a scalar corresponding to the optimal value of the minimization
problem $\Sigma$.
\end{itemize}

\subsection{Examples}
\subsubsection{State-feedback with control saturation constraint}
\label{ex1}
Consider the linear system
\[
\dot{x}=Ax+Bu   
\]
where $A$ is an $n\times n$ and $B$, an $n \times n_u$ matrix.
There exists a stabilizing state feedback $K$ 
such that for every initial condition $x(0)$ with $\| x(0)\| \leq 1$,
the resulting control satisfies $\| u(t)\|$ for all $t\geq 0$, if and only if
there exist an $n\times n$ matrix $Q$ and an
$n_u \times n$ matrix $Y$ satisfying the equality constraint
\[
Q-Q^T=0
\]
and the inequality constraints
\begin{eqnarray*}
Q&\geq&0\\
-AQ-QA^T-BY-Y^TB^T &>& 0 \\
\left( \begin{array} {cc}  u_{max}^2I &Y\\Y^T & Q \end{array} \right) &\geq & 0
\end{eqnarray*}
in which case one such $K$ can be constructed as $K=YQ^{-1}$.

To solve this problem using {\tt lmisolver},  we first need to construct
the evaluation function.
\begin{verbatim}
function [LME,LMI,OBJ]=sf_sat_eval(XLIST)
 [Q,Y]=XLIST(:)
 LME=Q-Q'
 LMI=list(-A*Q-Q*A'-B*Y-Y'*B',[umax^2*eye(Y*Y'),Y;Y',Q],Q-eye)
 OBJ=[]
\end{verbatim}
Note that {\tt OBJ=[]} indicates that
the problem considered is a feasibility problem, i.e., we are only
interested in finding a set of $X$'s that satisfy LME and LMI functions.


Assuming {\tt A}, {\tt B} and {\tt umax}
already exist in the environment, we can call {\tt lmisolver}, and
reconstruct the solution in Scilab, as follows:
\begin{verbatim}
--> Q_init=zeros(A);
--> Y_init=zeros(B');
--> XLIST0=list(Q_init,Y_init);
--> XLIST=lmisolver(XLIST0,sf_sat_eval);
--> [Q,Y]=XLIST(:)
\end{verbatim}
These Scilab commands can of course be encapsulated in
a Scilab function, say {\tt sf{\_}sat}. Then, To solve this problem,
all we need to do is type:
\begin{verbatim}
--> [Q,Y]=sf_sat(A,B,umax) 
\end{verbatim}
We call {\tt sf{\_}sat} the {\em solver function} for this problem.



\subsubsection{Control of jump linear systems}
\label{ex2}
We are given a linear system 
\[
\dot{x} = A(r(t))x+B(r(t))u,
\]
where $A$ is $n \times n$ and $B$ is $n \times n_u$.   The scalar
parameter $r(t)$ is a continuous-time Markov process taking values in
a finite set $\{1,\ldots,N\}$. 

The transition probabilities of the process $r$ are defined by a
``transition matrix'' $\Pi = (\pi_{ij})$, where $\pi_{ij}$'s are the 
transition probability rates from
the $i$-th mode to the $j$-th.  Such systems, referred to as ``jump
linear systems'', can be used to model linear systems subject to
failures.

We seek a state-feedback control law such that the resulting
closed-loop system is mean-square stable.  That is, for every initial
condition $x(0)$, the resulting trajectory of the closed-loop system
satisfies $\lim_{t \rightarrow \infty} \mbox{\bf E} \| x(t) \| ^2 = 0$.

The control law we look for is a mode-dependent linear state-feedback,
\ie\ it has the form $u(t) = K(r(t))x(t)$; $K(i)$'s are $n_u \times n$ matrices (the
unknowns of our control problem). 

It can be shown that this problem has a solution if and only if
there exist $n \times n$ matrices  $Q(1),\ldots,Q(N)$, and $n_u \times n$
matrices $Y(1),\ldots,Y(N)$, such that  
\begin{eqnarray*}
Q(i)- Q(i)^T&=& 0 , \\
\Tr Q(1)+ \ldots +\Tr Q(N) -1 &=& 0.
\end{eqnarray*}
and
\begin{eqnarray*}
\left[ \begin{array}{cc} Q(i) & Y(i)^T \\ Y(i) & I \end{array}\right] &>&0, \\
- \left[ A(i)Q(i)+Q(i)A(i)^{T} + B(i)Y(i)+Y(i)^TB(i)^{T}+
\dsp\sum_{j=1}^{N}\pi_{ji}Q(j) \right] &>& 0, \;\; i = 1,\ldots,N, \\
\end{eqnarray*}
If such matrices exist, a stabilizing
state-feedback is given by $K(i) = Y(i)Q(i)^{-1}$, $i=1,\ldots,N$.

In the above problem, the data matrices are $A(1),\ldots,A(N)$,
$B(1),\ldots,B(N)$ and the transition matrix $\Pi$.  The unknown 
matrices are $Q(i)$'s (which are symmetric $n \times n$ matrices) and
$Y(i)$'s (which are $n_u \times n$ matrices). In this case, both
the number of the data matrices and that of the unknown matrices
are a-priori unknown. 

The above problem is obviously a $\Sigma$ problem.  In this case,
we can let {\tt XLIST} be a list of two lists: one representing
the $Q$'s and the other, the $Y$'s.

The evaluation function required for invoking {\tt lmisolver} can be constructed as
follows:
\begin{verbatim}
 function [LME,LMI,OBJ]=jump_sf_eval(XLIST)
 [Q,Y]=XLIST(:)
 N=size(A); [n,nu]=size(B(1))
 LME=list(); LMI1=list(); LMI2=list()
 tr=0
 for i=1:N
    tr=tr+trace(Q(i))
    LME(i)=Q(i)-Q(i)'
    LMI1(i)=[Q(i),Y(i)';Y(i),eye(nu,nu)]
    SUM=zeros(n,n)
    for j=1:N
      SUM=SUM+PI(j,i)*Q(j)
    end
    LMI2(i)= A(i)*Q(i)+Q(i)*A(i)'+B(i)*Y(i)+Y(i)'*B(i)'+SUM
 end
 LMI=list(LMI1,LMI2)
 LME(N+1)=tr-1
 OBJ=[]
\end{verbatim}
Note that {\tt LMI} is also a list of lists containing the values
of the LMI matrices. This is just a matter of convenience.

Now, we can solve the problem in
Scilab as follows (assuming lists {\tt A} and {\tt B}, and  matrix
{\tt PI} have already been defined).

First we should initialize {\tt Q} and {\tt Y}. 
\begin{verbatim}
--> N=size(A);  [n,nu]=size(B(1)); Q_init=list(); Y_init=list();
--> for i=1:N, Q_init(i)=zeros(n,n);Y_init(i)=zeros(nu,n);end
\end{verbatim}
Then, we can use {\tt lmisolver} as follows:
\begin{verbatim}
--> XLIST0=list(Q_init,Y_init)
--> XLISTF=lmisolver(XLIST0,jump_sf_eval)
--> [Q,Y]=XLISTF(:);
\end{verbatim}

The above commands can be encapsulated in a solver function, say 
{\tt jump{\_}sf},
in which case we simply need to type:
\begin{verbatim}
--> [Q,Y]=jump_sf(A,B,PI)         
\end{verbatim}
to obtain the solution.


\subsubsection{Descriptor Lyapunov inequalities}
\label{ex3}
In the study of descriptor systems, it is sometimes 
necessary to find (or find out that it does not exist)
an $n\times n$ matrix $X$ satisfying
\begin{eqnarray*}
E^TX=X^TE&\geq&0\\
A^TX+X^TA+I&\leq&0
\end{eqnarray*}
where $E$ and $A$ are $n\times n$ matrices such that ${E,A}$ is a regular pencil.
In this problem, which clearly is a $\Sigma$ problem,
the LME functions play important role. The evaluation function
can be written as follows
\begin{verbatim}
function [LME,LMI,OBJ]=dscr_lyap_eval(XLIST)
X=XLIST(:)
LME=E'*X-X'*E
LMI=list(-A'*X-X'*A-eye,E'*X)
OBJ=[]
\end{verbatim}
and the problem can be solved by (assuming $E$ and $A$ are
already defined)
\begin{verbatim}
--> XLIST0=list(zeros(A))
--> XLISTF=lmisolver(XLIST0,dscr_lyap_eval)
--> X=XLISTF(:)
\end{verbatim}

\subsubsection{Mixed $H_2/H_{\infty}$ Control}
Consider the linear system
\begin{eqnarray*}
\dot{x}&=&Ax+B_1w+B_2u\\
z_1&=&C_1x+D_{11}w+D_{12}u\\
z_2&=&C_2x+D_{22}u
\end{eqnarray*}
The mixed $H_2/H_{\infty}$ control problem consists in finding
a stabilizing feedback which yields $\|T_{z_1w}\|_{\infty}<\gamma$ and
minimizes $\|T_{z_2w}\|_2$ where $\|T_{z_1w}\|_{\infty}$ and
$\|T_{z_2w}\|_2$ denote respectively the closed-loop transfer
functions from $w$ to $z_1$ and $z_2$. In \cite{khargo}, it is
shown that the solution to this problem can be expressed as
$K=LX^{-1}$ where $X$ and $L$ are obtained from the problem of
minimizing Trace($Y$) subject to:
\[
X-X^T=0,\;\; Y-Y^T=0,
\]
and
\begin{eqnarray*}
-\left( \begin{array} {cc}
AX+B_2L+(AX+B_2L)^T+B_1B_1^T & XC_1^T+L^TD_{12}^T+B_1D_{11}^T \\
C_1X+D_{12}L+D_{11}B_1^T & -\gamma^2 I + D_{11}D_{11}^T
\end{array} \right) & > & 0 \\
\left( \begin{array} {cc}Y & C_2X+D_{22}L\\(C_2X+D_{22}L)^T&X
\end{array} \right) & > & 0
\end{eqnarray*}

To solve this problem with {\tt lmisolver}, we define the 
evaluation function:
\begin{verbatim}
 function [LME,LMI,OBJ]=h2hinf_eval(XLIST)
 [X,Y,L]=XLIST(:)
 LME=list(X-X',Y-Y');
 LMI=list(-[A*X+B2*L+(A*X+B2*L)'+B1*B1',X*C1'+L'*D12'+B1*D11';...
               (X*C1'+L'*D12'+B1*D11')',-gamma^2*eye+D11*D11'],...
                                  [Y,C2*X+D22*L;(C2*X+D22*L)',X])
 OBJ=trace(Y);
\end{verbatim}
and use it as follows:
\begin{verbatim}
--> X_init=zeros(A); Y_init=zeros(C2*C2'); L_init=zeros(B2')
--> XLIST0=list(X_init,Y_init,L_init);
--> XLISTF=lmisolver(XLIST0,h2hinf_eval);
--> [X,Y,L]=XLISTF(:)
\end{verbatim}


\subsubsection{Descriptor Riccati equations}
In Kalman filtering for descriptor system
\begin{eqnarray*}
Ex(k+1)& = & Ax(k) + u(k) \\
y(k+1)&=& Cx(k+1) + r(k)
\end{eqnarray*}
where $u$ and $r$ are zero-mean, white Gaussian noise sequences with 
covariance $Q$ and $R$ respectively, one needs to obtain the positive
solution to the descriptor Riccati equation (see \cite{ramine})
\[
P=-\left( \begin{array}{ccc} 0 & 0 & I \end{array} \right)
\left( \begin{array}{ccc} APA^T + Q & 0 & E \\
                           0   & R & C \\
                            E^T  & C^T & 0 \end{array} \right)^{-1}
\left( \begin{array}{c} 0 \\ 0 \\ I \end{array} \right) .
\]

It can be shown that this problem can be formulated as a $\Sigma$
problem as follows: maximize Trace(P) under constraints
\[
P-P^T=0
\]
and 
\[
\left( \begin{array}{ccc} APA^T + Q & 0 & EP \\
                           0   & R & CP \\
                          P^TE^T  & P^TC^T & P \end{array} \right)
\geq 0 .
\]

The evaluation function is:
\begin{verbatim}
 function [LME,LMI,OBJ]=ric_dscr_eval(XLIST)
 LME=P-P'
 LMI=[A*P*A'+Q,zeros(A*C'),E*P;zeros(C*A'),R,C*P;P*E',P*C',P]
 OBJ=-trace(P)
\end{verbatim}
which can be used as follows (asuming $E$, $A$, $C$, $Q$ and $R$ are
defined and have compatible sizes--note that $E$ and $A$ need not be
square).
\begin{verbatim}
--> P_init=zeros(A'*A)
--> P=lmisolver(XLIST0,ric_dscr_eval)
\end{verbatim}

\subsubsection{Linear programming with equality constraints}
\label{ex4}
Consider the following classical optimization problem
\[
\begin{array}{cc}
\mbox{minimize} & e^Tx \\
\mbox{subject to} & Ax + b \geq 0, \\
& Cx+d = 0.
\end{array}
\]
where $A$ and $C$ are matrices and $e$, $b$ and $d$ are vectors with
appropriate dimensions. Here the sign $\geq$ is to be understood elementwise.

This problem can be formulated in LMITOOL as follows:
\begin{verbatim}
function [LME,LMI,OBJ]=linprog_eval(XLIST)
 [x]=XLIST(:)
 [m,n]=size(A)
 LME=C*x+d
 LMI=list()
 tmp=A*x+b 
 for i=1:m
    LMI(i)=tmp(i)
 end
 OBJ=e'*x
\end{verbatim}
and solved in Scilab by (assuming $A$, $C$, $e$, $b$ and $d$ and
an initial guess x0 exist in the environment):
\begin{verbatim}
--> x=lmisolver(x0,linprog_eval)
\end{verbatim}

\subsubsection{Sylvester Equation}
\label{ex5}
The problem of finding matrix $X$ satisfying
\[
AX+XB = C
\]
or 
\[
AXB = C
\]
where $A$ and $B$ are square matrices (of possibly different sizes) is
a well-known problem. We refer to the first equation as the continuous
Sylvester equation and the second, the discrete Sylvester equation.

These two problems can easily be formulated as $\Sigma$ problems as
follows:
\begin{verbatim}
 function [LME,LMI,OBJ]=sylvester_eval(XLIST)
 [X]=XLIST(:)
 if flag=='c' then 
   LME=A*X+X*B-C
 else 
   LME=A*X*B-C
 end                                               
 LMI=[]                                                          
 OBJ=[]                                                          
\end{verbatim}
with a solver function such as:
\begin{verbatim}
 function [X]=sylvester(A,B,C,flag)
 [na,ma]=size(A);[nb,mb]=size(B);[nc,mc]=size(C);                                
 if ma<>na|mb<>nb|nc<>na|mc<>nb then error("invalid dimensions");end           
 XLISTF=lmisolver(zeros(nc,mc),sylvester_eval)
 X=XLISTF(:)
\end{verbatim}
Then, to solve the problem, all we need to do is to (assuming $A$, $B$
and $C$ are defined)
\begin{verbatim}
--> X=sylvester(A,B,C,'c')
\end{verbatim}
for the continuous problem and 
\begin{verbatim}
--> X=sylvester(A,B,C,'d')
\end{verbatim}
for the discrete problem.


\newpage

\section{Function {\tt LMITOOL}}
\label{s_lmitool}
The purpose of \lmitool\ is to automate most of the steps required before
invoking {\tt lmisolver}.  In particular, it generates a *.sci
file including the solver function and the evaluation function or at
least their skeleton. The solver function is used to define the
initial guess and to modify optimization parameters (if needed).

{\tt lmitool} can be invoked with zero, one or three arguments.

\subsection{Non-interactive mode}
{\tt lmitool} can be invoked with three input arguments as follows:

\subsubsection{Syntax}
\begin{verbatim}
txt=lmitool(probname,varlist,datalist)
\end{verbatim}
where
\begin{itemize}
\item
{\tt probname}: a string containing the name of the problem,
\item
{\tt xlist}: a string containing the names of the unknown matrices
(separated by commas if there are more than one).
\item
{\tt dlist}: a string containing the names of  data matrices (separated
by commas if there are more than one). 
\item
{\tt txt}: a string providing information on what the user should do
next.
\end{itemize}

In this mode, {\tt lmitool} generates a file in the current
directory. The name of this file is obtained by adding ``.sci''
to the end of {\tt probname}. This file is the skeleton of
a solver function and the corresponding evaluation function.

\subsubsection{Example}\
Suppose we want to use {\tt lmitool} to solve the problem
presented in Section~\ref{ex1}. Invoking
\begin{verbatim}
-->txt=lmitool('sf_sat','Q,Y','A,B,umax')
\end{verbatim}
yields the output
\begin{verbatim}

--> txt  = 
 
!    To solve your problem, you need to                         !
!                                                               !
!1- edit file /usr/home/DrScilab/sf_sat.sci                     !
!                                                               !
!2- load (and compile) your functions:                          !
!                                                               !
!   getf('/usr/home/DrScilab/sf_sat.sci','c')                   !
!                                                               !
!3- Define A,B,umax and call sf_sat function:                   !
!                                                               !
!  [Q,Y]=sf_sat(A,B,umax)                                       !
!                                                               !
!To check the result, use [LME,LMI,OBJ]=sf_sat_eval(list(Q,Y))  !

\end{verbatim}
and results in the creation of the file '/usr/home/curdir/sf{\_}sat.sci'
with the following content:
\begin{verbatim}
  function [Q,Y]=sf_sat(A,B,umax)
 // Generated by lmitool on Tue Feb 07 10:30:35 MET 1995
   
   Mbound = 1e3;
   abstol = 1e-10;
   nu = 10;
   maxiters = 100;
   reltol = 1e-10;
   options=[Mbound,abstol,nu,maxiters,reltol];
    
 ///////////DEFINE INITIAL GUESS BELOW
 Q_init=...
 Y_init=...
 /////////// 
  
 XLIST0=list(Q_init,Y_init)
 XLIST=lmisolver(XLIST0,sf_sat_eval,options)
 [Q,Y]=XLIST(:)
  
  
  
 /////////////////EVALUATION FUNCTION////////////////////////////
  
 function [LME,LMI,OBJ]=sf_sat_eval(XLIST)
 [Q,Y]=XLIST(:)
  
 /////////////////DEFINE LME, LMI and OBJ BELOW
 LME=...
 LMI=...
 OBJ=...
\end{verbatim}
It is easy to see how a small amount of editing can 
do the rest!


\subsection{Interactive mode}
{\tt lmitool} can be invoked with zero or one input argument as
follows:

\subsubsection{Syntax}
\begin{verbatim}
txt=lmitool()
txt=lmitool(file)
\end{verbatim}
where
\begin{itemize}
\item
{\tt file}: is a string giving the name of an existing ``.sci'' file
generated by {\tt lmitool}.
\end{itemize}
In this mode, {\tt lmitool} is fully interactive. Using a succession
of dialogue boxes, user can completely define his problem. This
mode is very easy to use and its operation completely self explanatory.
Invoking {\tt lmitool} with one argument allows the user to start off 
with an existing file. This mode is useful for modifying existing files 
or when the new problem is not too much different
from a problem already treated by {\tt lmitool}.

\subsubsection{Example}
Consider the following estimation problem
\[
y = H x + V w
\]
where $x$ is unknown to be estimated, $y$ is known, $w$ is a
unit-variance zero-mean Gaussian vector, and
\[
H \in \mbox{\bf Co}\left\{ H(1),...,H(N) \right\},\;\;\;
V \in \mbox{\bf Co}\left\{ V(1),...,V(N) \right\}
\]
where {\bf Co} denotes the convex hull and $H(i)$ and $V(i)$, $i=1,...,N,$
are given matrices.

The objective is to find $L$ such that the estimate
\[
\hat{x}=Ly
\]
is unbiased and the worst case estimation error variance
$\mbox{E}(\|x-\hat{x}\|^2)$ is minimized.

It can be shown that this problem can be formulated as a $\Sigma$
problem as follows: minimize $\gamma$ subject to
\begin{eqnarray*}
I-LH(i) &=& 0 ,\;\;\;i=1,...,N,\\
X(i)-X(i)^T&=& 0,\;\;\;i=1,...,N,
\end{eqnarray*}
and
\begin{eqnarray*}
\left( \begin{array} {cc} I & (L(i)V(i))^T\\
                        L(i)V(i) & X(i) 
         \end{array}  \right)  \geq 0,\;\;\;\;i=1,...,N, \\
\gamma-\mbox{Trace}(X(i)) \geq 0,\;\;\;\;i=1,...,N.
\end{eqnarray*}

To use {\tt lmitool} for this problem, we invoke it as follows:
\begin{verbatim}
--> lmitool()
\end{verbatim}
This results is an interactive session which is partly illustrated in
following figures.

\begin{figure}[hb]
\centerline{\psfig{figure=fig2.eps,height=6cm}}
\caption{This window must be edited to define problem name and the
name of variables used.}
\label{f1}
\end{figure}

\newpage

\begin{figure}[ht]
\centerline{\psfig{figure=fig3.eps,height=6cm}}
\caption{For the example at hand the result of the editing should
look something like this.}
\end{figure}

%\newpage

\begin{figure}[hb]
\centerline{\psfig{figure=fig4.eps,height=12cm}}
\caption{This is the skeleton of the solver function and the
evaluation function generated by {\lmitool} using the names
defined previously.}
\end{figure}


\begin{figure}[hbtp]
\centerline{\psfig{figure=fig8.eps,height=15cm}}
\caption{After editing, we obtain.}
\vskip.5cm
\centerline{\psfig{figure=fig6.eps,height=3cm}}
\caption{A file is proposed in which the solver and evaluation
functions are to be saved. You can modify it if you want.}
\end{figure}



\newpage

\appendix
\section{How {\tt lmisolver} works}
\label{s-lmisolver-works}
The function {\tt lmisolver} works essentially in four steps:
\begin{enumerate}
\item
{\em Initial set-up.}  The sizes and structure of the initial guess are
used to set up the problem, and in particular the size of the unknown
vector.
\item
{\em Elimination of equality constraints.}  Making repeated calls
to the evaluation function, {\tt lmisolver} generates 
a canonical representation of the form
\[
 \begin{array}{ll} \mbox{minimize} & \tilde{c}^T z\\ 
 \mbox{subject to} & \tilde{F}_0 + z_1\tilde{F}_1 + \cdots +
z_{\tilde{m}} \tilde{F}_{\tilde{m}} \geq 0, \;\; Az + b = 0,
 \end{array}
\]
where $z$ contains the coefficients of all matrix variables. 
This step uses extensively  sparse matrices to speed up
the computation and reduce memory requirement.
\item
{\em Elimination of variables.} Then, {\tt lmisolver} 
eliminates the redundant variables.  The equality
constraints are eliminated by computing the null space $N$ of $A$ and
a solution $z_0$ (if any) of $Ax+b=0$.  At this stage, all solutions
of the equality constraints are parametrized by 
\[
z = Nx+z_0,
\]
where $x$ is a vector containing the independent variables.  The
computation of $N,z_0$ is done using sparse LU functions of Scilab.

Once the equality constraints are eliminated, the problem is
reformulated as
\[
 \begin{array}{ll} \mbox{minimize} & c^T x\\
 \mbox{subject to} & F_0 + x_1F_1 + \cdots + x_m F_m \geq 0,
 \end{array}
\]
where $c$ is a vector, and $F_0,\ldots,F_m$ are symmetric matrices,
and $x$ contains the {\em independent\/} elements in the matrix
variables $X_1,\ldots,X_M$.  (If the $F_i$'s are dependent, a column
compression is performed.)
\item
{\em Optimization.}
Finally, {\tt lmisolver} makes a call to the function {\tt semidef}
(an interface to {\bf SP} \cite{sp}). This phase is itself divided into a
feasibility phase and a minimization phase (only if the linear
objective function is not empty).  The feasibility phase is avoided if
the initial guess is found to be feasible. 

The function {\tt semidef} is called with the optimization
parameters {\tt abstol}, {\tt nu}, {\tt maxiters}, {\tt reltol}.  The
parameter {\tt M} is set above the value
\begin{verbatim}
Mbnd*max(sum(abs([F0 ... Fm])))
\end{verbatim}
For details about the optimization phase, and the meaning of the above
optimization parameters see manual page for {\tt semidef}.
\end{enumerate}

\section{Other versions}
{\tt LMITOOL} is also available on Matlab. The Matlab version can be
obtained by anonymous ftp from {\tt ftp.ensta.fr} under
{\tt /pub/elghaoui/lmitool}.

\newpage

\begin{thebibliography}{99}
\bibitem{sp} Vandenberghe, L., and S. Boyd, ``Semidefinite
Programming,'' Internal Report, Stanford University, 1994 (submitted
to SIAM Review).

\bibitem{BEFB:94} Boyd, S., L. El Ghaoui, E. Feron, and V. Balakrishnan,
{\em Linear Matrix Inequalities in Systems and Control Theory}, SIAM
books, 1994.

\bibitem{khargo} Khargonekar, P. P., and M. A. Rotea, ``Mixed
$H_2/H_{\infty}$ Control: a Convex Optimization Approach,'' {\em IEEE
Trans Aut. Contr.}, 39 (1991), pp. 824-837.

\bibitem{ramine} Nikoukhah, R., Willsky, A. S., and B. C. Levy,
``Kalman Filtering and Riccati Equations for Descriptor Systems,'' {\em IEEE
Trans Aut. Contr.}, 37 (1992), pp. 1325-1342.

\end{thebibliography}





\end{document}


