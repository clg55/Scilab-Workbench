% Copyright INRIA

We describe here the internal structure of Scilab, in particular the way
Fortran subroutine are hard interfaced with Scilab: description of the stack 
where all Scilab data are put and description of the internal coding of
Scilab data structures.

\section{From the Scilab call to the prompt}
We describe here the sequence of the ``master'' programs of Scilab which are
executed before the prompt and how Scilab is organized.

The program {\tt main} is {\tt scilab} (in the sub-directory 
{\tt routines/default}) which  begins with the call of {\tt inffic} 
(in the sub-directory {\tt routines/sun}) which initializes the names of the 
main files
needed by Scilab (for the help, save, graphics ...). Then the routine 
{\tt inisci} (in the sub-directory {\tt routines/system}) is called 
to initialize the data bases and some other tables. After that  the routine 
{\tt scirun} (in the sub-directory {\tt routines/system}) is called
and first executes the Scilab instruction given in scilab.star file.

\subsection{The main routines inisci ans scirun}
The initializations of the database are done by the include file\\
{\tt <scilab dir>/routines/stack.h}; the other initialization done by 
{\tt inisci} is {\tt nunit} which is the maximum number of logical
units. Numerous other initializations are done by this routine 
such as : the unit numbers for the input and the output,the predefined 
variables, the character set,...

After that {\tt scirun}  (in the sub-directory
{\tt routines/sytem}) which is a simple call of the routine {\tt parse} (in
the sub-directory {\tt routines/system}) followed by a call to one of the interfaces. This is done by the mean of the include file 
{\tt callinter.h} (in the sub-directory {\tt routines}). {\tt parse} is the
Scilab parsing function : after examination of a command, {\tt parse} returns
{\tt fun} which is the number of the interface to be called by {\tt scirun}.

\subsection{The Scilab parsing function and the interfaces}
After the analysis of a Scilab command by {\tt parse} a Scilab function
(written in Scilab language) or a fortran (or C) routine can be called.
In the last case the call is done by the corresponding interface : all the
interfaces are in the sub-directory {\tt routines/interf} and each of them 
contains the sequence of its routines.

The organization of this internal structure is represented by figure  
\ref{struct}.

%structure Scilab
\begin{figure}
\begin{center}
\begin{picture}(300,300)(-150,-150)
\put(-50,130){\framebox(100,20){\sc system}}
\put(-50,110){\framebox(100,20){Interpreter}}
\put(-50,90){\framebox(100,20){Variables handling}}
\put(-50,70){\framebox(100,20){Error handling}}
\put(0,70){\vector(0,-1){20}}
\put(-60,20){\framebox(120,30){Interfaces handling}}
\put(0,20){\vector(-4,-1){100}}
\put(0,20){\vector(4,-1){100}}
%
\put(0,20){\vector(0,-1){20}}
\put(-170,-30){\framebox(70,30){Interface 1}}
\put(-135,0){\line(0,1){80}}
\put(-135,80){\vector(1,0){85}}
\put(100,-30){\framebox(70,30){Interface n}}
\put(135,0){\line(0,1){80}}
\put(135,80){\vector(-1,0){85}}
\put(-125,-30){\vector(0,-1){20}}
\put(-145,-30){\vector(0,-1){20}}
\put(125,-30){\vector(0,-1){20}}
\put(145,-30){\vector(0,-1){20}}
\put(-170,-80){\framebox(70,30){Library 1}}
\put(100,-80){\framebox(70,30){Library  n}}
\put(-135,-80){\vector(0,-1){20}}
\put(135,-80){\vector(0,-1){20}}
\put(-10,-20){\huge \ldots}
\put(-10,-70){\huge \ldots}
\put(-170,-120){\framebox(340,20){Low level routines (BLAS,\ldots)}}
\end{picture}
\caption{Internal structure of Scilab}
\label{struct}
\end{center}
\end{figure}

\section{The databasis}
\subsection{The fortran structure }
The leading program of Scilab is written in fortran and so the database is
organized in fortran arrays. This database is composed of the 4 following 
arrays (in fact 3 arrays but one of them is interpreted in two different
manners):

%
\begin{itemize}

\item Names of the variables : \\
{\tt IDSTK(NSIZ,LSIZ)} integer array .  
{\tt IDSTK(1:NSIZ,K)} 
is the code of the name of the variable number {\tt K}.

\item Addresses of the starting location : \\
{\tt LSTK(LSIZ)}  vector of 
integers.     {\tt LSTK(K)} is the adress
of the starting location of the variable {\tt K} in the stack
 {\tt STK}, and ${\tt LSTK}({\tt K}+1)-1$ is the adress
of the last word of this variable in {\tt STK}.

\item Definitions of all the variables : \\
{\tt  STK(VSIZ)}  is the double float vector 
of the definitions of all the variables known in Scilab and the working area.

\item{\tt ISTK} vector of integers ``equivalent'' to {\tt STK} (occupying
the same place in the memory).

\end{itemize}
%

The description of the database is completed by 4 integers :



\begin{itemize}

\item Maximum number of variables: \\
{\tt  ISIZ}  is the dimension of the 
arrays  \verb!IDSTK!  and
\verb!LSTK!.  \verb!ISIZ!  is the maximum number of variables
(permanent and  temporary) which can be managed simultaneously
by the system (for example 500).

\item Dimension of the stack \verb!STK!:  \\
{\tt VSIZ} .  \verb!VSIZ!
is the size of the stack defined in double precision words 
containing the variables (permanent and temporary) and the working 
area (for example 2000000).  {\tt VSIZ} and
{\tt ISIZ} are constants which can only be changed by
modifying the assigned in the include file
 \verb!routines/stack.h!

\item Location limit of temporary variables: \\
{\tt  TOP}  pointer  on the arrays 
\verb!LSTK!   and
\verb!IDSTK!: the variables with a  number from  1 to \verb!TOP!  are
temporary variables   (parameter of a  fonction ,
evaluation of expressions, $\ldots$,).  \verb!LSTK(TOP+1)!
is the current first free address of the stack \verb!STK!.

\item Location limit of permanent variables: \\
{\tt  BOT}.  The  variables numbered from  \verb!BOT!  to  ${\tt
ISIZ-1}$ are  permanent variables   (variables created by an
assignment : \verb!a=expr...!).   \verb!LSTK(BOT)-1! is the last free
address of the stack \verb!STK!.    The relation ${\tt TOP+1}<{\tt
BOT}$ must be always satisfied (to avoid overwritting).  
The figure \ref{database} presents the 3 arrays of the database.
Then the figure  \ref{stack} describes the stack, the figure \ref{isadr} gives
the correspondance between the indices of the array {\tt STK-ISTK}.
This correspondance is again presented in figure \ref{stack2} and the detail
of the decomposition between the description of a variable and its values
is in figure \ref{stack3}.

\end{itemize}

\begin{figure}
\hspace{-1.cm}
\fbox{\begin{picture}(450.0,530.0)
\special{psfile=stackeps hscale=70.0 vscale=70.0}
\end{picture}}
\caption{The 3 arrays of the database}
\label{database}
\end{figure}



\newcommand{\grossebrace}{$ \left. \begin{array}{c}
\, \\
\, \\
\, \\
\, \\
\, \\
\, \\
\, \\
\, \\
\end{array}   \right\} {VARIABLE\ \  top-2} $}
\newcommand{\bigbrace}{$ \left. \begin{array}{c}
\, \\
\, \\
\, \\
\, \\
\, \\
\, \\
\, \\
\, \\
\, \\
\, \\
\end{array}   \right\} {WORKING\ \ de\ \ AREA} $}
\newcommand{\medbrace}{$ \left. \begin{array}{c}
\, \\
\, \\
\, \\
\, \\
\, \\
\, \\
\end{array}   \right\} {VARIABLE\ \ top-1} $}
\newcommand{\midbrace}{$ \left. \begin{array}{c}
\, \\
\, \\
\, \\
\, \\
\, \\
\, \\
\end{array}   \right\} {VARIABLE\ \ top} $}

\begin{figure}
\begin{center}
\begin{picture}(200,380)(50,0)
\put(0,380){\line(1,0){70}}
\put(0,280){\line(1,0){70}}
\put(0,200){\line(1,0){70}}
\put(0,120){\line(1,0){70}}
\put(0,0){\line(1,0){70}}
\put(50,360){\line(1,0){20}}
\put(50,260){\line(1,0){20}}
\put(50,180){\line(1,0){20}}
\put(70,360){\framebox(30,20){ISTK}}
\put(70,260){\framebox(30,20){ISTK}}
\put(70,180){\framebox(30,20){ISTK}}
\put(70,280){\framebox(30,80){STK}}
\put(70,200){\framebox(30,60){STK}}
\put(70,120){\framebox(30,60){STK}}
\put(70,0){\framebox(30,120){STK}}
\put(-30,370){Variable description}
\put(30,320){Values}
\put(-30,270){Variable description}
\put(30,230){Values}
\put(-30,190){Variable description}
\put(30,150){Values}
\put(180,330){\grossebrace}
\put(180,240){\medbrace}
\put(180,160){\midbrace}
\put(180,60){\bigbrace}
\put(105,375){$\leftarrow$iadr(lstk(top-2))}
\put(105,275){$\leftarrow$iadr(lstk(top-1))}
\put(105,195){$\leftarrow$iadr(lstk(top))}
\put(105,115){$\leftarrow$lstk(top+1)}
\put(105,2){$\leftarrow$lstk(bot)}
\end{picture}
\caption{Description of the stack}
\label{stack}
\end{center}
\end{figure}


\noindent{\bf Remark:}\\
 A double-float is equivalent to two integers.  Converting the address from
  {\tt STK}  to {\tt ISTK} is done through the  fonctions
\verb!iadr! and {\tt sadr}.

\medskip

\begin{figure}
\begin{center}
\begin{picture}(200,80)
\thicklines
\multiput(60,0)(40,0){3}{\line(0,1){60}}
\thinlines
\put(60,20){\line(1,0){80}}
\put(60,40){\line(1,0){80}}
\put(100,30){\line(1,0){40}}
\put(28,35){{\tt l}}
\put(35,35){\vector(1,0){20}}
\put(165,35){\vector(-1,0){20}}
\put(165,25){\vector(-1,0){20}}
\put(170,35){{\tt il1}}
\put(170,25){{\tt il2}}
\put(70,70){{\tt STK}}
\put(110,70){{\tt ISTK}}
\end{picture}
\caption{STK to ISTK conversion}
\label{isadr}
\end{center}
\end{figure}


\medskip

We have the following relations:

\begin{verbatim}
      il1 = iadr(l)
      l = sadr(il1)
      sadr(il2) = l + 1
\end{verbatim}


\begin{figure}
\hspace{-1.cm}
\fbox{\begin{picture}(450.0,530.0)
\special{psfile=stack2eps hscale=70.0 vscale=70.0}
\end{picture}}
\caption{Correspondance of the arrays}
\label{stack2}
\end{figure}


\begin{figure}
\hspace{-1.cm}
\fbox{\begin{picture}(450.0,530.0)
\special{psfile=stack3eps hscale=70.0 vscale=70.0}
\end{picture}}
\caption{Description of a variable location}
\label{stack3}
\end{figure}


The database is transmitted to the different routines by the labelled commons :

\begin{verbatim}
      COMMON /STACK/ STK
      COMMON /VSTK/ BOT,TOP,IDSTK,LSTK,LEPS,BBOT,BOT0
\end{verbatim}
defined in the include file  {\tt stack.h}.
{\tt LEPS} is the address of {\tt STK} where is stored the value of the 
machine precision  $b^{(1-t)}$   ($b$=base, $t$= length of the mantissa).

\subsection{Coding the different types of variables}
Let {\tt  k} be the number of the variable  considered  and {\tt  il=
iadr(LSTK(k))}: as seen before {\tt il}  is a pointer  towards the first word 
of the stack
\verb!ISTK!  corresponding to this variable.  \verb!ISTK(il)!  defines the type
of the variable. We will now consider the differents datatypes and present
their corresponding description and the organization of the part of the stack 
containing the values.


\subsubsection{Scalar matrix type}

\begin{itemize}

\item {\tt ISTK(il)} is equal to 1.

\item {\tt ISTK(il+1)} contains the line number $m$ of the matrix.

\item{\tt ISTK(il+2)} contains the column number $n$.

\item {\tt ISTK(il+3)} is $=0$ if the matrix coefficients are real and
$=1$ if they are complex numbers.

\end{itemize}

Let \verb!l1= sadr(il+4)!, then 

\begin{itemize}

\item  \verb!STK(l1:l1+m*n-1)!   are the real parts of the matrix elements,
 the element {\tt (i,j)} is stored in
\verb!STK(l1+(i-1)+(j-1)*m)!.

\end{itemize}

If {\tt ISTK(il+3)} is equal to 1, then

\begin{itemize}

\item {\tt STK(l1+m*n:l1+2*m*n-1)}  are the imaginary parts of the elements,
the part  $(i,j)$ is stored in \verb!STK(l1+m*n+(i-1)+(j-1)*m)!.

\end{itemize}

The figure \ref{matrc} presents the description of this type.


\begin{figure}
\begin{center}
\begin{picture}(120,300)(0,0)
\put(0,0){\framebox(90,100){\shortstack{Imaginary \\ part}}}
\put(0,100){\framebox(90,100){\shortstack{Real \\ part}}}
\put(0,200){\framebox(90,20){0 or 1}}
\put(0,220){\framebox(90,20){number of columns}}
\put(0,240){\framebox(90,20){number of lines}}
\put(0,260){\framebox(90,20){1}}
%
\put(90,0){\line(1,0){60}}
\put(90,100){\line(1,0){60}}
\put(90,200){\line(1,0){60}}
\put(90,280){\line(1,0){60}}
%
\put(100,270){il}
\put(100,250){il+1}
\put(100,190){l=sadr(il+4)}
\put(140,240){\large ISTK}
\put(140,150){\large STK}
\put(140,50){\large STK}
\end{picture}
\caption{Real or complex matrix}
\label{matrc}
\end{center}
\end{figure}

\subsubsection{Character string matrix}

If {\tt ISTK(il)} is equal to 10 :

\begin{itemize}

\item{\tt ISTK(il+1)} contains the number of the lines $m$ of the matrix,
\item{\tt ISTK(il+2)} contains the number of the columns $n$ of the matrix,
\item{\tt ISTK(il+3)} not used.

\end{itemize}

The character matrix datatype is represented by fig. \ref{strmat}.

\begin{figure}
\begin{center}
\begin{picture}(60,420)(50,-20)
\put(30,0){\framebox(60,80){\shortstack{characters \\ of \\ the string \\ numbered \\m.n}}}
\put(30,80){\framebox(60,80){\ldots}}
\put(30,160){\framebox(60,60){\shortstack{characters \\ of \\ the first \\ string}}}
\put(30,220){\framebox(60,20){pointer}}
\put(30,240){\framebox(60,20){\ldots}}
\put(30,260){\framebox(60,20){pointer}}
\put(30,280){\framebox(60,20){1}}
\put(30,300){\framebox(60,20){0}}
\put(30,320){\framebox(60,20){m}}
\put(30,340){\framebox(60,20){n}}
\put(30,360){\framebox(60,20){10}}
\put(90,0){\line(1,0){60}}
\put(90,220){\line(1,0){60}}
\put(90,300){\line(1,0){60}}
\put(90,80){\line(1,0){60}}
\put(90,160){\line(1,0){60}}
%
\put(30,290){\line(-1,0){10}}
\put(30,270){\line(-1,0){20}}
\put(30,230){\line(-1,0){30}}
\put(20,290){\line(0,-1){80}}
\put(10,270){\line(0,-1){120}}
\put(0,230){\line(0,-1){240}}
\put(20,210){\vector(1,0){10}}
\put(10,150){\vector(1,0){20}}
\put(0,-10){\vector(1,0){30}}
\put(100,360){il}
\put(100,350){il+1}
\put(100,290){il+4}
\put(100,230){il+4+m$\star$n}
\put(100,210){l1=il+m$\star$n+5}
\put(100,150){l2=l1-1+istk(il+3+2)}
\put(100,70){lmn=l1-1+istk(il+3+m$\star$n)}
\put(140,340){\large ISTK}
\put(140,260){\large ISTK}
\put(140,110){\large ISTK}
\end{picture}
\caption{Character string matrix. m : number of lines ,n : number of columns }
\label{strmat}
\end{center}
\end{figure}

In Scilab the characters are coded by integers (cf \ref{car}), the
function {\tt cvstr} (see {\tt routines/system/cvstr.f}) translates the
ASCII code to the Scilab code and conversly. 


\subsubsection{Polynomial matrix}

This datatype is represented by fig. \ref{matpol}.

\begin{figure}
\begin{center}
\begin{picture}(60,470)(50,-20)
\put(30,0){\framebox(60,80){\shortstack{coefficients \\ of the \\ polynomial \\ numbered \\m.n}}}
\put(30,80){\framebox(60,80){\ldots}}
\put(30,160){\framebox(60,60){\shortstack{coefficients \\ of the \\ first \\ polynomial}}}
\put(30,220){\framebox(60,20){pointer}}
\put(30,240){\framebox(60,20){\ldots}}
\put(30,260){\framebox(60,20){pointer}}
\put(30,280){\framebox(60,20){1}}
\put(30,300){\framebox(60,20){variable}}
\put(30,320){\framebox(60,20){the}}
\put(30,340){\framebox(60,20){of}}
\put(30,360){\framebox(60,20){code}}
\put(30,380){\framebox(60,20){0 or 1}}
\put(30,400){\framebox(60,20){m}}
\put(30,420){\framebox(60,20){n}}
\put(30,440){\framebox(60,20){2}}
\put(90,0){\line(1,0){60}}
\put(90,220){\line(1,0){60}}
\put(90,300){\line(1,0){60}}
\put(90,380){\line(1,0){60}}
\put(90,460){\line(1,0){60}}
\put(90,80){\line(1,0){60}}
\put(90,160){\line(1,0){60}}
%
\put(30,290){\line(-1,0){10}}
\put(30,270){\line(-1,0){20}}
\put(30,230){\line(-1,0){30}}
\put(20,290){\line(0,-1){80}}
\put(10,270){\line(0,-1){120}}
\put(0,230){\line(0,-1){240}}
\put(20,210){\vector(1,0){10}}
\put(10,150){\vector(1,0){20}}
\put(0,-10){\vector(1,0){30}}
\put(100,450){il}
\put(100,430){il+1}
\put(100,370){il+4}
\put(100,290){il+8}
\put(100,230){il+8+m$\star$n}
\put(100,210){l1=sadr(il+9+m$\star$n)}
\put(100,150){l2=l1+istk(il+7+2)-1}
\put(100,70){lmn=l1+istk(il+7+m$\star$n)-1}
\put(140,420){\large ISTK}
\put(140,340){\large ISTK}
\put(140,260){\large ISTK}
\put(140,110){\large  STK}
\end{picture}
\caption{Polynomial matrix. m : number of lines , n : number of columns }
\label{matpol}
\end{center}
\end{figure}


\subsubsection{Lists}

{\tt ISTK(il)=15}

\begin{itemize}

\item {\tt ISTK(il+1)} contains the number of elements $n$ of the list.

\noindent If {\tt ilp=il+2} and {\tt l=sadr(ilp+n+1)-1}, then:

\item{\tt  ISTK(ilp+(i-1))} contains the pointer {\tt li}  such that
{\tt STK(l+li)} is the first word of the element {\tt i}, the number of the
words (
de mots (in {\tt STK}) of the element numbered {\tt i}  is given by \\
 \verb!ni=ISTK(ilp+i) - ISTK(ilp+i-1)!.

\item{\tt STK(l+li:l+li+ni-1)}  contains the whole structure of the 
variable corresponding to this element.

\end{itemize}

This datatype is described by fig. \ref{list}.

\begin{figure}
\begin{center}
\begin{picture}(60,350)(50,-20)
\put(30,0){\framebox(60,80){\shortstack{structure\\ of the  \\ variable \\ numbered \\n}}}
\put(30,80){\framebox(60,80){\ldots}}
\put(30,160){\framebox(60,60){\shortstack{structure \\ of the \\first \\ variable}}}
\put(30,220){\framebox(60,20){pointer}}
\put(30,240){\framebox(60,20){\ldots}}
\put(30,260){\framebox(60,20){pointer}}
\put(30,280){\framebox(60,20){1}}
\put(30,300){\framebox(60,20){n}}
\put(30,320){\framebox(60,20){15}}
%
\put(90,0){\line(1,0){60}}
\put(90,220){\line(1,0){60}}
\put(90,300){\line(1,0){60}}
\put(90,340){\line(1,0){60}}
\put(90,160){\line(1,0){60}}
\put(90,80){\line(1,0){60}}
%
\put(30,290){\line(-1,0){10}}
\put(30,270){\line(-1,0){20}}
\put(30,230){\line(-1,0){30}}
\put(20,290){\line(0,-1){80}}
\put(10,270){\line(0,-1){120}}
\put(0,230){\line(0,-1){240}}
\put(20,210){\vector(1,0){10}}
\put(10,150){\vector(1,0){20}}
\put(0,-10){\vector(1,0){30}}
\put(100,320){il}
\put(100,310){il+1}
\put(100,290){il+2}
\put(100,230){il+2+n}
\put(100,210){l1=sadr(il+n+3)}
\put(100,150){l2=l1+ISTK(il+1+2)-1)}
\put(100,70){ln=l1+ISTK(il+1+n)-1)}
\put(140,310){\large ISTK}
\put(140,260){\large ISTK}
\put(140,190){\large STK}
\put(140,110){\large STK}
\put(140,30){\large STK}

\end{picture}
\caption{List. n : number of the elements of the list}
\label{list}
\end{center}
\end{figure}

\subsubsection{Functions}
{\tt ISTK(il)} is equal to 11. In this case its description is divided in
3 fields : the first one describes the output parameters of the function,
the second one is devoted to the input parameter and the last one contains
the set of instructions.

Si ${\tt ils}={\tt il}+1$:

\begin{itemize}

\item{\tt ISTK(ils)} contains the number $m$ of the output parameters.

\item{\tt ISTK(ils+1:ils+NSIZ*n)}   contains the names of the output variables
in Scilab code form compressed on {\tt NSIZ} integers.

\end{itemize}

Let ${\tt ile}={\tt ils}+NSIZ*n+1$, then

\begin{itemize}

\item{\tt ISTK(ile)} contains the number $m$ of the input parameters.

\item {\tt ISTK(ile+1:ile+NSIZ*m)}   contains the names of the output 
variables in Scilab code form compressed on {\tt NSIZ} integers.

\end{itemize}

Let ${\tt ilt}={\tt ile}+NSIZ*n+1$, then:

\begin{itemize}

\item{\tt   ISTK(ilt)}  contains the length (number of characters) $l$
of the code of the function.

\item{\tt ISTK(ilt+1:ilt+l)} the code of the function in Scilab code.

\end{itemize}

For the compiled functions \verb!ISTK(ilt+1:ilt+l)!   contains
a sequence of integers fefining the compiled code. The integers equal to
99 are lines separator.

\subsubsection{Library}

${\tt ISTK(il)}=14$

\begin{itemize}

\item{\tt ISTK(il+1)} contains the number {\tt nf} of  characters 
of the name of the file containing the functions.

\item{\tt  ISTK(il+2:il+1+nf)} contains the sequence of characters in
Scilab code form.

\end{itemize}

If ${\tt ilh}={\tt il+2+nf}$:

\begin{itemize}

\item{\tt ISTK(ilh)} contains the number {\tt nh} of  characters of
the name of the file containing the ``help''.

\item{\tt    ISTK(ilh+1:ilh+nh)} contains the Scilab code of the characters.

\end{itemize}

Let ${\tt iln}={\tt ilh+nh+1}$:

\begin{itemize}

\item{\tt   ISTK(iln)}  contains the number  {\tt   nm} of functions,
and  {\tt ISTK(iln+2i-1:iln+2i)} contains the compact code
of the name of the {\tt i}-th function, for {\tt i}  from 1 to {\tt
nm}.

\end{itemize}
\subsection{The code of the Scilab characters}
The following table \ref{Scicodes} gives the internal code of the the Scilab 
characters.
\label{car}


\begin{table}[htbp]
\begin{center}
\begin{tabular}{|c|c|}
\hline
CHARACTERS & SCILAB CODES \\ \hline
0-9&0-9\\ \hline
{\tt a-z} or {\tt A-Z} & 10-35 \\ \hline
{\tt \_} & 36 \\ \hline
{\tt \#} & 37 \\ \hline
\verb!~! & 38 \\ \hline
{\tt blank} & 40 \\ \hline
{\tt (} & 41 \\ \hline
{\tt )} & 42 \\ \hline
{\tt ;} & 43 \\ \hline
{\tt :} & 44 \\ \hline
{\tt +} & 45 \\ \hline
{\tt -} & 46 \\ \hline
{\tt *} & 47 \\ \hline
\verb!/! & 48 \\ \hline
\verb!\! or \verb!$! & 49 \\ \hline
{\tt =} & 50 \\ \hline
{\tt .} & 51 \\ \hline
{\tt ,} & 52 \\ \hline
{\tt '} or {\tt "} & 53 \\ \hline
{\tt [} or {\tt \{} & 54 \\ \hline
{\tt ]} or {\tt \}} & 55 \\ \hline
{\tt \%} & 56 \\ \hline
\verb!|!& 57 \\ \hline
\verb!&! & 58 \\ \hline
\verb!< ! or \verb!` ! & 59 \\ \hline
\verb!> !  & 60 \\ \hline
\verb!~ ! or {\tt @} & 61  \\ \hline
\end{tabular}
\end{center}
\caption{Scilab codes for known characters}
\label{Scicodes}
\end{table}

The upper-case characters and some equivalents are coded by the lower-case 
code with a sign change.


The function {\tt cvstr} (see {\tt routines/system/cvstr.f}) translates 
the code ASCII to the Scilab code and conversely.

\begin{verbatim}
      subroutine cvstr(n,line,str,job)
c!purpose
c     translates a character string written in Scilab code
c     to a standard string
c     the eol (99) are replaced by !
c
c!calling sequence
c     call cvstr(n,line,str,job)
c
c     with
c
c     n: integer, length of the string to be translated
c
c     line: vector of integers which are the codes of the characters
c
c     string: character, contains ASCII characters
c
c     job: integer, if equal to 1: code-->ascii
c                   if equal to 0: ascii-->code
\end{verbatim}


\section{Scilab Fortran Interfaces}

This section describes the rules to follow for writing an interface
allowing to add a new primitive to the system; of course we forget here
the parts depending on the host computer (compiler, linker,$\ldots $).
The Scilab structure is resumed in figure  \ref{struct}.   

\subsection{Interfaces handling}

The link between the Scilab primitives and the corresponding interfaces
is done by the routine  {\tt funtab}. 
\label{funtab}
This routine is {\bf automatically produced} by the program {\tt bin/newfun}
with the file  {\tt routines/default/fundef}.\label{fundef}
This routine handle two tables initialized by {\tt   DATA}.

The first table {\tt (funn(NSIZ,funl))} contains the coded names of the 
functions known by Scilab, \verb!funl! being the number of these functions.

The other table {\tt (funp(funl))}  define 2 integers \verb!fun! and
\verb!fin! for each known function; these 2 integers are represented by the
integer {\tt 100}$\star${\tt fun+fin}, where :

\begin{itemize}

\item{\tt fun} indicates the interface program implementing the function.

\item{\tt fin} indicates the function inside the interface program.

\end{itemize}

For adding a new primitive it is necessary to add its name and the 
value $100\star \tt fun+fin$ in the file
\verb!routines/default/fundef ! following the format specification.

Example :
\begin{verbatim}
abs                       601   0
atan                      625   0
cos                       624   0
\end{verbatim}

define the pointers towards the Scilab primitives {\tt abs}, {\tt
atan}, {\tt cos}.  

Running the program {\tt bin/newfun} or more easily the Makefile 
allows then the generation of the file {\tt funtab.f}.


\subsection{Interface routine}

When a Scilab function is called, the system calls the corresponding
interface program after having defined the 
variables \verb!fin!,  \verb!lhs! and \verb!rhs! in the common \verb!/com!/
and  configured  the database:  \verb!common/STACK/!   and
\verb!/VSTK/!.

\subsubsection{The variables lhs and rhs}

These variables indicate the numbers  of left-parameters (\verb!lhs!)
and right-parameters (\verb!rhs!) used for the call of the function.

\noindent{\sc   Example:}   \verb![x,y]=foo(a,b,c)!  gives  ${\tt
lhs}=2$ and ${\tt rhs}=3$.

The interface must check if these parameters numbers agree with the
allowed values for the functions. Different variants can be defined for 
a unique function by using the possibility of a variable length for
the parameters list.

In case of incompatibility for the parameter list, the interface program
calls the error handling program (\verb!error!) with the code  41   
(\verb!lhs!) or 42 (\verb!rhs!) and return the prompt.

\subsection{A working example}
We define here a fortran routine and then we interface it with Scilab.
We are in the case of the Scilab data types are not simple fortran types.
The code of this routine is the following :

\begin{verbatim}
      subroutine dmptr(pm,d,n,tr,dt)
c!purpose
c     Computes the trace of a square polynomial matrix pm
c!Calling sequence
c
c    subroutine dmptr(pm,d,n,tr,dt)
c    double precision pm(*),tr(*)
c    integer d(*),dt
c
c     pm : array of polynomial matrix coefficients:
c          pm=[coeff(pm(1,1)),coeff(pm(2,1)),...coeff(pm(n,n))]
c     d  : array of pointer on the first coefficient of pm(i,j)
c     d=[1,1+degree(pm(1,1)),1+degree(pm(1,1))+degree(pm(2,1)),...1+...+degree(pm(n,n))]
c     n  : size of pm matrix
c     tr : array of trace polynomial coefficients
c     dt : degre of trace polynomial
c!
      double precision pm(*),tr(*)
      integer d(*),dt,n
c
      integer deg
c     computes trace degree
      dt=0
      do 01 i=1,n
         ii=i+(i-1)*n
         dt=max(dt,d(ii+1)-d(ii))
 01   continue
c     initialize tr coefficients to 0.0d0
      call dset(dt,0.0d0,tr,1)
c sum of the diagonal elements of pm
      do 10 i=1,n
         ii=i+(i-1)*n
         deg=d(ii+1)-d(ii)
         write(*,*) i,ii,deg

         if(deg.gt.0) then
            do 05 k=1,deg
               tr(k)=tr(k)+pm(d(ii)-1+k)
 05         continue
         endif
 10   continue
      return
      end
\end{verbatim}

We now write the code {\tt newint} corresponding to the new interface
i.e. defining the Scilab command {\tt tr=trace(mp)}

\begin{verbatim}

      subroutine newint
      include 'routines/stack.h'
      double precision sr,si
      integer iadr, sadr, id(nsiz)
      iadr(l)=l+l-1
      sadr(l)=(l/2)+1
      rhs = max(0,rhs)
      lw = lstk(top+1)
      l0 = lstk(top+1-rhs)
C+++++++++++++++++++++++++++++++++++++++++++++++++++++++++++++++++++++
      if (fin .eq. 1) then
         if (rhs .ne. 1) then
            call error(39)
            return
         endif
         if (lhs .ne. 1) then
            call error(41)
            return
         endif
         il1 = iadr(lstk(top-rhs+1))
         if (istk(il1) .eq. 1) then
            if (istk(il1+1) .ne. istk(il1+2)) then
               err=1
               call error(20)
               return
            endif
            n1 = istk(il1+1)
            l1 = sadr(il1+4)
            it1 = istk(il1+3)
            sr = 0.0d0
            si = 0.0d0
            do 10 i = 1,n1
               sr = sr+stk(l1+(i-1)+(i-1)*n1)
 10         continue
            if(it1 .eq. 1) then
               do 11 i = 1,n1
                  si = si+stk(l1+n1*n1+(i-1)+(i-1)*n1)
 11            continue
            endif
            stk(l1) = sr
            if(it1 .eq. 1) stk(l1+1) = si
            istk(il1+1) = 1
            istk(il1+2) = 1
            if (si .eq. 0.0d0) istk(il1+3) = 0
            lstk(top+1) = l1+1+it1
         elseif (istk(il1) .eq. 2) then
            if (istk(il1+1) .ne. istk(il1+2)) then
               err = 1
               call error(20)
               return
            endif
            n1 = istk(il1+1)
            id1 = il1+8
            l1 = sadr(id1+n1*n1+1)
            it1 = istk(il1+3)
            idt=0
            do 20 i=1,n1
               ii=i+(i-1)*n1
               idt=max(idt,istk(id1+ii)-istk(id1-1+ii))
 20         continue
            lr = lw
            err = lr+idt*(it1+1) - lstk(bot)
            if (err .gt. 0) then
               call error(17)
               return
            endif
            call dmptr(stk(l1),istk(id1),n1,stk(lr),idt)
            if (it1 .ne. 0) then
               l1i = l1+istk(id1+n1*n1)-1
               lri = lr+idt
               call dmptr(stk(l1i),istk(id1),n1,stk(lr+idt),idt0)
            endif
            istk(il1+1) = 1
            istk(il1+2) = 1
            istk(il1+3) = it1
            istk(il1+8)=1
            istk(il1+9)=idt+1
            l1 = sadr(il1+10)
            call dcopy((idt+1)*(it1+1),stk(lr),1,stk(l1),1)
            lstk(top+1)=l1+(idt+1)*(it1+1)
         else
            buf='First argument is nor a matrix nor a polynomial matrix'
            call error(999)
            return
         endif
      endif
      end

\end{verbatim}

We now reconsider the previous code with the comments for the different steps
of the procedure 

\begin{verbatim}

      subroutine newint
C INCLUDING THE DATABASE PARAMETERS 
C REPLACE SCIDIR BY THE SCILAB PATH   
      include 'SCIDIR/routines/stack.h'
      double precision sr,si
      integer iadr, sadr, id(nsiz)
C DEFINITION OF THE ADDRESS CONVERTERS
      iadr(l)=l+l-1
      sadr(l)=(l/2)+1
      rhs = max(0,rhs)
C ADDRESSES OF THE BOUNDS OF THE LOCATIONS OF THE RIGHT HAND SIDE PARAMETERS
      l0 = lstk(top+1-rhs)
C+++++++++++++++++++++++++++++++++++++++++++++++++++++++++++++++++++++++++++
      if (fin .eq. 1) then
C BEGINNING OF THE CODE TO BE ADDED
C     SCILAB tr=trace(mp)
C     ==================
C CHECK NUMBER OF CALLING RIGHT HAND SIDE (rhs) ARGUMENTS
         if (rhs .ne. 1) then
            call error(39)
            return
         endif
C CHECK NUMBER OF CALLING LEFT HAND SIDE (lhs) ARGUMENTS
         if (lhs .ne. 1) then
            call error(41)
            return
         endif
C CHECK NOW VARIABLE mp (NUMBER 1)
         il1 = iadr(lstk(top-rhs+1))
         if (istk(il1) .eq. 1) then
C il1 IS THE TYPE OF THE VARIABLE (SEE FIG. 3.6)
C++++++++++++++ STANDARD MATRIX CASE
            if (istk(il1+1) .ne. istk(il1+2)) then
C     .       Non square matrix
               err=1
               call error(20)
               return
            endif
            n1 = istk(il1+1)
C l1 ADDRESS OF MATRIX ELEMENTS (REAL PART)
            l1 = sadr(il1+4)
C it1 REAL/COMPLEX FLAG (0 or 1)
            it1 = istk(il1+3)
C INLINE PROCEDURE TO COMPUTE THE MATRIX TRACE     
            sr = 0.0d0
            si = 0.0d0
            do 10 i = 1,n1
               sr = sr+stk(l1+(i-1)+(i-1)*n1)
 10         continue
            if(it1 .eq. 1) then
C     .    if complex computes imaginary part 
               do 11 i = 1,n1
                  si = si+stk(l1+n1*n1+(i-1)+(i-1)*n1)
 11            continue
            endif
C STORE RESULT IN PLACE OF mp     
            stk(l1) = sr
            if(it1 .eq. 1) stk(l1+1) = si
C SET RESULT SIZES FOR STACK HANDLING
            istk(il1+1) = 1
            istk(il1+2) = 1
            if (si .eq. 0.0d0) istk(il1+3) = 0    
C RETURN ADDRESS OF THE FIRST FREE POSITION IN THE STACK
            lstk(top+1) = l1+1+it1
C END OF STANDARD MATRIX CASE
         elseif (istk(il1) .eq. 2) then
C++++++++++++++POLYNOMIAL MATRIX CASE (SEE FIG. 3.8)
            if (istk(il1+1) .ne. istk(il1+2)) then
C     .       non square matrix
               err = 1
               call error(20)
               return
            endif
            n1 = istk(il1+1)
C id1 STARTING ADDRES OF POINTERS
            id1 = il1+8
C l1 STARTING ADDRESS OF THE COEFFICIENTS
            l1 = sadr(id1+n1*n1+1)
C it1 REAL/COMPLEX FLAG (0/1)
            it1 = istk(il1+3)
C COMPUTING THE SIZE OF THE RESULT     
            idt=0
            do 20 i=1,n1
               ii=i+(i-1)*n1
               idt=max(idt,istk(id1+ii)-istk(id1-1+ii))
 20         continue
C CKECKING AVAILABLE MEMORY
C SET RESULT POINTER TO THE FIRST FREE STACK ADDRESS
            lr = lw
C SET ERR TO THE NEGATIVE OF THE FREE SPACE
            err = lr+idt*(it1+1) - lstk(bot)
            if (err .gt. 0) then
C     .       Not enough memory
               call error(17)
               return
            endif     
C CALLING THE PROCEDURE TO COMPUTE THE MATRIX TRACE
C     .    Real part
            call dmptr(stk(l1),istk(id1),n1,stk(lr),idt)
            if (it1 .ne. 0) then
C     .       Imaginary part
               l1i = l1+istk(id1+n1*n1)-1
               lri = lr+idt
               call dmptr(stk(l1i),istk(id1),n1,stk(lr+idt),idt0)
            endif
C DEFINITION OF THE RETURN VARIABLE     
C SET THE RESULT HEADER (ISTK PART OF THE STACK)     
C     .    row size
            istk(il1+1) = 1
C     .    column size
            istk(il1+2) = 1
C     . real/complex flag
            istk(il1+3) = it1
C     . pointers
            istk(il1+8)=1
            istk(il1+9)=idt+1
C MOVE COMPUTED VALUE IN ITS FINAL PLACE     
            l1 = sadr(il1+10)
            call dcopy((idt+1)*(it1+1),stk(lr),1,stk(l1),1)
C RETURN ADDRESS OF THE FIRST FREE POSITION IN THE STACK     
            lstk(top+1)=l1+(idt+1)*(it1+1)
C     . End of polynomial matrix case
         else
C++++++++++++++INVALID ARGUMENT TYPE CASE
            buf='First argument is nor a matrix nor a polynomial matrix'
            call error(999)
            return
         endif
C END OF TRACE FUNCTION
      endif
      end

\end{verbatim}

After that we have to compile the routines {\tt dmptr.f} and {\tt newint.f}.
Then we use the Scilab function {\tt addinter} to link the new called fortran 
routine {\tt dmptr.o} and the new interface calling fortran routine 
{\tt newint.o} with Scilab. The last argument of {\tt addinter} is the
calling name for Scilab (for a complete description see the on-line help 
of {\tt addinter}). \\
During all the Scilab session, {\tt mytrace} remains defined and can be
used as predefined function. 

\begin{verbatim}
getf('SCI/macros/util/addinter.sci')
addinter('dmptr.o newint.o','newint','mytrace')
a=diag([%s+1,2,3,4])
mytrace(a)
\end{verbatim}


The result is :

\begin{verbatim}

-->getf('SCI/macros/util/addinter.sci')
 
-->addinter('dmptr.o newint.o','newint','mytrace')

linking newint_ defined in dmptr.o newint.o  with Scilab 

 
-->a=diag([%s+1,2,3,4])
 a  =
 
!   1 + s     0     0     0  !
!                            !
!   0         2     0     0  !
!                            !
!   0         0     3     0  !
!                            !
!   0         0     0     4  !
 
-->mytrace(a)
 ans  =
 
    10 + s   
 
\end{verbatim}

It may be preferable to add definitively a new interface to Scilab.
In this case the easiest way for a user is to give the name {\tt matusr} to the
interface entry point and to replace the standard 
{\tt <scilab dir>/routines/default/matusr.f} file by his own file. Then the 
user has to add the new function names in {\tt <scilab dir>/routines/default/fundef} and execute make.

