% Copyright INRIA

\chapter{Programming}
\label{ch4}
\index{programming}

	One of the most useful features of Scilab is its ability
to create and use functions.  This allows the development of specialized
programs which can be integrated into the Scilab package in a simple and
modular way through, for example, the use of libraries.  In this chapter we 
treat the following subjects:
\begin{itemize}
	\item Programming Tools
	\item Defining and Using Functions
        \item Definition of Operators for New Data Types
	\item{Debbuging}
\end{itemize}	
Creation of libraries is discussed in a later chapter.

\section{Programming Tools}

	Scilab supports a full list of programming tools
including loops, conditionals, case selection,
and creation of new environments.  Most programming tasks
should be accomplished in the environment of a function.
Here we explain what programming tools are available.

\subsection{Comparison Operators}\index{programming!comparison operators}
There exist five methods for making comparisons between the
values of data objects in Scilab. These comparisons are listed
in the following table.

\begin{center}
\begin{tabular}{|c|c|}
\hline
{\tt == } or {\tt =} & equal to \\ \hline
{\tt < } & smaller than \\ \hline
{\tt > } &  greater than \\ \hline
{\tt <= } & smaller or equal to \\ \hline
{\tt >= } & greater or equal to \\ \hline
{\tt <> } or {\verb!~=!} & not equal to \\ \hline
\end{tabular}
\end{center}

These comparison operators are used for evaluation of conditionals.

\subsection{Loops}\index{programming!loops}
Two types of loops exist in Scilab: the 
{\tt for}\index{for@{\tt for}} loop
and the 
{\tt while}\index{while@{\tt while}} loop.  The {\tt for} loop
steps through a vector of indices performing each time the
commands delimited by {\tt end}.  
\begin{verbatim}
 
--> x=1;for k=1:4,x=x*k,end
 x         =
 
    1.  
 x         =
 
    2.  
 x         =
 
    6.  
 x         =
 
    24.  
\end{verbatim}
The {\tt for} loop can iterate on any vector or matrix taking for values
the elements of the vector or the columns of the matrix.
\begin{verbatim}
 
--> x=1;for k=[-1 3 0],x=x+k,end
 x         =
 
    0.  
 x         =
 
    3.  
 x         =
 
    3.  
\end{verbatim}
The {\tt for} loop can also iterate on lists. The syntax is the same as for
matrices. The index takes as values the entries of the list.

\begin{verbatim}
-->l=list(1,[1,2;3,4],'str')

-->for k=l, disp(k),end
 
    1.  
 
!   1.    2. !
!   3.    4. !
 
 str
\end{verbatim}

	The {\tt while} loop repeatedly performs a sequence of commands 
until a condition is satisfied.
\begin{verbatim}
--> x=1; while x<14,x=2*x,end
 x         =
 
    2.  
 x         =
 
    4.  
 x         =
 
    8.  
 x         =
 
    16.  

\end{verbatim}

 A {\tt for} or {\tt while} loop can be ended by the command {\tt break} :
\begin{verbatim}

-->a=0;for i=1:5:100,a=a+1;if i > 10  then  break,end; end
 
-->a
 a  =
 
    3.  

\end{verbatim}
In nested loops, {\tt break} exits from the innermost loop.

\begin{verbatim}
-->for k=1:3; for j=1:4; if k+j>4 then break;else disp(k);end;end;end
 
    1.  
 
    1.  
 
    1.  
 
    2.  
 
    2.  
 
    3. 
\end{verbatim}
\subsection{Conditionals}\index{programming!conditionals}
Two types of conditionals exist in Scilab: the 
{\tt if}-{\tt then}-{\tt else}\index{if-then-else@{\tt if-then-else}}
conditional and the 
{\tt select}-{\tt case}\index{select-case@{\tt select}-{\tt case}} 
conditional.  The 
{\tt if}-{\tt then}-{\tt else} conditional
evaluates an expression and if true executes the
instructions between the {\tt then} statement and the {\tt else} statement
(or {\tt end} statement).
If false the statements between the {\tt else} and the {\tt end}
statement are executed.  The {\tt else} is not required. The {\tt elseif}
has the usual meaning and is a also a keyword recognized by the interpreter.
\begin{verbatim}
 
--> x=1
 x         =
 
    1.  
 
--> if x>0 then,y=-x,else,y=x,end
 y         =
 
  - 1.  
 
--> x=-1
 x         =
 
  - 1.  
 
--> if x>0 then,y=-x,else,y=x,end
 y         =
 
  - 1.  
\end{verbatim}


	The {\tt select}-{\tt case} conditional
compares an expression to several possible expressions and performs the
instructions following the first case which equals the initial expression.
\begin{verbatim}
 
--> x=-1
 x         =
 
  - 1.  
 
--> select x,case 1,y=x+5,case -1,y=sqrt(x),end
 y         =
 
    i    
\end{verbatim}
It is possible to include an {\tt else} statement for the condition
where none of the cases are satisfied.

\section{Defining and Using Functions}
\label{s4.2}
\index{programming!functions}
\index{functions}
\index{functions!definition}

	It is possible to define a function directly in
the Scilab environment, however, the most convenient way
is to create a file containing the function
with a text editor.  In this section we describe the
structure of a function and several Scilab commands which are
used almost exclusively in a function environment.

\subsection{Function Structure}
Function structure must obey the following format
\begin{verbatim}
function [y1,...,yn]=foo(x1,...,xm)
   .
   .
   .
\end{verbatim}
where {\tt foo} is the function name, the {\tt xi} are the $m$ input arguments
of the function,  the {\tt yj} are the $n$ output arguments from the function, and
the three vertical dots represent the list of instructions performed by
the function.  An example of a function
which calculates $k!$ is as follows\index{functions!{\tt getf}}\index{getf@{\tt getf}}
\begin{verbatim}
function [x]=fact(k)
  k=int(k);
  if k<1 then, 
     k=1; 
  end,
  x=1;
  for j=1:k,
     x=x*j;
  end,
\end{verbatim}
If this function is contained in a file called {\tt fact.sci} the function
must be ``loaded'' into  Scilab by the {\tt getf} command and before
it can be used:
\begin{verbatim}
--> exists('fact')
 ans       =
 
    0.  
 
--> getf('../macros/fact.sci')
 
--> exists('fact')
 ans       =
 
    1.  
 
--> x=fact(5)
 x         =
 
    120.  
\end{verbatim}
In the above Scilab session, the command 
{\tt exists}\index{functions!{\tt exists}}\index{exists@{\tt exists}} 
indicates that
{\tt fact} is not in the environment (by the $0$ answer to {\tt exist}).  The
function is loaded into the environment using {\tt getf} and now {\tt exists}
indicates that the function is there (the $1$ answer).  The example 
calculates $5!$. 

\subsection{Loading Functions}
Functions are usually defined in files. A file which contains a function
must obey the following format
\begin{verbatim}
function [y1,...,yn]=foo(x1,...,xm)
   .
   .
   .
\end{verbatim}
where {\tt foo} is the function name. The {\tt xi}'s  are the input 
parameters and the the {\tt yj}'s  are the  output parameters, and
the three vertical dots represent the set of instructions performed by
the function to evaluate the {\tt yj}'s, given the {\tt xi}'s.
Inputs and ouputs parameters can be {\it any} Scilab object
(including functions themeselves).

Functions are Scilab objects and should not be considered as files. To
be used in Scilab, functions defined in files  {\em must} be loaded by 
the command {\tt getf(filename)}. If the file {\tt filename} contains the
function {\tt foo}, the function {\tt foo} can be executed only if
it has been previously loaded by the command {\tt getf(filename)}.
A file may contain {\em several} functions. Functions can also be defined ``on line''
by the command {\tt deff}. This is useful if one wants to define
a function as the output parameter of a other function. 

Collections of functions can be organized as libraries (see {\tt lib}
command). Standard Scilab librairies (linear algebra, control,...) 
are defined in the subdirectories of {\tt SCIDIR/macros/}.

\subsection{Global and Local Variables}
If a variable in a function is not defined (and is not among the input
parameters) then it takes the value of a variable having the same name in the
calling environment. This variable however remains local in the sense that
modifying it within the function does not alter the variable in the calling
environment unless {\tt resume} is used (see below). Functions
can be invoked with less input or output parameters. Here is an example:
\begin{verbatim}
function [y1,y2]=f(x1,x2)
y1=x1+x2
y2=x1-x2

\end{verbatim}
\begin{verbatim}
-->[y1,y2]=f(1,1)
 y2  =
    0.  
 y1  =
    2.  

-->f(1,1)
 ans  =
    2.

-->f(1)
y1=x1+x2;
        !--error     4 
undefined variable : x2                      
at line       2 of function f

-->x2=1;

-->[y1,y2]=f(1)
 y2  =
    0.  
 y1  =
    2.  
 
-->f(1)
 ans  =
 
    2. 

\end{verbatim}

Note that it is not possible to call a function if one of the
parameter of the calling sequence is not defined:

\begin{verbatim}
function [y]=f(x1,x2)
if x1<0 then y=x1, else y=x2;end

\end{verbatim}
\begin{verbatim}

-->f(-1)
 ans  =
 
  - 1.  
 
-->f(-1,x2)
       !--error     4 
undefined variable : x2  

-->f(1)
 undefined variable : x2                      
at line       2 of function    f     called by :  
f(1)

-->x2=3;f(1)

-->f(1)
 ans  =
 
    3
                    
\end{verbatim}

Global variable are defined by the \verb!global! command. They can
be read and modified inside functions. Enter {\tt help global} for
details. 

\subsection{Special Function Commands}
Scilab has several special commands which are used almost exclusively
in functions.  These are the commands 

\begin{itemize}
	\item {\tt argn}\index{argn@{\tt argn}}: returns the number of input
and output arguments for the function
	\item {\tt error}\index{error@{\tt error}}: used to suspend the 
operation of a function, to print an error message, and to return to the
previous level of environment when an error is detected.  
	\item {\tt warning}\index{warning@{\tt warning}}, 
	\item {\tt pause}\index{pause@{\tt pause}}: temporarily suspends the 
operation of a function.
	\item {\tt break}\index{break@{\tt break}}: forces the end of a loop 
	\item {\tt return}\index{return@{\tt return}} or {\tt resume} : used 
to return to the calling environment and to pass local
variables from the function environment to the calling environment.  
\end{itemize}

The following example runs the following {\tt foo} function which
illustrates these commands.
\begin{verbatim}
function [z]=foo(x,y)
[out,in]=argn(0);
if x=0 then,
     error('division by zero');
end,
slope=y/x;
pause,
z=sqrt(slope);
s=resume(slope);
\end{verbatim}
\begin{verbatim} 
--> z=foo(0,1)
error('division by zero');
                          !--error 10000 
division by zero                                                                
at line       4 of function    foo      called by :  
 z=foo(0,1)
 
 
--> z=foo(2,1)
 
-1-> resume
 z         =
 
    0.7071068  
 
--> s
 s         =
 
    0.5  
 
\end{verbatim}
In the example, the first call to {\tt foo} passes an argument which cannot
be used in the calculation of the function.  The function discontinues
operation and indicates the nature of the error to the user.  The second call
to the function suspends operation after the calculation of {\tt slope}.
Here the user can examine values calculated inside of the function,
perform plots, and, in fact perform any operations 
allowed in Scilab.  The {\tt -1->} prompt indicates that the current
environment created by the {\tt pause} command is the environment 
of the function and not that of the calling environment.  Control is 
returned to the function by the command {\tt return}.  Operation of the
function can be stopped by the command {\tt quit} or {\tt abort}.
Finally the function terminates its calculation returning the
value of {\tt z}.  Also available in the environment is the variable
{\tt s} which is a local variable from the function which is passed to
the global environment.

\section{Definition of Operations on New Data Types}
\label{s4.3}
\index{operations!for new data types}

	It is possible to transparently define fundamental operations 
for new data types in Scilab (enter {\tt help overloading} for 
a full description of this feature).
That is, the user can give a sense to multiplication, division, addition, etc.
on any two data types which exist in Scilab.  As an example, two linear
systems (represented by lists)
can be added together to represent their parallel inter-connection
or can be multiplied together to represent their series inter-connection.
Scilab performs these user defined operations by searching for functions
(written by the user) which follow a special naming convention described
below.

	The naming convention Scilab uses to recognize operators 
defined by the user is determined by the following conventions.  The name
of the user defined function is composed of four (or possibly three)
fields.  The first field is always the symbol {\tt \%}.  
The third field is one of the characters in the following table
which represents the type of operation to be performed between the
two data types.

\begin{center}
\begin{tabular}{|c|c|}
\hline 
\multicolumn{2}{|c|}{Third field}
\\ \hline \hline
SYMBOL & OPERATION 
\\ \hline \hline
\verb!a! & \verb!+!  \\ \hline

\verb!b! & \verb!;!  (row separator)\\ \hline

\verb!c! & \verb![ ]! (matrix definition) \\ \hline

\verb!d! & \verb!./!  \\ \hline

\verb!e! & \verb!()! extraction: \verb!m=a(k)!  \\ \hline

\verb!i! & \verb!()! insertion:  \verb!a(k)=m!  \\ \hline

\verb!k! & \verb!.*.!  \\ \hline

\verb!l! & \verb!\!  left division \\ \hline

\verb!m! & \verb!*!  \\ \hline

\verb!p! & \verb!^!  exponent \\ \hline

\verb!q! & \verb!.\!  \\ \hline

\verb!r! & \verb!/!  right division \\ \hline

\verb!s! & \verb!-!  \\ \hline

\verb!t! & \verb!'!  (transpose) \\ \hline

\verb!u! & \verb!*.!  \\ \hline

\verb!v! & \verb!/.!  \\ \hline

\verb!w! & \verb!\.!  \\ \hline

\verb!x! & \verb!.*!  \\ \hline

\verb!y! & \verb!./.!  \\ \hline

\verb!z! & \verb!.\.!  \\ \hline
\end{tabular}
\end{center}

The second
and fourth fields represent the type of the first and second data objects,
respectively, 
to be treated by the function and are represented by the symbols given
in the following table.

\begin{center}
\begin{tabular}{|c|c|}
\hline  
\multicolumn{2}{|c|}{Second and Fourth fields}
\\ \hline \hline
SYMBOL & VARIABLE TYPE
\\ \hline \hline

\verb!s! & scalar  \\ \hline

\verb!p! & polynomial \\ \hline

\verb!l! & list (untyped)  \\ \hline

\verb!c! & character string   \\ \hline

\verb!m! & function \\ \hline

\verb!xxx! & list (typed)  \\ \hline 
\end{tabular}
\end{center}

A typed list is one in which the first
entry of the list is a character string where the first
characters of the string are represented by the 
{\tt xxx} in the above table.  For example a typed list
representing a linear system has the form 
{\tt tlist(['lss','A','B','C','D','X0','dt'],a,b,c,d,x0,'c')} and, thus, the {\tt xxx}
above is {\tt lss}.

	An example of the function name which multiplies two
linear systems together (to represent their series inter-connection)
is {\tt \%lss\_m\_lss}.  Here the first field is \%, the second field is 
{\tt lss} (linear state-space), the third field is {\tt m} ``multiply''
and the fourth one is {\tt lss}. A possible user function which performs
this multiplication is as follows
\begin{verbatim}
function [s]=%lss_m_lss(s1,s2)
[A1,B1,C1,D1,x1,dom1]=s1(2:7),
[A2,B2,C2,D2,x2]=s2(2:6),
B1C2=B1*C2,
s=lsslist([A1,B1C2;0*B1C2' ,A2],...
       [B1*D2;B2],[C1,D1*C2],D1*D2,[x1;x2],dom1),
\end{verbatim}
An example of the use of this function after having loaded it into
Scilab (using for example {\tt getf} or inserting it in a library) 
is illustrated in the following Scilab session
\begin{verbatim}
 
-->A1=[1 2;3 4];B1=[1;1];C1=[0 1;1 0];
 
-->A2=[1 -1;0 1];B2=[1 0;2 1];C2=[1 1];D2=[1,1];
 
-->s1=syslin('c',A1,B1,C1);
 
-->s2=syslin('c',A2,B2,C2,D2);
 
-->ssprint(s1)
 
.   | 1  2 |    | 1 |    
x = | 3  4 |x + | 1 |u   
 
    | 0  1 |    
y = | 1  0 |x   
 
-->ssprint(s2)
 
.   | 1 -1 |    | 1  0 |    
x = | 0  1 |x + | 2  1 |u   
 
y = | 1  1 |x + | 1  1 |u   
 
-->s12=s1*s2;   //This is equivalent to s12=%lss_m_lss(s1,s2)
 
-->ssprint(s12)
 
    | 1  2  1  1 |    | 1  1 |    
.   | 3  4  1  1 |    | 1  1 |    
x = | 0  0  1 -1 |x + | 1  0 |u   
    | 0  0  0  1 |    | 2  1 |    
 
    | 0  1  0  0 |    
y = | 1  0  0  0 |x   
\end{verbatim}
Notice that the use of {\tt \%lss\_m\_lss} is totally transparent in
that the multiplication of the two lists {\tt s1} and {\tt s2}
is performed using the usual multiplication operator {\tt *}.

The directory {\tt SCIDIR/macros/percent} contains all the functions
(a very large number...) which perform operations on linear systems
and transfer matrices. Conversions are automatically performed.
For example the code for the function {\tt \%lss\_m\_lss} is there (note
that it is much more complicated that the code given here!).
\section{Debbuging}
The simplest way to debug a Scilab function is to introduce
a {\tt pause} command in the function.
When executed the function stops at this point and prompts {\tt -1->} which 
indicates a different ``level''; another {\tt pause} gives {\tt -2->} ...
At the level 1 the Scilab commands are analog to a different session but
the user can display all the current variables present in Scilab,
which are inside or outside the function i.e. local in the function
or belonging to the calling environment. The execution of the function
is resumed by the command {\tt return} or {\tt resume} (the variables used
at the upper level are cleaned).
The execution of the function can be interrupted by {\tt abort}.
%and the current loop is terminated by {\tt break}.

It is also possible to insert breakpoints in functions. See the commands
{\tt setbpt}, {\tt delbpt}, {\tt disbpt}.
Finally, note that it is also possible to trap errors during the 
execution of a function: 
see the commands {\tt errclear} and {\tt errcatch}.
Finally the experts in  Scilab  can use the 
function {\tt debug(i)} where i=0,..,4 denotes a debugging level.
