% -*-LaTeX-*-
% Converted automatically from troff to LaTeX by tr2tex on Fri Feb 13 19:38:48 1998
% tr2tex was written by Kamal Al-Yahya at Stanford University
% (Kamal%Hanauma@SU-SCORE.ARPA)


%\documentstyle[troffman]{article}

%
% input file: example.1
%
\phead{errcatch}{1}{April\ 1993}{Scilab\ Group}{Scilab\ Function}


\Sdoc{errcatch}{ error trapping}\index{errcatch}\label{errcatch}

\Shead{CALLING SEQUENCE}
\begin{verbatim}
errcatch(n [,'action'] [,'option']) 
\end{verbatim}
\Shead{PARAMETERS}
\begin{scitem}
\item[{\tt n}]
: integer 
\item[{\tt action, option }]
: strings
\end{scitem}% end Env
\Shead{DESCRIPTION}
%
\tt errcatch %
\rm gives an "action" (error-handler)  to be 
performed when an error of type %
\tt n %
\rm occurs.
%
\tt n %
\rm has the followin meaning:
\par
if %
\tt n$>$0%
\rm , %
\tt n %
\rm is the error number to trap
\par
if %
\tt n$<$0 %
\rm  all errors are to be trapped
\par\noindent
%
\tt action %
\rm is one of the following character strings:
\ind{1\parindent}
\ind{2\parindent}
\begin{scitem}
\item[{\tt "pause"}]
: a pause is executed when trapping the error. This option is useful
for debugging purposes.
\item[{\tt "continue"}]
: next instruction in the function or exec files is executed, current
instruction is ignored. This option is useful for error recovery.
\item[{\tt "kill"}]
: default mode, all intermediate functions are killed, scilab goes
back to the level 0 prompt.
\item[{\tt "stop"}]
: interrupts the current Scilab session (useful when 
Scilab is called from an external program).
\ind{1\parindent}
\ind{0\parindent}
\end{scitem}% end Env
\par\noindent
%
\tt option %
\rm is the character string %
\tt 'nomessage' %
\rm for killing
error message.
\Seealso{SEE ALSO}
{\verb?errclear?} \pageref{errclear},{\verb? iserror?} \pageref{iserror}

