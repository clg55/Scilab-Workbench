\documentstyle[modif.article]{article}
\textheight=660pt 
\textwidth=470pt 
\topmargin=-27pt 
\oddsidemargin=0pt 
\evensidemargin=0pt 
\title{Travaux dirig\'es 3}
\author{J.Ph. Chancelier et M. Cohen de Lara}
\begin{document}
\maketitle 
\def\cmarg{\hspace{1cm}}

\def\tit#1{ \begin{center} \fbox{{\bf  #1}} \end{center}}
\centerline{Le fichier des macros de la s\'eance de T.D. 3}

\def\Mhlp{
\begin{flushleft}
{\sl 
\cmarg //$<$$>$=hlp()\\ 
\cmarg \verb@// Travaux diriges : bibliotheque de macros @\\ 
\cmarg \verb@// macros du T.P 3@\\ 
\cmarg \verb@//@\\ 
\cmarg \verb@// tdinit3  :   macro d'initialisation @\\ 
\cmarg \verb@//@\\ 
\cmarg \verb@//    PREMIERE PARTIE @\\ 
\cmarg \verb@// recur    :   equation recurrente bilineaire @\\ 
\cmarg \verb@//              pilotee par un bruit gaussien@\\ 
\cmarg \verb@// logr     :   calcul theorique de l'exposant de lyapunov@\\ 
\cmarg \verb@//@\\ 
\cmarg \verb@//    DEUXIEME PARTIE @\\ 
\cmarg \verb@// mine     :   extraction optimale d'un minerai d'une mine@\\ 
\cmarg \verb@//              a ciel ouvert@\\ 
\cmarg \verb@// trajopt  :   calcule et dessine la trajectoire et le controle @\\ 
\cmarg \verb@//              optimaux pour le probleme mine @\\ 
\cmarg \verb@// ff\_o       :   gain instantane apparaissant dans le critere @\\ 
\cmarg \verb@//              du programme mine.@\\ 
\cmarg \verb@// @\\ 
\cmarg \verb@//!@\\ 
\cmarg 0;\\ 
\cmarg //end}
\end{flushleft}}



\def\Mtdinitt{
\begin{flushleft}
{\sl 
\cmarg //$<$$>$=tdinit3()\\ 
\cmarg \verb@// Macro qui initialise les donnees du @\\ 
\cmarg \verb@// td3@\\ 
\cmarg \verb@//!@\\ 
\cmarg n1=20\\ 
\cmarg n2=20;\\ 
\cmarg te=0;\\ 
\cmarg xe=$-$0.1;\\ 
\cmarg k0=100;\\ 
\cmarg kc=1;\\ 
\cmarg $<$n1,n2,te,xe,k0,kc$>$={\bf resume}(...\\ 
\cmarg \hspace{2.0cm}n1,n2,te,xe,k0,kc);\\ 
\cmarg //end }
\end{flushleft}}



\def\Mrecur{
\begin{flushleft}
{\sl 
\cmarg //$<$y$>$=recur(x0,var,k,n)\\ 
\cmarg //$<$y$>$=recur(x0,var,k,n)\\ 
\cmarg \verb@// equation recurrente bilineaire @\\ 
\cmarg \verb@// x(i+1)=-x(i)*(k + sqrt(var)*br(i))@\\ 
\cmarg \verb@// partant de x0 et @\\ 
\cmarg \verb@// pilotee par un bruit gaussien de variance var.@\\ 
\cmarg \verb@//@\\ 
\cmarg \verb@// le programme dessine la trajectoire et @\\ 
\cmarg \verb@// retourne l'exposant de Liapunov empirique y @\\ 
\cmarg \verb@// ( x(i) est peu different de exp(y*i) )@\\ 
\cmarg \verb@//!@\\ 
\cmarg {\bf rand}('normal');\\ 
\cmarg br={\bf rand}(1,n);\\ 
\cmarg x={\bf ones}(1,n);\\ 
\cmarg x(1)=x0;\\ 
\cmarg {\bf for} i=2:n,x(i+1)=$-$x(i)$\star$(k+{\bf sqrt}(var)$\star$br(i));{\bf end};\\ 
\cmarg xclear();\\ 
\cmarg plot2d((1:n)',x',$<$$-$1$>$,"111"," suite x(n)",$<$0,$-$10,n,10$>$);\\ 
\cmarg y={\bf log}({\bf abs}(k$\star${\bf ones}(br)+{\bf sqrt}(var)$\star$br));\\ 
\cmarg y={\bf sum}(y)/n;\\ 
\cmarg //end}
\end{flushleft}}




\def\Mlogr{
\begin{flushleft}
{\sl 
\cmarg //$<$integr$>$=logr(k,var)\\ 
\cmarg //$<$integr$>$=logr(k,var)\\ 
\cmarg \verb@// calcul theorique de l'exposant de Liapunov y@\\ 
\cmarg \verb@//!@\\ 
\cmarg {\bf deff}('$<$y$>$=fff(x)',...\\ 
\cmarg \hspace{0.8cm}'y={\bf log}({\bf abs}('+{\bf string}(k)+'+x))$\star${\bf exp}($-$x$\star$$\star$2/(2$\star$'+{\bf string}(var)'+'))');\\ 
\cmarg integr={\bf intg}($-$100,100,fff)\\ 
\cmarg //end}
\end{flushleft}}






\def\Mmine{
\begin{flushleft}
{\sl 
\cmarg //$<$cout,feed$>$=mine(n1,n2,uvect)\\ 
\cmarg //$<$cout,feed$>$=mine(n1,n2,uvect)\\ 
\cmarg \verb@// extraction optimale d'un minerai d'une mine a ciel ouvert@\\ 
\cmarg \verb@// en avancant progressivement (k=1,2,...,n2) et en@\\ 
\cmarg \verb@// prelevant a l'abcisse k+1 la tranche de profondeur x(k+1)@\\ 
\cmarg \verb@// par une commande u a partir de la profondeur x(k)@\\ 
\cmarg \verb@// a l'abcisse k.@\\ 
\cmarg \verb@//@\\ 
\cmarg \verb@// la resolution du probleme se fait par programmation dynamique @\\ 
\cmarg \verb@// en calculant la commande optimale maximisant le critere :@\\ 
\cmarg \verb@//  @\\ 
\cmarg \verb@//   -- n2-1@\\ 
\cmarg \verb@//   \          @\\ 
\cmarg \verb@//   /        f(x(k),k) + V\_F(x,n2) @\\ 
\cmarg \verb@//   -- k=1@\\ 
\cmarg \verb@//  avec : x(k+1)=x(k) + u@\\ 
\cmarg \verb@//  x(k) est la profondeur de la mine a l'abcisse k (x=1@\\ 
\cmarg \verb@//      est le niveau du sol)@\\ 
\cmarg \verb@// la fonction gain instantane f(i,k) represente le benefice@\\ 
\cmarg \verb@//      si on creuse la profondeur i a l'abcisse k@\\ 
\cmarg \verb@//  V\_F(i,n2) est un cout final destine a forcer l'etat final@\\ 
\cmarg \verb@//       a valoir 1 (pour sortir de la mine ...)@\\ 
\cmarg \verb@//       V\_F(i,n2) vaut -10000 en tout les points i\ ne1 et 0@\\ 
\cmarg \verb@//       pour i=1@\\ 
\cmarg \verb@//@\\ 
\cmarg \verb@// le programme mine necessite de connaitre@\\ 
\cmarg \verb@//    n1    : l'etat est discretise en n1 points @\\ 
\cmarg \verb@//    n2    : nombre d'etapes@\\ 
\cmarg \verb@//    uvect : vecteur ligne des valeurs discretes que peut@\\ 
\cmarg \verb@//          prendre u (3 valeurs, on monte, on descend ou on avance)@\\ 
\cmarg \verb@// et le programme mine retourne alors deux matrices@\\ 
\cmarg \verb@//    cout(n1,n2) : valeurs de la fonction de Bellman@\\ 
\cmarg \verb@//    feed(n1,n2) : valeurs de la commande a appliquer@\\ 
\cmarg \verb@//          lorsque l'etat varie de 1 a n1 et @\\ 
\cmarg \verb@//          l'etape de 1 a n2.@\\ 
\cmarg \verb@//!@\\ 
\cmarg xgr=1:(n1+2)\\ 
\cmarg \verb@// tableau ou l'on stocke la fonction de Bellman@\\ 
\cmarg cout=0$\star${\bf ones}(n1+2,n2);\\ 
\cmarg \verb@// tableau ou l'on stocke le feedback u(x,t)@\\ 
\cmarg feed=0$\star${\bf ones}(n1,n2);\\ 
\cmarg \verb@// calul de la fonction de Bellman au  temps final@\\ 
\cmarg penal=10000; \\ 
\cmarg cout(:,n2)=$-$penal$\star${\bf ones}(n1+2,1);\\ 
\cmarg cout(2,n2)=0;\\ 
\cmarg \verb@// calcul retrograde de la fonction de Bellman et@\\ 
\cmarg \verb@// du controle optimal au temps temp par l'equation de Bellman @\\ 
\cmarg {\bf for} temp=n2:$-$1:2,\\ 
\cmarg \hspace{0.5cm}loc=0$\star${\bf ones}(n1+2,3);\\ 
\cmarg \hspace{0.5cm}{\bf for} i=1:3,\\ 
\cmarg \hspace{1.0cm}newx={\bf mini}({\bf maxi}(xgr+uvect(i)$\star${\bf ones}(xgr),1$\star${\bf ones}(xgr)),(n1+2)$\star${\bf ones}(xgr)),\\ 
\cmarg \hspace{1.0cm}loc(:,i)=cout(newx,temp)+ff\_o(xgr,temp$-$1),\\ 
\cmarg \hspace{0.5cm}{\bf end};\\ 
\cmarg \hspace{0.5cm}$<$mm,kk$>$={\bf maxi}(loc(:,1),loc(:,2),loc(:,3)),\\ 
\cmarg \hspace{0.5cm}cout(xgr,temp$-$1)=mm;\\ 
\cmarg \hspace{0.5cm}cout(1,temp$-$1)=$-$penal;\\ 
\cmarg \hspace{0.5cm}cout(n1+2,temp$-$1)=$-$penal;\\ 
\cmarg \hspace{0.5cm}feed(xgr,temp$-$1)=uvect(kk)';\\ 
\cmarg {\bf end}\\ 
\cmarg feed=feed(2:(n1+1),1:(n2$-$1));\\ 
\cmarg cout=cout(2:(n1+1),:);\\ 
\cmarg //end}
\end{flushleft}}



\def\Mtrajopt{
\begin{flushleft}
{\sl 
\cmarg //$<$$>$=trajopt(feed)\\ 
\cmarg //$<$$>$=trajopt(feed)\\ 
\cmarg \verb@// feed est la matrice de feedback calculee par mine @\\ 
\cmarg \verb@// trajopt calcule et dessine la trajectoire et le controle @\\ 
\cmarg \verb@// optimaux pour un point de depart (1,1)@\\ 
\cmarg \verb@//!@\\ 
\cmarg $<$n1,n2$>$={\bf size}(feed)\\ 
\cmarg xopt=0$\star${\bf ones}(1,n2)\\ 
\cmarg uopt=0$\star${\bf ones}(1,n2+1)\\ 
\cmarg xopt(1)=1;\\ 
\cmarg {\bf for} i=2:(n2+1),xopt(i)=feed(xopt(i$-$1),i$-$1)+xopt(i$-$1),\\ 
\cmarg \hspace{3.2cm}uopt(i$-$1)=feed(xopt(i$-$1),i$-$1),{\bf end}\\ 
\cmarg plot2d($<$1:(n2+1);1:(n2+1)$>$',$<$uopt;$-$xopt$>$',$<$$-$1,$-$2$>$,"111",...\\ 
\cmarg \hspace{1.8cm}"commande optimale@trajectoire optimale",...\\ 
\cmarg \hspace{1.8cm}$<$1,$-$10,n2,2$>$);\\ 
\cmarg //end}
\end{flushleft}}




\def\Mshowcost{
\begin{flushleft}
{\sl 
\cmarg //$<$$>$=showcost(n1,n2,teta,alpha)\\ 
\cmarg //$<$$>$=showcost(n1,n2,teta,alpha)\\ 
\cmarg \verb@// Montre en 3d la fonction de gain instantanee (ff)@\\ 
\cmarg \verb@// x: profondeur (n1)@\\ 
\cmarg \verb@// y: abscisse  (n2)@\\ 
\cmarg \verb@// en z : valeur de ff\_o(x,t)@\\ 
\cmarg \verb@//!@\\ 
\cmarg $<$lhs,rhs$>$={\bf argn}(0)\\ 
\cmarg {\bf if} rhs=2,teta=45,alpha=45,{\bf end}\\ 
\cmarg m=$<$$>$; \\ 
\cmarg {\bf for} i=1:n2,m=$<$m,ff\_o(1:n1,i)$>$,{\bf end}\\ 
\cmarg plot3d(1:n1,1:n2,m,teta,alpha,"profondeur @ abscisse @ gain inst",$<$2,1$>$);\\ 
\cmarg //end}
\end{flushleft}}



\def\Mffo{
\begin{flushleft}
{\sl 
\cmarg //$<$y$>$=ff\_o(x,t)\\ 
\cmarg //$<$y$>$=ff\_o(x,t)\\ 
\cmarg \verb@// gain instantane apparaissant dans le critere du@\\ 
\cmarg \verb@// programme mine.@\\ 
\cmarg \verb@// @\\ 
\cmarg \verb@// pour des raisons techniques, l'argument x doit@\\ 
\cmarg \verb@// etre un vecteur colonne et donc egalement la sortie y. @\\ 
\cmarg \verb@// en sortie y=< ff\_0(x(1),t),...ff\_o(x(n),t)>;@\\ 
\cmarg \verb@//!@\\ 
\cmarg xxloc={\bf ones}(x);\\ 
\cmarg y= k0$\star$(1$-$ (t$-$te)$\star$$\star$2/n2)$\star$xxloc + (x$-$xe$\star$xxloc)$\star$$\star$3/(n1$\star$$\star$3) $-$kc$\star$x\\ 
\cmarg y=y';\\ 
\cmarg //end}
\end{flushleft}}







\Mhlp

\Mtdinitt

\section{Equation r\'ecurrente bilin\'eaire}

\tit{recur}
Calcule les $n$ it\'er\'es d'une \'equation r\'ecurrente bilin\'eaire 
partant de $x0$ et pilot\'ee par un bruit gaussien de variance $var$.

\Mrecur

\tit{logr}
Valeur  th\'eorique de l'exposant de Liapunov.

\Mlogr

\centerline{{\sc Questions}}

\begin{itemize}
\item Taper la commande {\bf recur(?,?,?,?)} pour diff\'erentes valeurs des param\`etres. Qu'observe-t-on~?
\item Fixer {\bf var = 1} puis trouver une valeur approch\'ee du $k_{lim}$.
Calculer {\bf logr($k_{lim} , var$)}.
\end{itemize}
\bigskip


\section{Commande lin\'eaire optimale}

\def\Mhlp{
\begin{flushleft}
{\sl 
\cmarg //$<$$>$=hlp()\\ 
\cmarg \verb@// Travaux diriges : bibliotheque de macros @\\ 
\cmarg \verb@// macros du T.P 2@\\ 
\cmarg \verb@//@\\ 
\cmarg \verb@// tdinit   :  initialise les donnees du td @\\ 
\cmarg \verb@//@\\ 
\cmarg \verb@//    PREMIERE PARTIE @\\ 
\cmarg \verb@// pp       :  Un modele proie-predateur avec insecticide@\\ 
\cmarg \verb@// equilpp  :  calcule un point d'equilibre du systeme pp@\\ 
\cmarg \verb@//             pour la commande constante u@\\ 
\cmarg \verb@//@\\ 
\cmarg \verb@//    DEUXIEME PARTIE @\\ 
\cmarg \verb@// compet   :  Un modele de competition @\\ 
\cmarg \verb@// equilcom :  calcule un point d'equilibre du systeme compet@\\ 
\cmarg \verb@// lincomp  :  linearise le modele compet autour@\\ 
\cmarg \verb@//               du point d'equilibre donne par equilcom(ue)@\\ 
\cmarg \verb@// gaincom  :  calcul de la matrice k de gain pour @\\ 
\cmarg \verb@//               la commande du modele de competition @\\ 
\cmarg \verb@// gainobs  :  calcul de la matrice l de gain pour @\\ 
\cmarg \verb@//               l'observateur du modele de competition@\\ 
\cmarg \verb@// boucle   :  simule les trajectoires d'un systeme observe-controle@\\ 
\cmarg \verb@//          de dimension 2 avec bruit d'observation@\\ 
\cmarg \verb@//          utilise pour le modele de competition @\\ 
\cmarg \verb@// obscont  :  renvoit un systeme observe-controle.@\\ 
\cmarg \verb@// @\\ 
\cmarg \verb@// test     :  dynamique linearisee du systeme precedent @\\ 
\cmarg \verb@//               observateur-controleur (voir boucle)@\\ 
\cmarg \verb@//@\\ 
\cmarg \verb@//    TROISIEME PARTIE @\\ 
\cmarg \verb@// comric   :  calcul d'un gain de commande a l'aide @\\ 
\cmarg \verb@//          d'une equation de riccati@\\ 
\cmarg \verb@// obsric   :  calcul d'un gain d'observateur (de filtre)@\\ 
\cmarg \verb@//          a l'aide d'une equation de riccati@\\ 
\cmarg \verb@// exemple  : un essai de calcul de gains avec riccati@\\ 
\cmarg \verb@//          et simulations.@\\ 
\cmarg \verb@//@\\ 
\cmarg \verb@//!@\\ 
\cmarg 0;\\ 
\cmarg //end}
\end{flushleft}}




\def\Mtdinit{
\begin{flushleft}
{\sl 
\cmarg //$<$$>$=tdinit()\\ 
\cmarg //$<$$>$=tdinit()\\ 
\cmarg \verb@// constantes pour le modele proie-predateur@\\ 
\cmarg \verb@// et le modele de competition  @\\ 
\cmarg \verb@//!@\\ 
\cmarg ppr=1/100;\\ 
\cmarg ppk=1000;\\ 
\cmarg ppa=1/20000;\\ 
\cmarg ppb=1/10000;\\ 
\cmarg ppm=1/100;\\ 
\cmarg pps=1/200;\\ 
\cmarg ppl=500;\\ 
\cmarg $<$ppr,ppk,ppa,ppb,ppm,ppl,pps$>$={\bf resume}(ppr,ppk,ppa,ppb,ppm,ppl,pps);\\ 
\cmarg //end}
\end{flushleft}}



\def\Mpp{
\begin{flushleft}
{\sl 
\cmarg //$<$xdot$>$=pp(t,x,u)\\ 
\cmarg //$<$xdot$>$=pp(t,x,$[$u$]$)\\ 
\cmarg \verb@// Un modele proie-predateur avec insecticide@\\ 
\cmarg \verb@// x(1) : la proie (insecte nuisible)@\\ 
\cmarg \verb@// x(2) : le predateur (un gentil insecte)@\\ 
\cmarg \verb@// u    : taux de mortalite du a l'insecticide @\\ 
\cmarg \verb@//        qui agit a la fois sur la proie et le predateur@\\ 
\cmarg \verb@//        par defaut u=0 (pas d'insecticide)@\\ 
\cmarg \verb@//!@\\ 
\cmarg $<$lhs,rhs$>$={\bf argn}(0);\\ 
\cmarg {\bf if} rhs=2,u=0,{\bf end}\\ 
\cmarg xdot(1) = ppr$\star$x(1)$\star$(1$-$x(1)/ppk) $-$ ppa$\star$x(1)$\star$x(2) $-$ u$\star$x(1);\\ 
\cmarg xdot(2) = $-$ppm$\star$x(2)             + ppb$\star$x(1)$\star$x(2) $-$ u$\star$x(2);\\ 
\cmarg //end}
\end{flushleft}}



\def\Mequilpp{
\begin{flushleft}
{\sl 
\cmarg //$<$xe$>$=equilpp(ue)\\ 
\cmarg //$<$xe$>$=equilpp($[$ue$]$)\\ 
\cmarg \verb@// calcule un point d'equilibre xe du systeme pp@\\ 
\cmarg \verb@// pour la commande constante ue a choisir lors de l'appel de la macro@\\ 
\cmarg \verb@//  ue=0 par defaut@\\ 
\cmarg \verb@//!@\\ 
\cmarg $<$lhs,rhs$>$={\bf argn}(0);{\bf if} rhs=0,ue=0,{\bf end}\\ 
\cmarg xe(1) =  (ppm+ue)/ppb;\\ 
\cmarg xe(2) =  (ppr$\star$(1$-$xe(1)/ppk)$-$ue)/ppa;\\ 
\cmarg //end}
\end{flushleft}}



\def\Mcompet{
\begin{flushleft}
{\sl 
\cmarg //$<$xdot$>$=compet(t,x,u)\\ 
\cmarg //$<$xdot$>$=compet(t,x,$[$u$]$) u est optionnel\\ 
\cmarg \verb@// Un modele de competition @\\ 
\cmarg \verb@// x(1),x(2) deux populations vivant sur une meme ressource @\\ 
\cmarg \verb@// 1/u est le niveau de la ressource (vaut 1 par defaut)@\\ 
\cmarg \verb@// ex : deux souches de levure @\\ 
\cmarg \verb@// ex : des crustaces sur une algue @\\ 
\cmarg \verb@//!@\\ 
\cmarg $<$lhs,rhs$>$={\bf argn}(0);\\ 
\cmarg {\bf if} rhs=2,u=1,{\bf end}\\ 
\cmarg xdot=0$\star${\bf ones}(2,1);\\ 
\cmarg xdot(1) = ppr$\star$x(1)$\star$(1$-$x(1)/ppk) $-$ u$\star$ppa$\star$x(1)$\star$x(2) ,\\ 
\cmarg xdot(2) = pps$\star$x(2)$\star$(1$-$x(2)/ppl) $-$ u$\star$ppb$\star$x(1)$\star$x(2) ,\\ 
\cmarg //end}
\end{flushleft}}



\def\Mequilcom{
\begin{flushleft}
{\sl 
\cmarg //$<$xe$>$=equilcom(ue)\\ 
\cmarg //$<$xe$>$=equilcom($[$ue$]$)\\ 
\cmarg \verb@// calcule un point d'equilibre du systeme compet@\\ 
\cmarg \verb@// pour un niveau de ressource ue  (vaut 1 par defaut)@\\ 
\cmarg \verb@//!@\\ 
\cmarg $<$lhs,rhs$>$={\bf argn}(0);\\ 
\cmarg {\bf if} rhs=0,ue=1,{\bf end}\\ 
\cmarg mat=$<$  ppr/ppk , ue$\star$ppa; ue$\star$ppb , pps/ppl $>$\\ 
\cmarg cte=$<$ppr;pps $>$\\ 
\cmarg xe= {\bf inv}(mat)$\star$cte;\\ 
\cmarg //end}
\end{flushleft}}



\def\Mlincomp{
\begin{flushleft}
{\sl 
\cmarg //$<$f,g,h,linsy$>$=lincomp(ue)\\ 
\cmarg //$<$f,g,h,linsy$>$=lincomp(ue)\\ 
\cmarg \verb@// fournit les matrices f,g et h du modele compet avec sortie@\\ 
\cmarg \verb@// linearise  autour du point d'equilibre donne par equilpp(ue).@\\ 
\cmarg \verb@// la sortie y est la premiere composante x(1) du systeme.@\\ 
\cmarg \verb@// linsy : est une macro qui donne la dynamique du syt\ `eme @\\ 
\cmarg \verb@// lin\ 'eaire tangent @\\ 
\cmarg \verb@//!@\\ 
\cmarg $<$f,g,linsy$>$=tangent('compet',equilcom(ue),ue);\\ 
\cmarg h=$<$1,0$>$;\\ 
\cmarg //end}
\end{flushleft}}




\def\Mgaincom{
\begin{flushleft}
{\sl 
\cmarg //$<$$>$=gaincom(pole,ue)\\ 
\cmarg //$<$$>$=gaincom(pole,$[$ue$]$)\\ 
\cmarg \verb@// calcul de la matrice k de gain pour @\\ 
\cmarg \verb@// la commande du mod\ `ele de competition @\\ 
\cmarg \verb@// de sorte que la matrice f -g*k ait pour p\ \^oles @\\ 
\cmarg \verb@// le vecteur colonne pole@\\ 
\cmarg \verb@//!@\\ 
\cmarg $<$lhs,rhs$>$={\bf argn}(0);\\ 
\cmarg {\bf if} rhs=1,ue=1,{\bf end}\\ 
\cmarg $<$f,g,h$>$=lincomp(ue);\\ 
\cmarg k={\bf ppol}(f,g,pole)\\ 
\cmarg $<$f,g,h,k$>$={\bf resume}(f,g,h,k)\\ 
\cmarg //end}
\end{flushleft}}



\def\Mgainobs{
\begin{flushleft}
{\sl 
\cmarg //$<$$>$=gainobs(pole,ue)\\ 
\cmarg //$<$$>$=gainobs(pole,$[$ue$]$)\\ 
\cmarg \verb@// calcul de la matrice l de gain pour @\\ 
\cmarg \verb@// l'observateur du modele de competition@\\ 
\cmarg \verb@// de sorte que la matrice f -l*h ait pour poles @\\ 
\cmarg \verb@// le vecteur colonne pole @\\ 
\cmarg \verb@//!@\\ 
\cmarg $<$lhs,rhs$>$={\bf argn}(0);\\ 
\cmarg {\bf if} rhs=1,ue=1,{\bf end}\\ 
\cmarg $<$f,g,h$>$=lincomp(ue);\\ 
\cmarg l={\bf ppol}(f',h',pole)\\ 
\cmarg l=l'\\ 
\cmarg $<$f,g,h,l$>$={\bf resume}(f,g,h,l)\\ 
\cmarg //end}
\end{flushleft}}



\def\Mboucle{
\begin{flushleft}
{\sl 
\cmarg //$<$$>$=boucle(fch,abruit,xdim,npts,farrow)\\ 
\cmarg //$<$$>$=boucle(fch,$[$abruit,xdim,npts,farrow$]$)\\ 
\cmarg \verb@// Donne le portrait de phase et des trajectoires du syst\ `eme @\\ 
\cmarg \verb@// dynamique fch suppose etre un systeme dynamique observe-controle@\\ 
\cmarg \verb@// avec sortie bruit\ 'ee  de dimension d'etat 4 ( x:2 , xchap:2)@\\ 
\cmarg \verb@// zdot=fch(t,z) dans le cadre xdim=<xmin,ymin,ymax,ymax>@\\ 
\cmarg \verb@// @\\ 
\cmarg \verb@// Dans la forme d'appel par d\ 'efaut : boucle(fch) les valeurs@\\ 
\cmarg \verb@// du cadre  et les pas d'int\ 'egration sont demand\ 'es interactivement.@\\ 
\cmarg \verb@// La macro boucle  va chercher dans l'environnement @\\ 
\cmarg \verb@// global les valeurs de (xe,ue) le point d'equilibre@\\ 
\cmarg \verb@//   f,g,h les matrices du systeme linearise @\\ 
\cmarg \verb@//   l et k  les deux matrices de gain @\\ 
\cmarg \verb@// Arguments:@\\ 
\cmarg \verb@// fch  : nom du systeme a integrer. @\\ 
\cmarg \verb@//       si c'est une chaine de caract\ `ere, on peut lui donner les valeurs@\\ 
\cmarg \verb@//       'bcomp' : pour mod\ `ele de competition non-lineaire @\\ 
\cmarg \verb@//               observe-commande @\\ 
\cmarg \verb@//       'lcomp' : pour mod\ `ele linearise observe-comande@\\ 
\cmarg \verb@//       sinon on peut lui donner le nom d'une macro que l'on aura cree@\\ 
\cmarg \verb@//       avec la commande obscont @\\ 
\cmarg \verb@//  @\\ 
\cmarg \verb@// abruit : amplitude du bruit rajoute sur les observations@\\ 
\cmarg \verb@//    @\\ 
\cmarg \verb@// npts=<nombre-de-points,pas> ->  sert \ `a donner le nombre de points et @\\ 
\cmarg \verb@//          le pas pour l'int\ 'egration num\ 'erique.@\\ 
\cmarg \verb@// @\\ 
\cmarg \verb@// xdim=<xmin,ymin,xmax,ymax> -> sert \ `a donner le cadre du dessin @\\ 
\cmarg \verb@// farrow vaur 't' ou 'f' : s'il vaut 't' on rajoute des fleches@\\ 
\cmarg \verb@//  le long des trajectoires @\\ 
\cmarg \verb@//!@\\ 
\cmarg $<$lhs,rhs$>$={\bf argn}(0);\\ 
\cmarg \verb@// appel minimal @\\ 
\cmarg {\bf if} rhs$<$=4,farrow='f';{\bf end};\\ 
\cmarg {\bf if} rhs$<$=3,{\bf write}(\%io(2),'integration : nb points,pas du systeme '),\\ 
\cmarg \hspace{0.5cm}npts={\bf read}(\%io(1),1,2),\\ 
\cmarg {\bf end}\\ 
\cmarg {\bf if} rhs $<$= 2,\\ 
\cmarg \hspace{0.5cm}{\bf write}(\%io(2),'cadre du dessin : xmin,ymin,xmax,ymax'),\\ 
\cmarg \hspace{0.5cm}xdim={\bf read}(\%io(1),1,4),\\ 
\cmarg \verb@// Test sur le cadre @\\ 
\cmarg \hspace{0.5cm}{\bf if} xdim(3) $<$= xdim(1),\\ 
\cmarg \hspace{0.5cm}{\bf write}(\%io(2),'Erreur:  xmin $>$ xmax '),{\bf return},{\bf end}\\ 
\cmarg \hspace{0.5cm}{\bf if} xdim(4) $<$= xdim(2),\\ 
\cmarg \hspace{0.5cm}{\bf write}(\%io(2),'Erreur:  ymin $>$ ymax '),{\bf return},{\bf end}\\ 
\cmarg {\bf end} \\ 
\cmarg {\bf if} rhs $<$=1,abruit=0.0;{\bf end} \\ 
\cmarg tcal=npts(2)$\star$(0:npts(1))\\ 
\cmarg {\bf rand}('normal')\\ 
\cmarg br={\bf sqrt}(abruit)$\star${\bf rand}(1,npts(1)+1);\\ 
\cmarg {\bf if} {\bf type}(fch)=10;\\ 
\cmarg \verb@// Passage des constantes a un programme Fortran d'initialisation @\\ 
\cmarg \hspace{0.5cm}idisp=0;\\ 
\cmarg \hspace{0.5cm}pp\_c=$<$ppr,ppk,ppa,ppb,ppm,pps,ppl$>$\\ 
\cmarg \hspace{0.5cm}{\bf fort}('icomp',xe,1,'r',ue,2,'r',f,3,'r',g,4,'r',h,5,'r',...\\ 
\cmarg \hspace{1.5cm}k,6,'r',l,7,'r',br,8,'r',npts(2),9,'r',npts(1),10,'i',...\\ 
\cmarg \hspace{1.5cm}pp\_c,11,'r',idisp,12,'i','{\bf sort}');\\ 
\cmarg {\bf end} \\ 
\cmarg xset("window",0);xselect();xclear();\\ 
\cmarg \verb@// Boucle sur les points de depart @\\ 
\cmarg \hspace{0.5cm}goon=1\\ 
\cmarg \hspace{0.5cm}{\bf while} goon=1,\\ 
\cmarg \hspace{1.8cm}ftest=1;\\ 
\cmarg \hspace{1.8cm}{\bf while} ftest=1,\\ 
\cmarg \hspace{2.5cm}addtitle(fch);\\ 
\cmarg \hspace{2.5cm}plot2d($<$xdim(1);xdim(1);xdim(3)$>$,$<$xdim(2);xdim(4);xdim(4)$>$)\\ 
\cmarg \hspace{2.5cm}plot2d($<$xe(1)$>$,$<$xe(2)$>$,$<$2,4$>$,"111",...\\ 
\cmarg \hspace{3.5cm}"Point d''equilibre pour ue='+{\bf string}(ue),xdim);\\ 
\cmarg \hspace{2.5cm}{\bf write}(\%io(2),'Utilisez la souris : ');\\ 
\cmarg \hspace{2.5cm}{\bf write}(\%io(2),' $-$$>$ Bouton de droite pour quiter ');\\ 
\cmarg \hspace{2.5cm}{\bf write}(\%io(2),' $-$$>$ Bouton du milieu ou de gauche ');\\ 
\cmarg \hspace{2.5cm}{\bf write}(\%io(2),'      pour indiquer x0 ');\\ 
\cmarg \hspace{2.5cm}$<$n,x,y$>$=xclick()\\ 
\cmarg \hspace{2.5cm}{\bf if} n=2,goon=0;{\bf return};{\bf end}\\ 
\cmarg \hspace{2.5cm}$<$xx0,yy0,recc$>$=xchange(x,y,'i2f');\\ 
\cmarg \hspace{2.5cm}x0=$<$xx0,yy0$>$;\\ 
\cmarg \hspace{2.5cm}{\bf write}(\%io(2),'Utilisez la souris : ');\\ 
\cmarg \hspace{2.5cm}{\bf write}(\%io(2),' $-$$>$ Bouton de droite pour quiter ');\\ 
\cmarg \hspace{2.5cm}{\bf write}(\%io(2),' $-$$>$ Bouton du milieu ou de gauche ');\\ 
\cmarg \hspace{2.5cm}{\bf write}(\%io(2),'      pour indiquer xchap0 (observateur) ');\\ 
\cmarg \hspace{2.5cm}$<$n,x,y$>$=xclick()\\ 
\cmarg \hspace{2.5cm}{\bf if} n=2,goon=0;{\bf return};{\bf end}\\ 
\cmarg \hspace{2.5cm}$<$xx0,yy0,recc$>$=xchange(x,y,'i2f');\\ 
\cmarg \hspace{2.5cm}xchap0=$<$xx0,yy0$>$;\\ 
\cmarg \hspace{2.5cm}{\bf if} {\bf type}(fch)=10,\\ 
\cmarg \hspace{3.2cm}ftest=desorb1($<$x0,xchap0$>$',npts,fch,farrow,xdim);\\ 
\cmarg \hspace{2.5cm}{\bf else} \\ 
\cmarg \hspace{3.2cm}ftest=desorb1($<$x0,xchap0$>$',npts,{\bf list}(fch,abruit,...\\ 
\cmarg \hspace{6.2cm}npts(2),npts(1)),farrow,xdim);             \\ 
\cmarg \hspace{2.5cm}{\bf end}\\ 
\cmarg \hspace{2.5cm}{\bf if} ftest=1;{\bf write}(\%io(2),'conditions initiales hors du cadre'),{\bf end}\\ 
\cmarg \hspace{1.8cm}{\bf end}\\ 
\cmarg \hspace{0.5cm}{\bf end}\\ 
\cmarg //end}
\end{flushleft}}



\def\Mdesorbu{
\begin{flushleft}
{\sl 
\cmarg //$<$res$>$=desorb1(x0,n1,fch,farrow,xdim);\\ 
\cmarg //$<$res$>$=desorb1(x0,n1,fch,farrow,xdim);\\ 
\cmarg \verb@//!@\\ 
\cmarg res=0\\ 
\cmarg {\bf write}(\%io(2),'Calculs en cours')\\ 
\cmarg tcal=n1(2)$\star$(0:n1(1))\\ 
\cmarg xxx={\bf ode}(x0,0,tcal,fch);\\ 
\cmarg $<$nn1,nn2$>$={\bf size}(tcal);\\ 
\cmarg comcom=$-$k$\star$(xxx(3:4,:)$-$xe$\star${\bf ones}(1,nn2));\\ 
\cmarg \verb@//dessin de l'evolution conjointe de la deuxieme@\\ 
\cmarg \verb@//composante de l'etat et de son estimee (observateur)@\\ 
\cmarg xset("window",1);xclear();\\ 
\cmarg plot2d($<$tcal;tcal$>$',xxx($<$2,4$>$,:)',$<$$-$1,$-$2$>$,"111",...\\ 
\cmarg \hspace{1.8cm}"x2(t) @observateur de x2(t)",$<$0,xdim(2),n1(1)$\star$n1(2),xdim(4)$>$)\\ 
\cmarg xset("window",2);xclear();\\ 
\cmarg \verb@//dessin de la commande lineaire@\\ 
\cmarg plot2d($<$tcal$>$',$<$comcom$>$',$<$$-$1$>$,"121",...\\ 
\cmarg \hspace{1.8cm}"commande lineaire en fonction du temps (ecart par rapport a ue)")\\ 
\cmarg xset("window",0);xclear();\\ 
\cmarg \verb@//portrait de phase @\\ 
\cmarg plot2d(xxx($<$1,3$>$,:)',xxx($<$2,4$>$,:)',$<$$-$1,$-$2$>$,"111","(x1,x2)@observateur ",...\\ 
\cmarg xdim);\\ 
\cmarg //end }
\end{flushleft}}



\def\Mobscont{
\begin{flushleft}
{\sl 
\cmarg //$<$macr$>$=obscont(sysn)\\ 
\cmarg //$<$macr$>$=obscont(sysn)\\ 
\cmarg \verb@// @\\ 
\cmarg \verb@//cette macro renvoit le systeme observe-commande @\\ 
\cmarg \verb@//construit a partir du systeme sysn linearise autour de (xe,ue)@\\ 
\cmarg \verb@//@\\ 
\cmarg \verb@// sysn : chaine de caractere donnant le nom du systeme a commander @\\ 
\cmarg \verb@// gaincom,gainobs : vecteurs colonnes des gains demandes @\\ 
\cmarg \verb@//@\\ 
\cmarg \verb@// Retour : une nouvelle macro donnant le syst\ `eme observe-commande@\\ 
\cmarg \verb@// <x1dot>=macr(t,x1,abruit,pas,n)@\\ 
\cmarg \verb@//    x1=<x;xchap>,@\\ 
\cmarg \verb@// pour l'appel il faudra creer un vecteur de bruit de nom br @\\ 
\cmarg \verb@// d'autre part la macro cr\ 'ee va chercher dans l'environnement @\\ 
\cmarg \verb@// global les valeurs de (xe,ue) le point d'equilibre@\\ 
\cmarg \verb@//   f,g,h les matrices du systeme linearise @\\ 
\cmarg \verb@//   l et k  les deux matrices de gain @\\ 
\cmarg \verb@// exemple :@\\ 
\cmarg \verb@//    pas=10;n=200;@\\ 
\cmarg \verb@//    br=rand(1,n)@\\ 
\cmarg \verb@//    ode(<x0;xchap0>,0,pas*(0:n),list(macr,1.0,pas,n));@\\ 
\cmarg \verb@//!@\\ 
\cmarg {\bf deff}('$<$zdot$>$=macr(t,z,abr,pas,n)',...\\ 
\cmarg \hspace{1.0cm}$<$'u=ue$-$k(1)$\star$(z(3)$-$xe(1))$-$k(2)$\star$(z(4)$-$xe(2))';\\ 
\cmarg \hspace{1.2cm}'ff\_brui=abr$\star$br({\bf ent}({\bf mini}(t/pas+1,n)))';\\ 
\cmarg \hspace{1.2cm}'y=h(1)$\star$z(1)+h(2)$\star$z(2)+ff\_brui';\\ 
\cmarg \hspace{1.2cm}'xdot='+sysn+'(t,z(1:2),u)';\\ 
\cmarg \hspace{1.2cm}'xdot1=f$\star$(z(3:4)$-$xe) +g$\star$(u$-$ue)$-$l$\star$(h$\star$z(3:4)$-$y)';\\ 
\cmarg \hspace{1.2cm}'zdot=$<$xdot;xdot1$>$';$>$);\\ 
\cmarg //end}
\end{flushleft}}





\def\Mcomric{
\begin{flushleft}
{\sl 
\cmarg //$<$$>$=comric(q1,q2,r)\\ 
\cmarg //$<$$>$=comric(q1,q2,r)\\ 
\cmarg \verb@// pi est la solution de l'equation de riccati stationnaire @\\ 
\cmarg \verb@// f'*pi+pi*f+ q*eye(2)-pi*g*(1/r)*g'*pi=0@\\ 
\cmarg \verb@// pour calculer le gain r\^-1*g'*pi de la commande.@\\ 
\cmarg \verb@// ici, q et r sont des matrices de ponderation d'une fonction @\\ 
\cmarg \verb@// cout quadratique.@\\ 
\cmarg \verb@//!@\\ 
\cmarg $<$pi$>$={\bf ricc}(f,(1/r)$\star$g$\star$g',$<$q1,0;0,q2$>$,'cont');\\ 
\cmarg k= (1/r) $\star$g'$\star$pi;\\ 
\cmarg $<$k$>$={\bf resume}(k);\\ 
\cmarg //end}
\end{flushleft}}



\def\Mobsric{
\begin{flushleft}
{\sl 
\cmarg //$<$$>$=obsric(q1,q2,r)\\ 
\cmarg //$<$$>$=obsric(q1,q2,r)\\ 
\cmarg \verb@// p est la solution de l'equation de riccati stationnaire @\\ 
\cmarg \verb@// f*p+p*f'+ q*eye(2)-(1/r)*p*h'*h*p=0@\\ 
\cmarg \verb@// pour calculer le gain r\^-1*p*h' de l'observateur@\\ 
\cmarg \verb@// confondu avec le filtre de Kalman.@\\ 
\cmarg \verb@// ici, q est la variance du bruit d'etat, @\\ 
\cmarg \verb@//      r la variance du bruit de mesure. @\\ 
\cmarg \verb@//!@\\ 
\cmarg $<$p$>$={\bf ricc}(f',(1/r)$\star$h'$\star$h,$<$q1,0;0,q2$>$,'cont');\\ 
\cmarg l= (1/r) $\star$p$\star$h'\\ 
\cmarg $<$l$>$={\bf resume}(l);\\ 
\cmarg //end}
\end{flushleft}}




\def\Mexemple{
\begin{flushleft}
{\sl 
\cmarg //$<$$>$=exemple()\\ 
\cmarg \verb@// un essai de calcul de gains avec riccati@\\ 
\cmarg \verb@// et simulation.@\\ 
\cmarg \verb@//!@\\ 
\cmarg ue=1;\\ 
\cmarg xe=equilcom(ue);\\ 
\cmarg $<$f,g,h,vv$>$=lincomp(ue);\\ 
\cmarg obsric(3,1,1);\\ 
\cmarg comric(3,1,1);\\ 
\cmarg ue=1;\\ 
\cmarg xe=equilcom(ue);\\ 
\cmarg boucle('bcomp',0.1,$<$0,150,50,210$>$,$<$100,10$>$);\\ 
\cmarg //end}
\end{flushleft}}




\def\Mtest{
\begin{flushleft}
{\sl 
\cmarg //$<$f1,f2$>$=test(ue)\\ 
\cmarg //$<$f1,f2$>$=test(ue)\\ 
\cmarg \verb@// f1 est la dynamique linearisee @\\ 
\cmarg \verb@//autour du point d'equilibre donne par equilcom(ue)@\\ 
\cmarg \verb@//du systeme precedent observateur-controleur (voir simulcomp)@\\ 
\cmarg \verb@// f2 est la valeur theorique de f1 (voir cours)@\\ 
\cmarg \verb@//!@\\ 
\cmarg {\bf deff}('$<$yy,zdot$>$=fff(z,vv)',...\\ 
\cmarg \hspace{1.0cm}$<$'u=1$-$k$\star$z(3:4)';\\ 
\cmarg \hspace{1.2cm}'yy=0';\\ 
\cmarg \hspace{1.2cm}'xdot(1,1) = ppr$\star$z(1)$\star$(1$-$z(1)/ppk) $-$ u$\star$ppa$\star$z(1)$\star$z(2)';\\ 
\cmarg \hspace{1.2cm}'xdot(2,1) = pps$\star$z(2)$\star$(1$-$z(2)/ppl) $-$ u$\star$ppb$\star$z(1)$\star$z(2)';\\ 
\cmarg \hspace{1.2cm}'xdot1 = (f$-$l$\star$h$-$g$\star$k)$\star$z(3:4) + l$\star$(z(1)$-$xe(1))';\\ 
\cmarg \hspace{1.2cm}'zdot=$<$xdot;xdot1$>$'$>$);\\ 
\cmarg xe=equilcom(ue)\\ 
\cmarg $<$f1,g1,h1,k1$>$=lin(fff,$<$xe;0;0$>$,0),\\ 
\cmarg f2=$<$ f, $-$g$\star$k; l$\star$h,f$-$l$\star$h$-$g$\star$k$>$,\\ 
\cmarg //end}
\end{flushleft}}




\Mcomric

\Mobsric

\Mexemple

\Mtest

\section{Un probl\`eme de contr\^ole optimal}

\Mmine

\Mtrajopt

\Mshowcost

\Mff

\centerline{{\sc Questions}}

\begin{itemize}
\item Taper {\bf tdinit3()}. Choisir le nombre de couches $n1$, 
   le nombre d'\'etapes $n2$ et 
   l'ensemble des commandes admissibles sous forme d'un vecteur ligne 
   $uvect=<u_1,\ldots,u_n>$ ( par exemple $<-1,0,1>$ ). appeller la macro
	 par la commande {\bf $<$cout,feed$>$=mine(n1,n2,uvect)}.
\item Visualiser la commande et la trajectoire optimales en tapant 
	{\bf trajopt(feed)}. 
\item Changer les valeurs des param\`etres de la fonction gain instantan\'e 
   $n1$, $n2$, $te$, $xe$, $k0$,  $kc$ et relancer l'ex\'ecution de {\bf mine}.
\item Cr\'eer sa propre macro {\bf ff} donnant le gain instantan\'e apparaissant dans 
le programme {\bf mine} et relancer l'ex\'ecution de {\bf mine}.
\end{itemize}
\end{document}
 