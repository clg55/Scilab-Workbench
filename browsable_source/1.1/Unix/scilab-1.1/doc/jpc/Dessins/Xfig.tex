
\section{Une nouvelle option dans  transfig}

Les dessins Xfig peuvent \^etre ins\'er\'es dans un document
 \LaTeX\, gr\^ace a l'utilitaire {\bf transfig}. 
Une nouvelle option dans transfig est {\bf -L latexps}. Les commandes qui suivent~: 

\begin{verbatim}
transfig -L latexps -m 0.5 pointselle.fig
make 
\end{verbatim} 
vont cr\'eer  deux fichiers {\bf pointselle.tex } et {\bf pointselle.ps}. 
 Les objets de type  cha\^{\i}nes de caract\`eres du fichier XFig sont traduites 
 en \LaTeX\, et  le reste du dessin est traduit en Postscript. Ceci permet 
 ainsi par exemple d'obtenir des formules math\'ematiques dans un dessin.

La commande \verb+XFlatexpr+ r\'ealise l'op\'eration pr\'ec\'edente en une seule fois et change le fichier \LaTeX pour qu'il ait la m\^eme forme que dans 
 les cas pr\'ec\'edents. On utilisera donc la commande~:
\begin{verbatim}
XFlatexpr 0.5 pointselle.fig
\end{verbatim}
o\`u le deuxi\`eme argument est un facteur de magnification. Le dessin est alors mis dans \LaTeX par la comande~:
\begin{verbatim}
\input{pointselle.tex}
\dessin{Portrait de phase d'un point selle}{pointselle}
\end{verbatim} 
pour obtenir le dessin dans votre document \LaTeX. On obtient par exemple 
 la figure~(\ref{pointselle}).
 On peut bien sur \'editer le fichier poinsellle.tex pour rajouter en \LaTeX  
 diverses informations sur le dessin. 

\input{pointselle.tex}
\dessin{Portrait de phase d'un point selle}{pointselle}



