\chapter{Graphics}

This section introduces graphics in \Scilab. There are basically three
different graphics devices in \Scilab~:
\begin{itemize}
 \item[X11] Graphics device for the X11 window system 
 \item[Pos] Graphics device for Postscript Printers 
 \item[Xfig]Graphics device for the Xfig system
\end{itemize}
A driver can be selected with the command
\verb+driver('Driver-Name')+. The default driver is named "Rec" a
driver for the X11 window system which also records all the graphics
commands. 

In fact, one can ignore the existence of drivers and use the
functions \verb+xbasimp+, \verb+xs2fig+ in order to send a graphic 
to a printer or in a file for the \verb+Xfig+ system. For example, the
two commands~:
\begin{verbatim}
  plot(1:10) 
  xbasimp(0,'foo.ps')
\end{verbatim}

will plot \verb+y=x+ in an X11 window ( window 0) 
and then will replay the graphic
commands to produce, in the file {\tt 'foo.ps'}, a Postscript version of the
graphics code ( Since we can have many graphics windows, the first
number $0$ refers to it's name).

If you enlarge a graphic window, the command \verb+xbasr(window-name)+
is executed by \Scilab. This command clears the graphic window
\verb+window-name+ and replays the graphic commands associated to it. One can
call this function manually, in order to verify the associated
recorded graphics commands. 

\verb+xbasc(window-name)+ will clear
the recorded \Scilab graphics command associated to
the window \verb+window-name+ and will clear that X11-window.

One can use many graphics windows which can be created with buttons 
or with the commands \verb+xset+ or \verb+xselect+. The environment 
variable DISPLAY can be used to specify an X11 Display or one can use
the \verb+xinit+ function in order to open a graphic window on a
specific display. 

All the graphics are send to the "current graphic window", 
which can be set or raised with the help of specific X-$\Psi$lab
buttons or with the use of the functions \verb+xset+ or \verb+xselect+.

\section{Plotting}

\def\Figdir{figures/}
\subsection{A first example}
\def\exemple{d7.1}

\verbatok{figures/\exemple.code}
\end{verbatim}
\input{\Figdir \exemple.tex}
\dessin{\exemple}{\exemple}

This first example plots the function $f(t)\equiv \cos(t)$. The sampled values
are specified by two column vectors $t$ and $cos(t)$. 
Some parameters of the graphics are controlled by a graphic context 
( for example the line thickness) and other are controlled thru
graphics arguments. In this first example (Figure~(\ref{d7.1})),
we use all the default arguments.

\subsection{Superposing curves}
Suppose now that we want to specify the graphics boundaries, and that
we want to superpose two curves by calling \verb+plot2d+ twice (Figure~(\ref{d7.2}))~:
\def\exemple{d7.2}
\verbatok{figures/\exemple.code}
\end{verbatim}
\input{\Figdir \exemple.tex}
\dessin{\exemple}{\exemple}

This can be done with additional arguments. In the first call to
\verb+plot2d+, the key argument is the string \verb+"111"+~:
\begin{itemize}
 \item the first 1 means : add a caption for that curve 
 \item the second 1 means : uses the argument \verb+[0,-1,1,1]+ to set
the graphic boundaries
 \item the third 1 means : add x and y tics on the graphics.
\end{itemize}

After that call, the fixed boundaries are the defaults ones 
and on the second call the 0 character in \verb+"100"+ stands for :
use the default boundaries. Since a call to a graphic function doesn't
clear the graphic window we get superposed curves. 

One can also modify the graphic boundaries by calling the function
\verb+xsetech+ and can get the default values by calling the function
\verb+xgetech+. The basic graphics primitives also use the default 
boundaries. In the following example graphic boundaries are fixed with
a first call to \verb+plot2d+ and then a set of graphic primitives are
showed (Figure~(\ref{d7.9})).

\def\exemple{d7.9}
\verbatok{figures/\exemple.code}
\end{verbatim}
\input{\Figdir \exemple.tex}
\dessin{\exemple}{\exemple}

\subsection{Mixing 2D and 3D graphics}

When one uses 3D plotting function default graphic boundaries are
fixed, but in $R^3$. If one wants to use graphic primitives 
to add informations on 3D graphics the \verb+geom3d+ function can be
used to convert 3D coordinates to 2D-graphics coordinates. The
figure~(\ref{d7.10}) illustrates that feature.

\def\exemple{d7.10}
\verbatok{figures/\exemple.code}
\end{verbatim}
\input{\Figdir \exemple.tex}
\dessin{\exemple}{\exemple}

\subsection{Multiple plotting in one call}
\def\exemple{d7.3}

In fact \verb+plot2d+ can be directly used to plot multiple curves 
in one call. This is achieved by using matrix arguments and the 
the next example (Figure~(\ref{\exemple})) shows this possibility~:

\verbatok{figures/\exemple.code}
\end{verbatim}
\input{\Figdir \exemple.tex}
\dessin{\exemple}{\exemple}

\subsection{Sub-windows}
\def\exemple{d7.8}

It's also possible to make multiple plotting in the same 
graphic window (Figure~\ref{\exemple}).

\verbatok{figures/\exemple.code}
\end{verbatim}
\input{\Figdir \exemple.tex}
\dessin{Use of xsetech}{\exemple}

\subsection{A Set of figures}
\def\exemple{d7.11}

In This next example we give a brief summary of different plotting
functions for 2D or 3D graphics. The figure~(\ref{\exemple}) is 
obtained and inserted in this document with the help of the command 
\verb+Blatexprs+. 
\verbatok{figures/\exemple.gcode}
\end{verbatim}
\input{\Figdir \exemple.tex}
\dessin{\exemple}{\exemple}

\section{Printing and inserting \Scilab graphics in \LaTeX}
\subsection{Postscript File}
The Postscript files generated by \Scilab  can't be directly sent to a
Postscript printer, they need a preamble. Therefore, printing is done
thru the use of Unix scripts or programs which are provided with
\Scilab. The program \verb+Blpr+ is used to print a set of \Scilab Graphics 
on a single sheet of paper and is used as follows~:

\begin{verbatim}
 Blpr string-title file1.ps file2.ps ... | lpr 
\end{verbatim}

If you have the \verb+ghostview+ Postscript interpreter on your Unix
workstation you can do as well~:

\begin{verbatim}
 Blpr string-title file1.ps file2.ps ... | ghostview -
\end{verbatim}

in order to see the Postscript file before printing. The best result 
( best sized figures ) is obtained when printing two pictures on a
single page. 

\subsection{Postscript File in \LaTeX}

The \verb|Blatexpr| Unix shell and the programs \verb+Batexpr2+ and 
\verb+Blatexprs+ are provided in order to help inserting 
\Scilab graphics in \LaTeX.

Suppose that the file "sin.ps" was created with the help of
\verb+xbasimp+. Then typing the following statement under a Unix shell~:
\begin{verbatim}
 Blatexpr 1.0 1.0 sin.ps
\end{verbatim}
will create two files \verb+sin.ps.n+ and \verb+sin.tex+. The original
Postscript file is left unchanged. 
In order to get your picture you then insert the following \LaTeX code
in your \LaTeX document~:
\begin{verbatim}
\input fig.tex
\dessin{The caption of you picture}{The-label}
\end{verbatim}

The program \verb+Blatexprs+ does the same thing be is used to insert
a set of Postscript Figures in one \LaTeX picture.

\begin{verbatim}
  Blatexprs Bas2 file1.ps file2.ps .... filen
\end{verbatim}
In this case the figures are in the files \verb+file<i>+ and
\verb+Bas2+ is a file name in which the result will be generated.
This command will create two files \verb+Bas2.ps+ et \verb+Bas2.tex+.
and the picture is obtained with the \LaTeX statement.
\begin{verbatim}
\input Bas2.tex
\dessin{The caption of your picture}{The-label}
\end{verbatim}
The Postscipt files are inserted in \LaTeX with the help of the
\verb+\special+ command and with a syntax that works with the
\verb+dvips+ program.

The program \verb+Blatexpr2+ is used when you want two pictures side
by side.
\begin{verbatim}
  Blatexpr2 Fileres file1.ps file2.ps 
\end{verbatim}
\def\exemple{d7.12}
\input{\Figdir \exemple.tex}
\dessin{Blatexp2 Example}{\exemple}

It's sometime convenient to have a main \LaTeX document in a directory
and to store all the figures in a subdirectory. The proper way to
insert a  picture file in the main document, when the picture 
is stored in the subdirectory \verb+figures+, is the following~:

\begin{verbatim}
\def\Figdir{figures/} % My figures are in the figures/ subdirectory.
\input{\Figdir fig.tex}
\dessin{The caption of you picture}{The-label}
\end{verbatim}

The declaration \verb+\def\Figdir{figures/}+ is used twice, first to
find the file fig.tex ( when you use \verb+latex+ ), and
 second to produce a correct pathname for the
\verb+special+ \LaTeX command  found in \verb+fig.tex+. ( used at
dvips level).







