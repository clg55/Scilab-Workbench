\documentstyle[11pt]{article}
% Copyright ENPC
             \textheight=660pt 
             \textwidth=15cm
             \topmargin=-27pt 
             \oddsidemargin=0.7cm
             \evensidemargin=0.7cm
             \marginparwidth=60pt
             \title{Simulation of an n-link pendulum} 
             \author{J.Ph. Chancelier\thanks{Cergrene. Ecole Nationale des Ponts et Chauss\'ees, La Courtine  93167 Noisy le Grand C\'{e}dex }}
 
\begin{document}\maketitle

\def\tenrm{\size{10}{11pt}\shape{n}\family{cmr}\selectfont} 
\def\elvrm{\size{11}{12pt}\shape{n}\family{cmr}\selectfont} 
\def\sixrm{\size{6}{7pt}\shape{n}\family{cmr}\selectfont} 
\def\sevrm{\size{7}{8pt}\shape{n}\family{cmr}\selectfont} 
\def\dotx{\dot{x}}
\def\doty{\dot{y}}
\def\dotq{\dot{q}}
\def\dotth{\dot{\theta}}

\input npend.tex
\dessin{3-link Pendulum}{tlp}

\section{Equation of motion of an N-link pendulum}
The n-link pendulum is described on picture~(\ref{tlp}).
For the i-link~:
\begin{equation}
	E_i^c = 
	 1/2 m_i \left( (\dotx_{i-1} - l_i/2 \sin(\theta_i)\dotth_i)^2 +
	(\doty_{i-1}+ l_i/2 \cos(\theta_i)\dotth_i)^2 \right)
	+1/2 J_i \dot{\theta}_i^2
\end{equation}
with~:
\begin{equation}
	J_i= m_i l_i^2/12 
\end{equation}
\begin{equation}
	E_i^p= m_i g \left( y_{i-1} + l_i/2 \sin(\theta_i)\right)
\end{equation}
With the added point $(x_0,y_0)=(0,0)$.
The n-link pendulum has the following Lagrangian~:
\begin{eqnarray}
	L(q,\dot{q})&=& 
	\sum_{i=1}^n \left\{
	 1/2 m_i \left( (\dotx_{i-1} - l_i/2 \sin(\theta_i)\dotth_i)^2 +
	(\doty_{i-1}+ l_i/2 \cos(\theta_i)\dotth_i)^2 \right)
	+1/2 J_i \dot{\theta}_i^2 \right.\\
	&-& \left. \left( m_i g \left( y_{i-1} + l_i/2 \sin(\theta_i)\right)\right)
	\right\}
\end{eqnarray}
And is subject to the following set of constraints~:
\begin{equation}
	i=1,\ldots,n \quad \left\{
	\begin{array}{lcl}
	 x_i -x_{i-1} &=& l_i \cos(\theta_i) \\
	 y_i -y_{i-1} &=& l_i \sin(\theta_i) \\
	\end{array} \right.
\label{const}
\end{equation}

A time derivative of equation~(\ref{const}) leads to~:
\begin{equation}
	i=1,\ldots,n \quad \left\{
	\begin{array}{lcl}
	 \dotx_i -\dotx_{i-1} + l_i \sin(\theta_i)\dotth_i =0 \\
	 \doty_i -\doty_{i-1} - l_i \cos(\theta_i) \dotth_i =0 
	\end{array} \right.
\label{const}
\end{equation}
Which is of the form $A'(q) \dot{q} =0$.
If we solve the linear system $ A'(q)\tilde{q}=0 $ it's easy to see
that the solution depend on n-parameters $(\pi_i)_{i=1,n}$
\begin{eqnarray}
	\tilde{x}_i &=& - \sum_{k\le i} l_k \sin(\theta_k) \pi_k\\
	\tilde{y}_i &=&  \sum_{k\le i} l_k \cos(\theta_k) \pi_k\\
	\tilde{\theta}_i &=& \pi_k
\label{eqp}
\end{eqnarray}
In a  matrix form equation~(\ref{eqp}) gives $\tilde{q}=S(q)\pi$ and 
$ A(q)'S(q)=0$ 

Th Euler-Lagrange equations for the n-link pendulum can be derived
from The lagrangian and the constraints~:
\def\derivt{ {d \over dt}}
\def\derp#1#2{ { \partial #1 \over \partial #2}}
\begin{equation}
		\begin{array}{l} 
	\displaystyle \derivt ( \derp{L}{\dotq_i}(q(t),\dotq(t)) 
	- \derp{L}{q_i}(q(t),\dotq(t)) = (A(q)\lambda)_i \\
		\displaystyle \derivt q = \dotq
		\end{array}
\end{equation}

\end{document}



