\section{Scicos editor}
% -*-LaTeX-*-
% Converted automatically from troff to LaTeX by tr2tex on Tue Apr 22 11:31:15 1997
% tr2tex was written by Kamal Al-Yahya at Stanford University
% (Kamal%Hanauma@SU-SCORE.ARPA)


%\documentstyle[troffman]{article}

%
% input file: ../../man//scicos/scicos.man
%
\phead{scicos}{5}{Janvier\ 1996}{Scilab\ Group}{Scilab\ function}


\Sdoc{scicos}{ Block diagram editor and GUI to hybrid simulator scicosim}\index{scicos}\label{scicos}

\Shead{CALLING SEQUENCE}
sys=scicos()
sys=scicos(sys,[menus])
sys=scicos(file,[menus])
\Shead{PARAMETERS}
\begin{scitem}
\item[{\verb?sys?}]
: a Scicos data structure
\item[{\verb?file?}]
: a character string. The path of a file containing the binary image
of a Scicos data structure 
\item[{\verb?menus?}]
: a vector of character strings. It allows to select some of the
scicos menus. If %
\tt menus==[] %
\rm  scicos draws the diagram and the
contents of each super blocks in separate windows without menu
bar. This option is usefull to print diagram definition.

\end{scitem}% end Env
\Shead{DESCRIPTION}
scicos is a visual editor for constructing models of hybrid dynamical
systems. Invoking scicos with no argument opens up an empty scicos
window. Models can then be assembled, loaded, saved, compiled,
simulated, using GUI of scicos. The input and ouput arguments are only
useful for debugging purposes. scicos serves as an interface to the
various block diagram compilers and the hybrid simulator scicosim.

\Seealso{SEE ALSO}
{\verb?scicosim scicos_main scicos_menu ?} \pageref{scicosimscicosmainscicosmenu}
      








% -*-LaTeX-*-
% Converted automatically from troff to LaTeX by tr2tex on Tue Apr 22 11:31:15 1997
% tr2tex was written by Kamal Al-Yahya at Stanford University
% (Kamal%Hanauma@SU-SCORE.ARPA)


%\documentstyle[troffman]{article}


\section{Blocks}
% -*-LaTeX-*-
% Converted automatically from troff to LaTeX by tr2tex on Tue Apr 22 11:31:15 1997
% tr2tex was written by Kamal Al-Yahya at Stanford University
% (Kamal%Hanauma@SU-SCORE.ARPA)


%\documentstyle[troffman]{article}

%
% input file: ../../man//scicos/ABSBLK_f.man
%
\phead{ABSBLK\_f}{5}{Janvier\ 1996}{Scilab\ Group}{Scicos\ Block}


\Sdoc{ABSBLK\_f}{ Scicos abs block}\index{ABSBLK\_f}\label{ABSBLKf}

\Shead{DIALOG PARAMETERS}
None.
\Shead{DESCRIPTION}
This block realizes element-wise vector absolute value operation. 
This block has a single input and a single output port.
Port dimension is determined by the context.





%
% input file: ../../man//scicos/AFFICH_f.man
%
\phead{AFFICH\_f}{5}{Janvier\ 1997}{Scilab\ Group}{Scicos\ Block}


\Sdoc{AFFICH\_f}{ Scicos digital viewer}\index{AFFICH\_f}\label{AFFICHf}

\Shead{DIALOG PARAMETERS}
\begin{scitem}
\item[{\verb?font?}]
: integer, the selected font number (see xset)
\item[{\verb?fontsize?}]
: integer, the selected font size (set xset)
\item[{\verb?color?}]
: integer, the selected color for the text (see xset)
\item[{\verb?Total numer of digits?}]
: an integer greater than 3, the maximum number of digits used to
represent the number(sign,integer part and rational part)
\item[{\verb?rational part numer of digits?}]
: an integer greater or equal 0, the number of digits used to
represent the  rational part

\end{scitem}% end Env
\Shead{DESCRIPTION}
This block allows to display a value in a block shape during
simulation. The block must be located in the main scilab window.

Warning, each time the block is moved user may click on it to set it's
parameters. So display position is automatically updated.
\Seealso{SEE ALSO}
{\verb?SCOPE_f?} \pageref{SCOPEf},{\verb? xset?} \pageref{xset}

%
% input file: ../../man//scicos/ANDLOG_f.man
%
\phead{ANDLOG\_f}{5}{Janvier\ 1997}{Scilab\ Group}{Scicos\ Block}


\Sdoc{ANDLOG\_f}{ Scicos logical AND block}\index{ANDLOG\_f}\label{ANDLOGf}

\Shead{DIALOG PARAMETERS}
None.
\Shead{DESCRIPTION}
This block, with two event inputs and a regular output,
outputs +1 or -1 on its regular ouput depending on input events. 
\begin{scitem}
\item[{\verb?+1?}]
:
When events are synchronously present on both event input ports
\item[{\verb?-1?}]
:
When only one event is present.
\end{scitem}% end Env
\Seealso{SEE ALSO}
{\verb?IFTHEl_f?} \pageref{IFTHElf}
%
% input file: ../../man//scicos/ANIMXY_f.man
%
\phead{ANIMXY\_f}{5}{Janvier\ 1996}{Scilab\ Group}{Scicos\ Block}


\Sdoc{ANIMXY\_f}{ Scicos 2D animated visualization block}\index{ANIMXY\_f}\label{ANIMXYf}

\Shead{DESCRIPTION}
This block realizes the visualization of the evolution of the two
regular input signals by drawing the second input as a function of
the first at instants of events on the event input port.
\Shead{DIALOG PARAMETERS}
\begin{scitem}
\item[{\verb?Curve colors?}]
: an integer. It is the color number ($<$0)
or marker type ($>$0) used to draw the evolution of the input port
signal. See %
\tt xset() %
\rm for color (dash type) definitions.
\item[{\verb?Line or mark size?}]
: an integer. 
\item[{\verb?Output window number?}]
: The number of graphic window used for the display. It is often good
to use high values to avoid conflict with palettes and Super Block
windows. If you have more than one scope, make sure they don't have
the same window numbers (unless superposition of the curves is
desired).
\item[{\verb?Output window position?}]
: a 2 vector specifying the coordinates of the upper left corner of
the graphic window. Answer [] for default window position.
\item[{\verb?Output window size?}]
: a 2 vector specifying the width and height  of 
the graphic window. Answer [] for default window dimensions.
\item[{\verb?Xmin, Xmax?}]
: Minimum and maximum values of the first input; used to set up the X-axis
of the plot in the graphics window.
\item[{\verb?Ymin, Ymax?}]
: Minimum and maximum values of the second input; used to set up the 
Y-axis of the plot in the graphics window.
\item[{\verb?Buffer size?}]
: an integer. The number of points visible at any time.
\end{scitem}% end Env
\Shead{REMARKS}
%
\tt Output window number%
\rm , %
\tt Output window size%
\rm , %
\tt Output window
position %
\rm are only taken into account at the initialisation time of the
simulation. 


\Seealso{SEE ALSO}
{\verb?SCOPE_f EVENTSCOPE_f SCOPXY_f?} \pageref{SCOPEfEVENTSCOPEfSCOPXYf}





%
% input file: ../../man//scicos/CLINDUMMY_f.man
%
\phead{CLINDUMMY\_f}{5}{Janvier\ 1996}{Scilab\ Group}{Scicos\ Block}


\Sdoc{CLINDUMMY\_f}{ Scicos dummy continuous system with state}\index{CLINDUMMY\_f}\label{CLINDUMMYf}

\Shead{DESCRIPTION}
This block should be placed in any block diagram that contains a
treshold block but no continuous system with state. The reason 
for that is that it is the ode solver that find zero crossing
surfaces.
\Seealso{SEE ALSO}
{\verb?ZCROSS_f?} \pageref{ZCROSSf}








%
% input file: ../../man//scicos/CLKIN_f.man
%
\phead{CLKIN\_f}{5}{Janvier\ 1996}{Scilab\ Group}{Scicos\ Block}


\Sdoc{CLKIN\_f}{ Scicos Super Block event input port}\index{CLKIN\_f}\label{CLKINf}

\Shead{DESCRIPTION}
This block must only be used inside Scicos Super Blocks to represent
an event input port. 
\par\noindent
In a Super Block, the event input ports must be numbered from 1 to the
number of event input ports.
\Shead{DIALOG PARAMETERS}
\begin{scitem}
\item[{\verb?Port number?}]
: an integer defining the port number.
\end{scitem}% end Env
\Seealso{SEE ALSO}
{\verb?IN_f OUT_f CLKOUT_f?} \pageref{INfOUTfCLKOUTf}
%
% input file: ../../man//scicos/CLKOUT_f.man
%
\phead{CLKOUT\_f}{5}{Janvier\ 1996}{Scilab\ Group}{Scicos\ Block}


\Sdoc{CLKOUT\_f}{ Scicos Super Block event output port}\index{CLKOUT\_f}\label{CLKOUTf}

\Shead{DESCRIPTION}
This block must only be used inside Scicos Super Blocks to represent
an event output port. 
\par\noindent
In a Super\_Block, the event output ports must be numbered from 1 to the
number of event output ports.
\Shead{DIALOG PARAMETERS}
\begin{scitem}
\item[{\verb?Port number?}]
: an integer giving the port number.
\end{scitem}% end Env
\Seealso{SEE ALSO}
{\verb?IN_f OUT_f CLKIN_f?} \pageref{INfOUTfCLKINf}

%
% input file: ../../man//scicos/CLKSOM_f.man
%
\phead{CLKSOM\_f}{5}{Janvier\ 1996}{Scilab\ Group}{Scicos\ Block}


\Sdoc{CLKSOM\_f}{ Scicos event addition block}\index{CLKSOM\_f}\label{CLKSOMf}

\Shead{DIALOG PARAMETERS}
None.
\Shead{DESCRIPTION}
This block is an event addition block with up to three inputs. The
output reproduces 
the events on all the input ports. Strictly speaking, CLKSOM is not a
Scicos block because it is discarded at the compilation phase. The
inputs and output of CLKSOM are synchronized (this is impossible for
other basic blocks in Scicos). 









%
% input file: ../../man//scicos/CLKSPLIT_f.man
%
\phead{CLKSPLIT\_f}{5}{Janvier\ 1996}{Scilab\ Group}{Scicos\ Block}


\Sdoc{CLKSPLIT\_f}{ Scicos event split block}\index{CLKSPLIT\_f}\label{CLKSPLITf}

\Shead{DIALOG PARAMETERS}
None.
\Shead{DESCRIPTION}
This block is an event split block with an input and two outputs. The
outputs reproduces the event the input port on each output ports. 
Strictly speaking, CLKSPLIT is not a Scicos block because it is
discarded at the compilation phase. This block is automatically
created when creating a new link issued from a link.

The inputs and output of CLKSPLIT are synchronized. 









%
% input file: ../../man//scicos/CLOCK_f.man
%
\phead{CLOCK\_f}{5}{Janvier\ 1996}{Scilab\ Group}{Scicos\ Block}


\Sdoc{CLOCK\_f}{ Scicos periodic event generator}\index{CLOCK\_f}\label{CLOCKf}

\Shead{DESCRIPTION}
This block is a Super Block constructed by feeding back the output
of an event delay block into its input event port. The unique
output of this block generates a regular train of events.
\Shead{DIALOG PARAMETERS}
\begin{scitem}
\item[{\verb?Period?}]
: scalar. One over the frequency of the clock. Period is the time that
separates two output events.
\item[{\verb?Period?}]
: scalar. One over the frequency of the clock. Period is the time that
separates two output events.
\item[{\verb?Init time?}]
: scalar. Starting date. if negative the clock never starts.

\end{scitem}% end Env
\Seealso{SEE ALSO}
{\verb?EVTDLY_f?} \pageref{EVTDLYf}







%
% input file: ../../man//scicos/CLR_f.man
%
\phead{CLR\_f}{5}{Janvier\ 1996}{Scilab\ Group}{Scicos\ Block}


\Sdoc{CLR\_f}{ Scicos continuous-time linear system (SISO transfer function)}\index{CLR\_f}\label{CLRf}

\Shead{DIALOG PARAMETERS}
\begin{scitem}
\item[{\verb?Numerator?}]
: a polynomial in s.
\item[{\verb?Denominator?}]
: a polynomial in s.
\end{scitem}% end Env
\Shead{DESCRIPTION}
This block realizes a SISO linear system represented by its rational
transfer function %
\tt Numerator/Denominator%
\rm .
The rational function must be proper.
\Seealso{SEE ALSO}
{\verb?CLSS_f INTEGRAL_f?} \pageref{CLSSfINTEGRALf}





%
% input file: ../../man//scicos/CLSS_f.man
%
\phead{CLSS\_f}{5}{Janvier\ 1996}{Scilab\ Group}{Scicos\ Block}


\Sdoc{CLSS\_f}{ Scicos continuous-time linear state-space system}\index{CLSS\_f}\label{CLSSf}

\Shead{DESCRIPTION}
This block realizes a continuous-time linear state-space system.
\begin{verbatim}
xdot=A*x+B*u
y   =C*x+D*u
\end{verbatim}
The system is defined by the (A,B,C,D) matrices and the initial
state x0. The dimensions must be compatible.
\Shead{DIALOG PARAMETERS}
\begin{scitem}
\item[{\verb?A?}]
: square matrix. The A matrix
\item[{\verb?B?}]
: the %
\tt B %
\rm matrix, may be %
\tt [] %
\rm if system has no input
\item[{\verb?C?}]
: the %
\tt C %
\rm matrix , may be %
\tt [] %
\rm if system has no output
\item[{\verb?D?}]
: the %
\tt D %
\rm matrix, may be %
\tt [] %
\rm if system has no %
\tt D %
\rm term.
\item[{\verb?x0?}]
: vector. The initial state of the system.
\end{scitem}% end Env
\Seealso{SEE ALSO}
{\verb?CLR_f INTEGRAL_f ?} \pageref{CLRfINTEGRALf}





%
% input file: ../../man//scicos/CONST_f.man
%
\phead{CONST\_f}{5}{Janvier\ 1996}{Scilab\ Group}{Scicos\ Block}


\Sdoc{CONST\_f}{ Scicos constant value(s) generator}\index{CONST\_f}\label{CONSTf}

\Shead{DIALOG PARAMETERS}
\begin{scitem}
\item[{\verb?constants?}]
: a real vector. The vector size gives the number of output
ports. The value %
\tt constants(i) %
\rm is assigned to the ith port.
Note that output ports are numbered from top to bottom.
\end{scitem}% end Env
\Shead{DESCRIPTION}
This block is a constant value(s) generator.




%
% input file: ../../man//scicos/COSBLK_f.man
%
\phead{COSBLK\_f}{5}{Janvier\ 1996}{Scilab\ Group}{Scicos\ Block}


\Sdoc{COSBLK\_f}{ Scicos cosine block}\index{COSBLK\_f}\label{COSBLKf}

\Shead{DIALOG PARAMETERS}
None.
\Shead{DESCRIPTION}
This block realizes vector cosine operation. %
\tt y(i)=cos(u(i))%
\rm .
The port input and output  port sizes are equal and determined by the
context.
\Seealso{SEE ALSO}
{\verb?SINBLK_f GENSIN_f?} \pageref{SINBLKfGENSINf}








%
% input file: ../../man//scicos/CURV_f.man
%
\phead{CURV\_f}{5}{Janvier\ 1996}{Scilab\ Group}{Scicos\ Block}


\Sdoc{CURV\_f}{ Scicos block, tabulated function of time}\index{CURV\_f}\label{CURVf}

\Shead{DIALOG PARAMETERS}
Tabulated function is entered using a graphics curv editor (see edit\_curv)

\Shead{DESCRIPTION}
This block defines a tabulated function of the time. Between mesh
points block performs a linear interpolation. Outside tabulation block
outputs last tabulated value.

User may define the tabulation of the function using a curv editor.

\Seealso{SEE ALSO}
{\verb?edit_curv?} \pageref{editcurv}



%
% input file: ../../man//scicos/DELAYV_f.man
%
\phead{DELAYV\_f}{5}{Janvier\ 1997}{Scilab\ Group}{Scicos\ Block}


\Sdoc{DELAYV\_f}{ Scicos varying delay block}\index{DELAYV\_f}\label{DELAYVf}

\Shead{DIALOG PARAMETERS}

\Shead{DESCRIPTION}
%
% input file: ../../man//scicos/DELAY_f.man
%
\phead{DELAY\_f}{5}{Janvier\ 1997}{Scilab\ Group}{Scicos\ Block}


\Sdoc{DELAY\_f}{ Scicos  delay block}\index{DELAY\_f}\label{DELAYf}

\Shead{DIALOG PARAMETERS}
\begin{scitem}
\item[{\verb?Discretisation time step?}]
: positive scalar, delay discretisation time step
\item[{\verb?Register initial state?}]
: register initial state vector. Dimension must be greater or equal
2
\end{scitem}% end Env
\Shead{DESCRIPTION}
This block implements as a discretised delay. It is in fact a Scicos
super block formed by a shift register and a clock.

value of the delay is given by  the discretisation time step multiplied by the
number of state of the register minus one 
\Seealso{SEE ALSO}
{\verb?DELAYV_f EVTDLY_f REGISTER_f?} \pageref{DELAYVfEVTDLYfREGISTERf}
%
% input file: ../../man//scicos/DEMUX_f.man
%
\phead{DEMUX\_f}{5}{Janvier\ 1997}{Scilab\ Group}{Scicos\ Block}


\Sdoc{DEMUX\_f}{ Scicos demupliplexer block}\index{DEMUX\_f}\label{DEMUXf}

\Shead{DIALOG PARAMETERS}
\begin{scitem}
\item[{\verb?number of output ports?}]
:
positive integer less or equal 8.
\end{scitem}% end Env
\Shead{DESCRIPTION}
Given a vector valued input this block splits inputs over all vector
valued outputs. So %
\tt u=[y1;y2....;yn]%
\rm , where %
\tt yi %
\rm are numbered
from top to bottom. Input and Output port sizes are determined by the
context.
\Seealso{SEE ALSO}
{\verb?MUX_f?} \pageref{MUXf}
%
% input file: ../../man//scicos/DLRADAPT_f.man
%
\phead{DLRADAPT\_f}{5}{Janvier\ 1997}{Scilab\ Group}{Scicos\ Block}


\Sdoc{DLRADAPT\_f}{ Scicos discrete-time linear adaptative system}\index{DLRADAPT\_f}\label{DLRADAPTf}

\Shead{DIALOG PARAMETERS}
\begin{scitem}
\item[{\verb?Vector of p mesh points?}]
: a vector which defines %
\tt u2 %
\rm mesh points.
Numerator roots
: a matrix, each line gives the roots of the numerator at the
corresponding mesh point.
\item[{\verb?Denominator roots?}]
: a matrix, each line gives the roots of the denominator at the
corresponding mesh point.
\item[{\verb?gain?}]
: a vector, each vector entry gives the transfer gain at the
corresponding mesh point.
\item[{\verb?past inputs?}]
: a vector of initial value of past %
\tt degree(Numerator) %
\rm inputs
\item[{\verb?past outputs?}]
: a vector of initial value of past %
\tt degree(Denominator) %
\rm outputs

\end{scitem}% end Env
\Shead{DESCRIPTION}
This block realizes a SISO linear system represented by its rational
transfer function whose numerator and denominator roots are tabulated
functions of the second block input. The rational function must be
proper.

Roots are interpolated linearly between mesh points.

\Seealso{SEE ALSO}
{\verb?DLSS_f DLR_f?} \pageref{DLSSfDLRf}






%
% input file: ../../man//scicos/DLR_f.man
%
\phead{DLR\_f}{5}{Janvier\ 1996}{Scilab\ Group}{Scicos\ Block}


\Sdoc{DLR\_f}{ Scicos discrete-time linear system (transfer function)}\index{DLR\_f}\label{DLRf}


\Shead{DIALOG PARAMETERS}
\begin{scitem}
\item[{\verb?Numerator?}]
: a polynomial in z.
\item[{\verb?Denomiator?}]
: a polynomial in z.
\end{scitem}% end Env
\Shead{DESCRIPTION}
This block realizes a SISO linear system represented by its rational
transfer function (in the symbolic variable z).
The rational function must be proper. If it is not proper, the
direct feedthrough make the output depend continuously on the input,
so if the input  
is not the output of a discrete block, it may not be piece-wise constant.
\Seealso{SEE ALSO}
{\verb?DLSS_f DLRADAPT_f?} \pageref{DLSSfDLRADAPTf}






%
% input file: ../../man//scicos/DLSS_f.man
%
\phead{DLSS\_f}{5}{Janvier\ 1996}{Scilab\ Group}{Scicos\ Block}


\Sdoc{DLSS\_f}{ Scicos discrete-time linear state-space system}\index{DLSS\_f}\label{DLSSf}

\Shead{DESCRIPTION}
This block realizes a discrete-time linear state-space system.
The system is defined by the (A,B,C,D) matrices and the initial
state x0. The dimensions must be compatible. At the arrival of
an input event on the unique input event port, the state is
updated. If D is not zero, the
the output depends directly on the input, so if the input 
is not the output of a discrete block, it may not be piecewise constant.
\Shead{DIALOG PARAMETERS}
\begin{scitem}
\item[{\verb?A?}]
: square matrix. The A matrix
\item[{\verb?B?}]
: the B matrix
\item[{\verb?C?}]
: the C matrix
\item[{\verb?x0?}]
: vector. The initial state of the system.
\end{scitem}% end Env
\Seealso{SEE ALSO}
{\verb?DLR_f INTEGRAL_f CLSS_f DLSS_f?} \pageref{DLRfINTEGRALfCLSSfDLSSf}





%
% input file: ../../man//scicos/EVENTSCOPE_f.man
%
\phead{EVENTSCOPE\_f}{5}{Janvier\ 1996}{Scilab\ Group}{Scicos\ Block}


\Sdoc{EVENTFSCOPE\_f}{ Scicos visualization block}\index{EVENTFSCOPE\_f}\label{EVENTFSCOPEf}

\Shead{DESCRIPTION}
This block allows the visualization of event signals
on its event input port(s).
\Shead{DIALOG PARAMETERS}
\begin{scitem}
\item[{\verb?Number of inputs?}]
: an integer, the number of event input ports.
\item[{\verb?Curve colors?}]
: a vector of integers. The i-th element is the color number ($<$0)
or dash type ($>$0) used to draw the evolution of the i-th 
input port signal. See %
\tt plot2d %
\rm for color (dash type) definitions.
\item[{\verb?Output window number?}]
: The number of graphic window used for the display. It is often good
to use high values to avoid conflict with palettes and Super Block
windows. If you have more than one scope, make sure they don't have
the same window numbers (unless superposition of the curves is desired).
\item[{\verb?Refresh period?}]
: Maximum value on the X-axis (time). The plot is redrawn when time
reaches a multiple of this value.
\end{scitem}% end Env
\Seealso{SEE ALSO}
{\verb?SCOPXY_f SCOPE_f ANIMXY_f?} \pageref{SCOPXYfSCOPEfANIMXYf}




%
% input file: ../../man//scicos/EVTDLY_f.man
%
\phead{EVTDLY\_f}{5}{Janvier\ 1996}{Scilab\ Group}{Scicos\ Block}


\Sdoc{EVTDLY\_f}{ Scicos event delay block}\index{EVTDLY\_f}\label{EVTDLYf}

\Shead{DESCRIPTION}
One event is generated dt after an event enters the unique input
event port. Block may also generate an initial output event.
\Shead{DIALOG PARAMETERS}
\begin{scitem}
\item[{\verb?Delay?}]
: scalar. Time delay between input and output event.
\item[{\verb?Auto-exec?}]
:  scalar. If %
\tt Auto-exec$>$=0 %
\rm and block has yet received  no input event
block generates an initial output event at date %
\tt Auto-exec%
\rm .
\end{scitem}% end Env
\Seealso{SEE ALSO}
{\verb?CLOCK_f?} \pageref{CLOCKf}







%
% input file: ../../man//scicos/EVTGEN_f.man
%
\phead{EVTGEN\_f}{5}{Janvier\ 1996}{Scilab\ Group}{Scicos\ Block}


\Sdoc{EVTGEN\_f}{ Scicos event firing block}\index{EVTGEN\_f}\label{EVTGENf}

\Shead{DESCRIPTION}
One event is generated on the unique output event port. The time
of this event is t if t is larger than equal to zero, if not, no
event is generated.
\Shead{DIALOG PARAMETERS}

\begin{scitem}
\item[{\verb?Event time?}]
: scalar. date of the initial event
\end{scitem}% end Env
\Seealso{SEE ALSO}
{\verb?CLOCK_f EVTDLY_f?} \pageref{CLOCKfEVTDLYf}







%
% input file: ../../man//scicos/EVTSCOPE_f.man
%
\phead{EVTSCOPE\_f}{5}{Janvier\ 1996}{Scilab\ Group}{Scicos\ Block}


\Sdoc{EVTSCOPE\_f}{ Scicos event visualization block}\index{EVTSCOPE\_f}\label{EVTSCOPEf}

\Shead{DESCRIPTION}
This block realizes the visualization of  the input event signals.
\Shead{DIALOG PARAMETERS}
\begin{scitem}
\item[{\verb?Number of event inputs?}]
: an integer giving the number of event input ports
colors
: a vector of integers. The i-th element is the color number ($<$0)
or dash type ($>$0) used to draw the evolution of the i-th 
input port signal. See %
\tt xset %
\rm for color (dash type) definitions.
\item[{\verb?Output window number?}]
: The number of graphic window used for the display. It is often good
to use high values to avoid conflict with palettes and Super Block
windows. If you have more than one scope, make sure they don't have
the same window numbers (unless superposition of the curves is
desired).
Output window position
: a 2 vector specifying the coordinates of the upper left corner of
the graphic window. Answer [] for default window position.
\item[{\verb?Output window size?}]
: a 2 vector specifying the width and height  of 
the graphic window. Answer [] for default window dimensions.
\item[{\verb?Refresh period?}]
: Maximum value on the X-axis (time). The plot is redrawn when time
reaches a multiple of this value.
\end{scitem}% end Env
\Shead{REMARKS}
%
\tt Output window number%
\rm , %
\tt Output window size%
\rm , %
\tt Output window
position %
\rm are only taken into account at the initialisation time of the
simulation. 
\Seealso{SEE ALSO}
{\verb?SCOPXY_f EVENTSCOPE_f ANIMXY_f?} \pageref{SCOPXYfEVENTSCOPEfANIMXYf}




%
% input file: ../../man//scicos/EXPBLK_f.man
%
\phead{EXPBLK\_f}{5}{Janvier\ 1997}{Scilab\ Group}{Scicos\ Block}


\Sdoc{EXPBLK\_f}{ Scicos a\^{}u block}\index{EXPBLK\_f}\label{EXPBLKf}

\Shead{DIALOG PARAMETERS}
\begin{scitem}
\item[{\verb?a :?}]
real positive scalar
\end{scitem}% end Env
\Shead{DESCRIPTION}
This block realizes %
\tt y(i)=a\^{}u(i)%
\rm . The input and output port sizes
are determined by the context.






%
% input file: ../../man//scicos/GAIN_f.man
%
\phead{GAIN\_f}{5}{Janvier\ 1996}{Scilab\ Group}{Scicos\ Block}


\Sdoc{GAIN\_f}{ Scicos gain block}\index{GAIN\_f}\label{GAINf}

\Shead{DIALOG PARAMETERS}
\begin{scitem}
\item[{\verb?Gain?}]
: a real matrix. 
\end{scitem}% end Env
\Shead{DESCRIPTION}
This block is a gain block. The output is the Gain times the
regular input (vector). The dimensions of Gain determines the
input (number of columns) and output (number of rows) port sizes.





%
% input file: ../../man//scicos/GENERAL_f.man
%
\phead{GENERAL\_f}{5}{Janvier\ 1996}{Scilab\ Group}{Scicos\ Block}


\Sdoc{GENERAL\_f}{ Scicos general zero crossing detector}\index{GENERAL\_f}\label{GENERALf}

\Shead{DESCRIPTION}
Depending on the sign (just before the crossing) of the inputs and
the input numbers of the inputs that have crossed zero, an event is
programmed (or not) with a given delay, for each output. The number
of combinations grows so fast that this becomes unusable for blocks
having more than 2 or 3 inputs. For the moment this block is not
documented.
\Shead{DIALOG PARAMETERS}
\begin{scitem}
\item[{\verb?Size of regular input?}]
: integer.
\item[{\verb?Number of output events?}]
: integer.
\item[{\verb?the routing matrix?}]
: matrix. number of rows is the number of output events. The columns
correspond to each possible combination of signs and zero crossings
of the inputs. The entries of the matrix give the delay for generating
the output event ($<$0 no event is generated).
\end{scitem}% end Env
\Seealso{SEE ALSO}
{\verb?NEGTOPOS_f POSTONEG_f ZCROSS_f ?} \pageref{NEGTOPOSfPOSTONEGfZCROSSf}





%
% input file: ../../man//scicos/GENSIN_f.man
%
\phead{GENSIN\_f}{5}{Janvier\ 1996}{Scilab\ Group}{Scicos\ Block}


\Sdoc{GENSIN\_f}{ Scicos sinusoid generator}\index{GENSIN\_f}\label{GENSINf}

\Shead{DESCRIPTION}
This block is a sine wave generator: M*sin(F*t+P)
\Shead{DIALOG PARAMETERS}
\begin{scitem}
\item[{\verb?Magnitude?}]
: a scalar. The magnitude M.
\item[{\verb?Frequency?}]
: a scalar. The frequency F.
\item[{\verb?Phase?}]
: a scalar. The phase P.
\end{scitem}% end Env
\Seealso{SEE ALSO}
{\verb?GENSQR_f RAND_f SAWTOOTH_f?} \pageref{GENSQRfRANDfSAWTOOTHf}






%
% input file: ../../man//scicos/GENSQR_f.man
%
\phead{GENSQR\_f}{5}{Janvier\ 1996}{Scilab\ Group}{Scicos\ Block}


\Sdoc{GENSQR\_f}{ Scicos square wave generator}\index{GENSQR\_f}\label{GENSQRf}

\Shead{DESCRIPTION}
This block is a square wave generator: output takes values
0 and M. Everytime an event is received on the input event port,
the output switches from 0 to M, or M to 0.
\Shead{DIALOG PARAMETERS}
\begin{scitem}
\item[{\verb?Amplitude?}]
: a scalar M.
\end{scitem}% end Env
\Seealso{SEE ALSO}
{\verb?GENSIN_f SAWTOOTH_f RAND_f?} \pageref{GENSINfSAWTOOTHfRANDf}






%
% input file: ../../man//scicos/HALT_f.man
%
\phead{STOP\_f}{5}{Janvier\ 1996}{Scilab\ Group}{Scicos\ Block}


\Sdoc{CLOCK\_f}{ Scicos Stop block}\index{CLOCK\_f}\label{CLOCKf}

\Shead{DIALOG PARAMETERS}
\begin{scitem}
\item[{\verb?State on halt?}]
: scalar. A value to be placed in the state of the block. For
debugging purposes this allows to distinguish between different
halts.
\end{scitem}% end Env
\Shead{DESCRIPTION}
This block has a unique input event port. Upon the arrival of an
event,
the simulation is stopped and the main Scicos window is activated.
Simulation can be restarted or continued (Run button).








%
% input file: ../../man//scicos/IFTHEL_f.man
%
\phead{IFTHEL\_f}{5}{Janvier\ 1996}{Scilab\ Group}{Scicos\ Block}


\Sdoc{IFTHEL\_f}{ Scicos if then else block}\index{IFTHEL\_f}\label{IFTHELf}

\Shead{DIALOG PARAMETERS}
None.
\Shead{DESCRIPTION}
One event is generated on one of the output event ports when an
input event arrives. Depending on the sign of the regular input, 
the event is generated on the first or second output.








%
% input file: ../../man//scicos/INTEGRAL_f.man
%
\phead{INTEGRAL\_f}{5}{Janvier\ 1996}{Scilab\ Group}{Scicos\ Block}


\Sdoc{INTEGRAL\_f}{ Scicos simple integrator}\index{INTEGRAL\_f}\label{INTEGRALf}

\Shead{DESCRIPTION}
This block is an integrator. The output is the integral of the input.
\Shead{DIALOG PARAMETERS}
\begin{scitem}
\item[{\verb?Initial state?}]
: a scalar. The initial condition of the integrator.
\end{scitem}% end Env
\Seealso{SEE ALSO}
{\verb?CLSS_f CLR_f?} \pageref{CLSSfCLRf}







%
% input file: ../../man//scicos/INVBLK_f.man
%
\phead{INVBLK\_f}{5}{Janvier\ 1997}{Scilab\ Group}{Scicos\ Block}


\Sdoc{INVBLK\_f}{ Scicos inversion block}\index{INVBLK\_f}\label{INVBLKf}

\Shead{DIALOG PARAMETERS}
None.
\Shead{DESCRIPTION}
This block computes %
\tt y(i)=1/u(i)%
\rm . The input (output) size is
determined by the context
%
% input file: ../../man//scicos/IN_f.man
%
\phead{IN\_f}{5}{Janvier\ 1996}{Scilab\ Group}{Scicos\ Block}


\Sdoc{IN\_f}{ Scicos Super Block regular input port}\index{IN\_f}\label{INf}

\Shead{DESCRIPTION}
This block must only be used inside Scicos Super Blocks to represent
a regular input port. The input size is determined by the context.
\par\noindent
In a Super Block, regular input ports must be numbered from 1 to the
number of regular input ports.
\Shead{DIALOG PARAMETERS}
\begin{scitem}
\item[{\verb?Port number?}]
: an integer giving the port number.
\end{scitem}% end Env
\Seealso{SEE ALSO}
{\verb?CLKIN_f OUT_f CLKOUT_f?} \pageref{CLKINfOUTfCLKOUTf}
%
% input file: ../../man//scicos/LOGBLK_f.man
%
\phead{LOGBLK\_f}{5}{Janvier\ 1997}{Scilab\ Group}{Scicos\ Block}


\Sdoc{LOGBLK\_f}{ Scicos logarithm  block}\index{LOGBLK\_f}\label{LOGBLKf}

\Shead{DIALOG PARAMETERS}
\begin{scitem}
\item[{\verb?a : real scalar greater than 1?}]
\end{scitem}% end Env
\Shead{DESCRIPTION}
This block realizes %
\tt y(i)=log(u(i))/log(a)%
\rm . The input and output port sizes
are determined by the context.


%
% input file: ../../man//scicos/LOOKUP_f.man
%
\phead{LOOKUP\_f}{5}{Janvier\ 1996}{Scilab\ Group}{Scicos\ Block}


\Sdoc{LOOKUP\_f}{ Scicos Lookup table with graphical editor}\index{LOOKUP\_f}\label{LOOKUPf}

\Shead{DESCRIPTION}
This block realizes a non-linear function defined using a graphical
editor. 





%
% input file: ../../man//scicos/MAX_f.man
%
\phead{MAX\_f}{5}{Janvier\ 1997}{Scilab\ Group}{Scicos\ Block}


\Sdoc{MAX\_f}{ Scicos max block}\index{MAX\_f}\label{MAXf}

\Shead{DIALOG PARAMETERS}
None.
\Shead{DESCRIPTION}
The block outputs the maximum of the input vector:
%
\tt y=max(u1,...un)%
\rm .
The input vector size is determined by the context.

%
% input file: ../../man//scicos/MCLOCK_f.man
%
\phead{MCLOCK\_f}{5}{Janvier\ 1996}{Scilab\ Group}{Scicos\ Block}


\Sdoc{MCLOCK\_f}{ Scicos 2 frequency event clock}\index{MCLOCK\_f}\label{MCLOCKf}

\Shead{DESCRIPTION}
This block is a Super Block constructed by feeding back the outputs
of an MFCLCK block into its input event port. The two
outputs of this block generate regular train of events, the frequency
of the first input being equal to that of the second output diveided
by an integer n. The two outputs are synchronized (this is impossible
for standard blocks; this is a Super Block).
\Shead{DIALOG PARAMETERS}
\begin{scitem}
\item[{\verb?Basic period?}]
: scalar. equals 1/f, f being the highest frequency.
\item[{\verb?n ?}]
: an intger $>$1. the frequency of the first output event is f/n.
\end{scitem}% end Env
\Seealso{SEE ALSO}
{\verb?MFCLCK_f CLOCK_f?} \pageref{MFCLCKfCLOCKf}







%
% input file: ../../man//scicos/MFCLCK_f.man
%
\phead{MFCLCK\_f}{5}{Janvier\ 1996}{Scilab\ Group}{Scicos\ Block}


\Sdoc{MFCLCK\_f}{ Scicos basic block for frequency division of event clock}\index{MFCLCK\_f}\label{MFCLCKf}

\Shead{DESCRIPTION}
This block is used in the Super Block MCLOCK. The input event is
directed once evry n times to output 1 and the rest of the time
to output 2. There is a delay of "Basic period" in the transmission
of the event. If this period $>$0 then the second output is initially
fired. It is not if this period=0. In the latter case, the input
is driven by an event clock and in the former case, feedback can
be used.
\Shead{DIALOG PARAMETERS}
\begin{scitem}
\item[{\verb?Basic period?}]
: positive scalar.
\item[{\verb?n ?}]
: an integer greater than 1. 
\end{scitem}% end Env
\Seealso{SEE ALSO}
{\verb?MCLOCK_f CLOCK_f?} \pageref{MCLOCKfCLOCKf}







%
% input file: ../../man//scicos/MIN_f.man
%
\phead{MIN\_f}{5}{Janvier\ 1997}{Scilab\ Group}{Scicos\ Block}


\Sdoc{MIN\_f}{ Scicos min block}\index{MIN\_f}\label{MINf}

\Shead{DIALOG PARAMETERS}
None.
\Shead{DESCRIPTION}
The block outputs the minimum of the input vector:
%
\tt y=min(u1,...un)%
\rm .
The input vector size is determined by the context.

%
% input file: ../../man//scicos/MUX_f.man
%
\phead{MUX\_f}{5}{Janvier\ 1997}{Scilab\ Group}{Scicos\ Block}


\Sdoc{MUX\_f}{ Scicos mupliplexer block}\index{MUX\_f}\label{MUXf}

\Shead{DIALOG PARAMETERS}
\begin{scitem}
\item[{\verb?number of output ports?}]
:
integer greater or equal 1 and less than 8
\end{scitem}% end Env
\Shead{DESCRIPTION}
Given %
\tt n %
\rm vector valued inputs this block merges inputs in an single
output vector. So %
\tt y=[u1;u2....;un]%
\rm , where %
\tt ui %
\rm are numbered
from top to bottom. Input and Output port sizes are determined by the
context.
\Seealso{SEE ALSO}
{\verb?MUX_f?} \pageref{MUXf}
%
% input file: ../../man//scicos/NEGTOPOS_f.man
%
\phead{NEGTOPOS\_f}{5}{Janvier\ 1996}{Scilab\ Group}{Scicos\ Block}


\Sdoc{NEGTOPOS\_f}{ Scicos negative to positive detector}\index{NEGTOPOS\_f}\label{NEGTOPOSf}

\Shead{DESCRIPTION}
An output event is generated when the unique input crosses zero
with a positive derivative.
\Seealso{SEE ALSO}
{\verb?POSTONEG_f ZCROSS_f GENERAL_f?} \pageref{POSTONEGfZCROSSfGENERALf}





%
% input file: ../../man//scicos/OUT_f.man
%
\phead{OUT\_f}{5}{Janvier\ 1996}{Scilab\ Group}{Scicos\ Block}


\Sdoc{OUT\_f}{ Scicos Super Block regular output port}\index{OUT\_f}\label{OUTf}

\Shead{DIALOG PARAMETERS}
\begin{scitem}
\item[{\verb?Port number?}]
: an integer giving the port number.
\end{scitem}% end Env
\Shead{DESCRIPTION}
This block must only be used inside Scicos Super Blocks to represent
a regular output port. 
\par\noindent
In a Super Block, regular output ports must be numbered from 1 to the
number of regular output ports.

size of the output is determined by the context.

\Seealso{SEE ALSO}
{\verb?CLKIN_f IN_f CLKOUT_f?} \pageref{CLKINfINfCLKOUTf}
%
% input file: ../../man//scicos/POSTONEG_f.man
%
\phead{POSTONEG\_f}{5}{Janvier\ 1996}{Scilab\ Group}{Scicos\ Block}


\Sdoc{POSTONEG\_f}{ Scicos positive to negative detector}\index{POSTONEG\_f}\label{POSTONEGf}

\Shead{DESCRIPTION}
An output event is generated when the unique input crosses zero
with a negative derivative.
\Seealso{SEE ALSO}
{\verb?NEGTOPOS_f ZCROSS_f GENERAL_f?} \pageref{NEGTOPOSfZCROSSfGENERALf}





%
% input file: ../../man//scicos/POWBLK_f.man
%
\phead{POWBLK\_f}{5}{Janvier\ 1997}{Scilab\ Group}{Scicos\ Block}


\Sdoc{POWBLK\_f}{ Scicos u\^{}a block}\index{POWBLK\_f}\label{POWBLKf}

\Shead{DIALOG PARAMETERS}
\begin{scitem}
\item[{\verb?a :?}]
real scalar
\end{scitem}% end Env
\Shead{DESCRIPTION}
This block realizes %
\tt y(i)=u(i)\^{}a%
\rm . The input and output port sizes
are determined by the context.
%
% input file: ../../man//scicos/PROD_f.man
%
\phead{PROD\_f}{5}{Janvier\ 1996}{Scilab\ Group}{Scicos\ Block}


\Sdoc{PROD\_f}{ Scicos product block}\index{PROD\_f}\label{PRODf}

\Shead{DESCRIPTION}
The output is the product of the two scalar inputs.






%
% input file: ../../man//scicos/QUANT_f.man
%
\phead{QUANT\_f}{5}{Janvier\ 1997}{Scilab\ Group}{Scicos\ Block}


\Sdoc{QUANT\_f}{ Scicos Quantization block}\index{QUANT\_f}\label{QUANTf}

\Shead{DIALOG PARAMETERS}
\begin{scitem}
\item[{\verb?Step?}]
: scalar, Quantization step
\item[{\verb?Quantization method?}]
: scalar with possible values 1,2,3 or 4
\begin{scitem}
\item[{\verb?1?}]
: Round method
\item[{\verb?2?}]
: Truncation method
\item[{\verb?3?}]
:
Floor method
\item[{\verb?4?}]
:
Ceil method
\end{scitem}
\end{scitem}% end Env
\Shead{DESCRIPTION}
This block outputs the quantization of the input according to a choice
of methods
\par\noindent
for %
\tt Round %
\rm method 

%
\tt y(i)=Step*(int(u(i)/Step+0.5)-0.5) %
\rm if %
\tt u(i)$<$0%
\rm .

%
\tt y(i)=Step*(int(u(i)/Step-0.5)+0.5)%
\rm . if %
\tt u(i)$>$=0%
\rm .

\par\noindent
For truncation method

%
\tt y(i)=Step*(int(u(i)/Step+0.5)) %
\rm if %
\tt u(i)$<$0%
\rm .

%
\tt y(i)=Step*(int(u(i)/Step-0.5)) %
\rm if %
\tt u(i)$>$=0%
\rm .

\par\noindent
For floor method

%
\tt y(i)=Step*(int(u(i)/Step+0.5)) %
\rm .

\par\noindent
For ceil method

%
\tt y(i)=Step*(int(u(i)/Step-0.5)) %
\rm .



%
% input file: ../../man//scicos/RAND_f.man
%
\phead{RAND\_f}{5}{Janvier\ 1996}{Scilab\ Group}{Scicos\ Block}


\Sdoc{RAND\_f}{ Scicos random wave generator}\index{RAND\_f}\label{RANDf}

\Shead{DESCRIPTION}
This block is a random wave generator: output takes piecewise
constant random values. Everytime an event is received on the 
input event port, the output takes a new independent random value.
\Shead{DIALOG PARAMETERS}
\begin{scitem}
\item[{\verb?flag?}]
: 0 or 1. 0 for uniform distribution on [A,A+B] and 1 for normal
distribution N(A,B*B).
\item[{\verb?A?}]
: scalar
\item[{\verb?B?}]
: scalar
\end{scitem}% end Env
\Seealso{SEE ALSO}
{\verb?GENSIN_f SAWTOOTH_f GENSQR_f?} \pageref{GENSINfSAWTOOTHfGENSQRf}







%
% input file: ../../man//scicos/REGISTER_f.man
%
\phead{REGISTER\_f}{5}{Janvier\ 1996}{Scilab\ Group}{Scicos\ Block}


\Sdoc{REGISTER\_f}{ Scicos shift register block}\index{REGISTER\_f}\label{REGISTERf}

\Shead{DESCRIPTION}
This block realizes a shift register. At every input event, the
register is shifted one step.
\Shead{DIALOG PARAMETERS}
\begin{scitem}
\item[{\verb?Initial condition?}]
: a column vector. It contains the initial state of the register.
\end{scitem}% end Env
\Seealso{SEE ALSO}
{\verb?DELAY_f DELAYV_f EVTDLY_f?} \pageref{DELAYfDELAYVfEVTDLYf}







%
% input file: ../../man//scicos/RFILE_f.man
%
\phead{RFILE\_f}{5}{Janvier\ 1996}{Scilab\ Group}{Scicos\ Block}


\Sdoc{RFILE\_f}{ Scicos "read from file" block}\index{RFILE\_f}\label{RFILEf}

\Shead{DIALOG PARAMETERS}
\begin{scitem}
\item[{\verb?Time record Selection?}]
: an empty matrix or a positive integer. If an integer %
\tt i %
\rm is given
the  %
\tt i%
\rm th element of the read record is assumed to be the date of
the output event. If empty no output event exists.
\item[{\verb?Output record selection?}]
: a vector of positive integer. %
\tt [k1,..,kn]%
\rm ,The %
\tt ki%
\rm th
element of  the read record gives the value of %
\tt i%
\rm th output.
\item[{\verb?size of output?}]
: a scalar. This fixes the number of "value" readed.
\item[{\verb?Input file name?}]
: a character string defining the path of the file 
\item[{\verb?Input Format?}]
: a character string defining the Fortran format to use or nothing for
an unformatted (binary) write
\item[{\verb?Buffer size?}]
: To improve efficiency it is possible to buffer the input data. read
on the file is only done after each %
\tt Buffer size %
\rm call to the block.
\end{scitem}% end Env
\Shead{DESCRIPTION}
This block allows user to read  datas in a file, in formatted or binary
mode. %
\tt Output record selection %
\rm and %
\tt Time record Selection %
\rm allows the user to select data among file records.

Each call to the block advance one record in the file. 

\Seealso{SEE ALSO}
{\verb?WFILE_f?} \pageref{WFILEf}









%
% input file: ../../man//scicos/SAMPLEHOLD_f.man
%
\phead{SAMPLEHOLD\_f}{5}{Janvier\ 1997}{Scilab\ Group}{Scicos\ Block}


\Sdoc{SAMPLEHOLD\_f}{ Scicos Sample and hold block}\index{SAMPLEHOLD\_f}\label{SAMPLEHOLDf}

\Shead{DIALOG PARAMETERS}
None.
\Shead{DESCRIPTION}
Each time an input event arises block pick the input value an hold it
on the output up to next input event.
For periodic Sample and hold event input must be generated by a %
\tt Clock%
\rm .
\Seealso{SEE ALSO}
{\verb?DELAY_f CLOCK_f?} \pageref{DELAYfCLOCKf}

%
% input file: ../../man//scicos/SAT_f.man
%
\phead{SAT\_f}{5}{Janvier\ 1996}{Scilab\ Group}{Scicos\ Block}


\Sdoc{SAT\_f}{ Scicos Saturation block}\index{SAT\_f}\label{SATf}

\Shead{DESCRIPTION}
This block realizes the non-linear function: saturation.
\Shead{DIALOG PARAMETERS}
\begin{scitem}
\item[{\verb?Min?}]
: a scalar. Lower saturation bound
\item[{\verb?Max?}]
: a scalar. Upper saturation bound
\item[{\verb?Slope?}]
: a scalar. The slope of the line going through the origin 
and describing the behavior of the function around zero.
\end{scitem}% end Env
\Seealso{SEE ALSO}
{\verb?LOOKUP_f?} \pageref{LOOKUPf}





%
% input file: ../../man//scicos/SAWTOOTH_f.man
%
\phead{SAWTOOTH\_f}{5}{Janvier\ 1996}{Scilab\ Group}{Scicos\ Block}


\Sdoc{SAWTOOTH\_f}{ Scicos sawtooth wave generator}\index{SAWTOOTH\_f}\label{SAWTOOTHf}

\Shead{DESCRIPTION}
This block is a sawtooth wave generator: output is (t-t\_i)
from ti to t\_(i+1) where t\_i and t\_(i+1) denote the times of
two successive input events.
\Shead{DIALOG PARAMETERS}
None.
\Seealso{SEE ALSO}
{\verb?GENSIN_f GENSQR_f RAND_f?} \pageref{GENSINfGENSQRfRANDf}






%
% input file: ../../man//scicos/SCOPE_f.man
%
\phead{SCOPE\_f}{5}{Janvier\ 1996}{Scilab\ Group}{Scicos\ Block}


\Sdoc{SCOPE\_f}{ Scicos visualization block}\index{SCOPE\_f}\label{SCOPEf}

\Shead{DESCRIPTION}
This block realizes the visualization of the evolution of the signals
on the standard input port(s) at instants of events on the event
input port.
\Shead{DIALOG PARAMETERS}
\begin{scitem}
\item[{\verb?Number of inputs?}]
: an integer, the number of input ports
\item[{\verb?Curve colors?}]
: a vector of integers. The i-th element is the color number ($<$0)
or dash type ($>$0) used to draw the evolution of the i-th 
input port signal. See %
\tt plot2d %
\rm for color (dash type) definitions.
\item[{\verb?Output window number?}]
: The number of graphic window used for the display. It is often good
to use high values to avoid conflict with palettes and Super Block
windows. If you have more than one scope, make sure they don't have
the same window numbers (unless superposition of the curves is
desired).
\item[{\verb?Output window position?}]
: a 2 vector specifying the coordinates of the upper left corner of
the graphic window. Answer [] for default window position.
\item[{\verb?Output window size?}]
: a 2 vector specifying the width and height  of 
the graphic window. Answer [] for default window dimensions.
\item[{\verb?Ymin, Ymax?}]
: Minimum and maximum values of the input; used to set up the Y-axis
of the plot in the graphics window.
\item[{\verb?Refresh period?}]
: Maximum value on the X-axis (time). The plot is redrawn when time
reaches a multiple of this value.
\item[{\verb?Buffer size?}]
: To improve efficiency it is possible to buffer the input data. The
drawing is only done after each %
\tt Buffer size %
\rm call to the block.
\end{scitem}% end Env
\Shead{REMARKS}
%
\tt Output window number%
\rm , %
\tt Output window size%
\rm , %
\tt Output window
position %
\rm are only taken into account at the initialisation time of the
simulation. 

\Seealso{SEE ALSO}
{\verb?SCOPXY_f EVENTSCOPE_f ANIMXY_f?} \pageref{SCOPXYfEVENTSCOPEfANIMXYf}




%
% input file: ../../man//scicos/SCOPXY_f.man
%
\phead{SCOPXY\_f}{5}{Janvier\ 1996}{Scilab\ Group}{Scicos\ Block}


\Sdoc{SCOPXY\_f}{ Scicos visualization block}\index{SCOPXY\_f}\label{SCOPXYf}

\Shead{DESCRIPTION}
This block realizes the visualization of the evolution of the two
regular input signals by drawing the second input as a function of
the first at instants of events on the event input port.
\Shead{DIALOG PARAMETERS}
\begin{scitem}
\item[{\verb?Curve colors?}]
: an integer. It is the color number ($<$0)
or dash type ($>$0) used to draw the evolution of the input port
signal. See %
\tt plot2d %
\rm for color (dash type) definitions.
\item[{\verb?Line or mark size?}]
: an integer. 
\item[{\verb?Output window number?}]
: The number of graphic window used for the display. It is often good
to use high values to avoid conflict with palettes and Super Block
windows. If you have more than one scope, make sure they don't have
the same window numbers (unless superposition of the curves is
desired).
\item[{\verb?Output window position?}]
: a 2 vector specifying the coordinates of the upper left corner of
the graphic window. Answer [] for default window position.
\item[{\verb?Output window size?}]
: a 2 vector specifying the width and height  of 
the graphic window. Answer [] for default window dimensions.
\item[{\verb?Xmin, Xmax?}]
: Minimum and maximum values of the first input; used to set up the X-axis
of the plot in the graphics window.
\item[{\verb?Ymin, Ymax?}]
: Minimum and maximum values of the second input; used to set up the 
Y-axis of the plot in the graphics window.
\item[{\verb?Buffer size?}]
: To improve efficiency it is possible to buffer the input data. The
drawing is only done after each %
\tt Buffer size %
\rm call to the block.
\end{scitem}% end Env
\Shead{REMARKS}
%
\tt Output window number%
\rm , %
\tt Output window size%
\rm , %
\tt Output window
position %
\rm are only taken into account at the initialisation time of the
simulation. 

\Seealso{SEE ALSO}
{\verb?SCOPE_f EVENTSCOPE_f ANIMXY_f?} \pageref{SCOPEfEVENTSCOPEfANIMXYf}





%
% input file: ../../man//scicos/SELECT_f.man
%
\phead{SELECT\_f}{5}{Janvier\ 1996}{Scilab\ Group}{Scicos\ Block}


\Sdoc{SELECT\_f}{ Scicos select block}\index{SELECT\_f}\label{SELECTf}

\Shead{DIALOG PARAMETERS}
\begin{scitem}
\item[{\verb?number of inputs?}]
: a scalar. Number of regular and event inputs.
\item[{\verb?initial connected input?}]
: a integer. It must be between 1 and the number of inputs.
\end{scitem}% end Env
\Shead{DESCRIPTION}
The blocks routes one of the regular inputs to the unique regular
output. the choice of which input is to be routed is done, initially
by the "initial connected input" parameter. Then, everytime an input
event arrives on the i-th input event port, the i-th regular input
port is routed to the regular output.









%
% input file: ../../man//scicos/SINBLK_f.man
%
\phead{SINBLK\_f}{5}{Janvier\ 1996}{Scilab\ Group}{Scicos\ Block}


\Sdoc{SINBLK\_f}{ Scicos sine block}\index{SINBLK\_f}\label{SINBLKf}

\Shead{DIALOG PARAMETERS}
None.
\Shead{DESCRIPTION}
This block realizes vector cosine operation. %
\tt y(i)=sin(u(i))%
\rm .
The port input and output  port sizes are equal and determined by the
context.









%
% input file: ../../man//scicos/SOM_f.man
%
\phead{SOM\_f}{5}{Janvier\ 1996}{Scilab\ Group}{Scicos\ Block}


\Sdoc{SOM\_f}{ Scicos addition block}\index{SOM\_f}\label{SOMf}

\Shead{DIALOG PARAMETERS}
\begin{scitem}
\item[{\verb?Number of inputs?}]
: a scalar. Safe to leave at three even if you connect less inputs.
More than 3 is not allowed. Use Gain for more inputs.
\item[{\verb?Input signs?}]
: a (1x3) vector of +1 and -1. If -1, the corresponding input is
multipled by -1 before addition.
\end{scitem}% end Env
\Shead{DESCRIPTION}
This block is a sum. The output is the elementwise sum of the inputs.

Input ports are located at up, left or right and down position.
You must specify 3 gain numbers but if only two links are 
connected only the first values are used, port are numbered  anti-clock wise.
\Seealso{SEE ALSO}
{\verb?GAIN_f?} \pageref{GAINf}







%
% input file: ../../man//scicos/SPLIT_f.man
%
\phead{SPLIT\_f}{5}{Janvier\ 1996}{Scilab\ Group}{Scicos\ Block}


\Sdoc{SPLIT\_f}{ Scicos regular split block}\index{SPLIT\_f}\label{SPLITf}

\Shead{DIALOG PARAMETERS}
None.
\Shead{DESCRIPTION}
This block is a regular split block with an input and two outputs. The
outputs reproduces the input port on each output ports. 
Strictly speaking, SPLIT is not a Scicos block because it is
discarded at the compilation phase. This block is automatically
created when creating a new link issued from a link.

Port sizes are determined by the context.


%
% input file: ../../man//scicos/STOP_f.man
%
\phead{STOP\_f}{5}{Janvier\ 1996}{Scilab\ Group}{Scicos\ Block}


\Sdoc{CLOCK\_f}{ Scicos Stop block}\index{CLOCK\_f}\label{CLOCKf}

\Shead{DIALOG PARAMETERS}
\begin{scitem}
\item[{\verb?Block label?}]
: string. Label to be placed under the block.
\item[{\verb?State on halt?}]
: scalar. A value to be placed in the state of the block. For
debugging purposes this allows to distinguish between different
halts.
\end{scitem}% end Env
\Shead{DESCRIPTION}
This block has a unique input event port. Upon the arrival of an
event,
the simulation is stoped and the main Scicos window is activated.
Simulation can be restarted or continued (Run button).








%
% input file: ../../man//scicos/SUPER_f.man
%
\phead{SUPER\_f}{5}{Janvier\ 1996}{Scilab\ Group}{Scicos\ Block}


\Sdoc{SUPER\_f}{ Scicos Super block}\index{SUPER\_f}\label{SUPERf}

\Shead{DESCRIPTION}
This block opens up a new Scicos window for editing a new block
diagram. This diagram describes the internal functions of the
super block. 
\par\noindent
Super block  inputs and outputs (regular or event) are designed by
special (input or output)  blocks.
\par\noindent
Regular input blocks must be numbered from 1 to the number of regular
input ports. Regular input ports of the super block are numbered from
the top of the block shape to the bottom.
\par\noindent

Regular output blocks must be numbered from 1 to the number of regular
output ports. Regular output ports of the super block are numbered
from the top of the block shape to the bottom.
\par\noindent

Event input blocks must be numbered from 1 to the number of event
input ports. Event input ports of the super block are numbered from
the left of the block shape to the right.
\par\noindent
Event output blocks must be numbered from 1 to the number of event
output ports. Event output ports of the super block are numbered from
the left of the block shape to the right.
\Seealso{SEE ALSO}
{\verb?CLKIN_f OUT_f CLKOUT_f IN_f?} \pageref{CLKINfOUTfCLKOUTfINf}








%
% input file: ../../man//scicos/TANBLK_f.man
%
\phead{TANBLK\_f}{5}{Janvier\ 1996}{Scilab\ Group}{Scicos\ Block}


\Sdoc{TANBLK\_f}{ Scicos tan block}\index{TANBLK\_f}\label{TANBLKf}

\Shead{DIALOG PARAMETERS}
None.
\Shead{DESCRIPTION}
This block realizes vector tangent operation.
input (output) port size are determined by the context









%
% input file: ../../man//scicos/TCLSS_f.man
%
\phead{TCLSS\_f}{5}{Janvier\ 1996}{Scilab\ Group}{Scicos\ Block}


\Sdoc{TCLSS\_f}{ Scicos jump continuous-time linear state-space system}\index{TCLSS\_f}\label{TCLSSf}

\Shead{DESCRIPTION}
This block realizes a continuous-time linear state-space system
with the possibility of jumps in the state. The number of inputs
to this block is two. The first input is the regular input of the
linear system, the second carries the new value of the state which
is copied into the state when an event arrives at the unique event 
input port of this block. That means the state of the system jumps 
to the value present on the second input (of size equal to that of
the state).
The system is defined by the (A,B,C,D) matrices and the initial
state x0. The dimensions must be compatible. The sizes of inputs
and outputs are adjusted automatically.
\Shead{DIALOG PARAMETERS}
\begin{scitem}
\item[{\verb?A?}]
: square matrix. The A matrix
\item[{\verb?B?}]
: the B matrix
\item[{\verb?C?}]
: the C matrix
\item[{\verb?D?}]
: the D matrix
\item[{\verb?x0?}]
: vector. The initial state of the system.
\end{scitem}% end Env
\Seealso{SEE ALSO}
{\verb?CLSS_f CLR_f ?} \pageref{CLSSfCLRf}





%
% input file: ../../man//scicos/TEXT_f.man
%
\phead{TEXT\_f}{5}{Janvier\ 1997}{Scilab\ Group}{Scicos\ Block}


\Sdoc{TEXT\_f}{ Scicos  text drawing block}\index{TEXT\_f}\label{TEXTf}

\Shead{DIALOG PARAMETERS}
\begin{scitem}
\item[{\verb?txt?}]
: a character string, Text to be displayed
\item[{\verb?font?}]
: a positive integer less than 6, number of selected font (see xset)
\item[{\verb?siz?}]
: a positive integer, selected font size (see xset)
\end{scitem}% end Env
\Shead{DESCRIPTION}
This special block is only use to add text at any point of the diagram
window. It has no effect on the simulation.
%
% input file: ../../man//scicos/TIME_f.man
%
\phead{TIME\_f}{5}{Janvier\ 1996}{Scilab\ Group}{Scicos\ Block}


\Sdoc{TIME\_f}{ Scicos time generator}\index{TIME\_f}\label{TIMEf}

\Shead{DIALOG PARAMETERS}
None.
\Shead{DESCRIPTION}
This block is a time generator. The unique regular output is the
current time.





%
% input file: ../../man//scicos/TRASH_f.man
%
\phead{TRASH\_f}{5}{Janvier\ 1996}{Scilab\ Group}{Scicos\ Block}


\Sdoc{TRASH\_f}{ Scicos Trash block}\index{TRASH\_f}\label{TRASHf}

\Shead{DIALOG PARAMETERS}
None
\Shead{DESCRIPTION}
This block does nothing. It simply allows to safely connect the ouputs
of other blocks which should be ignored. This happens mostly for event
outputs that should produce no action. The input size is determined
by the context.








%
% input file: ../../man//scicos/WFILE_f.man
%
\phead{WFILE\_f}{5}{Janvier\ 1996}{Scilab\ Group}{Scicos\ Block}


\Sdoc{WFILE\_f}{ Scicos "write to file" block}\index{WFILE\_f}\label{WFILEf}

\Shead{DIALOG PARAMETERS}
\begin{scitem}
\item[{\verb?number of inputs?}]
: a scalar. This fixes the input size
\item[{\verb?Output file name?}]
: a character string defining the path of the file 
\item[{\verb?Output Format?}]
: a character string defining the Fortran format to use or nothing for
an unformatted (binary) write
\item[{\verb?Buffer size?}]
: To improve efficiency it is possible to buffer the input data. write
on the file is only done after each %
\tt Buffer size %
\rm call to the block.
\end{scitem}% end Env
\Shead{DESCRIPTION}
This block allows user to save datas in a file, in formatted or binary
mode.
Each call to the block corresponds to a record in the file. Each
record has the following form:
%
\tt [t,V1,...,Vn] %
\rm where %
\tt t %
\rm is the value of time when block is
called  and %
\tt Vi %
\rm is the ith input value
\Seealso{SEE ALSO}
{\verb?RFILE_f?} \pageref{RFILEf}









%
% input file: ../../man//scicos/ZCROSS_f.man
%
\phead{ZCROSS\_f}{5}{Janvier\ 1996}{Scilab\ Group}{Scicos\ Block}


\Sdoc{ZCROSS\_f}{ Scicos zero crossing detector}\index{ZCROSS\_f}\label{ZCROSSf}

\Shead{DESCRIPTION}
An output event is generated when all inputs (if more than one)
cross zero simultaneously.
\Shead{DIALOG PARAMETERS}
\begin{scitem}
\item[{\verb?Number of inputs?}]
: a positive integer.
\end{scitem}% end Env
\Seealso{SEE ALSO}
{\verb?POSTONEG_f ZCROSS_f GENERAL_f?} \pageref{POSTONEGfZCROSSfGENERALf}






% -*-LaTeX-*-
% Converted automatically from troff to LaTeX by tr2tex on Tue Apr 22 11:31:17 1997
% tr2tex was written by Kamal Al-Yahya at Stanford University
% (Kamal%Hanauma@SU-SCORE.ARPA)


%\documentstyle[troffman]{article}

%
% input file: ../../man//scicos/scifunc_block.man
%
\phead{scifunc\_block}{5}{Janvier\ 1996}{Scilab\ Group}{Scicos\ Block}


\Sdoc{scifunc\_block}{ Scicos block defined interactively}\index{scifunc\_block}\label{scifuncblock}

\Shead{DESCRIPTION}
This block can realize any type of Scicos block. The function of the
block is defined interactively using dialog boxes and in Scilab
language. During simulation, these instructions are interpreted by
Scilab; the simulation of diagrams that include these types of blocks 
is slower. For more information see Scicos reference manual.
\Shead{DIALOG PARAMETERS}
\begin{scitem}
\item[{\verb?number of inputs?}]
: a scalar. Number of regular input ports
\item[{\verb?number of outputs?}]
: a scalar. Number of regular output ports
\item[{\verb?number of input events?}]
: a scalar. Number of input event ports
\item[{\verb?number of output events?}]
: a scalar. Number of output event ports
\item[{\verb?Initial continuous state?}]
: a column vector.
\item[{\verb?Initial discrete state?}]
: a column vector.
\item[{\verb?System type?}]
: a string: z if the block is zero crossing; anything else if it is
a regular block (c or d for example to specify continuous or discrete). 
\item[{\verb?System parameter?}]
: column vector. Any parameters used in the block can be defined here
a column vector.
\item[{\verb?initial firing?}]
: vector. Size of this vector corresponds to the number of event
outputs. The value of the i-th entry specifies the time of the
preprogrammed event firing on the i-th output event port. If less than
zero, no event is preprogrammed.

\item[{\verb?Instructions?}]
: other dialogs are opened consecutively where used may input scilab
code associated to the computations needed (block initialization,
outputs, continuuous and 
discrete state, output events date, block ending),

\end{scitem}% end Env
\Seealso{SEE ALSO}
{\verb?GENERIC_f?} \pageref{GENERICf}





\section{Data Structures}
% -*-LaTeX-*-
% Converted automatically from troff to LaTeX by tr2tex on Tue Apr 22 11:31:18 1997
% tr2tex was written by Kamal Al-Yahya at Stanford University
% (Kamal%Hanauma@SU-SCORE.ARPA)


%\documentstyle[troffman]{article}

%
% input file: ../../man//scicos/scicos_main.man
%
\phead{scicos\_main}{Janvier\ 1997}{Scilab\ Group}{Scicos\ data\ structure}{}


\Sdoc{scicos\_main}{ scicos editor main  data structure}\index{scicos\_main}\label{scicosmain}

\Shead{DEFINITION}
\begin{verbatim}
scs_m=list(params,o_1,....,o_n)


\end{verbatim}
\Shead{PARAMETERS}
\begin{scitem}
\item[{\verb?params?}]
: scilab list, %
\tt params=list([w,h,Xshift,Yshift], title, tolerances,
tf, context)) %
\rm \begin{scitem}
\item[{\verb?w ?}]
: real scalar,scicos editor window width
\item[{\verb?h ?}]
: real scalar,scicos editor window height
\item[{\verb?Xshift ?}]
: real scalar, diagram drawing x offset within scicos editor window
\item[{\verb?Yshift ?}]
: real scalar, diagram drawing y offset within scicos editor window
\item[{\verb?title ?}]
: character string, diagram title and default save file name
\item[{\verb?tolerances ?}]
: 1 x 4 vector %
\tt [atol,rtol,ttol,maxt]%
\rm , where  %
\tt atol,rtol %
\rm are
respectively asolute and relative tolerances for the ode solver,
%
\tt ttol %
\rm is the  minimal distance between to differents events time
and %
\tt maxt %
\rm is maximum integration time interval for a single call to the
ode solver.
\item[{\verb?tf?}]
: real scalar, final time for simulation.
\item[{\verb?context?}]
: vector of character strings, defines scilab instructions used to define
formal scilab variables  used in block definitions.
\end{scitem}
\item[{\verb?o\_i?}]
: block or link  or deleted object data structure (see
%
\tt scicos\_block %
\rm and %
\tt scicos\_link%
\rm ). deleted object data
structure is defined by %
\tt list('Deleted')%
\rm .
\item[{\verb?scs\_m?}]
: main scicos structure
\end{scitem}% end Env
\Shead{DESCRIPTION}
scicos editor uses and modifies the  scicos editor main  data
structure to keep all information relative to the edited
diagram. Scicos compiler uses it as a input.

\Seealso{SEE ALSO}
{\verb?scicos?} \pageref{scicos},{\verb?  scicos_block?} \pageref{scicosblock},{\verb? scicos_link ?} \pageref{scicoslink}








% -*-LaTeX-*-
% Converted automatically from troff to LaTeX by tr2tex on Tue Apr 22 11:31:18 1997
% tr2tex was written by Kamal Al-Yahya at Stanford University
% (Kamal%Hanauma@SU-SCORE.ARPA)


%\documentstyle[troffman]{article}

%
% input file: ../../man//scicos/scicos_block.man
%
\phead{scicos\_block}{Janvier\ 1997}{Scilab\ Group}{Scicos\ data\ structure}{}


\Sdoc{scicos\_block}{ scicos block data structure}\index{scicos\_block}\label{scicosblock}

\Shead{DEFINITION}
\begin{verbatim}
blk=list('Block',graphics,model,void,gui)
\end{verbatim}
\Shead{PARAMETERS}
\begin{scitem}
\item[{\verb?"Block"?}]
: keyword used to define list as a scicos block representation
\item[{\verb?graphics?}]
: scilab list, graphic properties data structure
\item[{\verb?model?}]
: scilab list, system properties data structure. 

\item[{\verb?void?}]
: unused, reserved for future use.
\item[{\verb?gui?}]
: character string, the name of the graphic user interface function
(generally written in scilab) associated with the block.
\item[{\verb?blk?}]
: scilab list, scicos block data structure
\end{scitem}% end Env
\Shead{DESCRIPTION}
Scicos editor creates and uses for each block a data structure
containing all information relative to the graphic interface and
simulation part of the block. Each of them are stored in the scicos
editor main data structure. Index of these in scicos
editor main data structure is given by the creation order.


If block is a super block %
\tt model(9) %
\rm contains a data structure
similar to the %
\tt scicos\_main %
\rm data structure.

\Seealso{SEE ALSO}
{\verb?scicos?} \pageref{scicos},{\verb? scicos_main?} \pageref{scicosmain},{\verb? scicos_graphic?} \pageref{scicosgraphic},{\verb? scicos_model?} \pageref{scicosmodel}









% -*-LaTeX-*-
% Converted automatically from troff to LaTeX by tr2tex on Tue Apr 22 11:31:18 1997
% tr2tex was written by Kamal Al-Yahya at Stanford University
% (Kamal%Hanauma@SU-SCORE.ARPA)


%\documentstyle[troffman]{article}

%
% input file: ../../man//scicos/scicos_graphics.man
%
\phead{scicos\_graphics}{Janvier\ 1997}{Scilab\ Group}{Scicos\ data\ structure}{}


\Sdoc{scicos\_graphics}{ scicos block graphics data structure}\index{scicos\_graphics}\label{scicosgraphics}

\Shead{DEFINITION}
\begin{verbatim}
graphics=list(orig,sz,orient,exprs,pin,pout,pevtin,pevtout,gr_i)
\end{verbatim}
\Shead{PARAMETERS}
\begin{scitem}
\item[{\verb?orig?}]
: 2 x 1 vector, the coordinate of down-left point of the block shape.
\item[{\verb?sz?}]
: vector %
\tt [w,h]%
\rm , where %
\tt w %
\rm is the width and %
\tt h %
\rm the
height of the block shape.
\item[{\verb?orient?}]
: boolean, the block orientation. if true the input ports are on the
left of the box and output ports are on the right. if false  the input ports are on the
right of the box and output ports are on the left.  
\item[{\verb?exprs?}]
: column vector of strings,  contains expressions answered by the user
at block set time.
\item[{\verb?pin?}]
: column vector of integers. If %
\tt pin(k)$<$$>$0 %
\rm then  %
\tt k%
\rm th input
port is connected to the %
\tt pin(k)$<$$>$0 %
\rm block, else the port is
unconnected. If no input port exist %
\tt pin==[]%
\rm .
\item[{\verb?pout?}]
: column vector of integers.  If %
\tt pout(k)$<$$>$0 %
\rm then  %
\tt k%
\rm th output
port is connected to the %
\tt pout(k)$<$$>$0 %
\rm block, else the port is
unconnected. If no output port exist %
\tt pout==[]%
\rm .
\item[{\verb?pevtin?}]
: column vector of ones. If %
\tt pevin(k)$<$$>$0 %
\rm then  %
\tt k%
\rm th event input
port is connected to the %
\tt pevin(k)$<$$>$0 %
\rm block, else the port is
unconnected. If no event input port exist %
\tt pevin==[]%
\rm .
\item[{\verb?pevtout?}]
: column vector of integers.  If %
\tt pevtout(k)$<$$>$0 %
\rm then  %
\tt k%
\rm th
event output port is connected to the %
\tt evtpout(k)$<$$>$0 %
\rm block, else the port is
unconnected. If no event  output port exist %
\tt pevtout==[]%
\rm .
\item[{\verb?gr\_i?}]
: column vector of strings,  contains scilab instrucions used to
customize the block graphical aspect. This field may be set with
%
\tt "Icon" %
\rm sub\_menu.
\item[{\verb?graphics?}]
: scilab list, scicos block graphics data structure.
\end{scitem}% end Env
\Shead{DESCRIPTION}
scicos block graphics data structure contains all information relative
to graphical display of the block and to user dialog. Fields may be
fixed by block definition or set as a result of user dialog or connections. 

\Seealso{SEE ALSO}
{\verb?scicos?} \pageref{scicos},{\verb?  scicos_model?} \pageref{scicosmodel},{\verb? scicos_main ?} \pageref{scicosmain}








% -*-LaTeX-*-
% Converted automatically from troff to LaTeX by tr2tex on Tue Apr 22 11:31:18 1997
% tr2tex was written by Kamal Al-Yahya at Stanford University
% (Kamal%Hanauma@SU-SCORE.ARPA)


%\documentstyle[troffman]{article}

%
% input file: ../../man//scicos/scicos_model.man
%
\phead{scicos\_model}{Janvier\ 1997}{Scilab\ Group}{Scicos\ data\ structure}{}


\Sdoc{scicos\_model}{ scicos block functionality  data structure}\index{scicos\_model}\label{scicosmodel}

\Shead{DEFINITION}
\begin{verbatim}
model=list(sim,in,out,evtin,evtout,state,dstate,rpar,ipar,..
                          blocktype,firing,dep_ut,label,import)
\end{verbatim}
\Shead{PARAMETERS}
\begin{scitem}
\item[{\verb?sim?}]
: list(fun,typ) or fun. In the latest case typ is supposed to be 0.
\begin{scitem}
\item[{\verb?fun?}]
: character string, the name of the block simulation function (a 
linked C or Fortran procedure  or a scilab function).
\item[{\verb?typ?}]
: integer, calling sequence type of simulation function (see
documentation for more precision).
\end{scitem}
\item[{\verb?in?}]
: column vector of integers, input port sizes indexed from top to
bottom of the block. If no input port exist %
\tt in==[]%
\rm .
\item[{\verb?out?}]
: column vector of integers, output port sizes indexed from top to
bottom of the block. If no output port exist %
\tt in==[]%
\rm .
\item[{\verb?evtin?}]
: column vector of ones, the size of %
\tt evtin %
\rm gives the number of
event input ports. If no event input port exists %
\tt evtin %
\rm must be equal
to %
\tt []%
\rm .
\item[{\verb?evtout?}]
: column vector of ones, the size of %
\tt evtout %
\rm gives the number of
event output ports. If no event output port exists %
\tt evtout %
\rm must be equal
to %
\tt []%
\rm .
\item[{\verb?state?}]
: column vector, the initial continuous state of the block. Must be
%
\tt [] %
\rm if no continuous state.
\item[{\verb?dstate?}]
: column vector, the initial discrete state of the block. Must be
%
\tt [] %
\rm if no discrete state.
\item[{\verb?rpar?}]
: column vector, the vector of floating point block parameters. Must be
%
\tt [] %
\rm if no floating point parameters.
\item[{\verb?ipar?}]
: column vector, the vector of integer  block parameters. Must be
%
\tt [] %
\rm if no integer parameters.
\item[{\verb?blocktype?}]
: a character with possible values:
\begin{scitem}
\item[{\verb?:?}]
'c'  block output depend continuously of the time.
\item[{\verb?:?}]
'd' block output changes only on input events.
\item[{\verb?:?}]
'z' zero crossing block
\item[{\verb?:?}]
'l' logical block
\end{scitem}
\item[{\verb?firing?}]
: a vector whose size is equal to the size of %
\tt evtout%
\rm $>$ It contains
output initial event dates (Events generated before any input event
arises). Negative values stands for no initial event on the
corresponding port.
\item[{\verb?dep\_ut?}]
: 1x 2 vector of boolean %
\tt [dep\_u, dep\_t]%
\rm , %
\tt dep\_u %
\rm must be
true if output depends continuously of the input, %
\tt dep\_t %
\rm must be
true if output depends continuously of the time.
\item[{\verb?label?}]
: a character string, used as an identifier. 
\item[{\verb?import?}]
: Unused.
\item[{\verb?model?}]
: scilab list, scicos block model data structure.
\end{scitem}% end Env
\Shead{DESCRIPTION}
scicos block model data structure contains all information relative
to the simulation functionality of the block. Fields may be
fixed by block definition or set. 

If block is a super block, the fields
%
\tt state%
\rm ,%
\tt dstate%
\rm ,%
\tt ipar%
\rm ,%
\tt blocktype%
\rm ,%
\tt firing%
\rm ,
%
\tt dep\_ut%
\rm , are unused. The %
\tt rpar %
\rm field contains a
data structure similar to the %
\tt scicos\_main %
\rm data structure.


\Seealso{SEE ALSO}
{\verb?scicos?} \pageref{scicos},{\verb?  scicos_model?} \pageref{scicosmodel},{\verb? scicos_main ?} \pageref{scicosmain}








% -*-LaTeX-*-
% Converted automatically from troff to LaTeX by tr2tex on Tue Apr 22 11:31:18 1997
% tr2tex was written by Kamal Al-Yahya at Stanford University
% (Kamal%Hanauma@SU-SCORE.ARPA)


%\documentstyle[troffman]{article}

%
% input file: ../../man//scicos/scicos_link.man
%
\phead{scicos\_link}{Janvier\ 1997}{Scilab\ Group}{Scicos\ data\ structure}{}


\Sdoc{scicos\_link}{ scicos link data structure}\index{scicos\_link}\label{scicoslink}

\Shead{DEFINITION}
\begin{verbatim}
lnk=list('Link',xx,yy,'drawlink',' ',[0,0],ct,from,to)
\end{verbatim}
\Shead{PARAMETERS}
\begin{scitem}
\item[{\verb?"Link"?}]
: keyword used to define list as a scicos link representation
\item[{\verb?xx?}]
: vector of x coordinates of the link path.
\item[{\verb?yy?}]
: vector of y coordinates of the link path.
\item[{\verb?ct?}]
: 2 x 1 vector, %
\tt [color,typ] %
\rm where %
\tt color %
\rm defines the color
used for the link drawing and %
\tt typ %
\rm defines its type (0 for
regular link ,1 for event link).
\item[{\verb?from?}]
: 2 x 1 vector, %
\tt [block,port] %
\rm where %
\tt block %
\rm is the index of
the block at the origin of the link and %
\tt port %
\rm is the index of the
port.
\item[{\verb?to?}]
: 2 x 1 vector, %
\tt [block,port] %
\rm where %
\tt block %
\rm is the index of
the block at the end of the link and %
\tt port %
\rm is the index of the
port.
\end{scitem}% end Env
\Shead{DESCRIPTION}
Scicos editor creates and uses for each link a data structure
containing all information relative to the graphic interface and
interconnection information. Each of them are stored in the scicos
editor main data structure. Index of these in scicos
editor main data structure is given by the creation order.
\Seealso{SEE ALSO}
{\verb?scicos?} \pageref{scicos},{\verb? scicos_main?} \pageref{scicosmain},{\verb? scicos_graphic?} \pageref{scicosgraphic},{\verb? scicos_model?} \pageref{scicosmodel}


% -*-LaTeX-*-
% Converted automatically from troff to LaTeX by tr2tex on Tue Apr 22 11:31:18 1997
% tr2tex was written by Kamal Al-Yahya at Stanford University
% (Kamal%Hanauma@SU-SCORE.ARPA)


%\documentstyle[troffman]{article}

%
% input file: ../../man//scicos/scicos_cpr.man
%
\phead{scicos\_cpr}{Janvier\ 1997}{Scilab\ Group}{Scicos\ data\ structure}{}


\Sdoc{scicos\_cpr}{ scicos compiled diagram data structure}\index{scicos\_cpr}\label{scicoscpr}

\Shead{DEFINITION}
\begin{verbatim}
cpr=list(state,sim,cor,corinv)
\end{verbatim}
\Shead{PARAMETERS}
\begin{scitem}
\item[{\verb?state?}]
: scilab %
\tt tlist %
\rm contains  initial state. 
\begin{scitem}
\item[{\verb?state('x')?}]
: continuous  state vector. 
\item[{\verb?state('z')?}]
: discrete  state vector. 
\item[{\verb?state('tevts')?}]
: vector of  event dates
\item[{\verb?state('evtspt')?}]
: vector of event pointers
\item[{\verb?state('pointi')?}]
: pointer to next event
state('npoint')
: not used yet
state('outtb')
: vector of inputs/outputs initial values.
\end{scitem}
\item[{\verb?sim?}]
: scilab %
\tt tlist%
\rm .  Usually generated by scicos
%
\tt Compile %
\rm menu. Some useful entries are:
\begin{scitem}
\item[{\verb?sim('rpar')?}]
: vector of blocks' floating point parameters
\item[{\verb?sim('rpptr')?}]
: (nblk+1) x 1 vector of integers, 
 %
\tt sim('rpar')(rpptr(i):(rpptr(i+1)-1)) %
\rm is the vector of floating
point parameters of the %
\tt i%
\rm th block. 
\item[{\verb?sim('ipar')?}]
: vector of blocks' integer parameters
\item[{\verb?sim('ipptr')?}]
: (nblk+1) x 1 vector of integers, 
 %
\tt sim('ipar')(ipptr(i):(ipptr(i+1)-1)) %
\rm is the vector of integer
parameters of the %
\tt i%
\rm th block. 
\item[{\verb?sim('funs')?}]
: vector of strings containing the names of each block simulation function
\item[{\verb?sim('xptr')?}]
: (nblk+1) x 1 vector of integers, 
 %
\tt state('x')(xptr(i):(xptr(i+1)-1)) %
\rm is the continuous state
vector  of the %
\tt i%
\rm th block. 
\item[{\verb?sim('zptr')?}]
: (nblk+1) x 1 vector of integers, 
 %
\tt state('z')(zptr(i):(zptr(i+1)-1)) %
\rm is the discrete state
vector  of the %
\tt i%
\rm th block. 
\item[{\verb?sim('inpptr')?}]
: (nblk+1) x 1 vector of integers, %
\tt inpptr(i+1)-inpptr(i) %
\rm gives
the number of input ports. %
\tt inpptr(i)%
\rm th points to the
beginning of %
\tt i%
\rm th block inputs  within the indirection table %
\tt inplnk%
\rm .
\item[{\verb?sim('inplnk')?}]
: nblink x 1 vector of integers, %
\tt inplnk(inpptr(i)-1+j) %
\rm is the
index of the link connected to the  %
\tt j%
\rm th input port of the %
\tt i%
\rm th block.
where %
\tt j %
\rm goes from %
\tt 1 %
\rm to %
\tt inpptr(i+1)-inpptr(i))%
\rm .
\item[{\verb?sim('outptr')?}]
: (nblk+1) x 1 vector of integers, %
\tt outptr(i+1)-outptr(i) %
\rm gives
the number of output ports. %
\tt outptr(i)%
\rm th points to the
beginning of %
\tt i%
\rm th block outputs  within the indirection table %
\tt outlnk%
\rm .
\item[{\verb?sim('outlnk')?}]
: nblink x 1 vector of integers, %
\tt outlnk(outptr(i)-1+j) %
\rm is the
index of the link connected to the  %
\tt j%
\rm th output port of the %
\tt i%
\rm th block.
where %
\tt j %
\rm goes from %
\tt 1 %
\rm to %
\tt outptr(i+1)-outptr(i))%
\rm .
\item[{\verb?sim('lnkptr')?}]
: (nblink+1) x 1 vector of integers, %
\tt k%
\rm th  entry points to the
beginning of region  within %
\tt outtb %
\rm dedicated to link indexed %
\tt k%
\rm .
\item[{\verb?sim('funs')?}]
: vector of strings containing the names of each block simulation function
\item[{\verb?sim('funtyp')?}]
: vector of  block block types.
\end{scitem}
\item[{\verb?cor    ?}]
: is a list with same recursive structure as scs\_m each leaf 
contains the index of associated block in %
\tt cpr %
\rm data structure.
\item[{\verb?corinv ?}]
: corinv(i) is the path of %
\tt i %
\rm th block defined in %
\tt cpr %
\rm data structure
          in the %
\tt scs\_m %
\rm data structure.

\end{scitem}% end Env
\Shead{DESCRIPTION}
scicos compiled diagram data structure contains all information needed
by %
\tt scicosim %
\rm to simulate the system. relative
\Seealso{SEE ALSO}
{\verb?scicos?} \pageref{scicos},{\verb?  scicos_model?} \pageref{scicosmodel},{\verb? scicos_main?} \pageref{scicosmain},{\verb? scicosim?} \pageref{scicosim}








\section{Usefull Function}
% -*-LaTeX-*-
% Converted automatically from troff to LaTeX by tr2tex on Tue Apr 22 11:31:19 1997
% tr2tex was written by Kamal Al-Yahya at Stanford University
% (Kamal%Hanauma@SU-SCORE.ARPA)


%\documentstyle[troffman]{article}

%
% input file: ../../man//scicos/standard_define.man
%
\phead{standard\_define}{Janvier\ 1997}{Scilab\ Group}{Scicos\ function}{}


\Sdoc{standard\_define}{ scicos block initial definition  function}\index{standard\_define}\label{standarddefine}

\Shead{CALLING SEQUENCE}
\begin{verbatim}
o=standard_define(sz,model,dlg,gr_i)
\end{verbatim}
\Shead{PARAMETERS}
\begin{scitem}
\item[{\verb?o?}]
: scicos block data structure (see scicos\_block)
\item[{\verb?sz?}]
: 2 vector, giving the initial block width and height 
\item[{\verb?model?}]
: initial model data structure definition (see scicos\_model)
\item[{\verb?dlg?}]
: vector of character strings,initial parameters expressions
\item[{\verb?gr\_i?}]
: vector of character strings, initial icon definition instructions 
\end{scitem}% end Env
\Shead{DESCRIPTION}
This function creates the initial block data structure given the
initial size  %
\tt sz%
\rm , this intial model definition  %
\tt model%
\rm , the
initial parameters expressions  %
\tt dlg %
\rm and initial icon definition
instructions  %
\tt gr\_i %
\rm \Seealso{SEE ALSO}
{\verb?scicos_model?} \pageref{scicosmodel}


%
% input file: ../../man//scicos/standard_draw.man
%
\phead{standard\_draw}{Janvier\ 1997}{Scilab\ Group}{Scicos\ function}{}


\Sdoc{standard\_draw}{ scicos block drawing function}\index{standard\_draw}\label{standarddraw}

\Shead{CALLING SEQUENCE}
\begin{verbatim}
standard_draw(o)
\end{verbatim}
\Shead{PARAMETERS}
\begin{scitem}
\item[{\verb?o?}]
: scicos block data structure (see scicos\_block)
\end{scitem}% end Env
\Shead{DESCRIPTION}
%
\tt standard\_draw %
\rm is the scilab function used to display
standard blocks in interfacing functions. 
\par\noindent
It draws a block with a rectangular shape with any number of regular
or event input respectively on the left and right faces of the block
(if not flipped), event input or output respectively on the top and
bottom faces of the block. Number of ports, size, origin, orientation,
background color, icon of the block are taken from the block data
structure %
\tt o%
\rm .
\Seealso{SEE ALSO }
{\verb?scicos_block?} \pageref{scicosblock}
%
% input file: ../../man//scicos/standard_input.man
%
\phead{standard\_input}{Janvier\ 1997}{Scilab\ Group}{Scicos\ function}{}


\Sdoc{standard\_input}{ get scicos block input port positions}\index{standard\_input}\label{standardinput}

\Shead{CALLING SEQUENCE}
\begin{verbatim}
[x,y,typ]=standard_input(o)
\end{verbatim}
\Shead{PARAMETERS}
\begin{scitem}
\item[{\verb?o?}]
: scicos block data structure (see scicos\_block)
\item[{\verb?x?}]
: vector of x coordinates of the block regular and event input ports
\item[{\verb?y?}]
: vector of y coordinates of the block regular and event output ports
\item[{\verb?typ?}]
: vector of input ports types (+1 : regular port; -1:event port)

\end{scitem}% end Env
\Shead{DESCRIPTION}
%
\tt standard\_input %
\rm is the scilab function used to get
standard blocks input port position anf types in interfacing
functions. 
\par\noindent
Port positions are computed, each time they are required, as a
function of block dimensions.

\Seealso{SEE ALSO }
{\verb?scicos_block?} \pageref{scicosblock}
%
% input file: ../../man//scicos/standard_origin.man
%
\phead{standard\_origin}{Janvier\ 1997}{Scilab\ Group}{Scicos\ function}{}


\Sdoc{standard\_origin}{ scicos block origin function}\index{standard\_origin}\label{standardorigin}

\Shead{CALLING SEQUENCE}
\begin{verbatim}
[x,y]=standard_draw(o)
\end{verbatim}
\Shead{PARAMETERS}
\begin{scitem}
\item[{\verb?o?}]
: scicos block data structure (see scicos\_block)
\item[{\verb?x?}]
: x coordinate of the block orign (bottom left corner)
\item[{\verb?y?}]
: y coordinate of the block orign (bottom left corner)
\end{scitem}% end Env
\Shead{DESCRIPTION}
%
\tt standard\_origin %
\rm is the scilab function used to get
standard blocks position in interfacing functions. 

\Seealso{SEE ALSO }
{\verb?scicos_block?} \pageref{scicosblock}
%
% input file: ../../man//scicos/standard_output.man
%
\phead{standard\_output}{Janvier\ 1997}{Scilab\ Group}{Scicos\ function}{}


\Sdoc{standard\_output}{ get scicos block output port positions}\index{standard\_output}\label{standardoutput}

\Shead{CALLING SEQUENCE}
\begin{verbatim}
[x,y,typ]=standard_output(o)
\end{verbatim}
\Shead{PARAMETERS}
\begin{scitem}
\item[{\verb?o?}]
: scicos block data structure (see scicos\_block)
\item[{\verb?x?}]
: vector of x coordinates of the block regular and event output ports
\item[{\verb?y?}]
: vector of y coordinates of the block regular and event output ports
\item[{\verb?typ?}]
: vector of output ports types (+1 : regular port; -1:event port)

\end{scitem}% end Env
\Shead{DESCRIPTION}
%
\tt standard\_output %
\rm is the scilab function used to get
standard blocks output port position anf types in interfacing
functions. 
\par\noindent
Port positions are computed, each time they are required, as a
function of block dimensions.

\Seealso{SEE ALSO }
{\verb?scicos_block?} \pageref{scicosblock}

% -*-LaTeX-*-
% Converted automatically from troff to LaTeX by tr2tex on Tue Apr 22 11:31:19 1997
% tr2tex was written by Kamal Al-Yahya at Stanford University
% (Kamal%Hanauma@SU-SCORE.ARPA)


%\documentstyle[troffman]{article}

%
% input file: ../../man//scicos/scicosim.man
%
\phead{scicosim}{Janvier\ 1997}{Scilab\ Group}{Scicos\ function}{}


\Sdoc{scicosim}{ scicos simulation function}\index{scicosim}\label{scicosim}

\Shead{CALLING SEQUENCE}
\begin{verbatim}
[state,t]=scicosim(state,0,tf,sim,'start' [,tolerances])
[state,t]=scicosim(state,tcur,tf,sim,'run' [,tolerances])
[state,t]=scicosim(state,tcur,tf,sim,'finish' [,tolerances])
\end{verbatim}
\Shead{PARAMETERS}
\begin{scitem}
\item[{\verb?state?}]
: scilab %
\tt tlist %
\rm contains scicosim  initial state. Usually generated by scicos
%
\tt Compile %
\rm or %
\tt Run %
\rm menus (see scicos\_cpr for more details).
\item[{\verb?tcur ?}]
: initial simulation  time
\item[{\verb?tf?}]
: final simulation time (Unused with options  %
\tt 'start' %
\rm and  %
\tt 'finish' %
\rm \item[{\verb?sim?}]
: scilab %
\tt tlist%
\rm .  Usually generated by scicos
%
\tt Compile %
\rm menu (see scicos\_cpr  for more details). 
\item[{\verb?tolerances?}]
: 4 vector %
\tt [atol,rtol,ttol,deltat] %
\rm where %
\tt atol,rtol %
\rm  are respectively the
absolute and relative tolerances for ode solver (see ode), %
\tt ttol %
\rm is the precision on event dates.  %
\tt deltat %
\rm is maximum integration
interval for each call to ode solver.
\item[{\verb?t?}]
: final reached time
\end{scitem}% end Env
\Shead{DESCRIPTION}
Simulator for scicos compiled diagram. Usually %
\tt scicosim %
\rm is
called by %
\tt scicos %
\rm to perform simulation of a diagram. 

But %
\tt scicosim %
\rm may also be called outside scicos. Typical usage in
such a case may be:
\begin{scitem}
\item[{\verb?1?}]
Use scicos to define a block diagram, compile it.
\item[{\verb?2?}]
Save the compiled diagram using %
\tt Save,SaveAs %
\rm scicos menus . 
\item[{\verb?3?}]
In scilab, load saved file using %
\tt load %
\rm function. You get
variables %
\tt scicos\_ver%
\rm , %
\tt scs\_m%
\rm , %
\tt cpr%
\rm ,
%
\tt needcompile%
\rm .
\end{scitem}% end Env
\par\noindent
%
\tt scs\_m %
\rm is the diagram scicos main data structure.
\par\noindent
if %
\tt needcompile==\%f %
\rm  %
\tt cpr %
\rm is the data structure
%
\tt list(state,sim,cor,corinv) %
\rm  
\begin{scitem}
\item[{\verb?4?}]
Extract %
\tt state%
\rm , %
\tt sim %
\rm out of %
\tt cpr %
\rm \item[{\verb?5?}]
Execute %
\tt [state,t]=scicosim(state,0,tf,sim,'start' [,tolerances]) %
\rm for initialisation.
\item[{\verb?6?}]
Execute %
\tt [state,t]=scicosim(state,0,tf,sim,'run' [,tolerances]) %
\rm for simulation from %
\tt 0 %
\rm to %
\tt tf%
\rm . Many successives such calls may be
performed changing initial and final time.
\item[{\verb?7?}]
Execute %
\tt [state,t]=scicosim(state,0,tf,sim,'finish' [,tolerances]) %
\rm at the very end of the simulation to close files,...

For advanced user it is possible to "manually" change some parameters
or state values

\end{scitem}% end Env
\Seealso{SEE ALSO}
{\verb?scicos ode odedc impl scicos_cpr?} \pageref{scicosodeodedcimplscicoscpr}
      








% -*-LaTeX-*-
% Converted automatically from troff to LaTeX by tr2tex on Tue Apr 22 11:31:19 1997
% tr2tex was written by Kamal Al-Yahya at Stanford University
% (Kamal%Hanauma@SU-SCORE.ARPA)


%\documentstyle[troffman]{article}

%
% input file: ../../man//scicos/curblock.man
%
\phead{curblock}{Janvier\ 1997}{Scilab\ Group}{Scicos\ function}{}


\Sdoc{curblock}{ get current block index in a scicos simulation function}\index{curblock}\label{curblock}

\Shead{CALLING SEQUENCE}
\begin{verbatim}
k=curblock()
\end{verbatim}
\Shead{PARAMETERS}
\begin{scitem}
\item[{\verb?k?}]
: integer, index of the block corresponding to the scilab simulation
function where this function is called.
\end{scitem}% end Env
\Shead{DESCRIPTION}
During simulation it may be interesting to get the index of the current
block to trace execution, to get its label, to animate the block icon
according to simulation...
\par\noindent
For block with a computational function written in Scilab, scilab
primitive function %
\tt curblock() %
\rm  allows to get the index of the
current block in the compiled data structure.
\par\noindent
To obtain path to the block in the scicos main structure user may uses
the %
\tt corinv %
\rm table (see scicos\_cpr).
\par\noindent
For block with a computational function written in C user may uses the
C function  %
\tt k=C2F(getcurblock)()%
\rm . Where %
\tt C2F %
\rm is the C
compilation macro defined in %
\tt $<$SCIDIR$>$/routines/machine.h %
\rm \par\noindent
For block with a computational function written in Fortran user may uses the
integer  function  %
\tt k=getcurblock()%
\rm .
\Seealso{SEE ALSO}
{\verb?getblocklabel getscicosvars setscicosvars scicos_cpr scicos_main?} \pageref{getblocklabelgetscicosvarssetscicosvarsscicoscprscicosmain}


% -*-LaTeX-*-
% Converted automatically from troff to LaTeX by tr2tex on Tue Apr 22 11:31:19 1997
% tr2tex was written by Kamal Al-Yahya at Stanford University
% (Kamal%Hanauma@SU-SCORE.ARPA)


%\documentstyle[troffman]{article}

%
% input file: ../../man//scicos/getblocklabel.man
%
\phead{getblocklabel}{Janvier\ 1997}{Scilab\ Group}{Scicos\ function}{}


\Sdoc{getblocklabel}{ get label of a scicos block at running time}\index{getblocklabel}\label{getblocklabel}

\Shead{CALLING SEQUENCE}
\begin{verbatim}
label=getblocklabel()
label=getblocklabel(k)
\end{verbatim}
\Shead{PARAMETERS}
\begin{scitem}
\item[{\verb?k?}]
: integer, index of the block. if %
\tt k %
\rm is omitted %
\tt k%
\rm is
supposed to be equal to   %
\tt curblock()%
\rm . 
\item[{\verb?label?}]
: a character string, The label of %
\tt k%
\rm th block (see  %
\tt Label %
\rm button
in %
\tt Block %
\rm menu.
\end{scitem}% end Env
\Shead{DESCRIPTION}
For display or debug purpose it may be usefull to give label to
particular blocks of a diagram. This may be done using scicos editor 
(%
\tt Label %
\rm button in %
\tt Block %
\rm menu). During simulation, value of
these labels may be obtained in any scilab block with %
\tt getblocklabel %
\rm scilab primitive function. 
\par\noindent
For C or fortran computational functions, user may use
 %
\tt C2F(getlabel) %
\rm to get a block label. See
%
\tt routines/scicos/import.c %
\rm file for more details
\par\noindent

Block indexes are those relative to the compile structure%
\tt cpr%
\rm . 

\Seealso{SEE ALSO}
{\verb?curblock getscicosvars setscicosvars?} \pageref{curblockgetscicosvarssetscicosvars}

% -*-LaTeX-*-
% Converted automatically from troff to LaTeX by tr2tex on Tue Apr 22 11:31:20 1997
% tr2tex was written by Kamal Al-Yahya at Stanford University
% (Kamal%Hanauma@SU-SCORE.ARPA)


%\documentstyle[troffman]{article}

%
% input file: ../../man//scicos/getscicosvars.man
%
\phead{getscicosvars}{5}{Janvier\ 1996}{Scilab\ Group}{Scicos\ function}


\Sdoc{getscicosvars}{ get scicos data structure while running}\index{getscicosvars}\label{getscicosvars}

\Shead{CALLING SEQUENCE}
v=getscicosvars(name)
\Shead{PARAMETERS}
\begin{scitem}
\item[{\verb?name?}]
: a character string, the name of the required structure
\item[{\verb?v ?}]
: vector of the structure value
\end{scitem}% end Env
\Shead{DESCRIPTION}
This function may be used in a scilab block to get value of some
particular global data while running. It allows to write diagram
monitoring blocks. 
\par\noindent
for example  the instruction  %
\tt disp(getscicosvars('x')) %
\rm  displays 
the entire continuous state of the diagram.
\begin{verbatim}
x=getscicosvars('x');
xptr=getscicosvars('xptr');
disp(x(xptr(k):xptr(k+1)-1))
\end{verbatim}
displays the continuous state of the %
\tt k %
\rm block

\ignore{
\begin{verbatim}
|=========================================================|
| name     |  data structure definition                   |
|=========================================================|
|'x'       | continuous state                             |
|'xptr'    | continuous state splitting vector            |
|'z'       | discrete state                               |
|'zptr'    | discrete  state splitting vector             |
|'rpar'    | real parameters vector                       |
|'rpptr'   | rpar  splitting vector                       |
|'ipar'    | integer parameters vector                    |
|'ipptr'   | ipar  splitting vector                       |
|'outtb'   | vector of all input/outputs values           |
|'inpptr'  | inplnk splitting vector                      |
|'outptr'  | outlnk splitting vector                      |
|'inplnk'  | vector of input port values adress in lnkptr |
|'outlnk'  | vector of output port values adress in lnpkpr|
|'lnkptr'  | outtb splitting vector                       |
|=========================================================|
\end{verbatim}
}
\par\noindent
%Thisis for LaTeX 
 \begin{tabular}{|r|l|} \hline name& data structure definition \\  \hline  'x' &    continuous state \\  \hline  'xptr' &    continuous state splitting vector\\  \hline  'z' &    discrete state \\  \hline  'zptr' &    discrete  state splitting vector\\  \hline  'rpar' &  real parameters vector \\  \hline  'rpptr' &    rpar  splitting vector\\  \hline  'ipar' &     integer parameters vector \\ \hline  'ipptr' &    ipar  splitting vector\\  \hline  'outtb' &    vector of all input/outputs values \\ \hline  'inpptr' &   inplnk splitting vector \\ \hline  'outptr' &   outlnk splitting vector  \\ \hline  'inplnk' &   vector of input port values adress in lnkptr \\ \hline  'outlnk' &   vector of output port values adress in lnpkpr \\ \hline  'lnkptr' &   outtb splitting vector \\ \hline \end{tabular}\par\noindent
See %
\tt scicos\_cpr %
\rm for more detail on these data structures.
\par\noindent
For C or fortran computational function the C procedure
%
\tt C2F(getscicosvars) %
\rm may used. See
%
\tt routines/scicos/import.c %
\rm file for more details.
\Seealso{SEE ALSO}
{\verb?setscicosvars  scicosim  curblock scicos_cpr getblocklabel?} \pageref{setscicosvarsscicosimcurblockscicoscprgetblocklabel}
%
% input file: ../../man//scicos/setscicosvars.man
%
\phead{setscicosvars}{5}{Janvier\ 1996}{Scilab\ Group}{Scicos\ function}


\Sdoc{setscicosvars}{ set scicos data structure while running}\index{setscicosvars}\label{setscicosvars}

\Shead{CALLING SEQUENCE}
setscicosvars(name,v)
\Shead{PARAMETERS}
\begin{scitem}
\item[{\verb?name?}]
: a character string, the name of the required structure
\item[{\verb?v ?}]
: vector of the new structure value
\end{scitem}% end Env
\Shead{DESCRIPTION}
This function may be used in a scilab block to set value of some
particular global data while running. It allows to write diagram
supervisor blocks.
\par\noindent
for example  the instructions  
\begin{verbatim}
x=getscicosvars('x');
xptr=getscicosvars('xptr');
x(xptr(k):xptr(k+1)-1)=xk
setscicosvars('x',x)
\end{verbatim}
Changes the continuous state of the %
\tt k %
\rm block to %
\tt xk%
\rm .

\ignore{
\begin{verbatim}
|=========================================================|
| name     |  data structure definition                   |
|=========================================================|
|'x'       | continuous state                             |
|'xptr'    | continuous state splitting vector            |
|'z'       | discrete state                               |
|'zptr'    | discrete  state splitting vector             |
|'rpar'    | real parameters vector                       |
|'rpptr'   | rpar  splitting vector                       |
|'ipar'    | integer parameters vector                    |
|'ipptr'   | ipar  splitting vector                       |
|'outtb'   | vector of all input/outputs values           |
|'inpptr'  | inplnk splitting vector                      |
|'outptr'  | outlnk splitting vector                      |
|'inplnk'  | vector of input port values adress in lnkptr |
|'outlnk'  | vector of output port values adress in lnpkpr|
|'lnkptr'  | outtb splitting vector                       |
|=========================================================|
\end{verbatim}
}
\par\noindent
%Thisis for LaTeX 
 \begin{tabular}{|r|l|} \hline name& data structure definition \\  \hline  'x' &    continuous state \\  \hline  'xptr' &    continuous state splitting vector\\  \hline  'z' &    discrete state \\  \hline  'zptr' &    discrete  state splitting vector\\  \hline  'rpar' &  real parameters vector \\  \hline  'rpptr' &    rpar  splitting vector\\  \hline  'ipar' &     integer parameters vector \\ \hline  'ipptr' &    ipar  splitting vector\\  \hline  'outtb' &    vector of all input/outputs values \\ \hline  'inpptr' &   inplnk splitting vector \\ \hline  'outptr' &   outlnk splitting vector  \\ \hline  'inplnk' &   vector of input port values adress in lnkptr \\ \hline  'outlnk' &   vector of output port values adress in lnpkpr \\ \hline  'lnkptr' &   outtb splitting vector \\ \hline \end{tabular}\par\noindent
See %
\tt scicos\_cpr %
\rm for more detail on these data structures.
\par\noindent
For C or fortran computational function the C procedure
%
\tt C2F(setscicosvars) %
\rm may used. See
%
\tt routines/scicos/import.c %
\rm file for more details.
\par\noindent
Warning: The use of this function requires a deep knowledge on how
scicosim works,may be used very carefully. Unpredicted parameters,
state, link values changes  may produce erroneous simulations.
\Seealso{SEE ALSO}
{\verb?getscicosvars  scicosim  curblock scicos_cpr getblocklabel?} \pageref{getscicosvarsscicosimcurblockscicoscprgetblocklabel}

