% -*-LaTeX-*-
% Converted automatically from troff to LaTeX by tr2tex on Wed Jul 16 16:49:17 1997
% tr2tex was written by Kamal Al-Yahya at Stanford University
% (Kamal%Hanauma@SU-SCORE.ARPA)


%\documentstyle[troffman]{article}

%
% input file: example.1
%
\phead{ss2ss}{1}{April\ 1993}{Scilab\ Group}{Scilab\ Function}

 
\Sdoc{ss2ss}{ state-space to state-space conversion, feedback, injection}\index{ss2ss}\label{ss2ss}

\Shead{CALLING SEQUENCE}
\begin{verbatim}
[Sl1,right,left]=ss2ss(Sl,T, [F, [G , [flag]]])
\end{verbatim}
\Shead{PARAMETERS}
\begin{scitem}
\item[{\tt Sl}]
: linear system (%
\tt syslin %
\rm list) in state-space form
\item[{\tt T }]
: square (non-singular) matrix
\item[{\tt Sl1, right, left}]
: linear systems (syslin lists) in state-space form
\item[{\tt F}]
: real matrix (state feedback gain)
\item[{\tt G}]
: real matrix (output injection gain)
\end{scitem}% end Env
\Shead{DESCRIPTION}
Returns the linear system %
\tt Sl1=[A1,B1,C1,D1] %
\rm where %
\tt A1=inv(T)*A*T, B1=inv(T)*B, C1=C*T, D1=D%
\rm .
\par\noindent
Optional parameters %
\tt F %
\rm and %
\tt G %
\rm are state feedback
and output injection respectively. 
\par\noindent
For example,
%
\tt Sl1=ss2ss(Sl,T,F) %
\rm returns %
\tt Sl1 %
\rm with:
\par\noindent
\ignore{
\begin{verbatim} 
              | inv(T)*(A+B*F)*T  ,    inv(T)*B |
        Sl1 = |                                 |
              |     (C+D*F)*T     ,      D      |
\end{verbatim}
}
 \[ \mbox{\verb+Sl1+} = \left( \begin{array}{cc} T^{-1}(A+BF)T & T^{-1} (B) \\	        (C+DF)T & D \end{array} \right) \] and %
\tt right %
\rm is a non singular linear system such that %
\tt Sl1=Sl*right%
\rm .
\par\noindent
%
\tt Sl1*inv(right) %
\rm is a factorization of %
\tt Sl%
\rm .
\par\noindent
\par\noindent
%
\tt Sl1=ss2ss(Sl,T,0*F,G) %
\rm returns %
\tt Sl1 %
\rm with:
\ignore{
\begin{verbatim} 
              | inv(T)*(A+G*C)*T  , inv(T)*(B+G*D) |
        Sl1 = |                                    |
              |      C*T          ,      D         |
\end{verbatim}
}
 \[ \mbox{\verb+Sl1+} = \left( \begin{array}{cc} T^{-1}(A+GC)T & T^{-1} (B+GD) \\	        CT & D \end{array} \right) \] \par\noindent
and %
\tt left %
\rm is a non singular linear system such that %
\tt Sl1=left*Sl %
\rm (%
\tt right=Id %
\rm if %
\tt F=0%
\rm ).

When both %
\tt F %
\rm and %
\tt G %
\rm are given, %
\tt Sl1=left*Sl*right%
\rm .
\begin{scitem}
\item[{\tt -}]
When %
\tt flag %
\rm is used and %
\tt flag=1 %
\rm an output injection 
as follows is used 
\ignore{
\begin{verbatim}
	 | inv(T)*(A + GC)*T  , inv(T)*(B+GD,-G) |
	 |      C*T           ,       (D   , 0)  |
\end{verbatim}
}
 \[ \mbox{\verb+Sl1+} = \left( \begin{array}{cc} T^{-1}(A+GC)T & 		T^{-1} \left(B+GD,-G\right) \\	        CT & \left(D , 0 \right)\end{array} \right) \] and then a feedback is performed, %
\tt F %
\rm must be of size %
\tt (m+p,n) %
\rm 
\ignore{
( x is in R\^{}n , y in R\^{}p, u in R\^{}m ). 
}
 ( $x$ is in $R^n$, $y$ in $R^p$, $u$ in $R^m$ ).%
\tt right %
\rm and %
\tt left %
\rm have the following property:
\begin{verbatim}
	 Sl1 =  left*sysdiag(sys,eye(p,p))*right 
\end{verbatim}
\item[{\tt -}]
When %
\tt flag %
\rm is used and %
\tt flag=2 %
\rm a feedback (%
\tt F %
\rm must be of 
size %
\tt (m,n)%
\rm ) is performed and then the above output injection is applied.
%
\tt right %
\rm and %
\tt left %
\rm have 
the following property:
\begin{verbatim} 
	 Sl1 = left*sysdiag(sys*right,eye(p,p)))
\end{verbatim}

\end{scitem}% end Env
\Shead{EXAMPLE}
\begin{verbatim}
Sl=ssrand(2,2,5); trzeros(Sl)       // zeros are invariant:
Sl1=ss2ss(Sl,rand(5,5),rand(2,5),rand(5,2)); 
trzeros(Sl1), trzeros(rand(2,2)*Sl1*rand(2,2))
// output injection [ A + GC, (B+GD,-G)]
//                  [   C   , (D   , 0)]
p=1,m=2,n=2; sys=ssrand(p,m,n);

// feedback (m,n)  first and then output injection.

F1=rand(m,n);
G=rand(n,p);
[sys1,right,left]=ss2ss(sys,rand(n,n),F1,G,2);

// Sl1 equiv left*sysdiag(sys*right,eye(p,p)))

res=clean(ss2tf(sys1) - ss2tf(left*sysdiag(sys*right,eye(p,p))))

// output injection then feedback (m+p,n) 
F2=rand(p,n); F=[F1;F2];
[sys2,right,left]=ss2ss(sys,rand(n,n),F,G,1);

// Sl1 equiv left*sysdiag(sys,eye(p,p))*right 

res=clean(ss2tf(sys2)-ss2tf(left*sysdiag(sys,eye(p,p))*right))

// when F2= 0; sys1 and sys2 are the same 
F2=0*rand(p,n);F=[F1;F2];
[sys2,right,left]=ss2ss(sys,rand(n,n),F,G,1);

res=clean(ss2tf(sys2)-ss2tf(sys1))
\end{verbatim}
\Seealso{SEE ALSO}
{\verb?projsl?} \pageref{projsl},{\verb? feedback?} \pageref{feedback}




