\documentstyle[11pt]{article}
             \textheight=660pt 
             \textwidth=15cm
             \topmargin=-27pt 
             \oddsidemargin=0.7cm
             \evensidemargin=0.7cm
             \marginparwidth=60pt
             \title{Biblioth\`eque Basile} 
             \author{J.Ph. Chancelier\thanks{Cergrene. Ecole Nationale des Ponts et Chauss\'ees, La Courtine  93167 Noisy le Grand C\'{e}dex }}
	
\begin{document}
	
\maketitle	
\section{Introduction}
Nous d\'ecrivons ici les biblioth\`eques de macro rajout\'ees dans Basile 
et permettant d'utiliser la biblioth\`eque graphique \'ecrite en C 
appell\'ee ``Missile'' (JPh Chancelier). Cette biblioth\`eque est pour l'instant 
 interfac\'ee en utilisant les primitive interf et fort de Basile. 
Les macros acc\'essibles dans Basile sont d\'ecrites ci-dessous, elles 
 utilisent toutes fort pour s'interfacer avec une fonction C remplissant la m\^eme fonction.
\begin{describe}
\item[fonction g\'en\'erales de gestion des \'ecrans et des contextes graphiques]
 \begin{enumerate}
  \item driver   : choix du driver 
  \item xinit    : initialisation du driver 
  \item xclear   : efface la fen\^etre courante ou une fenetre de numero donne
  \item xpause   : attente d'un caract\`ere au clavier 
  \item xselect  : choix d'une fen\^etre 
  \item xclick        : choix d'un point \`a la souris 
  \item xclea    : clear a specific zone 
  \item xload    : dynamic load of the X-11 driver (Hacker)
	\item   xend     : fin d'une session graphique 
	\item   xset     : fixe des valeurs du contexte graphique.
	\item   xget     : pour obtenir des valeurs du contexte graphique.
	\item   xlfont   : charge une nouvelle famille de font
\end{enumerate}
\item[Fonction de dessin]
\begin{enumerate]
	\item   xpoly,xpolys,xfpoly,xfpolys   : dessin ou remplissage d'un ou de plusieurs polygones.
	\item   xsegs    : dessins d'un ensemble de segments.
	\item   xarrows  : dessins d'un ensemble de fl\`eches.
	\item   xrect, xfrect, xrects  : 
	\item   xarc,   xfarc, xarcs    :dessin ou remplissage d'un ou plusieurs cercle ou arc de cercle. 
	\item   xstring  : dessin d'une chaine de caractere 
	\item   xstringl : calcul d'un rectangle entourant une chaine de caracteres  
	\item   xnumb    : dessin d'un ensemble de nombres 
	\item   xaxis    : dessin d'un ``axe''
\end{enumerate}
\item[Fonction de haut niveau]
\begin{enumerate}
\item     fplot3d, plot3d  :  dessin 3D d'une surface 
\item     param3d :  dessin 3D d'une courbe param\'etrique
\item     plot2d,plot2d1,plot2d2,plot2d3,plo2d4  :  dessin 2D de courbes
\item     fchamp, champ  : champ de vecteur dans le plan 
\item     fcontour, fcontour: courbes de niveau d'une surface.
\item     xgrid   :  rajout d'une grille sur un dessin 2D 
\item     xtitle  :  rajout d'un titre sur un dessin 2D 
\item     xchange :  transformation de syst\`eme de coordonn\'ees 
\item     errbar  : rajout de barres d'erreur sur un plot2d 
\end{enumerate}
\end{describe}

\section{fonction g\'en\'erales de gestion des \'ecrans et des contextes graphiques}
\begin{verbatim}
driver(dr_name)
\end{verbatim}
Choisit un driver Graphique : \verb+dr_name+ peut prendre les valeurs 
\begin{itemize}
\item X11 pour XWindow
\item Pos pour Postscript
\item Fig pour Xfig
\item Rec pour XWindow + m\'emorisation des commandes.
\end{itemize}
Seuls X11 et Pos sont compl\'etement finis.

\begin{verbatim}
xinit(name)
\end{verbatim}
Une fois un driver choisit cette fonction doit \^etre appell\'ee.
 Pour XWindow \verb+name+ doit \^etre un nom de display par exemple 
 ``unix:0.0'', pour Postscript ou Xfig \verb+name+ est un nom de fichier. 
 Exemple d'utilisation~:
\begin{verbatim}
driver(``X11''); 
xinit('unix:0.0') ; // cree une fenetre graphique
...
driver(``Pos'');    
xinit('file'); // ouver un fichier pour le code Postscript
...
xend(); // ferme le fichier Postscript 
...     // les ordres graphiques qui suivent iront a l'ecran 
...     // si l'on n'ouvre pas un nouveau fichier 
driver(``X11'');
...    // on utilise a nouveau XWindow il est inutile de rappeller 
       // xinit car on a deja une fenetre graphique
\end{verbatim}
\begin{verbatim}
xend()
\end{verbatim}
Pour Postscript ou Xfig cette fonction ferme le fichier graphique courant
\begin{verbatim}
xclear();	
\end{verbatim}
pour XWindow : efface la fenetre graphique courante, pour les autres drivers 
 cette fonction ne fait rien.
\begin{verbatim}
xpause()	
\end{verbatim}
Pour XWindow cette fonction attend que l'utilisateur tape sur une touche au 
 clavier pour continuer.
\begin{verbatim}
xselect()	
\end{verbatim}
Pour XWindow : cette fonction met au premier plan la fenetre graphique courante
\begin{verbatim}
<c,x,y>=xclick()	
\end{verbatim}
Pour Xwindow cette fonction attend que l'utilisateur clique dans la fenetre 
graphique courante. elle renvoit le num\'ero du bouton appuy\'e et les coordonn\'ees 
du point ou l'on a cliqu\'e la souris en pixel.

\begin{verbatim}
<>=xarc(x,y,w,h,a1,a2)
\end{verbatim}
Dessine une ellipse ou une portion d'ellipse contenue dans le rectangle 
 d\'efini par (x,y,w,h). (x,y) sont les coordonn\'ees du point en haut � gauche du rectangle 
 w est la longuer et h la hauteur, toutes ces dimensions sont donn\'ees en pixel. 
 a1 et a2 sont deux nombre r\'eel, la portiob d'ellipse ou de cercle est dessin\'ee
  entre les angle a1/64 en degr\'e et a2/64  en degr\'e. Le trac\'e se fait avec 
  le style courant (cfre xset)

\begin{verbatim}
<>=xfarc(x,y,w,h,a1,a2)
\end{verbatim}
Renplit  une ellipse ou une portion d'ellipse contenue dans le rectangle 
 d\'efini par (x,y,w,h). (x,y) sont les coordonn\'ees du point en haut � gauche du rectangle 
 w est la longuer et h la hauteur, toutes ces dimensions sont donn\'ees en pixel. 
 a1 et a2 sont deux nombre r\'eel, la portion d'ellipse ou de cercle est dessin\'ee
  entre les angle a1/64 en degr\'e et a2/64  en degr\'e.
  Le remplissage se fait avec le niveau de gris courant (cfre xset)

\begin{verbatim}
<>=xarcs(arcs,fill)
\end{verbatim}
Dessine ou remplit un ensemble d'ellipse ou de secteurs d'ellipses. 
arcs est une matrice Basile~:  
\begin{verbatim}
arcs= < x,y,w,h,a1,a2 ; x,y,w,h,a1,a2 ; .....> ' 
\end{verbatim}
dont chaque colonne d\'efinit une ellipse (cfre xarc). 
 fill est un vecteur basile de m�me taille que le nombre de colonne 
 de la matrice arcs.  Ce vecteur donne le pattern a utiliser 
  pour le remplissage de l'ellipse, si fill[i] >= blanc +1 l'ellipse est juste 
  trac\'ee avec le style courant 
  sinon elle est remplit avec le pattern fill[i]. blanc est le num\'ero du pattern blanc 
  ( cfre xset)


<>=xpoly(xv,yv,dtype,close)

Dessine un polygone ou une polyligne d\'efinis par les points (xv(i),yv(i))
si close=1  la polyligne sera ferm\'ee; dtype peut prendre deux valeurs 
itemize
item "lines" : on utilise le style de trac\'e courant pour le dessin 
item "marks" : on utilise la marque courante pour le dessin 

<>=xfpoly(xv,yv,close)
<>=xfpoly(xv,yv,close)

remplit un polygone ou une polyligne d\'efinis par les points (xv(i),yv(i))
avec le pattern courant. 
si close=1  la polyligne sera ferm\'ee; 

<>=xpolys(xpols,ypols,draw)
<>=xpolys(xpols,ypols,[draw])

Dessine une famille de polygones. xpolys et ypols sont des matrices 
(p,q) qui d\'efinissent q polygones a p points.  xpols(:,i),ypols(:,i)  
deffinissent les points du polygone i et draw(i) donne alors le mode ou style de 
trac\'e 
si draw(i) >=0 on utilisera la marque num\'ero draw(i) pour le trac\'e
si draw(i) < 0  on utlisera un trac\'e d\'efini par le style abs(draw(i)). 
cfre xset pour plus d'explication. 

<>=xfpolys(xpols,ypols,fill)
<>=xfpolys(xpols,ypols,[fill])
Remplit une famille de polygones. xpolys et ypols sont des matrices 
(p,q) qui d\'efinissent q polygones a p points.  xpols(:,i),ypols(:,i)  
deffinissent les points du polygone i et fill(i) donne alors le mode de 
remplissage.
si fill(i) <= blanc. le pattern max(fill(i),0) sera utilis\'e pour le remplissage
si fill(i) == blanc+1. le polygone est juste trac\'e avec le style de ligne courant 
( si le polgone donn\'e est une polyligne le contour n'est pas automatiquement
ferm\'e
si fill(i) >= blanc+2. le polygone est remplit avec le pattern .... 
puis son contour automatiquement ferm\'e est trac\'e .


<>=xrect(x,y,w,h)
<>=xrect(x,y,w,h)
 Dessine le re
