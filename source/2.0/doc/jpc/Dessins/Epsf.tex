
\section{Fichiers Epsf  ins\'er\'es en \LaTeX} 

\paragraph{}Cette premi\`ere partie traite de l'insertion de dessins r\'ealis\'es avec CricketDraw sur Macintosh. Il suffit de sauvegarder une version de 
votre dessin CricketDraw en format Epsf. 
On supposera que cette version Epsf a pour nom {\bf DesEpsf.epsf} et qu'elle a \'et\'e transf\'er\'ee sous ce nom sur un Sun. 
Vous utilisez alors la commande unix~:
\begin{verbatim}
TeXEpsf 1.0 1.0 DesEpsf.epsf
\end{verbatim}
Vous obtenez alors un fichier {\bf DesEpsf.tex} que vous placerez dans votre 
document \LaTeX\, comme suit~:

\begin{verbatim}
\input DesEpsf.tex
\dessin{Legende du dessin}{deslabel}
\end{verbatim}
Le dessin obtenu est le dessin~(\ref{deslabel}). Le deuxi\`eme argument 
 de la macro dessin est une r\'ef\'erence \LaTeX pour votre dessin (un label). Les arguments 1.0 et 1.0 sont respectivement les facteurs d'homoth\'etie r\'ealis\'es sur le dessin pour l'axe des x et l'axe des y.

\paragraph{Remarques}
\begin{itemize}
\item La commande \verb+TeXEpsf 1.0 1.0 DesEpsf.epsf+ modifie votre fichier \verb+DesEpsf.epsf+ ( en fait uniquement au premier appel \`a \verb+TeXEpsf+). Vous ne pouvez plus alors imprimer directement votre fichier Postscript avec  la commande unix \verb+lpr TDesEpsf.epsf+. Pour rendre \`a votre fichier
 \verb+DesEpsf.epsf+ sa forme originale il suffit de d\'etruire les lignes pr\'ec\'edant la ligne 
\begin{verbatim} 
%!PS-Adobe-2.0 
\end{verbatim}
et de rajouter \`a la fin du fichier la commande \verb+showpage+.
\item En \'editant le fichier {\bf DesEpsf.tex}, on peut supprimer le cadre  en retirant la commande {\bf fbox}. On peut \'egalement changer la taille du dessin en modifiant les valeurs de {\bf hscale} et {\bf vscale} qui sont des facteurs d'\'echelle sur le dessin. On n'oubliera pas de modifier 
 \'egalement les dimensions donn\'ees en argument de la macro {\bf picture}.
\end{itemize}
\input DesEpsf.tex
\dessin{Legende du dessin}{deslabel}

\section{Fichiers Postscript cr\'ees par Lotus}

\paragraph{}Soit {\bf DesLotus.ps} un fichier Postscript cr\'e\'e par Lotus, vous utilisez alors la commande unix~:
\begin{verbatim}
LotusEpsf 1.0 1.0 DesLotus.ps
\end{verbatim}
Vous obtenez alors un fichier {\bf DesLotus.tex} que vous placerez dans votre 
document \LaTeX comme suit~:

\begin{verbatim}
\input DesLotus.tex
\dessin{Dessin Lotus}{deslabel1}
\end{verbatim}
Le dessin obtenu est le dessin~(\ref{deslabel1}). Le deuxi\`eme argument 
 de la macro dessin est une r\'ef\'erence \LaTeX pour votre dessin (un label). Les arguments 1.0 et 1.0 sont respectivement les facteurs d'homoth\'etie r\'ealis\'es sur le dessin pour l'axe des x et l'axe des y.

\input DesLotus.tex
\dessin{Dessin Lotus}{deslabel1}

Les remarques du paragraphe pr\'ec\'edent s'appliquent aussi ici. 
La commande \verb+LotusEpsf+ modifie votre fichier Postscript
 ( en fait uniquement au premier appel \`a \verb+LotusEpsf+).
 Pour rendre \`a votre fichier sa forme originale il suffit de d\'etruire les lignes pr\'ec\'edant la ligne 
\begin{verbatim} 
%!
%%Creator: Lotus
\end{verbatim}
et de decommenter l'appel \`a la macro sp \`a la fin de votre fichier (
 remplacer \verb+%sp+ par \verb+sp+.
 
