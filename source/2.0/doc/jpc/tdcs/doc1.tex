
\documentstyle[modif.article]{article}

\textheight=660pt 
\textwidth=470pt 
\topmargin=-27pt 
\oddsidemargin=0pt 
\evensidemargin=0pt 
\def\cmarg{\hspace{1cm}}
\title{Travaux dirig\'es 1}
\author{J.Ph. Chancelier et M. Cohen de Lara}

\begin{document}
\maketitle
\vspace{2cm}

\begin{itemize}
\item Entrer sous {\em Basile},
\item taper la commande {\bf xinit ('unix:0.0')} pour pouvoir disposer
d'une fen\^etre graphique.
 \end{itemize}

\centerline{{\bf Le fichier des macros de la s\'eance de T.D. 1}}

\def\tit#1{ \begin{center} \fbox{{\bf  #1}} \end{center}}

% Ce fichier est genere par Basile ne pas l'editer 

\def\Mhlp{
\begin{flushleft}
{\sl 
\cmarg //$<$$>$=hlp()\\ 
\cmarg \verb@//@ Travaux diriges 1 : bibliotheque de macros \\ 
\cmarg \verb@//@     \\ 
\cmarg \verb@//@    PREMIERE PARTIE \\ 
\cmarg \verb@//@ part1    :  initialise les donnees de la partie 1\\ 
\cmarg \verb@//@ linear   :  dynamique d'un systeme lineaire \\ 
\cmarg \verb@//@ linper   :  systeme lineaire avec perturbation  quadratique \\ 
\cmarg \verb@//@ cycllim  :  dynamique d'un systeme avec cercle limite \\ 
\cmarg \verb@//@    DEUXIEME PARTIE \\ 
\cmarg \verb@//@ part2    :  initialise les donnees de la partie 2.\\ 
\cmarg \verb@//@ bioreact :  modele de bioreacteur\\ 
\cmarg \verb@//@ mu       : taux specifique de croissance (utilise par bioreact)\\ 
\cmarg \verb@//@    TROISIEME PARTIE \\ 
\cmarg \verb@//@ part3    :  initialise les donnees de la partie 3.\\ 
\cmarg \verb@//@ lincom   :  systeme lineaire commande par feedback lineaire d'etat\\ 
\cmarg \verb@//@            (u=-k*x)\\ 
\cmarg \verb@//@!\\ 
\cmarg 0;\\ 
\cmarg //end}
\end{flushleft}}



\def\Mpartu{
\begin{flushleft}
{\sl 
\cmarg //$<$$>$=part1()\\ 
\cmarg //$<$$>$=part1()\\ 
\cmarg \verb@//@ Macro qui initialise les donnees de \\ 
\cmarg \verb@//@ la partie 1\\ 
\cmarg \verb@//@!\\ 
\cmarg \verb@//@ matrice du systeme lineaire \\ 
\cmarg alin=$<$$-$1,0;0,$-$2$>$;\\ 
\cmarg \verb@//@ donnees pour la perturbation quadratique \\ 
\cmarg qeps=1/10.;\\ 
\cmarg q1linper=qeps$\star$$<$1,0;0,1$>$;\\ 
\cmarg q2linper=q1linper;\\ 
\cmarg rlinper=$<$1,$-$1$>$';\\ 
\cmarg $<$alin,qeps,q1linper,q2linper,rlinper$>$={\bf resume}(...\\ 
\cmarg \hspace{1.2cm}alin,qeps,q1linper,q2linper,rlinper);\\ 
\cmarg //end }
\end{flushleft}}



\def\Mlinear{
\begin{flushleft}
{\sl 
\cmarg //$<$xdot$>$=linear(t,x)\\ 
\cmarg //$<$xdot$>$=linear(t,x)\\ 
\cmarg \verb@//@------------------------------------------------\\ 
\cmarg \verb@//@ un systeme lineaire xdot=alin*x\\ 
\cmarg \verb@//@ la matrice du systeme est une variable globale\\ 
\cmarg \verb@//@ alin \\ 
\cmarg \verb@//@-----------------------------------------------\\ 
\cmarg \verb@//@!\\ 
\cmarg xdot=alin$\star$x,\\ 
\cmarg //end}
\end{flushleft}}



\def\Mlinper{
\begin{flushleft}
{\sl 
\cmarg //$<$xdot$>$=linper(t,x)\\ 
\cmarg //$<$xdot$>$=linper(t,x)\\ 
\cmarg \verb@//@-----------------------------------\\ 
\cmarg \verb@//@ systeme lineaire avec perturbation \\ 
\cmarg \verb@//@ quadratique \\ 
\cmarg \verb@//@------------------------------------\\ 
\cmarg \verb@//@!\\ 
\cmarg xdot= alin$\star$x+(1/2)$\star$qeps$\star$$<$(x')$\star$q1linper$\star$x;(x')$\star$q2linper$\star$x$>$+rlinper\\ 
\cmarg //end}
\end{flushleft}}



\def\Mcycllim{
\begin{flushleft}
{\sl 
\cmarg //$<$xdot$>$=cycllim(t,x)\\ 
\cmarg //$<$xdot$>$=cycllim(t,x)\\ 
\cmarg \verb@//@-----------------------------------\\ 
\cmarg \verb@//@ xdot=a*x+qeps(1-||x||**2)x\\ 
\cmarg \verb@//@------------------------------------\\ 
\cmarg \verb@//@!\\ 
\cmarg xdot= $<$ 0 ,$-$1 ; 1 , 0$>$$\star$x+ qeps$\star$(1$-$(x'$\star$x))$\star$x;\\ 
\cmarg //end}
\end{flushleft}}



\def\Mpartd{
\begin{flushleft}
{\sl 
\cmarg //$<$$>$=part2()\\ 
\cmarg //$<$$>$=part2()\\ 
\cmarg \verb@//@ Initialisation de la partie 2\\ 
\cmarg \verb@//@  bioreact \\ 
\cmarg \verb@//@!\\ 
\cmarg k=2.0\\ 
\cmarg debit=1.0\\ 
\cmarg x2in=3;\\ 
\cmarg $<$k,debit,x2in$>$={\bf resume}(k,debit,x2in);\\ 
\cmarg //end}
\end{flushleft}}




\def\Mbioreact{
\begin{flushleft}
{\sl 
\cmarg //$<$xdot$>$=bioreact(t,x)\\ 
\cmarg //$<$xdot$>$=bioreact(t,x)\\ 
\cmarg \verb@//@ modele de bioreacteur\\ 
\cmarg \verb@//@   x(1): concentration de biomasse \\ 
\cmarg \verb@//@   x(2): ''            de sucre \\ 
\cmarg \verb@//@!\\ 
\cmarg xdot(1)=mu(x(2))$\star$x(1)$-$ debit$\star$x(1);\\ 
\cmarg xdot(2)=$-$k$\star$mu(x(2))$\star$x(1)$-$debit$\star$x(2)+debit$\star$x2in;\\ 
\cmarg //end}
\end{flushleft}}



\def\Mmu{
\begin{flushleft}
{\sl 
\cmarg //$<$y$>$=mu(x)\\ 
\cmarg //$<$y$>$=mu(x)\\ 
\cmarg \verb@//@ mu : taux specifique de croissance \\ 
\cmarg \verb@//@!\\ 
\cmarg y=x/(1+x);\\ 
\cmarg //end }
\end{flushleft}}



\def\Mpartt{
\begin{flushleft}
{\sl 
\cmarg //$<$$>$=part3()\\ 
\cmarg //$<$$>$=part3()\\ 
\cmarg \verb@//@ Initialisation de la partie 3\\ 
\cmarg \verb@//@ lincom portfeed champfeed \\ 
\cmarg \verb@//@!\\ 
\cmarg a={\bf eye}(2);\\ 
\cmarg b=$<$1,1$>$';\\ 
\cmarg $<$a,b$>$={\bf resume}(a,b);\\ 
\cmarg //end}
\end{flushleft}}



\def\Mlincom{
\begin{flushleft}
{\sl 
\cmarg //$<$xdot$>$=lincom(t,x,k)\\ 
\cmarg \verb@//@ systeme lineaire commande par \\ 
\cmarg \verb@//@ feedback lineaire d'etat (u=-k*x)\\ 
\cmarg \verb@//@!\\ 
\cmarg xdot= a$\star$x +b$\star$($-$k$\star$x);\\ 
\cmarg //end }
\end{flushleft}}





\Mhlp

\section{Premi\`ere partie : stabilit\'e}
\tit{part1}

En tapant la commande {\bf part1()}, vous chargez des valeurs par d\'efaut
pour les macros de la premi\`ere partie du T.D.
\medskip

\Mpartu

\tit{linear}

Cette macro d\'efinit le champ de vecteurs lin\'eaire $f(x) = alin \star x$.
\medskip

\Mlinear

\centerline{{\sc Questions}}

\begin{itemize}
\item Taper la commande {\bf fchamp(linear,0,-5:1:5,-5:1:5)}. 
Visualiser quelques trajectoires gr\^ace \`a la commande {\bf portrait(linear)} (on pourra prendre 100 pour le nombre de points et 0.1 pour le pas).
\item Taper {\bf help spec}, puis {\bf spec(alin)}. Commenter.
\item Choisir d'autres matrices {\bf alin} en tapant une commande du
type  {\bf alin}= $<1 \;  2;3 \;  4>$. Quels sont les types de comportements
possibles pour les trajectoires au voisinage de 0~?
\end{itemize}

\tit{linper}

Cette macro d\'efinit un champ de vecteurs non lin\'eaire, perturb\'e du champ 
pr\'ec\'edent.
\medskip

\Mlinper

\centerline{{\sc Questions}}

\begin{itemize}
\item Taper la commande {\bf fchamp(linper,0,-5:1:5,-5:1:5)}. 
 Visualiser quelques trajectoires
gr\^ace \`a la commande {\bf portrait(linper)}.
\item Prendre la matrice {\bf alin}= $<0 \; 1 ; -1 \; 0 >$. 
Faire varier la perturbation en tapant une commande du
type  {\bf qeps}= $5e-02$. Quels sont les ph\'enom\`enes
observ\'es lorsque {\bf qeps} cro\^{\i}t~?
\end{itemize}

\tit{cycllim}

Cette macro d\'efinit un champ de vecteurs qui admet un cycle limite.
\medskip

\Mcycllim

\centerline{{\sc Questions}}

\begin{itemize}
\item Taper la commande {\bf fchamp(cycllim,0,-2:0.2:2,-2:0.2:2)}. 
 Visualiser quelques trajectoires
gr\^ace \`a la commande {\bf portrait(cycllim)}.
\item D\'ecrire les ph\'enom\`enes et les objets g\'eom\'etriques
observ\'es. Que se passe-t-il lorsque le point initial d'une
trajectoire est situ\'e sur le cercle limite~? Et en son voisinage~?
\item Changer le signe de {\bf qeps} et observer (on pourra prendre {\bf qeps =-0.5}, nombre de points=150 et pas=0.05). Que peut-on en d\'eduire
sur la robustesse du syst\`eme pour\\ {\bf qeps} $=0$~?
 
\end{itemize}

\section{deuxi\`eme partie : mod\`ele de bior\'eacteur }

\tit{part2}

En tapant la commande {\bf part2()}, vous chargez des valeurs par d\'efaut
pour les macros de la deuxi\`eme partie du T.D.
\medskip
\Mpartd

\tit{bioreact}

Cette macro d\'efinit un champ de vecteurs qui mod\'elise la dynamique
d'un bior\'eacteur.
\medskip

\Mbioreact
\medskip

\centerline{{\sc Questions}}

\begin{itemize}
\item Taper la commande {\bf fchamp(bioreact,0,0:0.2:4,0:0.2:4)}.  
Visualiser quelques trajectoires
gr\^ace \`a la commande {\bf portrait(bioreact)}.
\item Combien y a-t-il de points d'\'equilibre~?
Quelle est leur nature~? ( utiliser la macro {\bf tangent} pour obtenir le 
 syst\`eme lin\'earis\'e)
\item Changer progressivement la valeur du d\'ebit (commande)
et observer l'\'evolution des points d'\'equilibre.
\item V\'erifier th\'eoriquement ces r\'esultats.
\end{itemize}

\tit{mu}

Cette fonction intervient dans l'expression du champ {\bf bioreact}.
\medskip

\Mmu

\section{troisi\`eme partie : feedback lin\'eaire}

\tit{part3}

En tapant la commande {\bf part3()}, vous chargez des valeurs par d\'efaut
pour les macros de la troisi\`eme partie du T.D.
\medskip

\Mpartt

\tit{lincom}

Cette macro d\'efinit un syst\`eme lin\'eaire command\'e par
un feedback lin\'eaire d'\'etat $u=-K \star x$.
\medskip

\Mlincom

\begin{itemize}
\item Taper la commande {\bf fchamp(list(lincom,$<$0,0$>$),-5:1:5,-5:1:5)},
 pour visualiser le champ de vecteur original ( la notation {\bf list(lincom,k)} permet de sp\'ecifier une valeur de la matrice de gain k) . 
Quelle est la nature du point d'\'equilibre $0$~?
\item Changer la valeur du gain $K$ du feedback de sorte que le
point d'\'equilibre $0$ soit asymptotiquement stable. utilisez {\bf fchamp(lisp(lincom,k),\ldots)} et {\bf portrait(list(lincom,k))} pour la visualisation du champ et des portraits de phase.
\item Donner un moyen th\'eorique de trouver des matrices de gain
$K$ telles que $0$ soit un point asymptotiquement stable du syst\`eme
 lin\'eaire boucl\'e.
\item Reprendre les questions avec la matrice diagonale 
\[
	a =\left( \begin{array}{cc} 1 & 0 \\ 0 & 2 \end{array} \right)
\]
\end{itemize}

\end{document}
