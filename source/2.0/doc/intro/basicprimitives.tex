\chapter{Basic Primitives}
\label{ch5}
This chapter briefly describes some basic primitives of $\Psi$lab.
More precise information is given in the manual 
(see the directory {\tt SCIDIR/man/LaTex-doc}).
\section{The Environment and Input/Output}
	In this chapter we describe the most important aspects
of the environment of $\Psi$lab, how to automatically
perform certain operations when entering $\Psi$lab,
and how to read and write data
to and from the $\Psi$lab environment.

\subsection{The Environment}
\label{s5.1}
\index{environment}

	At the moment that the user calls $\Psi$lab the environment
of $\Psi$lab is loaded with a certain number of variables and primitives.
The command {\tt who}\index{who@{\tt who}} lists the variables 
which are currently available in $\Psi$lab.  

The {\tt who} command also indicates how many elements and variables
are available for use.  The user can obtain on-line help on any of 
the functions listed by what by typing {\tt help}\index{help@{\tt help}}.

	Variables can be saved in a external binary
file by using {\tt save}\index{save@{\tt save}}.  
Similarly, variables previously saved can be
reloaded into $\Psi$lab using {\tt load}\index{load@{\tt load}}.

Notice that after the command {\tt clear}\index{clear@{\tt clear}} 
that the variables {\tt x} and
{\tt y} no longer exist in the environment.  
The command {\tt save} without any variable arguments saves the entire 
$\Psi$lab environment.  Similarly,
the command {\tt clear} used
without any arguments clears all of the variables, macros, and libraries
in the environment.

	Macros which exist in files can be loaded by using 
{\tt getf}\index{getf@{\tt getf}}.  

Libraries of macros are loaded
using {\tt lib}\index{lib@{\tt lib}}\index{libraries}.  

The list of macros available in the library can be had by using {\tt disp}.

\subsection{Startup Commands by the User}
\label{s5.2}
\index{startup by user}
\index{starup.sci@{\tt startup.sci}}

	When $\Psi$lab is called the user can automatically load
into the environment macros, libraries, variables, and perform
commands using the the file {\tt .scilab} in the home directory.  
This is particularly useful when the user wants to run $\Psi$lab programs
in the background (such as in batch mode).  Another useful aspect
of the {\tt .scilab} file is when the certain macros or libraries
are often used.  In this case the command {\tt getf} can be used
in the {\tt .scilab} file to automatically load the desired 
macros and libraries whenever $\Psi$lab is invoked.

\subsection{Input and Output}
\label{s5.3}
\index{input}
\index{output}

	Although the commands {\tt save} and {\tt load} are
convenient, one has much more control over the transfer of
data between files and $\Psi$lab by using the commands 
{\tt read}\index{read@{\tt read}}
and {\tt write}\index{write@{\tt write}}.  
These two commands work similarly to the
read and write commands found in Fortran.  The syntax of these
two commands is as follows.  
\begin{verbatim}
 
--> x=[1 2 %pi;%e 3 4]
 x         =
 
!   1.           2.    3.1415927 !
!   2.7182818    3.    4.        !
 
--> write('x.dat',x)
 
--> clear x
 
--> xnew=read('x.dat',2,3)
 xnew      =
 
!   1.           2.    3.1415927 !
!   2.7182818    3.    4.        !
\end{verbatim}
Notice that the {\tt read} specifies the number of rows and columns
of the matrix {\tt x}.
Complicated formats can be specified. 

\section{Help}
On-line help is available either by clicking on the {\tt help}
button or by entering {\tt help item} (where {\tt item} is usually the 
name of a macro or primitive). The help facility is based on the 
Unix {\tt xman} command. {\tt apropos item} looks for {\tt item} 
in the {\tt whatis} file. This facility is equivalent to the Unix 
{\tt whatis} command. To add a new item in the manual is easy: just
look at the {\tt SCIDIR/man/Man-General} directory. Note that
the directory {\tt SCIDIR/man/Man-Contrib} is for the inclusion
of new help item contributions. 
The $\Psi$lab \LaTeX  manual is automatically obtained from the
manual items by a {\tt Makefile}. 
See the directory {\tt SCIDIR/man/Latex-doc}. Note that the command
{\tt manedit} opens an help file with an editor (default editor
is {\tt emacs}).


\section{Non-Linear Calculation}
\label{ch6}
\index{non-linear calculation}

	$\Psi$lab has several powerful non-linear primitives for simulation
or optimization.
\index{simulation}
\index{optimization}
\subsection{Externals}
\index{external}
An "external" is a macro or Fortran routine which is used as an argument
of some high-level primitives (such as {\tt ode}, {\tt optim}...).
The calling sequence of the external (macro or routine) is imposed by
the high-level primitive which sets the arguments of the external.
For example the external macro {\tt costfunc} is an argument of 
the {\tt optim} primitive. Its calling sequence must be: 
{\tt [f,g,ind]=costfunc(x,ind)} as imposed by the {\tt optim} primitive.
The following non-linear primitives in $\Psi$lab need externals: 
{\tt ode}, {\tt optim}, {\tt impl}, {\tt dassl}, {\tt intg}, {\tt fsolve}
For problems where computation time is important, it is recommended
to code the externals as Fortran subroutines. Examples of such
subroutines are given in the directory {\tt SCIDIR/routines/default}.
When such a subroutine is written it must be linked to $\Psi$lab.
This link operation can be ``dynamic'' (see {\tt link}). It is
also possible to introduce the code in a specific interface (see e.g.
{\tt fydot.f} in {\tt SCIDIR/routines/default} and rebuild a new $\Psi$lab
by a {\tt make}.

\subsection{Non-Linear primitives}
The main simulation primitives
in $\Psi$lab allow integrating a broad range of non-linear system functions of
both explicit and implicit nature. It is also possible to integrate
a system of differential equations with a stopping time: 
integration is performed until the trajectory reaches a given surface.
In the following we illustrate the basic {\tt ode} syntax by example.

	The example illustrates the basic usage of {\tt ode}.
Here we simulate the motion of a double pendulum.  
Let $\theta_1$ and $\theta_2$ be the angles that the first
and second pendulum elements make with the vertical and
$r_1$, $r_2$, $m_1$
and $m_2$ be the lengths and masses of 
the respective pendulum elements.  
The non-linear
differential equation which describes the motion of the pendulum
is:
\begin{equation}
\left[\begin{array}{cc}
mr_1&m_2r_2\cos\theta\\
r_1\cos\theta_1&r_2
\end{array}\right]
\left[\begin{array}{c}
\ddot{\theta}_1\\
\ddot{\theta}_2
\end{array}\right]
=
\left[\begin{array}{c}
-m_2r_2\dot{\theta}_2^2\sin\theta\\
r_1\dot{\theta}_1^2\sin\theta
\end{array}\right]
-
\left[\begin{array}{c}
m\sin\theta_1\\
\sin\theta_2
\end{array}\right]
g
\label{e6.1}
\end{equation}
where $g$ is the acceleration due to gravity, $m=m_1+m_2$, and 
$\theta=\theta_1+\theta_2$.
The following $\Psi$lab session simulates the movement
of the pendulum for $1.5$ seconds where $g=9.8m/s$,
$r_1=r_2=1m$, and $m1=m2=1kg$.  The initial conditions are
$\theta_1=\pi/2$, $\theta_2=\pi/2$, $\dot{\theta}_1=0$, and 
$\dot{\theta}_2=0$.
\begin{verbatim}
--> m1=1;m2=1;r1=1;r2=1;
 
--> g=9.8;
 
--> t0=0;
 
--> t=0:.1:1.5;
 
--> z0=[%pi/2;%pi/2;0;0];
 
--> getf('../macros/dpend.sci','c');
 
--> z=ode(z0,t0,t,dpend);
 
--> pp(z)
\end{verbatim}
The result of the above session is plotted in Figure~\ref{f6.1}.
Note that {\tt ode} takes four arguments: a vector initial condition
{\tt z0}, the initial time {\tt t0}, a vector of times {\tt t}
for which the integrated values {\tt z} are desired, and the macro
name {\tt dpend} which exists in the environment of $\Psi$lab and which
calculates the value of the derivative {\tt zd} give the time {\tt t} 
and the value of {\tt z} at this time.

\input{\Figdir d6.1.tex}
\dessin{Simulation of a Double Pendulum}{f6.1}

Note that the {\tt dpend} macro calculates the derivative of
the state of the double pendulum following (\ref{e6.1}).

\begin{verbatim}
function [zdot]=dpend(time,z)
   th1=z(1);th2=z(2);th1d=z(3);th2d=z(4);
   s12=sin(th1-th2);c12=cos(th1-th2);
   m12=m1+m2;s1=sin(th1);s2=sin(th2);
   mat=[m12*r1 m2*r2*c12;...
        r1*c12 r2];
   vec=-[m12*g*s1+m2*r2*th2d*th2d*s12;...
         g*s2-r1*th1d*th1d*s12];
   res=mat\vec;
   th1dd=res(1);th2dd=res(2);
   zdot=[th1d;th2d;th1dd;th2dd];

function []=pp(z)
   th1=z(1,:);th2=z(2,:);
   rc1=r1*cos(th1);rc2=r2*cos(th2);
   rs1=r1*sin(th1);rs2=r2*sin(th2);
   rs12=rs1+rs2;rc12=rc1+rc2;
   rect=[-2.1,-3.1,2.1,1.1];
   for k=1:maxi(size(th1));
      plot2d([0 rs1(k) rs12(k)]',-[0 rc1(k) rc12(k)]',...
		[-1],"011",' ',rect,[10,3,10,3]);
   end,

\end{verbatim}

	The $\Psi$lab ``external'' macro which calculates the derivative
(in this example {\tt dpend}) must have a certain format.
This format is as follows
\begin{verbatim}
function [xdot]=f(time,x)
   .
   .
   .
\end{verbatim}
The format specifies that the argument list contain two variables.
The first variable {\tt time} is a scalar which represents time, the second
variable {\tt x} is a vector (or matrix) of the state values.  The returned
value {\tt xdot} is the calculated derivative and has the same dimensions
as {\tt x}.  For differential equations of order greater than one
state augmentation is used. 

The optimization primitive 
{\tt optim}\index{optim@{\tt optim}} in $\Psi$lab
is capable finding locally optimum solutions to a wide range
of non-linear problems.  In the following we illustrate 
the basic {\tt optim} syntax by example. 

	The example used to illustrate the use of {\tt optim} is
the classic ray-tracing problem of a plane wave traveling from
one material to another.  Here we pose the problem slightly differently.
Imagine a lifeguard on a beach and a person calling for help 
in the water.  The lifeguard's speed on land is different from
his speed in water and, consequently, he would like to choose the
point along the border between beach and water which minimizes
the travel time to the calling person.  This problem can be analytically
resolved when the border between beach and water is a straight
line.  The problem may not be analytically resolvable when the function
of the border is more complicated than a straight line.

	The problem is posed as follows.  Given that the lifeguard 
is at the origin of a Cartesian coordinate system, that the person
to be saved is at the coordinates $(x,y)$, and that the border
between the beach and the water can be described by the function
$d(x)$ we have that the lifeguard's time of travel is
\begin{equation}
T(p)=\frac{\sqrt{p^2+d^2(p)}}{v_b}+\frac{\sqrt{(x-p)^2+(y-d(p))^2}}{v_w}
\label{e6.2}
\end{equation}
where $p$ is the $x$-axis coordinate of the point along the border through
which the lifeguard passes and
where $v_s$ and $v_w$ are the respective 
speeds of the lifeguard on the beach and in the water.
Here we assume that the $x$ coordinate give by $p$ uniquely defines
a point on the border between the beach and the water.

	We know that a locally optimum solution must make 
$dT(p)/dt=0$ and we can calculate the derivative of $T$ as
a function of the derivative of $d$
\begin{equation}
\frac{d}{dt}T(p)=\frac{p+dd_p}{v_s\sqrt{p^2+d^2}}
                 +\frac{(p-x)+(d-y)d_p}{v_w\sqrt{(x-p)^2+(y-d)^2}}
\label{e6.3}
\end{equation}
where $d_p$ is the partial of $d$ with respect to $p$.  The following
$\Psi$lab macros make the calculations in (\ref{e6.2}) and (\ref{e6.3}).
\begin{verbatim}
function [t,t_p,ind]=swim(p,ind)
   [d,d_p]=dpfun(p);
   as=sqrt(d^2+p^2);
   aw=sqrt((x-p)^2+(y-d)^2);
   t=as/vs+aw/vw;
   t_p=(p+d*d_p)/(vs*as)+((p-x)+(d-y)*d_p)/(vw*aw);


function [d,d_p]=dpfun(p);
   d=1.75-exp(log(1.55)*p);
   d_p=-log(1.55)*exp(log(1.55)*p);
\end{verbatim}
where the functions $d$ and $d_p$ are calculated by the macro {\tt dpfun}.
These two macros are used in the following $\Psi$lab session by the
primitive {\tt optim} to find the optimum point $p$.
\begin{verbatim}
 
--> x=1;y=1;vs=5;vw=1;
 
--> getf('../macros/swim.sci');
 
--> getf('../macros/plotopt.sci');
 
--> [ts,ps,tps]=optim(swim,0)
 end of optimization
 
 tps       =
 
    4.580D-16  
 ps        =
 
    0.6144065  
 ts        =
 
    0.8303513  
 
--> popt(ps);
\end{verbatim}
Here it can be seen that the primitive {\tt optim} takes two 
arguments and returns three values.  The first argument is the
name of the $\Psi$lab macro which calculates the value of time (the cost)
and the derivative of the time with respect to the value of $p$.
The second argument is a first guess for the value of $p$.
The returned values are, respectively, the optimal cost {\tt ts}
associated to the value of $p$ which is the value in {\tt ps}, and,
finally, the value of the gradient at this point which is in
{\tt tps}.

	The optimal path for the lifeguard is illustrated in
Figure~\ref{f6.2}.  The curved line represents the shape of the
shoreline and the dotted line represents the lifeguards optimal
path.
%
\input{\Figdir d6.2.tex}
\dessin{Optimal Lifeguard Path}{f6.2}
%


\section{Fortran or C Interface}
\index{Fortran}
\index{interfacing fortran}
\index{C}

 $\Psi$lab in a Unix environment can be easily interfaced with Fortran 
or C subroutines. 
The simplest way is using the dynamic link primitive 
{\tt link}\index{link@{\tt link}} and the subroutine
call primitive {\tt fort}\index{fort@{\tt fort}}. Details are
given in the $\Psi$lab manual.

Fortran subprograms can also be linked to $\Psi$lab by using an external 
Fortran program call {\tt interf}\index{interf@{\tt interf}}. See
the program {\tt interf.f} in the directory {\tt SCIDIR/default} where
examples are given.

These facilities are particularly useful
for simulation and optimization of non-linear problems or for
introducing programs automatically generated by MAPLE.  There are
two steps for interfacing dynamically a subroutine written 
in fortran with $\Psi$lab. 
These two steps are demonstrated in what follows.

	To link a fortran subroutine with $\Psi$lab it is first necessary
to have a file which contains the executable object file of the subroutine.
We introduce the Fortran subroutine {\tt fibb} which calculates the
$n^{th}$ Fibbonacci number, $f_n$ (i.e., the sequence $0,1,1,2,3,5,8,\ldots$
where $f_n=f_{n-1}+f_{n-2}$).
\begin{verbatim}
      subroutine fibb(fn,n)
      n0=0
      n1=1
      if(n.ge.3)then
      do 10 k=1,n-2
         fn=n0+n1
         n0=n1
         n1=fn
 10   continue
      else
         fn=n-1
      endif
      return
      end
\end{verbatim}
Assuming that the object file for {\tt fibb} is available in the
file {\tt fibb.o} the following $\Psi$lab session shows how to use
the subroutine from $\Psi$lab.
\begin{verbatim}
-->unix(" make fibb.o");
 
-->link('fibb.o','fibb')
 
-->link()
 ans       =
 
 fibb                   
 
-->n=6
 n         =
 
    6.  
 
-->fn=fort('fibb',n,2,'i','out',[1,1],1,'r')
 fn        =
 
    5.  
\end{verbatim}
The primitive {\tt link} normally takes two arguments where the
first argument is the name of the object file and the second argument
is the name of the subroutine call.  Note that the primitive \verb!c_link!
may be used to know  all the previously linked subroutines.  
The use of {\tt fort} to call the subroutine {\tt fibb}
is a bit complicated.  The arguments are divided into four groups.
The first argument in {\tt 'fibb'} is the name of the subroutine called.
The argument {\tt 'out'} divides the remaining arguments into two 
groups.  The group of arguments between {\tt 'fibb'} and {\tt 'out'}
is the list of input arguments, their positions in the call to {\tt fibb},
and their data type.  The group of arguments to the right of {\tt 'out}
are the dimensions of the output variables, their positions in the call
to {\tt fibb}, and their data type.
The possible data types are real, integer, and double precision which
are indicated, respectively, by the strings {\tt 'r'}, {\tt 'i'}, and
{\tt 'd'}.  The positions of the two arguments for {\tt fibb} are
{\tt n} at position $2$ and {\tt fn} at positon $1$ which explains
the above call values.  The vector {\tt [1,1]} indicates that the output
{\tt fn} is a $1\times 1$ matrix.
The routines which are linked to $\Psi$lab can also access internal 
$\Psi$lab variables (see the routine {\tt matz.f}
in the directory {\tt SCIDIR/routines/system2}).
\section{X-window dialog}
It may be convenient to open a specific X-window window for entering
interactively parameters inside a macro or for a demo.
This facility is possible thanks to e.g. the macros \verb!x_dialog!,
\verb!x_choose!, \verb!x_mdialog!, \verb!x_matrix! and \verb!x_message!.
The demos which can be executed by clicking on the {\tt demo} button
provide simple examples of the use of these macros.

\section{Maple interface}
The Maple procedure {\tt maple2scilab} allows to numerically evaluate
any Maple expression in $\Psi$lab.
On input this procedure takes a Maple matrix made of symbolic
expressions. When {\tt maple2scilab} is invoked in Maple,
a Fortran program is automatically generated by Maple 
(via {\tt Macrofort}) together with a $\Psi$lab macro which 
allows to (dynamically) call this subroutine.
The parameters of the macro are the Maple symbols and thus the Maple
matrix can be evaluated numerically in $\Psi$lab. The use of Fortran
is for speed purposes since the formal Maple expression can be
extremely complicated.

\section{System Interconnection}
	The purpose of this section is to illustrate some
of the more sophisticated aspects of $\Psi$lab by way of an example.
The example shows how $\Psi$lab can be used to symbolically represent
the inter-connection of multiple systems which in turn can 
then be used to numerically evaluate the performance of the
inter-connected systems.  The symbolic representation of the
inter-connected systems is done with a macro called {\tt bloc2exp}
\index{bloc2exp@{\tt bloc2exp}}
and the evaluation of the resulting system is done with
{\tt evstr}\index{evstr@{\tt evstr}}.

	The example illustrates the symbolic inter-connection of the 
systems shown in Figure~\ref{fa.1}\index{linear systems!inter-connection of}.  
%
%\begin{figure}
\begin{center}
\begin{picture}(360,204)

\put(80,85){\framebox(40,30){\tt Model}}
\put(150,85){\framebox(40,30){\tt Reg}}
\put(240,85){\framebox(40,30){\tt Proc}}
\put(120,140){\framebox(40,30){\tt FF}}
\put(190,30){\framebox(40,30){\tt Sensor}}
\put(210,100){\framebox(0,0){\tt +}}

\put(35,95){\tt U}
\put(315,95){\tt Y}
\put(315,125){\tt UP}

\put(50,100){\vector(1,0){30}}
\put(120,100){\vector(1,0){30}}
\put(190,100){\vector(1,0){15}}
\put(215,100){\vector(1,0){25}}
\put(210,155){\vector(0,-1){50}}
\put(295,45){\vector(-1,0){65}}
\put(65,155){\vector(1,0){55}}
\put(170,45){\vector(0,1){40}}
\put(225,130){\vector(1,0){85}}
\put(280,100){\vector(1,0){30}}

\put(65,100){\line(0,1){55}}
\put(160,155){\line(1,0){50}}
\put(170,45){\line(1,0){20}}
\put(225,100){\line(0,1){30}}
\put(295,100){\line(0,-1){55}}

\put(210,100){\circle{10}}
\put(48,100){\circle{2}}
\put(310,100){\circle{2}}
\put(310,130){\circle{2}}
\put(65,100){\circle*{2}}
\put(225,100){\circle*{2}}
\put(295,100){\circle*{2}}

\end{picture}
\end{center}
%\caption{Inter-Connected Systems}
\label{fa.1}
%\end{figure}
%
Figure~\ref{fa.1} illustrates the classic regulator problem\index{regulator}
where the block labeled {\tt Proc} is to be controlled
using feedback from the {\tt Sensor} block and {\tt Reg} block.
The {\tt Reg} block compares the output from the {\tt Model}
block to the output from the {\tt Sensor} block to decide how to
regulate the {\tt Proc} block.  There is also a feed-forward
block which filters the input signal, {\tt U} to the {\tt Proc}
block.  The outputs of the system are {\tt Y} and {\tt UP}.

	The system illustrated in Figure~\ref{fa.1} can
be represented in $\Psi$lab by using the macro {\tt bloc2exp}.
The use of {\tt bloc2exp} is illustrated in the following $\Psi$lab
session.
There a two kinds of objects: 'transfer' and 'links'. The example
considered here admits 5 transfers and 7 links.
First the transfer are defined in a symbolic manner. Then links
are defined and an 'interconnected system' is defined as
a specific list. The macro {\tt bloc2exp} evaluates symbolically
the global transfer and {\tt evstr} evaluates numerically
the global transfer function once the systems are given ``values'', i.e.
are defined as $\Psi$lab linear systems.
%
\begin{verbatim}
 
-->model=2;reg=3;proc=4;sensor=5;ff=6;somm=7;
 
-->tm=list('transfer','model');
 
-->tr=list('transfer',['reg(:,1)','reg(:,2)']);
 
-->tp=list('transfer','proc');
 
-->ts=list('transfer','sensor');
 
-->tf=list('transfer','ff');
 
-->tsum=list('transfer',['1','1']);
 
-->lum=list('link','input',[-1],[model,1],[ff,1]);
 
-->lmr=list('link','model output',[model,1],[reg,1]);
 
-->lrs=list('link','regulator output',[reg,1],[somm,1]);
 
-->lfs=list('link','feed-forward output',[ff,1],[somm,1]);
 
-->lsp=list('link','proc input',[somm,1],[proc,1],[-2]);
 
-->lpy=list('link','proc output',[proc,1],[sensor,1],[-1]);
 
-->lsup=list('link','sensor output',[sensor,1],[reg,2]);
 
-->syst=...
 list('blocd',tm,tr,tp,ts,tf,tsum,lum,lmr,lrs,lfs,lsp,lpy,lsup);
 
-->[sysf,names]=bloc2exp(syst)
 names     =
 
 
       names>1
 
 input   
 
       names>2
 
!proc output  !
!             !
!proc input   !
 sysf      =
 
!proc*((eye-reg(:,2)*sensor*proc)\(-(-ff-reg(:,1)*model)))  !
!                                                           !
!(eye-reg(:,2)*sensor*proc)\(-(-ff-reg(:,1)*model))         !
\end{verbatim}
%
Note that the argument to {\tt bloc2exp} is a list of lists.  The 
first element of the list {\tt syst} is actually the character chain
{\tt 'blocd'} which indicates that the list represents a block-diagram
inter-connection of systems.  Each list entry in the list {\tt syst}
represents a block or an inter-connection in Figure~\ref{fa.1}.
The form of a list which represents a block begins with a character
chain {\tt 'transfer'} followed by a matrix of character chains
which give the symbolic name of the block.  If the block is multi-input
multi-output the matrix of character chains must be represented as
is illustrated for the block {\tt Reg}.  

	The inter-connections between blocks is also represented by lists.  
The first element of the list is the character chain {\tt 'link'}.
The second element of the inter-connection is its symbolic name.
The third element of the inter-connection is the input to the connection.
The remaining elements are all the outputs of the connection.
Each input and output to an inter-connection is a vector which
contains as its first element the block number (i.e., the {\tt model}
block is assigned the number $2$).  The second element of the vector
is the port number for the block (for the case of multi-input multi-output
blocks).  If an inter-connection is not attached to anything the value
of the block number is negative (as for example the inter-connection
labeled {\tt 'input'} or is omitted.

	The result of the {\tt bloc2exp} macro is a list of names
which give the unassigned inputs and outputs of the system and
the symbolic transfer function of the system given by {\tt sysf}.
The symbolic names in {\tt sysf} can be associated to polynomials
and evaluated using the macro {\tt evstr}.  This illustrated in the
following $\Psi$lab session.
%
\begin{verbatim}
 
-->s=poly(0,'s');
 
-->ff=1;sensor=1;model=1;
 
-->proc=s/(s^2+3*s+2);
 
-->reg=[1/s 1/s];
 
-->sys=evstr(sysf)
 sys       =
 
!     1 + s          !
!   ----------       !
!             2      !
!   1 + 3s + s       !
!                    !
!              2   3 !
!   2 + 5s + 4s + s  !
!   ---------------- !
!           2   3    !
!     s + 3s + s     !

\end{verbatim}
%
The resulting polynomial transfer function relates the input
of the block system to the two outputs.  Note that the output
of {\tt evstr} is the rational polynomial matrix {\tt sys}
whereas the output of {\tt bloc2exp} is a matrix of character chains.

The symbolic evaluation which is given here is not very efficient
with large interconnected systems. The macro {\tt bloc2ss}
performs the previous calculation in state-space format.
Each system is given now in state-space 
as a {\tt syslin} list or simply as a gain (constant matrix). 
Note {\tt bloc2ss} performs the necessary conversions if this 
is not done by the user. Each system must be given a value before
bloc2ss is called. All the calculations are made in state-space
representation even if the linear systems are given in transfer form.

\section{Converting $\Psi$lab macros to Fortran routines}

$\Psi$lab provides a compiler to transform some $\Psi$lab macros
into Fortran subroutines. The routines which are thus obtained
make use of the routines which are in the Fortran libraries.
All the basic matrix operations are available.

Let us consider the following $\Psi$lab macro:
\begin{verbatim}

function [x]=macr(a,b,n)
z=n+m+n,
c(1,1)=z,
c(2,1)=z+1,
c(1,2)=2,
c(2,2)=0,
if n=1 then,
 x=a+b+a,
else,
x=a+b-a'+b,
end,
y=a(3,z+1)-x(z,5),
x=2*x*x*2.21,
sel=1:5,
t=a*b,
for k=1:n,
 z1=z*a(k+1,k)+3,
end,
t(sel,5)=a(2:4,7),
x=[a b;-b' a']

\end{verbatim}

It is translated into Fortran by the macro {\tt mac2for}.
Each input parameter of the subroutine is described by a list
which contains its type and its dimensions. Here, we have 3
input variables {\tt a,b,c} which are, say, {\tt integer,
double precision, double precision} and of dimensions
{\tt (m,m), (m,m), (1,1)}: this information is gathered
in the following list:
\begin{verbatim}
l=list();
l(1)=list('1','m','m');
l(2)=list('1','m','m');
l(3)=list('0','1','1');
\end{verbatim}
The call to {\tt mac2for} is made as follows:
\begin{verbatim}
comp(macr);
mac2for(macr2lst(macr),l)
\end{verbatim}
The output of this command is a stand-alone Fortran subroutine.
\begin{verbatim}
       subroutine macr(a,b,n,x,m,work,iwork)
c!
c  automatic translation
.
.
.
       double precision a(m,m),b(m,m),x(m+m,m+m),y,z1,24(m,m),work(*)
       integer n,m,z,c(2,2),sel(5),k,iwork(*)
.
.
.      
       call dmcopy(b,m,x(1,m+1),m+m,m,m)
       call dmcopy(work(iw1),m,x(m+1,1),m+m,m,m)
       call dmcopy(work(iw1),m,x(m+1,m+1),m+m,m,m)
       return
c
       end


\end{verbatim}
This routine can be linked to $\Psi$lab and interactively called.

