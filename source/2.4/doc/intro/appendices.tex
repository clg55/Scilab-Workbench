% Copyright INRIA

\appendix
\chapter{Demo session}
We give here the Scilab session corresponding to the first demo.

  \long\def\Checksifdef#1#2#3{%
     \expandafter\ifx\csname #1\endcsname\relax#2\else#3\fi}
  \Checksifdef{Figdir}{\gdef\Figdir{}}{}
  \def\dessin#1#2{
  \begin{figure}[hbtp]
  \begin{center}
  %If you prefer cm use the followin two lines
  %\setlength{\unitlength}{1mm}
  %\fbox{\begin{picture}(105,74)
  \fbox{\begin{picture}(300.0,212.0)
  \special{psfile=\Figdir demo1.epsf hscale=100.0 vscale=100.0}
  \end{picture}}
  \end{center}
  \caption{\label{#2}#1}
  \end{figure}}


\chapter{System interconnexion}

%\section{System Interconnection}
	The purpose of this appendix is to illustrate some
of the more sophisticated aspects of Scilab by the way of an example.
The example shows how Scilab can be used to symbolically represent
the inter-connection of multiple systems which in turn can 
then be used to numerically evaluate the performance of the
inter-connected systems.  The symbolic representation of the
inter-connected systems is done with a function called {\tt bloc2exp}
\index{bloc2exp@{\tt bloc2exp}}
and the evaluation of the resulting system is done with
{\tt evstr}\index{evstr@{\tt evstr}}.

	The example illustrates the symbolic inter-connection of the 
systems shown in Figure~\ref{fa.1}\index{linear systems!inter-connection of}.  
%
\begin{figure}
\begin{center}
\begin{picture}(360,204)

\put(80,85){\framebox(40,30){\tt Model}}
\put(150,85){\framebox(40,30){\tt Reg}}
\put(240,85){\framebox(40,30){\tt Proc}}
\put(120,140){\framebox(40,30){\tt FF}}
\put(190,30){\framebox(40,30){\tt Sensor}}
\put(210,100){\framebox(0,0){\tt +}}

\put(35,95){\tt U}
\put(315,95){\tt Y}
\put(315,125){\tt UP}

\put(50,100){\vector(1,0){30}}
\put(120,100){\vector(1,0){30}}
\put(190,100){\vector(1,0){15}}
\put(215,100){\vector(1,0){25}}
\put(210,155){\vector(0,-1){50}}
\put(295,45){\vector(-1,0){65}}
\put(65,155){\vector(1,0){55}}
\put(170,45){\vector(0,1){40}}
\put(225,130){\vector(1,0){85}}
\put(280,100){\vector(1,0){30}}

\put(65,100){\line(0,1){55}}
\put(160,155){\line(1,0){50}}
\put(170,45){\line(1,0){20}}
\put(225,100){\line(0,1){30}}
\put(295,100){\line(0,-1){55}}

\put(210,100){\circle{10}}
\put(48,100){\circle{2}}
\put(310,100){\circle{2}}
\put(310,130){\circle{2}}
\put(65,100){\circle*{2}}
\put(225,100){\circle*{2}}
\put(295,100){\circle*{2}}

\end{picture}
\end{center}
\caption{Inter-Connected Systems}
\label{fa.1}
\end{figure}
%
Figure~\ref{fa.1} illustrates the classic regulator problem\index{regulator}
where the block labeled {\tt Proc} is to be controlled
using feedback from the {\tt Sensor} block and {\tt Reg} block.
The {\tt Reg} block compares the output from the {\tt Model}
block to the output from the {\tt Sensor} block to decide how to
regulate the {\tt Proc} block.  There is also a feed-forward
block which filters the input signal {\tt U} to the {\tt Proc}
block.  The outputs of the system are {\tt Y} and {\tt UP}.

	The system illustrated in Figure~\ref{fa.1} can
be represented in Scilab by using the function {\tt bloc2exp}.
The use of {\tt bloc2exp} is illustrated in the following Scilab
session.
There a two kinds of objects: ``transfer'' and ``links''. The example
considered here admits 5 transfers and 7 links.
First the transfer are defined in a symbolic manner. Then links
are defined and an ``interconnected system'' is defined as
a specific list. The function {\tt bloc2exp} evaluates symbolically
the global transfer and {\tt evstr} evaluates numerically
the global transfer function once the systems are given ``values'', i.e.
are defined as Scilab linear systems.
%
\begin{verbatim}
 
-->model=2;reg=3;proc=4;sensor=5;ff=6;somm=7;
 
-->tm=list('transfer','model');tr=list('transfer',['reg(:,1)','reg(:,2)']);
 
-->tp=list('transfer','proc');ts=list('transfer','s
ensor');
 
-->tf=list('transfer','ff');tsum=list('transfer',['1','1']);
 
-->lum=list('link','input',[-1],[model,1],[ff,1]);
 
-->lmr=list('link','model output',[model,1],[reg,1]);
 
-->lrs=list('link','regulator output',[reg,1],[somm,1]);
 
-->lfs=list('link','feed-forward output',[ff,1],[somm,2]);
 
-->lsp=list('link','proc input',[somm,1],[proc,1],[-2]);
 
-->lpy=list('link','proc output',[proc,1],[sensor,1],[-1]);
 
-->lsup=list('link','sensor output',[sensor,1],[reg,2]);
 
-->syst=...
 list('blocd',tm,tr,tp,ts,tf,tsum,lum,lmr,lrs,lfs,lsp,lpy,lsup);
 
-->[sysf,names]=bloc2exp(syst)
 names     =
 
 
       names>1
 
 input   
 
       names>2
 
!proc output  !
!             !
!proc input   !
 sysf      =
 
!proc*((eye-reg(:,2)*sensor*proc)\(-(-ff-reg(:,1)*model)))  !
!                                                           !
!(eye-reg(:,2)*sensor*proc)\(-(-ff-reg(:,1)*model))         !
\end{verbatim}
%
Note that the argument to {\tt bloc2exp} is a list of lists.  The 
first element of the list {\tt syst} is (actually) the character string
{\tt 'blocd'} which indicates that the list represents a block-diagram
inter-connection of systems.  Each list entry in the list {\tt syst}
represents a block or an inter-connection in Figure~\ref{fa.1}.
The form of a list which represents a block begins with a character
string {\tt 'transfer'} followed by a matrix of character strings
which gives the symbolic name of the block.  If the block is multi-input
multi-output the matrix of character strings must be represented as
is illustrated by the block {\tt Reg}.  

	The inter-connections between blocks is also represented by lists.  
The first element of the list is the character string {\tt 'link'}.
The second element of the inter-connection is its symbolic name.
The third element of the inter-connection is the input to the connection.
The remaining elements are all the outputs of the connection.
Each input and output to an inter-connection is a vector which
contains as its first element the block number (for instance the {\tt model}
block is assigned the number $2$).  The second element of the vector
is the port number for the block (for the case of multi-input multi-output
blocks).  If an inter-connection is not attached to anything the value
of the block number is negative (as for example the inter-connection
labeled {\tt 'input'} or is omitted.

	The result of the {\tt bloc2exp} function is a list of names
which give the unassigned inputs and outputs of the system and
the symbolic transfer function of the system given by {\tt sysf}.
The symbolic names in {\tt sysf} can be associated to polynomials
and evaluated using the function {\tt evstr}.  This is illustrated in the
following Scilab session.
%
\begin{verbatim}
 
-->s=poly(0,'s');ff=1;sensor=1;model=1;proc=s/(s^2+3*s+2);
 
-->reg=[1/s 1/s];sys=evstr(sysf)
 sys       =
 
!     1 + s          !
!   ----------       !
!             2      !
!   1 + 3s + s       !
!                    !
!              2   3 !
!   2 + 5s + 4s + s  !
!   ---------------- !
!           2   3    !
!     s + 3s + s     !

\end{verbatim}
%
The resulting polynomial transfer function links the input
of the block system to the two outputs.  Note that the output
of {\tt evstr} is the rational polynomial matrix {\tt sys}
whereas the output of {\tt bloc2exp} is a matrix of character strings.

The symbolic evaluation which is given here is not very efficient
with large interconnected systems. The function {\tt bloc2ss}
performs the previous calculation in state-space format.
Each system is given now in state-space 
as a {\tt syslin} list or simply as a gain (constant matrix). 
Note {\tt bloc2ss} performs the necessary conversions if this 
is not done by the user. Each system must be given a value before
bloc2ss is called. All the calculations are made in state-space
representation even if the linear systems are given in transfer form.

\chapter{Converting Scilab functions to Fortran}

\section{Converting Scilab Functions to Fortran Routines}

Scilab provides a compiler (under development) to transform some Scilab 
functions
into Fortran subroutines. The routines which are thus obtained
make use of the routines which are in the Fortran libraries.
All the basic matrix operations are available.

Let us consider the following Scilab function:
\begin{verbatim}

function [x]=macr(a,b,n)
z=n+m+n,
c(1,1)=z,
c(2,1)=z+1,
c(1,2)=2,
c(2,2)=0,
if n=1 then,
 x=a+b+a,
else,
x=a+b-a'+b,
end,
y=a(3,z+1)-x(z,5),
x=2*x*x*2.21,
sel=1:5,
t=a*b,
for k=1:n,
 z1=z*a(k+1,k)+3,
end,
t(sel,5)=a(2:4,7),
x=[a b;-b' a']

\end{verbatim}

which can be translated into Fortran by using the function {\tt mac2for}.
Each input parameter of the subroutine is described by a list
which contains its type and its dimensions. Here, we have three
input variables {\tt a,b,c} which are, say, {\tt 
double precision, double precision, integer} with dimensions
{\tt (m,m), (m,m), (1,1)}. This information is gathered
in the following list:
\begin{verbatim}
l=list();
l(1)=list('1','m','m');
l(2)=list('1','m','m');
l(3)=list('0','1','1');
\end{verbatim}
The call to {\tt mac2for} is made as follows:
\begin{verbatim}
comp(macr);
mac2for(macr2lst(macr),l)
\end{verbatim}
The output of this command is a string containing a stand-alone Fortran 
subroutine.
\begin{verbatim}
       subroutine macr(a,b,n,x,m,work,iwork)
c!
c  automatic translation
.
.
.
       double precision a(m,m),b(m,m),x(m+m,m+m),y,z1,24(m,m),work(*)
       integer n,m,z,c(2,2),sel(5),k,iwork(*)
.
.
.      
       call dmcopy(b,m,x(1,m+1),m+m,m,m)
       call dmcopy(work(iw1),m,x(m+1,1),m+m,m,m)
       call dmcopy(work(iw1),m,x(m+1,m+1),m+m,m,m)
       return
c
       end


\end{verbatim}
This routine can be linked to Scilab and interactively called.

