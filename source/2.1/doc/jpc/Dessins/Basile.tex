

\section{Fichiers obtenus avec Basile ou Neoclo  ins\'er\'es en \LaTeX} 

\paragraph{}Cette partie traite de l'insertion de dessins Postscript 
 obtenus soit avec Basile soit avec Xneoclo. Soit par exemple {\bf DesBasile.ps} 
 un tel dessin. 

On utilise la commande 
\begin{verbatim}
Blatexpr  xs ys  DesBasile.ps
\end{verbatim}
Pour r\'eduire le dessin de $1/2$ suivant l'axe horizontal et de $0.7$ suivant l'axe vertical ceci donne~:
\begin{verbatim}
Blatexpr 0.5  0.7  DesBasile.ps
\end{verbatim}
On obtient alors deux  fichiers {\bf DesBasile.tex} 
 et {\bf DesBasile.ps.n}. Il suffit d'utiliser dans votre document 
 \LaTeX\, la commande~:
\begin{verbatim}
\input DesBasile.tex 
\dessin{Fonction $\sin(x)$}{deslabel2}
\end{verbatim}
pour obtenir le dessin~(\ref{deslabel2}) dans l'impression finale

\input DesBasile.tex 
\dessin{Fonction $\sin(x)$}{deslabel2}

Votre fichier Postscript original n'est pas modifi\'e.

La fonction Blatexprs vous permet de regrouper directement plusieurs 
dessins issus de Basile ou Neoclo dans une figure \LaTeX. la syntaxe d'utilisation est la suivante~:
\begin{verbatim}
  Blatexprs Bas2 fichier1.ps fichier2.ps .... fichiern
\end{verbatim}
Cette commande va cr\'eer deux fichiers \verb+Bas2.ps+ et \verb+Bas2.tex+. Pour 
obtenir la figure~(\ref{labdeux}) dans Basile utiliser~:
\begin{verbatim}
\input Bas2.tex 
\dessin{Deux Dessins}{labdeux}
\end{verbatim}
\input Bas2.tex 
\dessin{Deux Dessins}{labdeux}



