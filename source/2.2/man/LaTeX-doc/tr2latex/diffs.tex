\documentstyle[tr2tex,11pt]{article}
\begin{document}
\def\troff{{\it troff}}
\def\Troff{{\it Troff}}
\def\ditroff{{\it ditroff}}
\def\Ditroff{{\it Ditroff}}

\title{Differences between \TeX\ and troff typesetters}

Outlined below are the differences between \TeX\ and \troff\/ that
I know from experience with the two languages most of which
obtained while writing {\bf tr2tex}.
Most of them are advantages \TeX\ has over \troff.

There are actually more than one type of \TeX, the most
used ones are \LaTeX and plain \TeX\
\Troff\/ can also be loaded with various macro packages to
produce variations to plain \troff. Also, \ditroff\/ (device independent \troff)
is becoming more and more the standard of \troff.
The following comparison is made mainly between \LaTeX\ and {\bf ms} \troff.
It will be mentioned if in a particular case \ditroff\/ makes a difference.

\begin{itemize}
\item \TeX\ is not system-dependent. \Troff\/ is a Unix tool.

\item In \Troff, tables and equations are handled by preprocessors
while in \TeX\ they are simultaneously processed with text.

\item \Troff\/'s commands have to start at the beginning of the line
and start with a dot. Equation symbols are recognized when delimited by space.
All \TeX\ commands, in math or non-math mode, start with a backslash
and they don't have to be placed at the beginning of the line.

\item \TeX\ and \LaTeX\ commands are more verbose than \troff\/'s.
This can be an advantage or disadvantage depending on the user.

\item \TeX\ processes {\it boxes} such as lines and paragraphs as one unit.
This means it can distribute {\it badness} over that box.
For example, when a spacing between two lines needs to be large, because
a line has large symbols, it will slightly stretch the
line spacing in other lines to make the large spacing not look
too bad. This also makes it avoid orphan lines.

\item The input to \LaTeX\ is a more structured document with scopes. It makes it
easier to proofread. \Troff\/'s input is less structured.

\item Some  \TeX\ drivers make it very easy to include prepared
graphs with the text. Including graphs with \troff\/'s text is more difficult.

\item Many fonts can be loaded in a \TeX\ document (up to 32); in \troff,
the limit is 4 (\ditroff\/ does not have this restriction).

\item Non-math macros are defined just like math macros in \TeX.
In \troff, {\bf define} is used for math definitions and {\bf .de} is
used for non-math macros.
\Troff\/'s macros can be made up from anything while
\TeX\ macros cannot have non-letters which is a nuisance sometimes.

\item There is no limit on the page size in \TeX. The size is limited only by
the output device. \Troff\/'s paper size is limited.

\item \TeX\ is interactive, while \troff\/ is not. However, not many people can
benefit from this feature since they have to be skillful in answering
its questions.

\item \TeX\ and \LaTeX\ give a {\bf l\,o\,n\,g} ambiguous log file that does not
exactly tell what the error is. \Troff\/ does not give error messages.

\item \LaTeX\ automatically numbers equations and figures, etc. A powerful
cross-referencing technique relieves the user from worrying about having
the right sequence of equations' and figures' numbers. This feature
is not available in \troff.

\item The documentation in the \LaTeX\ manual is excellent.
\Troff\/'s documentation is scattered over many references.
The documentation in the \TeX Book is very technical and the average reader
may find it unreadable.

\item At hanauma, we have a previewer for TeX on the SUN, but don't have it
for \troff.
\end{itemize}
\end{document}
