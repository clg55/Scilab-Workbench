\chapter{Introduction}

\section{What is Scilab}
\index{signals!saving and loading}

 Since the introduction of the ``classic'' (Fortran) MATLAB 
by C. Moler in 1982 there have been a number of interactive scientific 


software 
packages which have been developed for system control and 
signal processing applications.

Developed at INRIA, Scilab which is one of the most elaborate of these packages
is freely distributed in source code format (see the file {\tt notice.tex}). 
and runs in Unix/Xwindow environments. 
Its libraries and most of the interpreter are written in Fortran for 
compatibility with numerical librairies. 
The graphic facilities and the Unix interface are written in C. 
Scilab is made of three distinct parts: an interpreter, 
libraries of functions (Scilab procedures) and libraries of Fortran 
and C routines. 
These routines (which, strictly speaking, do not belong to Scilab but
are interactively called by the interpreter) are of 
independent interest and most of them are available through Netlib. 
A few of them have been slightly modified for better compatibility
with Scilab's interpreter.
A useful tool distributed with Scilab is {\tt intersci} which
is a set of routines that allow users to easily add new primitives to
Scilab i.e. to add new modules of Fortran or C code into Scilab making
it easy to customize.
 
A key feature of the MATLAB syntax is its ability to handle matrices: 
basic matrix manipulations such as concatenation, 
extraction or transpose are immediately performed as well as basic operations
such as addition or mutiplication. 
Scilab's aims are the following: first to use the MATLAB syntax 
for more complex objects than numerical matrices,
(e.g. automatic control people may want to manipulate transfer matrices) and
second to be an open interface to numerical libraries (e.g. a specific  
routine can be either called dynamically from Scilab or included 
in the package as a new primitive). 

Scilab is an interactive, interpreted software package (with a syntax
similar to the MATLAB one) which has a number of powerful features:
%
\begin{itemize}
	\item lists
	\item symbolic manipulation of polynomials and polynomial matrices
        \item symbolic manipulation of linear and non-linear systems
	\item non-linear calculation: simulation and optimization
	\item easy interfacing with fortran and C codes
\end{itemize}
%

 The list structure allows a natural symbolic representation of complicated
mathematical objects such as transfer functions and linear systems
(see Section~\ref{s2.5}).

	Polynomials, polynomials matrices and transfer matrices 
are also defined and 
Scilab allows the definition and manipulation of these objects in a
natural, symbolic fashion (see Section~\ref{s2.4}). 
The syntax used for manipulating these matrices
is identical to that used for manipulating constant vectors and matrices.

	Scilab provides a variety of powerful primitives for
the analysis of non-linear systems.  
Integration of explicit and implicit systems can be accomplished 
numerically.  There exist numerical optimization 
facilities for non linear optimization (including
non differentiable optimization), quadratic optimization and 
linear optimization.

	Scilab has an open programming environment where the
creation of functions and libraries of functions is completely in the
hands of the user (see Chapter~\ref{ch4}).    
Functions are recognized as data objects in Scilab and, thus, can be 
manipulated or created as other data objects.  For example, functions
can be passed as arguments of other functions.

In addition Scilab supports a character string data type 
which, in particular, allows the automatic creation of functions.
Matrices of character strings are also manipulated with the same
syntax as ordinary matrices.
	Finally, Scilab is easily interfaced with Fortran or C 
subprograms.  This allows use of standardized 
packages and libraries in the interpreted environment of Scilab.

	The general philosophy of Scilab is to provide the following
sort of computing environment:
\begin{itemize}
   \item To have data types which are varied and flexible.
   \item To have a syntax which is natural and easy to use.
   \item To provide a reasonable set of primitives which serve
	   as a basis for a wide variety of calculations.
   \item To have an open programming environment where new
	   primitives are easily added.
   \item To support library development through ``toolboxes'' of
         functions devoted to specific
	   applications (linear control, signal processing, 
	   networks analysis, non-linear control, etc.)
\end{itemize}

	The objective of this introduction manual is to give the user 
an idea of what Scilab can do. On line documentation on all
Scilab functions is available.




\section{Software Organization}

Scilab is divided into a set of directories. The main directory
\verb!SCIDIR! contains the files {\tt scilab.star} (startup file), the
copyright file {\tt notice.tex}, and the file \verb!configure! 
(see(\ref{install})).
The subdirectories are the following:
\begin{itemize}

\item{{\tt bin} is the directory of the executable files.
The executable code of Scilab, {\tt scilex}, is there. In
particular, this directory contains Shell scripts 
for managing or printing Postscript/\LaTeX~ files produced by Scilab }

\item{{\tt demos} is the directory of Scilab demos. The file 
{\tt alldems.dem} allows to add a new demo which can be run by 
clicking in ``demo''. This directory contains the codes corresponding
to various demos. They are often useful for inspiring new users.
Note that running a graphic function without input parameter
provides an example of use for this function (for instance {\tt
plot2d()} displays an example for using {\tt plot2d}  function). }

\item{{\tt doc} is the directory of the Scilab documentation: \LaTeX, dvi 
and Postscript files. 
This documentation is {\tt SCIDIR/doc/intro/intro.tex}. See 
also the manual (on-line {\tt help}) in the directory {\tt SCIDIR/man}}

\item{{\tt geci} contains source code and binaries for GeCI which is an 
interactive communication manager created in order to
manage remote executions of softwares and allow exchanges of messages
beetwen those softwares. It offers the possibility to exploit numerous
machines on a network, as a virtual computer, by creating a
distributed group of independent softwares. GeCI is used for the link
of Xmetanet with Scilab.}


\item{{\tt imp} is the directory of the routines managing the Postscript files
for print.}

\item{{\tt libs} contains Scilab libraries (compiled code).}

\item{{\tt macros} contains the libraries of Scilab functions
which are available on-line. New libraries can easily be added 
(see the Makefile). This directory is divided into a number of subdirectories
which contain ``Toolboxes'' for control, signal processing, etc... Strictly
speaking Scilab is not organized in toolboxes : functions of a specific
subdirectory can call functions of other directories; so, for example, the 
subdirectory ``signal'' is not self-contained but its functions are all devoted
to signal processing.}

\item{{\tt man} is the directory containing the manual (Unix manual), divided 
into submanuals, corresponding to the on-line help and to 
a \LaTeX format of the Scilab reference manual. The \LaTeX~ code is produced 
by a 
translation of the Unix format Scilab manual (see the subdirectory
{\tt Man-General}).}
To get information about an item enter {\tt help item} in Scilab
or use the help window facility obtained with help button.
To get functions corresponding to a key word enter {\tt apropos
key-word} or use apropos in the help window.

\item{{\tt maple} is the directory which contains the source code of Maple
functions which allow the transfer of Maple objects
into Scilab functions. For efficiency, the transfer
is made through Fortran code generation.}

\item{{\tt routines} is a directory with contains the source code of all
the numerical routines. The subdirectory {\tt default} contains the
source code of routines which are useful to customize Scilab.}
In particular ``external'' routines for ODE/DAE solvers or optimization
should be included here (see e.g. the file {\tt fydot.f}, interface
Scilab-Fortran for ode simulation). Note that if, for example, you want
to solve an ode, the right hand side function can be a Scilab function or a C 
or 
Fortran subroutine. This Fortran subroutine can be dynamically linked
to Scilab or put into the specific file {\tt fydot.f} of the
{\tt default} directory. This function is then inside your version of Scilab.

\item{{\tt intersci} contains the facility provided for add new Fortran or C 
primitives to Scilab.}


\item{{\tt scripts} is the directory which contains the source code of 
shell scripts files.}

\item{{\tt tests} : this directory contains evaluation programs for testing 
Scilab's installation on a machine. The file ``demos.tst'' tests all the 
demos.}

\item{{\tt tmp} : some examples written by users for courses ... have been
added in this directory. }

\item{{\tt util} contains some utility functions for calling Scilab as a
fortran routine or for making the documentation }

\item{{\tt xless} is a file browsing tool developped at Berkeley University.}

\item{{\tt xmetanet} is the directory which contains {\tt xmetanet}, a 
graphic display for networks. Type {\tt metanet()} in Scilab to use it.}


\end{itemize}

\section{Installing Scilab. System Requirements}
\label{install}
Scilab is distributed in source code format; binaries for several popular 
Unix-XWindow systems are also available:
Dec Alpha (OSF 3.0), Dec Mips (ULTRIX 4.2), Sun Sparc stations (Sun OS 4.1.3), 
Sun Sparc stations (Sun Solaris 2.3), HP9000 (HP-UX 9.01),
SGI Mips Irix 5.2, IBM-RS6000 (AIX 3.2), PC 486 (Slackware Linux 2.0.2 -- XFree86 3.1). 

The installation requirements are the following :

- for the source version: Scilab requires approximately 75Mb of 
disk storage to unpack and install (all sources included). 
You need X Window (X11R4 or X11R5, C compiler and Fortran
compiler (or f2c). If you run X11R4, you also need Athena 
Widgets libraries libXaw.a and libXmu.a.
  
- for the binary version: the minimum for running Scilab (without
sources) is about 20 Mb when decompressed.
The versions for Dec Alpha, Dec Mips, Sun OS, HP9000 and 
IBM-RS6000 are statically linked and in principle
do not require a fortran compiler.
The versions for Sun Solaris, SGI and PC Linux are dynamically linked.

The main part of the memory in Scilab is a pile corresponding to
the usual Fortran behaviour. In some parts Scilab is using dynamic allocation 
(in particular for the sparse matrices). 
We have chosen 2 mega-words (double float) for the size of the pile. 
Of course with the source code version a user can easily change this size and
(decrease or) increase it up to the memory of his computer (parameter 
{\tt vsiz} in the file {\tt routines/stack.h}).
 
\section{Scilab at a Glance. A Tutorial}

\subsection{Getting Started}

Scilab is called by typing {\tt scilab} in the directory {\tt SCIDIR/bin}
where {\tt SCIDIR} denotes the directory where Scilab is installed.
Scilab can be launched in another directory with the same command and
a corresponding search path.
This shell script  runs Scilab 
in an Xwindow environment (this script file can be invoked with
specific parameters).
You will immediatly get the Scilab window with the following banner and 
prompt represented by the {\tt -->} : 


\bigskip

\begin{verbatim}
                           ===========
                           S c i l a b
                           ===========
 
 
                  Scilab-2.1  ( 10 February 1995 ) 
                  Copyright (C) 1989-95 INRIA 
 
 
Startup execution:
  loading initial environment
   
 -->

\end{verbatim}


A first contact with Scilab can be made by clicking 
on {\tt Demos} with the left mouse button and clicking then on 
{\tt Introduction to SCILAB }: the 
execution of the session is then done by entering empty lines and can be
stopped with the buttons {\tt Stop} and {\tt Abort}.

  Several libraries
(see the {\tt SCIDIR/scilab.star} file) are automatically loaded.
	
To give the user an idea of some of the capabilities of Scilab
we will give later a sample session in Scilab.\\

\subsection{Editing a command line}

Before the sample session, we briefly present how to edit a command line.
You can enter a command line by typing after the prompt or clicking with the 
mouse on a part on a window and recall it at the prompt in the Scilab
window. At this moment you have the classical Emacs commands at your 
disposal for modifying a command (Ctrl-$<$chr$>$  means hold the CONTROL key 
while typing the character $<$chr$>$), for example:

\bigskip


%
\begin{itemize}
	\item Ctrl-p 	recall previous line
	\item Ctrl-n  	recall next line
	\item Ctrl-b  	move backward one character
	\item Ctrl-f  	move forward one character
	\item Delete  	delete previous character
	\item Ctrl-h  	delete previous character
	\item Ctrl-d  	delete one character (at cursor)
	\item Ctrl-a  	move to beginning of line
	\item Ctrl-e  	move to end of line
	\item Ctrl-k  	delete to the end of the line
	\item Ctrl-u  	cancel current line
	\item Ctrl-y  	yank the text previously deleted
	\item !prev   	recall the last command line which begins by prev
	\item Ctrl-c  	interrupt Scilab and pause after carriage return. 
	(Only functions can be interrupted). Clicking on the stop 
	button enters a Ctrl-c.
\end{itemize}
%

As said before you can also cut and paste using the mouse. This way will be
useful if you type your Scilab  commands in an editor. Another way to ``load'' files
containing Scilab statements
is available with the {\tt File Operations} button.

\subsection{Buttons}
The Scilab window has the following buttons.\\

%
\begin{itemize}
	 \item Stop  	interrupts execution of Scilab and enters in 
{\tt pause} mode
	\item Resume  	continues execution after a {\tt pause} entered as a command or generated by the {\tt Stop} button
	\item Abort 	aborts execution after one (or several) {\tt pause}, 
and returns to top-level prompt
	\item Restart  	clears all variables and executes startup files
	\item Quit  	quits Scilab
	\item Kill  	kills Scilab shell script
	\item Demos  	for interactive run of some demos
	\item File Operations  	facility for loading functions or data into
Scilab, or executing script files. Note the following change w.r.t. the
previous release : using this button implied to 
change the working directory to the directory of the location of the loaded
file. This fact could be confusing and the use of this button does not
change anymore the working directory.
	\item Help : invokes on-line help with the tree of the man and the 
names of the corresponding items. It is possible to type directly 
{\tt help <item>} in the Scilab window.
	\item +- : increases or decreases the number of the active window
	\item Raise Window : exposes the window corresponding to the indicated number and creates one or several windows if necessary
	\item Set Window : the window corresponding to the indicated number becomes active (and creates one or several windows if necessary)
\end{itemize}
%
Note that the command {\tt SCIDIR/bin/scilab -nw} invokes Scilab
in the ``no-window'' mode.

\subsection{Customizing your Scilab}
As usual for many softwares the parameters of the different windows opened by 
Scilab can be easily changed. The way for doing that is to edit the files 
contained in the sub-directory X11-defaults. The first possibility is 
to directly change these files but the same modifications will be needed
for the further releases. The right way is to copy the right lines with the
modifications in the {\tt .Xdefaults} file of one's own home directory.
These modifications are activated by starting again Xwindow or with the
command {\tt xrdb .Xdefaults}. Scilab will read the {\tt .Xdefaults} file: 
the lines of this
file will cancel and replace the corresponding lines of X11-defaults.

A simple example : 

\begin{verbatim}
Xscilab.color*Scrollbar.background:red
Xscilab*vpane.height: 500
Xscilab*vpane.width:  500
\end{verbatim}

in {\tt .Xdefaults} will change the 500x650 window to a square window of 
500x500 and the  
scrollbar background color changes from green to red.


\subsection{Sample Session for Beginners}

We present now some simple commands. A command ends with a semi-colon or a
carriage return. At the carriage return all the commands typed since the last 
prompt are interpreted. The semi-colon before the prompt is optional.  

\noindent.\dotfill.

\verbatok{\Diary d1p1.dia}
\end{verbatim}

Give the values of 1 and 2 to the variables {\tt a} and { \tt A} .  The 
semi-colon at the end of the command suppresses the display of the result.
Note that Scilab is case-sensitive.
Then two commands are processed and the second result is displayed because
it is not followed by a semi-colon. The last command shows how to write a
command on several lines by using ``{\tt ...}''. This sign is only needed
in the on-line typing for avoiding the effect of the carriage return.
 The chain of characters which follow the {\tt //} is not interpreted by
Scilab (it is a comment line).

\noindent.\dotfill.

\verbatok{\Diary d1p2.dia}
\end{verbatim}

We get the list of previously defined variables {\tt a b c A}  together
with the initial environment composed of the different libraries and
some specific ``permanent'' variables.

Below is an example of an expression which mixes constants with existing
variables.  The result is retained in the standard default variable 
{\tt ans}\index{ans@{\tt ans}}.

\noindent.\dotfill.

\verbatok{\Diary d1p3.dia}
\end{verbatim}

Calling a function (or primitive) with a vector argument.  The response
is a complex vector.

\noindent.\dotfill.

\verbatok{\Diary d1p4.dia}
\end{verbatim}

A  more complicated command which creates a polynomial. 

\noindent.\dotfill.

\verbatok{\Diary d1p5.dia}
\end{verbatim}

Definition of a polynomial matrix. The syntax for polynomial matrices
is the same as the one for matrices of constants. Calculation of the
determinant of the polynomial matrix by the {\tt det} function. 


\noindent.\dotfill.

\verbatok{\Diary d1p6.dia}
\end{verbatim}

Definition of a matrix of rational polynomials.  The internal representation
of {\tt f} is a list {\tt list('r',num,den)} where {\tt num} and 
{\tt den} are two matrix polynomials.

\noindent.\dotfill.

\verbatok{\Diary d1p7.dia}
\end{verbatim}


Here we move into a new environment using the command 
{\tt pause}\index{pause@{\tt pause}}
and we obtain the new prompt {\tt -1->} which indicates the level
of the new environment (level 1).  All variables that are available
in the first environment are also available in the new environment.  Variables
created in the new environment can be returned to the original environment
by using {\tt return}\index{return@{\tt return}}.  
Use of {\tt return} without an argument 
destroys all the variables created in the new environment before returning
to the old environment. The {\tt pause} facility is very useful 
for debugging purposes.

\noindent.\dotfill.

\verbatok{\Diary d1p8.dia}
\end{verbatim}


Definition of a rational polynomial by extraction of an element
of the matrix {\tt f} defined above.  This is followed by the evaluation
of the rational polynomial at the vector of complex frequency values defined
by {\tt frequencies}.  The evaluation of the polynomial is done by
the primitive {\tt freq}. {\tt numer(f21)} is the numerator
polynomial and {\tt denom(f21)} is the denominator polynomial.
The  visualization of the resulting evaluation
is made by using the command {\tt plot2d} (see Figure~\ref{f1.1}).

\noindent.\dotfill.

\verbatok{\Diary d1p9.dia}
\end{verbatim}

The function {\tt horner} allows the
user to make a (possibly symbolic) change of variables for a polynomial (for example, to
perform the bilinear transformation as seen above).

\noindent.\dotfill.

\verbatok{\Diary d1p10.dia}
\end{verbatim}

Definition of a linear system in state-space representation.
The function {\tt syslin} defines here the continuous time ({\tt 'c'}) system
{\tt Sl} with state-space matrices ({\tt A,B,C}). The function
{\tt ss2tf} transforms {\tt Sl} into transfer matrix representation.

\noindent.\dotfill.

\verbatok{\Diary d1p11.dia}
\end{verbatim}

Definition of the rational matrix {\tt R}. {\tt Sl} is the
continuous-time linear system with (improper) transfer matrix
R. {\tt tf2ss} puts {\tt Sl} in state-space representation with a
polynomial {\tt D} matrix. Note that linear systems are represented
by special lists (with 7 entries).

\noindent.\dotfill.

\verbatok{\Diary d1p12.dia}
\end{verbatim}

{\tt sl1} is the linear system in transfer matrix representation
obtained by the parallel inter-connection of {\tt Sl} and {\tt 2*Sl +eye}.
The same syntax is valid with {\tt Sl} in state-space representation.

\noindent.\dotfill.

\verbatok{\Diary d1p13.dia}
\end{verbatim}

On-line definition of a function, called {\tt compen} which calculates the 
state space representation
({\tt Cl}) of a linear system controlled by an observer with gain {\tt Ko}
and a controller with gain {\tt Kr}.  Note that matrices are constructed
in block form using other matrices.  The function {\tt compen} is then
compiled by {\tt comp}.

\noindent.\dotfill.

\verbatok{\Diary d1p14.dia}
\end{verbatim}

Call to the function {\tt compen} defined above where the gains were
calculated by a call to the primitive {\tt ppol} which performs pole
placement.
The resulting {\tt f} matrix is displayed and the placement
of its poles is checked using the primitive {\tt spec} which calculates
the eigenvalues of a matrix. (The function {\tt compen} is defined here
on-line by {\tt deff} as an example of function which receive a linear system 
({\tt Sl}) as input and returns a linear system ({\tt Cl}) as output.
In general Scilab functions are defined in files and loaded in Scilab
by {\tt getf}).

\noindent.\dotfill.

\verbatok{\Diary d1p15.dia}
\end{verbatim}

Relation with the Unix environment and an error message: command is not 
interpretable by the system since the
variable {\tt q} is unknown.

\noindent.\dotfill.


\verbatok{\Diary d1p16.dia}
\end{verbatim}

Definition of a column vector of character strings defining a Fortran
subroutine. The routine is compiled (needs a compiler), dynamically 
linked to Scilab, and interactively called by the function {\tt myplus}.

\noindent.\dotfill.

\verbatok{\Diary d1p17.dia}
\end{verbatim}

Definition of a function which calculates a first order vector differential
{\tt f(t,y)}.  This is followed by the definition of the constant {\tt a}
used in the function and the function is compiled.  The primitive {\tt ode}
then integrates the differential equation defined by {\tt f(t,y)}
for $y(0)=\langle 1;0\rangle$ at $t=0$ and where the solution is given
at the time values $t=0,.02,.04,\ldots,20$.  The result is plotted in
Figure~\ref{f1.2} where the first element of the integrated vector is 
plotted against the second element of this vector.

\noindent.\dotfill.

\verbatok{\Diary d1p18.dia}
\end{verbatim}

Definition of a matrix containing character strings. By default, the 
operation of symbolic multiplication of two matrices of character
strings is not defined in Scilab.  The (on-line)
function definition for {\tt \%cmc} defines the multiplication of 
matrices of character strings (note that the double quote is necessary
because the body of the {\tt deff} contains quotes inside of quotes).
The {\tt \%} which begins the function definition for {\tt \%cmc}
allows the definition of an operation which did not previously 
exist in Scilab, and the name {\tt cmc} means ``chain multiply chain''.
This example is not very useful: it is simply given to show
how operations can be defined on complex data structures.

\noindent.\dotfill.

\verbatok{\Diary d1p19.dia}
\end{verbatim}

A simple example which illustrates the passing of a function as an argument
to another function. Scilab functions are objects which may be defined, loaded,
or manipulated as other objects such as matrices or lists.

\noindent.\dotfill.

\begin{verbatim}
-->quit
\end{verbatim}

Exit from Scilab.

\noindent.\dotfill.
\input{\Figdir d1-7.tex}
\dessin{A Simple Response}{f1.1}

\input{\Figdir d1-14.tex}
\dessin{Phase Plot}{f1.2}


