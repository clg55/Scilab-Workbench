
\def\Mhlp{
\begin{flushleft}
{\sl 
\cmarg //$<$$>$=hlp()\\ 
\cmarg \verb@//@ Travaux diriges 1 : bibliotheque de macros \\ 
\cmarg \verb@//@     \\ 
\cmarg \verb@//@    PREMIERE PARTIE \\ 
\cmarg \verb@//@ part1    :  initialise les donnees de la partie 1\\ 
\cmarg \verb@//@ linear   :  dynamique d'un systeme lineaire \\ 
\cmarg \verb@//@ linper   :  systeme lineaire avec perturbation  quadratique \\ 
\cmarg \verb@//@ cycllim  :  dynamique d'un systeme avec cercle limite \\ 
\cmarg \verb@//@    DEUXIEME PARTIE \\ 
\cmarg \verb@//@ part2    :  initialise les donnees de la partie 2.\\ 
\cmarg \verb@//@ bioreact :  modele de bioreacteur\\ 
\cmarg \verb@//@ mu       : taux specifique de croissance (utilise par bioreact)\\ 
\cmarg \verb@//@    TROISIEME PARTIE \\ 
\cmarg \verb@//@ part3    :  initialise les donnees de la partie 3.\\ 
\cmarg \verb@//@ lincom   :  systeme lineaire commande par feedback lineaire d'etat\\ 
\cmarg \verb@//@            (u=-k*x)\\ 
\cmarg \verb@//@!\\ 
\cmarg 0;\\ 
\cmarg //end}
\end{flushleft}}



\def\Mpartu{
\begin{flushleft}
{\sl 
\cmarg //$<$$>$=part1()\\ 
\cmarg //$<$$>$=part1()\\ 
\cmarg \verb@//@ Macro qui initialise les donnees de \\ 
\cmarg \verb@//@ la partie 1\\ 
\cmarg \verb@//@!\\ 
\cmarg \verb@//@ matrice du systeme lineaire \\ 
\cmarg alin=$<$$-$1,0;0,$-$2$>$;\\ 
\cmarg \verb@//@ donnees pour la perturbation quadratique \\ 
\cmarg qeps=1/10.;\\ 
\cmarg q1linper=qeps$\star$$<$1,0;0,1$>$;\\ 
\cmarg q2linper=q1linper;\\ 
\cmarg rlinper=$<$1,$-$1$>$';\\ 
\cmarg $<$alin,qeps,q1linper,q2linper,rlinper$>$={\bf resume}(...\\ 
\cmarg \hspace{1.2cm}alin,qeps,q1linper,q2linper,rlinper);\\ 
\cmarg //end }
\end{flushleft}}



\def\Mlinear{
\begin{flushleft}
{\sl 
\cmarg //$<$xdot$>$=linear(t,x)\\ 
\cmarg //$<$xdot$>$=linear(t,x)\\ 
\cmarg \verb@//@------------------------------------------------\\ 
\cmarg \verb@//@ un systeme lineaire xdot=alin*x\\ 
\cmarg \verb@//@ la matrice du systeme est une variable globale\\ 
\cmarg \verb@//@ alin \\ 
\cmarg \verb@//@-----------------------------------------------\\ 
\cmarg \verb@//@!\\ 
\cmarg xdot=alin$\star$x,\\ 
\cmarg //end}
\end{flushleft}}



\def\Mlinper{
\begin{flushleft}
{\sl 
\cmarg //$<$xdot$>$=linper(t,x)\\ 
\cmarg //$<$xdot$>$=linper(t,x)\\ 
\cmarg \verb@//@-----------------------------------\\ 
\cmarg \verb@//@ systeme lineaire avec perturbation \\ 
\cmarg \verb@//@ quadratique \\ 
\cmarg \verb@//@------------------------------------\\ 
\cmarg \verb@//@!\\ 
\cmarg xdot= alin$\star$x+(1/2)$\star$qeps$\star$$<$(x')$\star$q1linper$\star$x;(x')$\star$q2linper$\star$x$>$+rlinper\\ 
\cmarg //end}
\end{flushleft}}



\def\Mcycllim{
\begin{flushleft}
{\sl 
\cmarg //$<$xdot$>$=cycllim(t,x)\\ 
\cmarg //$<$xdot$>$=cycllim(t,x)\\ 
\cmarg \verb@//@-----------------------------------\\ 
\cmarg \verb@//@ xdot=a*x+qeps(1-||x||**2)x\\ 
\cmarg \verb@//@------------------------------------\\ 
\cmarg \verb@//@!\\ 
\cmarg xdot= $<$ 0 ,$-$1 ; 1 , 0$>$$\star$x+ qeps$\star$(1$-$(x'$\star$x))$\star$x;\\ 
\cmarg //end}
\end{flushleft}}



\def\Mpartd{
\begin{flushleft}
{\sl 
\cmarg //$<$$>$=part2()\\ 
\cmarg //$<$$>$=part2()\\ 
\cmarg \verb@//@ Initialisation de la partie 2\\ 
\cmarg \verb@//@  bioreact \\ 
\cmarg \verb@//@!\\ 
\cmarg k=2.0\\ 
\cmarg debit=1.0\\ 
\cmarg x2in=3;\\ 
\cmarg $<$k,debit,x2in$>$={\bf resume}(k,debit,x2in);\\ 
\cmarg //end}
\end{flushleft}}




\def\Mbioreact{
\begin{flushleft}
{\sl 
\cmarg //$<$xdot$>$=bioreact(t,x)\\ 
\cmarg //$<$xdot$>$=bioreact(t,x)\\ 
\cmarg \verb@//@ modele de bioreacteur\\ 
\cmarg \verb@//@   x(1): concentration de biomasse \\ 
\cmarg \verb@//@   x(2): ''            de sucre \\ 
\cmarg \verb@//@!\\ 
\cmarg xdot(1)=mu(x(2))$\star$x(1)$-$ debit$\star$x(1);\\ 
\cmarg xdot(2)=$-$k$\star$mu(x(2))$\star$x(1)$-$debit$\star$x(2)+debit$\star$x2in;\\ 
\cmarg //end}
\end{flushleft}}



\def\Mmu{
\begin{flushleft}
{\sl 
\cmarg //$<$y$>$=mu(x)\\ 
\cmarg //$<$y$>$=mu(x)\\ 
\cmarg \verb@//@ mu : taux specifique de croissance \\ 
\cmarg \verb@//@!\\ 
\cmarg y=x/(1+x);\\ 
\cmarg //end }
\end{flushleft}}



\def\Mpartt{
\begin{flushleft}
{\sl 
\cmarg //$<$$>$=part3()\\ 
\cmarg //$<$$>$=part3()\\ 
\cmarg \verb@//@ Initialisation de la partie 3\\ 
\cmarg \verb@//@ lincom portfeed champfeed \\ 
\cmarg \verb@//@!\\ 
\cmarg a={\bf eye}(2);\\ 
\cmarg b=$<$1,1$>$';\\ 
\cmarg $<$a,b$>$={\bf resume}(a,b);\\ 
\cmarg //end}
\end{flushleft}}



\def\Mlincom{
\begin{flushleft}
{\sl 
\cmarg //$<$xdot$>$=lincom(t,x,k)\\ 
\cmarg \verb@//@ systeme lineaire commande par \\ 
\cmarg \verb@//@ feedback lineaire d'etat (u=-k*x)\\ 
\cmarg \verb@//@!\\ 
\cmarg xdot= a$\star$x +b$\star$($-$k$\star$x);\\ 
\cmarg //end }
\end{flushleft}}



