\chapter{Graphics}
This section introduces graphics in Scilab. 

\section{The Graphics Window}
On the right side of the {\tt Delete} button, a checker give the number 
{\tt wn} of the graphics window. This number is given at the end of the
title of the window, e.g. {\tt ScilabGraphic1}. The button 
{\tt Raise (Create) Window}
raises the window {\tt wn} if it exists and creates it if not. The 
button {\tt Set (Create) Window} activates the window {\tt wn} if it exists 
and simultaneously creates it if necessary.
The {\tt Delete} button closes the window {\tt wn} if it exists.
The execution of a plotting command automatically creates a window if 
necessary.

There are 4 buttons on the graphics window:
\begin{itemize}
	\item \verb+3D Rot.+: for applying a rotation with the mouse to a
3D plot. This button is inhibited for a 2D plot.
	\item \verb+2D Zoom+: zooming on a 2D plot. This command can be
recursively invoked. This button  has no for a 3D plot.
	\item \verb+UnZoomx+: return to the initial plot (not to the plot
corresponding to the previous zoom in case of multiple zooms).
	\item \verb+File+: this button opens different commands and menus.

The first one is simple : {\tt Clear} simply rubs out the plot of the window.

The next command {\tt Print...} opens a selection panel for getting a
paper output of the plot.

The {\tt Export} command opens a panel selection for getting a copy of the 
plot on a file with a specified format (Postscript, Latex).

The {\tt save} command directly saves the plot on a file with a specified name.
This file can be loaded later in Scilab for replotting.

The {\tt Delete} is the same command (close) than the previous one but simply 
applied to its window. 

\end{itemize}

 
\section{The Media}
There are different graphics devices in Scilab which can be used to send
graphics to windows or paper. The default is \verb+ScilabGraphic0+ window .

The basic Scilab graphics commands are :
%
\begin{itemize}
	\item \verb+driver+: selects a graphic driver
	\item \verb+xinit+: initializes a graphic driver
	\item \verb+xclear+: clears one or more graphic windows
	\item \verb+xpause+: a pause in milliseconds
	\item \verb+xselect+: raises the current graphic window
	\item \verb+xclick+: waits for a mouse click
	\item \verb+xend+: closes a graphic session
\end{itemize}
%

\noindent The different devices are:

%
\begin{itemize}
	\item \verb+X11+	: graphics device for the X11 window system 
	\item \verb+Rec+	: an X Window driver (X11) which also records all the graphic commands. This is the default
	\item \verb+Wdp+	: an X11 driver without recorded graphics; the graphics are done on a pixmap and are send to the graphic window with the
command  {\tt xset("wshow")}. The pixmap is cleared with the command
{\tt xset("wwpc")} or with the usual command {\tt xbasc()}
	\item \verb+Pos+	: graphics device for Postscript printers 
	\item \verb+Fig+	: graphics device for the Xfig system
\end{itemize}
%

In fact, in many cases, one can ignore the existence of drivers and use the
functions \verb+xbasimp+, \verb+xs2fig+ in order to send a graphic 
to a printer or in a file for the \verb+Xfig+ system. For example with :
\begin{verbatim}
-->driver('Pos')
 
-->xinit('foo.ps')
 
-->plot(1:10)
 
-->xend()
 
-->driver('Rec')
 
-->plot(1:10)
 
-->xbasimp(0,'foo1.ps')

\end{verbatim}

\noindent we get two identical Postscript files : {\tt 'foo.ps'} and {\tt 'foo1.ps.0'}
(the appending 0 is the number of the active window where the plot has been 
done).

The default for plotting is the superposition; this can be avoided with
the command {\tt xbasc(window-number)} which clears the recorded Scilab 
graphics command associated with the window \verb+window-number+ and clears 
this window (see the warning below for the difference between {\tt xbasc}
and {\tt xclear}).

If you enlarge a graphic window, the command \verb+xbasr(window-number)+
is executed by Scilab. This command clears the graphic window
\verb+window-number+ and replays the graphic commands associated with it. One 
can
call this function manually, in order to verify the associated
recorded graphics commands. 

Any number of graphics windows can be created with buttons 
or with the commands \verb+xset+ or \verb+xselect+. The environment 
variable DISPLAY can be used to specify an X11 Display or one can use
the \verb+xinit+ function in order to open a graphic window on a
specific display. 

\section{2D Plotting}

\subsection{Basic 2D Plotting}

The simplest 2D plot is {\tt plot(x,y)} or {\tt plot(y)}: this is the plot of 
{\tt y} as function of {\tt x} where {\tt x} and {\tt y} are 2 vectors; if 
{\tt x} is missing, it is replaced by the vector {\tt (1,...,size(y))}.
If {\tt y} is a matrix, its rows are plotted. There are optional arguments.


A first example is given by the following commands and the result is
represented on figure \ref{d7-1}:


\verbatok{diary/simple.dia}
\end{verbatim}

\input{\Figdir d7-1.tex}
\dessin{First example of plotting}{d7-1}

\noindent The generic 2D multiple plot is 


\noindent {\tt  plot2di(str,x,y,[style,strf,leg,rect,nax])} 

with {\tt i=missing,1,2,3,4}.

For the different values of {\tt i} we have:

	{\tt i=missing}	: piecewise linear plotting

	{\tt i=1}	: as previous with possible logarithmic scales

	{\tt i=2}	: piecewise constant drawing style

	{\tt i=3}	: vertical bars

	{\tt i=4}	: arrows style (e.g. ode in a phase space)


-Parameter {\tt str} : it is the string {\tt "abc"} :
 
	{\tt str} is empty if {\tt i} is missing.

	{\tt a=e} : means empty; the values of {\tt x} are not used;
	   (The user must give a dummy value to {\tt x}).

	{\tt a=o} : means one; the x-values are the same for all the curves

	{\tt a=g} : means general. 

	{\tt b=l} : a logarithmic scale is used on the X-axis

	{\tt c=l} : a logarithmic scale is used on the Y-axis

-Parameters {\tt x,y} : two matrices of the same size {\tt [nl,nc]} ({\tt nc} 
is the number of curves and {\tt nl} is the number of points of each curve)

-Parameter {\tt style} : it is a real vector of size {\tt (1,nc)}; the style 
to use for curve j is defined by {\tt size(j)} (when only one curve is drawn 
{\tt style} can specify the style and a position to use for the caption).

-Parameter {\tt strf}  : it is a string of length 3 {\tt "xyz"} corresponding 
to :

	    {\tt x=1} : captions  displayed 

	    {\tt y=1} : the argument {\tt rect} is used to specify the 
		boundaries of the plot.\\  
		{\tt rect=[xmin,ymin,xmax,ymax]}

	    {\tt y=2}	: the boundaries of the	plot are computed 

	    {\tt y=0}	: the current boundaries 

	    {\tt z=1}	: an axis is drawnand  the number of tics can be specified by the {\tt nax} argument

	    {\tt z=2}	: the plot is only surrounded by a box

-Parameter {\tt leg } : it is the string of the captions for the different 
plotted curves . This string is composed of fields separated by the {\tt @} 
symbol: for example  {\tt ``module@phase''} (see example below). These 
strings are
displayed under the plot with  small segments recalling the styles of the 
corresponding curves.

-Parameter {\tt rect } : it is a vector of 4 values specifying the boundaries 
of the plot {\tt rect=[xmin,ymin,xmax,ymax]}.

For different plots the simple commands without any argument show a demo 
(e.g {\tt plot2d3()} ).

\subsection{Specialized 2D Plottings}
%
\begin{itemize}
        \item \verb+champ+	: vector field in $R^{2}$
	\item \verb+fchamp+	: for a vector field in $R^{2}$ defined by a 
function
	\item \verb+fplot2d+	: 2D plotting of a curve described by a 
function
	\item \verb+fgrayplot+	: gray level on a 2D plot
	\item \verb+errbar+	: creates a plot with error bars
\end{itemize}
%

\subsection{Captions and Presentation}
%
\begin{itemize}
	\item \verb+xgrid+	: adds a grid on a 2D graphic
	\item \verb+xtitle+	: adds	title and axis names on	a 2D graphic
	\item \verb+titlepage+	: graphic title page
	\item \verb+plotframe+	: graphic frame with scaling and grid
\end{itemize}
%

The command {\tt plotframe} is used to add a grid and graduations by choosing
the number of graduations and getting rounded numbers.



\subsection{Plotting Some Geometric Figures}
\subsubsection{Polylines Plotting}
\begin{itemize}
	\item \verb+xsegs+	: draws	a set of unconnected segments 
	\item \verb+xrect+	: draws	a single rectangle
	\item \verb+xfrect+	: fills a single rectangle
	\item \verb+xrects+	: fills or draws a set	of rectangles
	\item \verb+xpoly+	: draws	a polyline
	\item \verb+xpolys+	: draws a set of polylines
	\item \verb+xfpoly+	: fills a polygon
	\item \verb+xfpolys+	: fills a set of polygons
	\item \verb+xarrows+	: draws a set of unconnected arrows
	\item \verb+xfrect+	: fills a single rectangle
	\item \verb+xclea+	: erases a rectangle on a graphic window
\end{itemize}
%
\subsubsection{Curves Plotting}
\begin{itemize}
	\item \verb+xarc+	: draws	an ellipsis
	\item \verb+xfarc+	: fills	an ellipsis
	\item \verb+xarcs+	: fills	or draws a set of ellipsis
\end{itemize}
%

\subsection{Writing by Plotting }
%
\begin{itemize}
	\item  \verb+xstring+	: draws a string or a matrix of strings
        \item  \verb+xstringl+	: computes a rectangle which surrounds a string
	\item  \verb+xstringb+	: draws a string in a specified box
	\item  \verb+xnumb+	: draws a set of numbers
\end{itemize}
%

\subsection{Manipulating the Plot and Graphics Context}
\subsubsection{Graphics Context}

Some parameters of the graphics are controlled by a graphic context 
( for example the line thickness) and others are controlled through
graphics arguments. In the first example (Figure~(\ref{d7-1})),
we use all the default arguments.
%
\begin{itemize}
	\item  {\tt xset}	: to set graphic context values.
Some examples of the use of {\tt xset} : 

(i)-{\tt xset("use color",flag)} changes to color or gray plot according to 
the values (1 or 0) of {\tt flag}.

(ii)-{\tt xset("window",window-number)} sets the current window to the window 
{\tt window-number} and creates the window if it doesn't exist.

(iii)-{\tt xset("wpos",x,y)} fixes the position of the upper left point of 
the graphic window.

The choice of the font, the width and height of the window, the driver...
can be done by {\tt xset}.

	\item  {\tt xget}	: to get informations about the current graphic context.
All the values of the parameters fixed by {\tt xset} can be obtained by 
{\tt xget}.

        \item  {\tt xlfont}	: to load a new family of fonts from the XWindow Manager
\end{itemize}
%

\subsubsection{Some Manipulations}

Coordinates transforms :
\begin{itemize}
	\item \verb+isoview+	: isometric scale without window change 

allows an isometric scale in the window of previous plots without changing
the window size (see the example below).
 
	\item \verb+square+	: isometric scale with resizing the window

the window is resized according to the parameters of the command.

        \item \verb+scaling+	: scaling on data
        \item \verb+rotate+	: rotation

\verb+scaling+ and \verb+rotate+ executes respectively an affine transform and
a geometric rotation of a 2-lines-matrix corresponding to the {\tt (x,y) }
values of a set of points.
 
        \item \verb+xgetech, xsetech+	: change of scale inside the graphic window

The current graphic scale can be fixed by a high level plot command. You may
want to get this parameter or to fix it directly : this is the role of 
\verb+xgetech, xsetech+ .

\end{itemize}
%

\section{Some Examples}

We give here a sequence of commands corresponding to the different capabilities
of the 2D plot and the generated figures. 
This first example corresponds to the figure~ \ref{fcapab}. 

\verbatok{\Figdir example1-2.code}
\end{verbatim}

\input{\Figdir example1.tex}
\dessin{Simple 2D Plot}{fsimple}

\input{\Figdir example2.tex}
\dessin{Some Capabilities of 2D Plot}{fcapab}

\input{\Figdir frame.tex}
\dessin{Use of plotframe}{fpframe}

We give now the sequence of the commands for obtaining the 
figure~ \ref{fgeom}.

\verbatok{\Figdir example3.code}
\end{verbatim}

\input{\Figdir example3x.tex}
\dessin{Geometric Graphics and Comments}{fgeom}

\section{3D Plotting}
\subsection{Generic 3D Plotting}
%
\begin{itemize}
	\item  \verb+plot3d+ : 3D plotting of a matrix of points : plot3d(x,y,z) with x,y,z 3 matrices, z being the values for the points with coordinates x,y. Other
arguments are optional
	\item   \verb+plot3d1+ : 3d plotting	of a matrix of points with gray levels
	\item   \verb+fplot3d+ : 3d plotting	of a surface described by a function; z is given by an external z=f(x,y)
	\item   \verb+fplot3d1+ : 3d	plotting of a surface described	by a function with gray	levels
\end{itemize}
%
 
\subsection{Specialized 3D Plotting}
\begin{itemize}
	\item \verb+param3d+	: plots parametric curves in 3d space
	\item \verb+contour+	: level curves for a 3d function given by a 
matrix
	\item \verb+grayplot10+ : gray level on a 2d plot
	\item \verb+fcontour10+ : level curves for a 3d function given by a 
function
	\item \verb+hist3d+	: 3d histogram
	\item \verb+secto3d+	: conversion of a surface description from 
sector to plot3d compatible data
	\item \verb+eval3d+	: evaluates a function on a regular grid. 
(see also feval)
\end{itemize}
%
 

\subsection{Mixing 2D and 3D graphics}

When one uses 3D plotting function, default graphic boundaries are
fixed, but in $R^3$. If one wants to use graphic primitives 
to add informations on 3D graphics, the \verb+geom3d+ function can be
used to convert 3D coordinates to 2D-graphics coordinates. The
figure~ \ref{d7-10} illustrates this feature.

%\def\exemple{d7-10}
\verbatok{\Figdir d7-10.code}
\end{verbatim}

\input{\Figdir d7-10.tex}
\dessin{2D and 3D plot}{d7-10}

\subsection{Sub-windows}
%\def\exemple{d7-8}

It is also possible to make multiple plotting in the same 
graphic window (Figure~\ref{d7-8}).

\verbatok{\Figdir d7-8.code}
\end{verbatim}

\input{\Figdir d7-8.tex}
\dessin{Use of xsetech}{d7-8}


\subsection{A Set of Figures}
%\def\exemple{d7a11}

In this next example we give a brief summary of different plotting
functions for 2D or 3D graphics. The figure~ \ref{d7a11} is 
obtained and inserted in this document with the help of the command 
\verb+Blatexprs+.
 
\verbatok{\Figdir d7a11.gcode}
\end{verbatim}

\input{\Figdir d7a11.tex}
\dessin{Group of figures}{d7a11}

\section{Printing and Inserting Scilab Graphics in \LaTeX}

We describe here the use of programs (Unix shells) for handling Scilab 
graphics and printing the results. These programs are located in the
sub-directory {\tt bin} of Scilab.

\subsection{Window to Paper}
The simplest command to get a paper copy of a plot is to click on the
{\tt print} button of the ScilabGraphic window.

\subsection{Creating a Postscript File}
We have seen at the beginning of this chapter that the simplest way to get 
a Postscript file containing a Scilab plot is :

\begin{verbatim}
-->driver('Pos')
 
-->xinit('foo.ps')
 
-->plot3d1();
 
-->xend()
 
-->driver('Rec')
 
-->plot3d1()
 
-->xbasimp(0,'foo1.ps')

\end{verbatim}


The Postscript files ({\tt foo.ps } or {\tt foo1.ps }) generated by Scilab 
cannot be directly sent to a
Postscript printer, they need a preamble. Therefore, printing is done
through the use of Unix scripts or programs which are provided with
Scilab. The program \verb+Blpr+ is used to print a set of Scilab Graphics 
on a single sheet of paper and is used as follows~:

\begin{verbatim}
 Blpr string-title file1.ps file2.ps > result
\end{verbatim}

You can then print the file  {\tt result} with the classical Unix command :

\begin{verbatim}
lpr -Pprinter-name result
\end{verbatim}

\noindent or use the \verb+ghostview+ Postscript interpreter on your Unix
workstation to see the result.

You can avoid the file {\tt result} with a pipe, replacing {\tt  > result}
by the printing command {\tt | lpr} or the previewing command 
{\tt | ghostview -}.

The best result (best sized figures) is obtained when printing two 
pictures on a single page. 


\subsection{Including a Postscript File in \LaTeX}

The \verb|Blatexpr| Unix shell and the programs \verb+Batexpr2+ and 
\verb+Blatexprs+ are provided in order to help inserting Scilab graphics 
in \LaTeX .

Taking the previous file {\tt foo.ps} and  typing the following statement 
under a Unix shell~:
\begin{verbatim}
Blatexpr 1.0 1.0 foo.ps
\end{verbatim}
creates two files \verb+foo.epsf+ and \verb+foo.tex+. The original
Postscript file is left unchanged. 
To include the figure in a \LaTeX~ document you should insert the following 
\LaTeX~ code in your \LaTeX~ document~:

\begin{verbatim}
\input foo.tex
\dessin{The caption of your picture}{The-label}
\end{verbatim}

You can also see your figure by using  the  Postscript
previewer \verb+ghostview+.

The program \verb+Blatexprs+ does the same thing: it is used to insert
a set of Postscript figures in one \LaTeX picture.

In the following example, we begin by using the Postscript driver {\tt Pos}
and then initialize successively 4 Postscript files 
{\tt fig1.ps, ..., fig4.ps} for
4 different plots and at the end return to the driver {\tt Rec} (X11 driver
with record).

\verbatok{\Diary mulplot.dia}
\end{verbatim}

Then we execute the command :

\begin{verbatim}
  Blatexprs multi fig1.ps fig2.ps fig3.ps fig4.ps
\end{verbatim}

\noindent and we get 2 files {\tt multi.tex} and  {\tt multi.ps} and you can 
include the result in a \LaTeX~ source file by :

\begin{verbatim}
\input multi.tex
\dessin{The caption of your picture}{The-label}
\end{verbatim}

Note that the second line {\tt dessin...} is absolutely necessary and you
have of course to give the absolute path for the input file if you are 
working in another directory (see below). The file {\tt  multi.tex} is only 
the definition of the command {\tt dessin} with 2 parameters : the caption 
and the label; the command {\tt dessin} can be used with one or two
empty arguments {\tt `` ``} if you want to avoid the caption or the label.   

The Postscipt files are inserted in \LaTeX~ with the help of the
\verb+\special+ command and with a syntax that works with the
\verb+dvips+ program.

The program \verb+Blatexpr2+ is used when you want two pictures side
by side.
\begin{verbatim}
  Blatexpr2 Fileres file1.ps file2.ps 
\end{verbatim}

%\def\exemple{d7-12}
\input{\Figdir d7-12.tex}
\dessin{Blatexp2 Example}{d7-12}

It is sometimes convenient to have a main \LaTeX~ document in a directory
and to store all the figures in a subdirectory. The proper way to
insert a  picture file in the main document, when the picture 
is stored in the subdirectory \verb+figures+, is the following~:

\begin{verbatim}
\def\Figdir{figures/} % My figures are in the {\tt figures/ } subdirectory.
\input{\Figdir fig.tex}
\dessin{The caption of you picture}{The-label}
\end{verbatim}

The declaration \verb+\def\Figdir{figures/}+ is used twice, first to
find the file {\tt fig.tex} (when you use \verb+latex+), and
 second to produce a correct pathname for the
\verb+special+ \LaTeX~ command  found in \verb+fig.tex+. (used at
dvips level).

-WARNING : the default driver is {\tt Rec}, i.e. all the graphic commands are
recorded, one record corresponding to one window. The {\tt xbasc()} command
erases the plot on the active window and all the records corresponding to
this window. The {\tt  clear} button has the same effect; the {\tt xclear}
command erases the plot but the record is preserved. So you almost never need
to use the {\tt xbasc()} or {\tt  clear} commands. If you use such a command 
and if you re-do a plot you may have a surprising result (if you forget
that the environment is wiped out); the scale only is preserved and so you 
may have the ``window-plot'' and the ``paper-plot'' completely different.


\subsection{Postscript by Using Xfig}

Another useful way to get a Postscript file for a plot is to use Xfig.
By the simple command {\tt xs2fig(active-window-number,file-name)} you get
a file in Xfig syntax. 

This command needs the use of the driver {\tt Rec}.

The window ScilabGraphic0 being  active, if you enter :

\begin{verbatim}

-->t=-%pi:0.3:%pi;
 
-->plot3d1(t,t,sin(t)'*cos(t),35,45,'X@Y@Z',[2,2,4]);
 
-->xs2fig(0,'demo.fig');   

\end{verbatim}

you get the file {\tt demo.fig} which contains the plot 
of window 0. 

Then you can use Xfig and after the modifications you want, get a Postscript 
file that you can insert in a \LaTeX~ file. The following figure is the result
of Xfig after adding some comments.

\begin{figure}
\fbox{\hspace*{0.cm} \begin{picture}(400.0,350.0)(100.0,200.0)
\hspace{-0.8cm}
\special{psfile=\Figdir demo.ps}
\end{picture}}
\caption{Encapsulated Postscript by Using Xfig}
\label{xfig2ps}
\end{figure}


\subsection{Encapsulated Postscript Files}
As it was said before, the use of {\tt Blatexpr} creates 2 files : a 
{\tt .tex} file to be inserted in the \LaTeX~ file and a {\tt .epsf} file.

It is possible to get the encapsulated Postscript file corresponding to a 
{\tt .ps} file by using the command {\tt BEpsf}.

Notice that the {\tt .epsf} file generated by {\tt Blatexpr} is not an 
encapsulated Postscript file : it has no bounding box and {\tt BEpsf}
generates a {\tt .eps} file which is an encapsulated Postscript file with 
a bounding box.
