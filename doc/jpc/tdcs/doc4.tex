
\documentstyle[modif.article]{article}
\textheight=660pt 
\textwidth=470pt 
\topmargin=-27pt 
\oddsidemargin=0pt 
\evensidemargin=0pt 
\title{Travaux dirig\'es 4}
\author{J.Ph. Chancelier et E. Parent}
\begin{document}
\maketitle 
\def\cmarg{\hspace{1cm}}

\centerline{Le fichier des macros de la s\'eance de T.D. 4}


\def\Mhlp{
\begin{flushleft}
{\sl 
\cmarg //$<$$>$=hlp()\\ 
\cmarg \verb@//@ Travaux diriges : bibliotheque de macros \\ 
\cmarg \verb@//@ macros du T.P 4\\ 
\cmarg \verb@//@ Un probleme de controle optimal sur l'alunisage d'une fusee\\ 
\cmarg \verb@//@\\ 
\cmarg \verb@//@ finit    :  initialise les donnees du td \\ 
\cmarg \verb@//@\\ 
\cmarg \verb@//@ sfusee   :  simule l'evolution de la trajectoire pour une loi \\ 
\cmarg \verb@//@          de commande bang-bang\\ 
\cmarg \verb@//@\\ 
\cmarg \verb@//@ fuseegrad:  effectue des iterations de gradient pour chercher \\ 
\cmarg \verb@//@          une loi de commande optimale\\ 
\cmarg \verb@//@ equad    : integre le systeme d'equations primales-duales\\ 
\cmarg \verb@//@ ubang    : genere une loi de commande bang-bang\\ 
\cmarg \verb@//@ fusee    : dynamique de la fusee\\ 
\cmarg \verb@//@ fuseep   : dynamique de l'etat adjoint\\ 
\cmarg \verb@//@! \\ 
\cmarg 0\\ 
\cmarg //end}
\end{flushleft}}




\def\Mfinit{
\begin{flushleft}
{\sl 
\cmarg //$<$$>$=finit()\\ 
\cmarg \verb@//@ Initialisation de parametres relatif au probleme \\ 
\cmarg \verb@//@ de l'alunissage\\ 
\cmarg \verb@//@k     : acceleration de poussee de la fusee\\ 
\cmarg \verb@//@gamma : acceleration de la pesanteur sur la lune \\ 
\cmarg \verb@//@umax  : debit maximum d'ejection des gaz \\ 
\cmarg \verb@//@mcap  : masse de la capsule \\ 
\cmarg \verb@//@cpen  : penalisation dans la fonction cout de l'etat final \\ 
\cmarg \verb@//@!\\ 
\cmarg k=100\\ 
\cmarg gamma=1\\ 
\cmarg umax = 1\\ 
\cmarg mcap = 10\\ 
\cmarg cpen =100;\\ 
\cmarg $<$k,gamma,umax,mcap,cpen$>$={\bf resume}(k,gamma,umax,mcap,cpen)\\ 
\cmarg //end }
\end{flushleft}}



\def\Msfusee{
\begin{flushleft}
{\sl 
\cmarg //$<$$>$=sfusee(tau,h0,v0,m0,Tf)\\ 
\cmarg //$<$$>$=sfusee(tau,h0,v0,m0,Tf)\\ 
\cmarg \verb@//@  \\ 
\cmarg \verb@//@ calcule la trajectoire de la fusee soumise a \\ 
\cmarg \verb@//@ une commande bang-bang\\ 
\cmarg \verb@//@ tau est la date de commutation \\ 
\cmarg \verb@//@ h0 : la hauteur initiale \\ 
\cmarg \verb@//@ v0 : la vitesse initiale ( negative si chute)\\ 
\cmarg \verb@//@ m0 : la masse initiale ( carburant + capsule)\\ 
\cmarg \verb@//@ Tf : l'horizon d'integration \\ 
\cmarg \verb@//@!\\ 
\cmarg \verb@//@ Premiere phase : chute libre\\ 
\cmarg n=20;\\ 
\cmarg ind=1:n;\\ 
\cmarg t= ind$\star$tau/n;\\ 
\cmarg m(ind)= m0$\star${\bf ones}(1,n);\\ 
\cmarg v(ind)=$-$gamma$\star$(t)+v0$\star${\bf ones}(1,n);\\ 
\cmarg h(ind)= $-$ gamma$\star$(t.$\star$t)/2 +  v0$\star$(t) + h0$\star${\bf ones}(1,n);\\ 
\cmarg m = $<$ m0,m$>$\\ 
\cmarg v=  $<$ v0,v$>$\\ 
\cmarg h= $<$h0,h$>$\\ 
\cmarg t= $<$ 0 t$>$\\ 
\cmarg \verb@//@ Deuxieme phase : frein plein gaz\\ 
\cmarg n1=40;\\ 
\cmarg ind=1:n1;\\ 
\cmarg ind1=0:(n1$-$1)\\ 
\cmarg t1= ind1$\star$Tf/(n1$-$1) +tau$\star$ ((n1$-$1)$\star${\bf ones}(1,n1)$-$ind1)/(n1$-$1);\\ 
\cmarg m1(ind)= ( m0+umax$\star$tau)$\star${\bf ones}(1,n1) $-$umax$\star$(t1);\\ 
\cmarg mcapsul=mcap$\star${\bf ones}(1,n1);\\ 
\cmarg m1={\bf maxi}(m1,mcapsul);\\ 
\cmarg v1(ind)= $-$ gamma$\star$(t1)+ v0$\star${\bf ones}(1,n1) $-$k $\star${\bf log}( m1(ind)/m0);\\ 
\cmarg h1(ind)= $-$ gamma$\star$(t1.$\star$t1)/2 +  v0$\star$(t1) + (h0$-$k$\star$tau)$\star${\bf ones}(1,n1)...\\ 
\cmarg \hspace{1.5cm}+(k/umax)$\star$m1(ind).$\star${\bf log}(m1(ind)/m0)+k$\star$t1;\\ 
\cmarg m=$<$m,m1$>$;\\ 
\cmarg v=$<$v,v1$>$;\\ 
\cmarg h=$<$h,h1$>$;\\ 
\cmarg t=$<$t,t1$>$;\\ 
\cmarg \verb@//@ a revoir \\ 
\cmarg $<$m1,m2$>$={\bf maxi}(h,0$\star$h);\\ 
\cmarg m2=2$\star${\bf ones}(m2)$-$m2;\\ 
\cmarg $<$n1,n2$>$={\bf size}(m2);\\ 
\cmarg ialu=1;\\ 
\cmarg {\bf for} i=1:n2,{\bf if} m2(i)=0,ialu=$<$ialu,i$>$,{\bf end},{\bf end}\\ 
\cmarg ialu=ialu(2);\\ 
\cmarg {\bf write}(\%io(2),t(ialu),'('' Date alunissage'',f7.2)')\\ 
\cmarg {\bf write}(\%io(2),m(ialu),'('' Masse  alunissage'',f7.2)')\\ 
\cmarg {\bf write}(\%io(2),v(ialu),'('' Vitesse alunissage'',f7.2)')\\ 
\cmarg xset("window",0)\\ 
\cmarg xclear();\\ 
\cmarg \verb@//@ Dessin \\ 
\cmarg $<$q1,q2$>$={\bf size}(h)\\ 
\cmarg h1=0$\star${\bf ones}(h);\\ 
\cmarg \verb@//@h1(ialu:q2)=maxi(h)*ones(1,q2-ialu+1);\\ 
\cmarg \verb@//@\\ 
\cmarg plot2d($<$t$>$',$<$h$>$',$<$$-$1;1$>$,"111","distance par rapport au sol",...\\ 
\cmarg \hspace{1.0cm}$<$0,0,tf,{\bf maxi}(h)$>$)\\ 
\cmarg xset("window",1)\\ 
\cmarg {\bf if} xget("window")=0 , xinit('unix:0.0'),xset("window",1),{\bf end}\\ 
\cmarg xclear();\\ 
\cmarg plot2d($<$t;t$>$',$<$v;0$\star$v$>$',$<$$-$1;$-$1$>$,"121",...\\ 
\cmarg \hspace{1.8cm}"vitesse de la fusee (si + v ascent.)@0");\\ 
\cmarg \verb@//@recherche de la date d'arrivee au sol \\ 
\cmarg //end}
\end{flushleft}}



\def\Mfuseegrad{
\begin{flushleft}
{\sl 
\cmarg //$<$ukp1$>$=fuseegrad(niter,ukp1,pasg)\\ 
\cmarg //$<$ukp1$>$=fuseegrad(niter,ukp1,pasg)\\ 
\cmarg \verb@//@ niter : nombre d'iteration de gradient a faire a partir\\ 
\cmarg \verb@//@ de ukp1 solution initiale de taille 135\\ 
\cmarg \verb@//@ pasg  : le pas de gradient choisit\\ 
\cmarg \verb@//@ la valeur renvoyee est la derniere loi de commande obtenue.\\ 
\cmarg \verb@//@ l'optimum s'obtient avec ubang(135,50)\\ 
\cmarg \verb@//@ (optimum du pb non penalise)\\ 
\cmarg \verb@//@!\\ 
\cmarg \verb@//@ fenetres graphiques\\ 
\cmarg xset("window",0);xclear();\\ 
\cmarg xset("window",1);\\ 
\cmarg {\bf if} xget("window")=0 , xinit('unix:0.0'),xset("window",1),{\bf end}\\ 
\cmarg xclear();\\ 
\cmarg xset("window",2);\\ 
\cmarg {\bf if} xget("window")=0 , xinit('unix:0.0'),xset("window",2),{\bf end}\\ 
\cmarg xclear();\\ 
\cmarg \verb@//@ on s'arrete a tf=135\\ 
\cmarg tf=135\\ 
\cmarg $<$n1,n2$>$={\bf size}(ukp1)\\ 
\cmarg {\bf if} n2 $<$$>$135, {\bf print}(\%io(2),"uk doit etre un vecteur (1,135)")\\ 
\cmarg \hspace{1.2cm}{\bf return},{\bf end}\\ 
\cmarg \verb@//@ Calculs de gradient et dessins\\ 
\cmarg {\bf for} i=1:niter, $<$c,xk,pk,ukp1$>$=fcout(tf,ukp1,pasg),\\ 
\cmarg {\bf write}(\%io(2),c,'(''Cout : '',f20.2)');\\ 
\cmarg {\bf write}(\%io(2),xk(3,135),'(''Masse de la fusee : '',f20.2)');\\ 
\cmarg xset("window",0);\\ 
\cmarg tt=1:tf;\\ 
\cmarg plot2d(tt',xk(1,:)',$<$$-$1$>$,"111","Trajectoire",$<$1,0,tf,5200$>$);\\ 
\cmarg xset("window",1);\\ 
\cmarg plot2d(tt',xk(3,:)',$<$$-$1$>$,"111","Evolution de la masse",$<$1,10,tf,100$>$);\\ 
\cmarg xset("window",2);\\ 
\cmarg plot2d(tt',ukp1',$<$$-$1$>$,"111","Commande",$<$1,$-$1,tf,2$>$);\\ 
\cmarg {\bf end}\\ 
\cmarg //end}
\end{flushleft}}



\def\Mequad{
\begin{flushleft}
{\sl 
\cmarg //$<$xk,pk$>$=equad(tf,uk)\\ 
\cmarg //$<$xk,pk$>$=equad(tf,uk)\\ 
\cmarg \verb@//@ pour une loi de commande u(t)  stockee dans uk, calcule \\ 
\cmarg \verb@//@ la trajectoire xk associee et l'etat adjoint pk\\ 
\cmarg \verb@//@!\\ 
\cmarg xk={\bf ode}($<$5220;$-$5;100$>$,1,1:tf,0.01,0.01,fusee);\\ 
\cmarg {\bf deff}('$<$y$>$=gg(t,x)','y=$-$fuseep('+{\bf string}(tf)+'$-$t,x)');\\ 
\cmarg \verb@//@ condition finales pour l'equation adjointe\\ 
\cmarg \verb@//@ en fait on minimise -m(tf)**2+...\\ 
\cmarg pk={\bf ode}($<$2$\star$cpen$\star$xk(1,tf);2$\star$cpen$\star$xk(2,tf);$-$xk(3,tf)$>$,0.01,0.01:tf,...\\ 
\cmarg \hspace{1.0cm}1,1,gg);\\ 
\cmarg pk(1,:)=pk(1,tf:$-$1:1);\\ 
\cmarg pk(2,:)=pk(2,tf:$-$1:1);\\ 
\cmarg pk(3,:)=pk(3,tf:$-$1:1);\\ 
\cmarg //end}
\end{flushleft}}



\def\Mubang{
\begin{flushleft}
{\sl 
\cmarg //$<$uk$>$=ubang(tf,tcom)\\ 
\cmarg //$<$uk$>$=ubang(tf,tcom)\\ 
\cmarg \verb@//@ genere une loi bang-bang qui vaut 0 de 0 a tcom\\ 
\cmarg \verb@//@ et 1 de tcom a tf\\ 
\cmarg \verb@//@!\\ 
\cmarg uk=0$\star${\bf ones}(1,tf)\\ 
\cmarg uk(tcom:tf)=1$\star${\bf ones}(1,tf$-$tcom+1);\\ 
\cmarg //end}
\end{flushleft}}



\def\Mfcout{
\begin{flushleft}
{\sl 
\cmarg //$<$c,xk,pk,ukp1$>$=fcout(tf,uk,pasg)\\ 
\cmarg //$<$c,xk,pk,ukp1$>$=fcout(tf,uk,pasg)\\ 
\cmarg \verb@//@ pour une loi de commande uk\\ 
\cmarg \verb@//@ Calcule la fonction cout que l'on cherche a minimiser \\ 
\cmarg \verb@//@ c = -m(tf) + C*( h(tf)**2 + v(tf)**2)\\ 
\cmarg \verb@//@ (on veut minimiser la consommation et atteindre la \\ 
\cmarg \verb@//@ cible h=0 avec une vitess nulle obtenue par penalisation)\\ 
\cmarg \verb@//@ la trajectoire associee \\ 
\cmarg \verb@//@ Calcule aussi une nouvelle loi de commande par une methode \\ 
\cmarg \verb@//@ de gradient\\ 
\cmarg \verb@//@!\\ 
\cmarg $<$xk,pk$>$=equad(tf,uk);\\ 
\cmarg c= $-$ xk(3,tf) +cpen$\star$(xk(1,tf)$\star$$\star$2 +xk(2,tf)$\star$$\star$2);\\ 
\cmarg grad =   k$\star$pk(2,:)./xk(3,:) $-$pk(3,:);\\ 
\cmarg \verb@//@gradient projete su [0,umax]\\ 
\cmarg ukp1={\bf maxi}({\bf mini}(uk$-$ pasg$\star$grad,umax$\star${\bf ones}(1,tf)),0$\star${\bf ones}(1,tf));\\ 
\cmarg //end}
\end{flushleft}}



\def\Mfusee{
\begin{flushleft}
{\sl 
\cmarg //$<$xdot$>$=fusee(t,x)\\ 
\cmarg //$<$xdot$>$=fusee(t,x)\\ 
\cmarg \verb@//@ dynamique de la fusee \\ 
\cmarg \verb@//@!\\ 
\cmarg xd= x(2);\\ 
\cmarg {\bf if} x(3)$<$= 10, md=0\\ 
\cmarg yd= $-$gamma;\\ 
\cmarg ,{\bf else} md= $-$pousse(t),\\ 
\cmarg yd= k$\star$pousse(t)/x(3)$-$gamma;\\ 
\cmarg {\bf end};\\ 
\cmarg xdot=$<$xd;yd;md$>$;\\ 
\cmarg //end}
\end{flushleft}}



\def\Mpousse{
\begin{flushleft}
{\sl 
\cmarg //$<$ut$>$=pousse(t)\\ 
\cmarg //$<$ut$>$=pousse(t)\\ 
\cmarg \verb@//@ la loi de commande u(t) constante par morceaux \\ 
\cmarg \verb@//@ construite sur la loi de comande discrete uk\\ 
\cmarg \verb@//@!\\ 
\cmarg $<$n1,n2$>$={\bf size}(uk);\\ 
\cmarg ut=uk({\bf mini}({\bf maxi}({\bf ent}(t),1),n2));\\ 
\cmarg //end}
\end{flushleft}}



\def\Mtraj{
\begin{flushleft}
{\sl 
\cmarg //$<$xt$>$=traj(t)\\ 
\cmarg //$<$xt$>$=traj(t)\\ 
\cmarg \verb@//@ approximation constante par morceaux de l'evolution de la masse\\ 
\cmarg \verb@//@ construite sur xk : trajectoire discrete.\\ 
\cmarg \verb@//@! \\ 
\cmarg xt=xk(3,{\bf maxi}({\bf ent}(t),1));\\ 
\cmarg \verb@//@\\ 
\cmarg //end}
\end{flushleft}}



\def\Mfuseep{
\begin{flushleft}
{\sl 
\cmarg //$<$pdot$>$=fuseep(t,p)\\ 
\cmarg //$<$pdot$>$=fuseep(t,p)\\ 
\cmarg \verb@//@equation adjointe\\ 
\cmarg \verb@//@!\\ 
\cmarg xp=0 \\ 
\cmarg yp=$-$p(1);\\ 
\cmarg zp= p(2)$\star$k$\star$pousse(t)/(traj(t)$\star$$\star$2);\\ 
\cmarg pdot=$<$xp;yp;zp$>$\\ 
\cmarg //end}
\end{flushleft}}





\Mhlp

\Mfinit

\section{Simulation de l'alunissage}

\paragraph{}La macro {\bf sfusee} simule l'evolution de la dynamique 
 de la fus\'ee soumise \`a une loi de commande bang-bang.
\[
    \left\{ \begin{array}{lcl}
	\dot{h} & = & v \\
	\dot{v} & = & k*u(t)/m(t) -\gamma\\
	\dot{m} & = & - u(t) \end{array}\right.
\]
\begin{center}
	{\bf Question}
\end{center}
A la main faire varier le point de commutation de la loi bang-bang de fa\c{c}on \`a realiser l'objectif suivant : se poser en h=0 avec une vitesse nulle en 
 ayant consomm\'e le moins possible de carburant ( c'est \`a dire avec le $m(Tf)$ le plus grand possible.).

\Msfusee

\section{Recherche d'une loi de commande optimale par gradient}

\begin{center}
	{\bf Question}
\end{center}

On suppose ici que la date optimale pour se poser est connue ($tf=135$)
et on se propose de minimiser par un m\'ethode de gradient le crit\`ere  
$J(u)$~:
\[
	J(u)= -m(Tf) + {1\over \epsilon} (h(tf)^2+v(tf)^2)
\]
Expliquer l'utilisation de se crit\`ere et retrouver la formule de calcul 
du gradient \`a utiliser. La macro {\bf fuseegrad} effectue des it\'erations 
 de gradient \`a partir d'une loi de commande initiale. 
 Chercher un pas de gradient qui fait diminuer le crit\`ere, regarder
 l'influence de la p\'enalisation des conditions finales.

\Mfuseegrad

\Mequad

\Mubang

\Mfcout

\Mfusee

\Mpousse

\Mtraj

\Mfuseep

\end{document}